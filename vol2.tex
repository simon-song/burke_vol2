%%%%%%%%%%%%%%%%%%%%%%%%%%%%%%%%%%%%%%%%%%%%%%%%%%%%%%%%%%%%%%%%%%%%%%%
\chapter*[Speech on American Taxation]{Speech on American Taxation
\\ \vspace{0.1cm}\large{April 19, 1774}}
%\label{chap:vindication}
\addcontentsline{toc}{chapter}{SPEECH ON AMERICAN TAXATION, April 19, 1774}

%%%%%%%%%%%%%%%%%%%%%%%%%%%%%%%%%%%%%%%%%
\begin{center}
  \textbf{\large PREFACE} \par 
\end{center}
\addcontentsline{toc}{section}{PREFACE}

The following speech has been much the subject of conversation, and the desire of having it printed was last summer very general. The means of gratifying the public curiosity were obligingly furnished from the notes of some gentlemen, members of the last Parliament.

This piece has been for some months ready for the press. But a delicacy, possibly over-scrupulous, has delayed the publication to this time. The friends of administration have been used to attribute a great deal of the opposition to their measures in America to the writings published in England. The editor of this speech kept it back, until all the measures of government have had their full operation, and can be no longer affected, if ever they could have been affected, by any publication.

Most readers will recollect the uncommon pains taken at the beginning of the last session of the last Parliament, and indeed during the whole course of it, to asperse the characters and decry the measures of those who were supposed to be friends to America, in order to weaken the effect of their opposition to the acts of rigor then preparing against the colonies. The speech contains a full refutation of the charges against that party with which Mr. Burke has all along acted. In doing this, he has taken a review of the effects of all the schemes which have been successively adopted in the government of the plantations. The subject is interesting; the matters of information various and important; and the publication at this time, the editor hopes, will not be thought unseasonable.

%%%%%%%%%%%%%%%%%%%%%%%%%%%%%%%%%%%%%%%%%
\begin{center}
  \textbf{\large SPEECH} \par 
\end{center}
\addcontentsline{toc}{section}{SPEECH}

\begin{itpars}
During the last session of the last Parliament, on the 19th of April, 1774, Mr. Rose Fuller, member for Rye, made the following motion:—

"That an act made in the seventh year of the reign of his present Majesty, intituled, 'An act for granting certain duties in the British colonies and plantations in America; for allowing a drawback of the duties of customs upon the exportation from this kingdom of coffee and cocoa-nuts, of the produce of the said colonies or plantations; for discontinuing the drawbacks payable on china earthenware exported to America; and for more effectually preventing the clandestine running of goods in the said colonies and plantations, might be read."

And the same being read accordingly, he moved,—

"That this House will, upon this day sevennight, resolve itself into a committee of the whole House, to take into consideration the duty of three-pence per pound weight upon tea, payable in all his Majesty's dominions in America, imposed by the said act; and also the appropriation of the said duty."

On this latter motion a warm and interesting debate arose, in which Mr. Burke spoke as follows.
\end{itpars}

\vspace{0.3cm}

Sir,—I agree with the honorable gentleman
%[1] 
\footnote{Charles Wolfran Cornwall, Esq., lately appointed one of the Lords of the Treasury.}
who spoke last, that this subject is not new in this House. Very disagreeably to this House, very unfortunately to this nation, and to the peace and prosperity of this whole empire, no topic has been more familiar to us. For nine long years, session after session, we have been lashed round and round this miserable circle of occasional arguments and temporary expedients. I am sure our heads must turn and our stomachs nauseate with them. We have had them in every shape; we have looked at them in every point of view. Invention is exhausted; reason is fatigued; experience has given judgment; but obstinacy is not yet conquered.

The honorable gentleman has made one endeavor more to diversify the form of this disgusting argument. He has thrown out a speech composed almost entirely of challenges. Challenges are serious things; and as he is a man of prudence as well as resolution, I dare say he has very well weighed those challenges before he delivered them. I had long the happiness to sit at the same side of the House, and to agree with the honorable gentleman on all the American questions. My sentiments, I am sure, are well known to him; and I thought I had been perfectly acquainted with his. Though I find myself mistaken, he will still permit me to use the privilege of an old friendship; he will permit me to apply myself to the House under the sanction of his authority, and, on the various grounds he has measured out, to submit to you the poor opinions which I have formed upon a matter of importance enough to demand the fullest consideration I could bestow upon it.

He has stated to the House two grounds of deliberation: one narrow and simple, and merely confined to the question on your paper; the other more large and more complicated,—comprehending the whole series of the Parliamentary proceedings with regard to America, their causes, and their consequences. With regard to the latter ground, he states it as useless, and thinks it may be even dangerous, to enter into so extensive a field of inquiry. Yet, to my surprise, he had hardly laid down this restrictive proposition, to which his authority would have given so much weight, when directly, and with the same authority, he condemns it, and declares it absolutely necessary to enter into the most ample historical detail. His zeal has thrown him a little out of his usual accuracy. In this perplexity, what shall we do, Sir, who are willing to submit to the law he gives us? He has reprobated in one part of his speech the rule he had laid down for debate in the other, and, after narrowing the ground for all those who are to speak after him, he takes an excursion, himself, as unbounded as the subject and the extent of his great abilities.

Sir, when I cannot obey all his laws, I will do the best I can. I will endeavor to obey such of them as have the sanction of his example, and to stick to that rule which, though not consistent with the other, is the most rational. He was certainly in the right, when he took the matter largely. I cannot prevail on myself to agree with him in his censure of his own conduct. It is not, he will give me leave to say, either useless or dangerous. He asserts, that retrospect is not wise; and the proper, the only proper subject of inquiry, is "not how we got into this difficulty, but how we are to get out of it." In other words, we are, according to him, to consult our invention, and to reject our experience. The mode of deliberation he recommends is diametrically opposite to every rule of reason and every principle of good sense established amongst mankind. For that sense and that reason I have always understood absolutely to prescribe, whenever we are involved in difficulties from the measures we have pursued, that we should take a strict review of those measures, in order to correct our errors, if they should be corrigible,—or at least to avoid a dull uniformity in mischief, and the unpitied calamity of being repeatedly caught in the same snare.

Sir, I will freely follow the honorable gentleman in his historical discussion, without the least management for men or measures, further than as they shall seem to me to deserve it. But before I go into that large consideration, because I would omit nothing that can give the House satisfaction, I wish to tread the narrow ground to which alone the honorable gentleman, in one part of his speech, has so strictly confined us.

He desires to know, whether, if we were to repeal this tax, agreeably to the proposition of the honorable gentleman who made the motion, the Americans would not take post on this concession, in order to make a new attack on the next body of taxes; and whether they would not call for a repeal of the duty on wine as loudly as they do now for the repeal of the duty on tea. Sir, I can give no security on this subject. But I will do all that I can, and all that can be fairly demanded. To the experience which the honorable gentleman reprobates in one instant and reverts to in the next, to that experience, without the least wavering or hesitation on my part, I steadily appeal: and would to God there was no other arbiter to decide on the vote with which the House is to conclude this day!

When Parliament repealed the Stamp Act in the year 1766, I affirm, first, that the Americans did not in consequence of this measure call upon you to give up the former Parliamentary revenue which subsisted in that country, or even any one of the articles which compose it. I affirm also, that, when, departing from the maxims of that repeal, you revived the scheme of taxation, and thereby filled the minds of the colonists with new jealousy and all sorts of apprehensions, then it was that they quarrelled with the old taxes as well as the new; then it was, and not till then, that they questioned all the parts of your legislative power, and by the battery of such questions have shaken the solid structure of this empire to its deepest foundations.

Of those two propositions I shall, before I have done, give such convincing, such damning proof, that, however the contrary may be whispered in circles or bawled in newspapers, they never more will dare to raise their voices in this House. I speak with great confidence. I have reason for it. The ministers are with me. They at least are convinced that the repeal of the Stamp Act had not, and that no repeal can have, the consequences which the honorable gentleman who defends their measures is so much alarmed at. To their conduct I refer him for a conclusive answer to his objection. I carry my proof irresistibly into the very body of both Ministry and Parliament: not on any general reasoning growing out of collateral matter, but on the conduct of the honorable gentleman's ministerial friends on the new revenue itself.

The act of 1767, which grants this tea-duty, sets forth in its preamble, that it was expedient to raise a revenue in America for the support of the civil government there, as well as for purposes still more extensive. To this support the act assigns six branches of duties. About two years after this act passed, the ministry, I mean the present ministry, thought it expedient to repeal five of the duties, and to leave (for reasons best known to themselves) only the sixth standing. Suppose any person, at the time of that repeal, had thus addressed the minister:
%[2] 
\footnote{Lord North, then Chancellor of the Exchequer.}
"Condemning, as you do, the repeal of the Stamp Act, why do you venture to repeal the duties upon glass, paper, and painters' colors? Let your pretence for the repeal be what it will, are you not thoroughly convinced that your concessions will produce, not satisfaction, but insolence in the Americans, and that the giving up these taxes will necessitate the giving up of all the rest?" This objection was as palpable then as it is now; and it was as good for preserving the five duties as for retaining the sixth. Besides, the minister will recollect that the repeal of the Stamp Act had but just preceded his repeal; and the ill policy of that measure, (had it been so impolitic as it has been represented,) and the mischiefs it produced, were quite recent. Upon the principles, therefore, of the honorable gentleman, upon the principles of the minister himself, the minister has nothing at all to answer. He stands condemned by himself, and by all his associates old and new, as a destroyer, in the first trust of finance, of the revenues,—and in the first rank of honor, as a betrayer of the dignity of his country.

Most men, especially great men, do not always know their well-wishers. I come to rescue that noble lord out of the hands of those he calls his friends, and even out of his own. I will do him the justice he is denied at home. He has not been this wicked or imprudent man. He knew that a repeal had no tendency to produce the mischiefs which give so much alarm to his honorable friend. His work was not bad in its principle, but imperfect in its execution; and the motion on your paper presses him only to complete a proper plan, which, by some unfortunate and unaccountable error, he had left unfinished.

I hope, Sir, the honorable gentleman who spoke last is thoroughly satisfied, and satisfied out of the proceedings of ministry on their own favorite act, that his fears from a repeal are groundless. If he is not, I leave him, and the noble lord who sits by him, to settle the matter as well as they can together; for, if the repeal of American taxes destroys all our government in America,—he is the man!—and he is the worst of all the repealers, because he is the last.

But I hear it rung continually in my ears, now and formerly,—"The preamble! what will become of the preamble, if you repeal this tax?"—I am sorry to be compelled so often to expose the calamities and disgraces of Parliament. The preamble of this law, standing as it now stands, has the lie direct given to it by the provisionary part of the act: if that can be called provisionary which makes no provision. I should be afraid to express myself in this manner, especially in the face of such a formidable array of ability as is now drawn up before me, composed of the ancient household troops of that side of the House and the new recruits from this, if the matter were not clear and indisputable. Nothing but truth could give me this firmness; but plain truth and clear evidence can be beat down by no ability. The clerk will be so good as to turn to the act, and to read this favorite preamble.

"Whereas it is expedient that a revenue should be raised in your Majesty's dominions in America, for making a more certain and adequate provision for defraying the charge of the administration of justice and support of civil government in such provinces where it shall be found necessary, and towards further defraying the expenses of defending, protecting, and securing the said dominions."

You have heard this pompous performance. Now where is the revenue which is to do all these mighty things? Five sixths repealed,—abandoned,—sunk,—gone,—lost forever. Does the poor solitary tea-duty support the purposes of this preamble? Is not the supply there stated as effectually abandoned as if the tea-duty had perished in the general wreck? Here, Mr. Speaker, is a precious mockery:—a preamble without an act,—taxes granted in order to be repealed,—and the reasons of the grant still carefully kept up! This is raising a revenue in America! This is preserving dignity in England! If you repeal this tax, in compliance with the motion, I readily admit that you lose this fair preamble. Estimate your loss in it. The object of the act is gone already; and all you suffer is the purging the statute-book of the opprobrium of an empty, absurd, and false recital.

It has been said again and again, that the five taxes were repealed on commercial principles. It is so said in the paper in my hand:
%[3] 
\footnote{Lord Hillsborough's Circular Letter to the Governors of the Colonies, concerning the repeal of some of the duties laid in the Act of 1767.}
a paper which I constantly carry about; which I have often used, and shall often use again. What is got by this paltry pretence of commercial principles I know not; for, if your government in America is destroyed by the repeal of taxes, it is of no consequence upon what ideas the repeal is grounded. Repeal this tax, too, upon commercial principles, if you please. These principles will serve as well now as they did formerly. But you know that either your objection to a repeal from these supposed consequences has no validity, or that this pretence never could remove it. This commercial motive never was believed by any man, either in America, which this letter is meant to soothe, or in England, which it is meant to deceive. It was impossible it should: because every man, in the least acquainted with the detail of commerce, must know that several of the articles on which the tax was repealed were fitter objects of duties than almost any other articles that could possibly be chosen,—without comparison more so than the tea that was left taxed, as infinitely less liable to be eluded by contraband. The tax upon red and white lead was of this nature. You have in this kingdom an advantage in lead that amounts to a monopoly. When you find yourself in this situation of advantage, you sometimes venture to tax even your own export. You did so soon after the last war, when, upon this principle, you ventured to impose a duty on coals. In all the articles of American contraband trade, who ever heard of the smuggling of red lead and white lead? You might, therefore, well enough, without danger of contraband, and without injury to commerce, (if this were the whole consideration,) have taxed these commodities. The same may be said of glass. Besides, some of the things taxed were so trivial, that the loss of the objects themselves, and their utter annihilation out of American commerce, would have been comparatively as nothing. But is the article of tea such an object in the trade of England, as not to be felt, or felt but slightly, like white lead, and red lead, and painters' colors? Tea is an object of far other importance. Tea is perhaps the most important object, taking it with its necessary connections, of any in the mighty circle of our commerce. If commercial principles had been the true motives to the repeal, or had they been at all attended to, tea would have been the last article we should have left taxed for a subject of controversy.

Sir, it is not a pleasant consideration, but nothing in the world can read so awful and so instructive a lesson as the conduct of ministry in this business, upon the mischief of not having large and liberal ideas in the management of great affairs. Never have the servants of the state looked at the whole of your complicated interests in one connected view. They have taken things by bits and scraps, some at one time and one pretence, and some at another, just as they pressed, without any sort of regard to their relations or dependencies. They never had any kind of system, right or wrong; but only invented occasionally some miserable tale for the day, in order meanly to sneak out of difficulties into which they had proudly strutted. And they were put to all these shifts and devices, full of meanness and full of mischief, in order to pilfer piecemeal a repeal of an act which they had not the generous courage, when they found and felt their error, honorably and fairly to disclaim. By such management, by the irresistible operation of feeble councils, so paltry a sum as three-pence in the eyes of a financier, so insignificant an article as tea in the eyes of a philosopher, have shaken the pillars of a commercial empire that circled the whole globe.

Do you forget that in the very last year you stood on the precipice of general bankruptcy? Your danger was indeed great. You were distressed in the affairs of the East India Company; and you well know what sort of things are involved in the comprehensive energy of that significant appellation. I am not called upon to enlarge to you on that danger, which you thought proper yourselves to aggravate, and to display to the world with all the parade of indiscreet declamation. The monopoly of the most lucrative trades and the possession of imperial revenues had brought you to the verge of beggary and ruin. Such was your representation; such, in some measure, was your case. The vent of ten millions of pounds of this commodity, now locked up by the operation of an injudicious tax, and rotting in the warehouses of the Company, would have prevented all this distress, and all that series of desperate measures which you thought yourselves obliged to take in consequence of it. America would have furnished that vent, which no other part of the world can furnish but America, where tea is next to a necessary of life, and where the demand grows upon the supply. I hope our dear-bought East India Committees have done us at least so much good, as to let us know, that, without a more extensive sale of that article, our East India revenues and acquisitions can have no certain connection with this country. It is through the American trade of tea that your East India conquests are to be prevented from crushing you with their burden. They are ponderous indeed; and they must have that great country to lean upon, or they tumble upon your head. It is the same folly that has lost you at once the benefit of the West and of the East. This folly has thrown open folding-doors to contraband, and will be the means of giving the profits of the trade of your colonies to every nation but yourselves. Never did a people suffer so much for the empty words of a preamble. It must be given up. For on what principle does it stand? This famous revenue stands, at this hour, on all the debate, as a description of revenue not as yet known in all the comprehensive (but too comprehensive!) vocabulary of finance,—a preambulary tax. It is, indeed, a tax of sophistry, a tax of pedantry, a tax of disputation, a tax of war and rebellion, a tax for anything but benefit to the imposers or satisfaction to the subject.

Well! but whatever it is, gentlemen will force the colonists to take the teas. You will force them? Has seven years' struggle been yet able to force them? Oh, but it seems "we are in the right. The tax is trifling,—in effect it is rather an exoneration than an imposition; three fourths of the duty formerly payable on teas exported to America is taken off,—the place of collection is only shifted; instead of the retention of a shilling from the drawback here, it is three-pence custom paid in America." All this, Sir, is very true. But this is the very folly and mischief of the act. Incredible as it may seem, you know that you have deliberately thrown away a large duty, which you held secure and quiet in your hands, for the vain hope of getting one three fourths less, through every hazard, through certain litigation, and possibly through war.

The manner of proceeding in the duties on paper and glass, imposed by the same act, was exactly in the same spirit. There are heavy excises on those articles, when used in England. On export, these excises are drawn back. But instead of withholding the drawback, which might have been done, with ease, without charge, without possibility of smuggling, and instead of applying the money (money already in your hands) according to your pleasure, you began your operations in finance by flinging away your revenue; you allowed the whole drawback on export, and then you charged the duty, (which you had before discharged,) payable in the colonies, where it was certain the collection would devour it to the bone,—if any revenue were ever suffered to be collected at all. One spirit pervades and animates the whole mass.

Could anything be a subject of more just alarm to America than to see you go out of the plain highroad of finance, and give up your most certain revenues and your clearest interest, merely for the sake of insulting your colonies? No man ever doubted that the commodity of tea could bear an imposition of three-pence. But no commodity will bear three-pence, or will bear a penny, when the general feelings of men are irritated, and two millions of people are resolved not to pay. The feelings of the colonies were formerly the feelings of Great Britain. Theirs were formerly the feelings of Mr. Hampden, when called upon for the payment of twenty shillings. Would twenty shillings have ruined Mr. Hampden's fortune? No! but the payment of half twenty shillings, on the principle it was demanded, would have made him a slave. It is the weight of that preamble, of which you are so fond, and not the weight of the duty, that the Americans are unable and unwilling to bear.

It is, then, Sir, upon the principle of this measure, and nothing else, that we are at issue. It is a principle of political expediency. Your act of 1767 asserts that it is expedient to raise a revenue in America; your act of 1769, which takes away that revenue, contradicts the act of 1767, and, by something much stronger than words, asserts that it is not expedient. It is a reflection upon your wisdom to persist in a solemn Parliamentary declaration of the expediency of any object, for which, at the same time, you make no sort of provision. And pray, Sir, let not this circumstance escape you,—it is very material, —that the preamble of this act which we wish to repeal is not declaratory of a right, as some gentlemen seem to argue it: it is only a recital of the expediency of a certain exercise of a right supposed already to have been asserted; an exercise you are now contending for by ways and means which you confess, though they were obeyed, to be utterly insufficient for their purpose. You are therefore at this moment in the awkward situation of fighting for a phantom,—a quiddity,—a thing that wants, not only a substance, but even a name,—for a thing which is neither abstract right nor profitable enjoyment.

They tell you, Sir, that your dignity is tied to it. I know not how it happens, but this dignify of yours is a terrible incumbrance to you; for it has of late been ever at war with your interest, your equity, and every idea of your policy. Show the thing you contend for to be reason, show it to be common sense, show it to be the means of attaining some useful end, and then I am content to allow it what dignity you please. But what dignity is derived from the perseverance in absurdity is more than ever I could discern. The honorable gentleman has said well,—indeed, in most of his general observations I agree with him,—he says, that this subject does not stand as it did formerly. Oh, certainly not! Every hour you continue on this ill-chosen ground, your difficulties thicken on you; and therefore my conclusion is, remove from a bad position as quickly as you can. The disgrace, and the necessity of yielding, both of them, grow upon you every hour of your delay.

But will you repeal the act, says the honorable gentleman, at this instant, when America is in open resistance to your authority, and that you have just revived your system of taxation? He thinks he has driven us into a corner. But thus pent up, I am content to meet him; because I enter the lists supported by my old authority, his new friends, the ministers themselves. The honorable gentleman remembers that about five years ago as great disturbances as the present prevailed in America on account of the new taxes. The ministers represented these disturbances as treasonable; and this House thought proper, on that representation, to make a famous address for a revival and for a new application of a statute of Henry the Eighth. We besought the king, in that well-considered address, to inquire into treasons, and to bring the supposed traitors from America to Great Britain for trial. His Majesty was pleased graciously to promise a compliance with our request. All the attempts from this side of the House to resist these violences, and to bring about a repeal, were treated with the utmost scorn. An apprehension of the very consequences now stated by the honorable gentleman was then given as a reason for shutting the door against all hope of such an alteration. And so strong was the spirit for supporting the new taxes, that the session concluded with the following remarkable declaration. After stating the vigorous measures which had been pursued, the speech from the throne proceeds:—

"You have assured me of your firm support in the prosecution of them. Nothing, in my opinion, could be more likely to enable the well-disposed among my subjects in that part of the world effectually to discourage and defeat the designs of the factious and seditious than the hearty concurrence of every branch of the legislature in the resolution of maintaining the execution of the laws in every part of my dominions."

After this no man dreamt that a repeal under this ministry could possibly take place. The honorable gentleman knows as well as I, that the idea was utterly exploded by those who sway the House. This speech was made on the ninth day of May, 1769. Five days after this speech, that is, on the thirteenth of the same month, the public circular letter, a part of which I am going to read to you, was written by Lord Hillsborough, Secretary of State for the Colonies. After reciting the substance of the king's speech, he goes on thus:—

"I can take upon me to assure you, notwithstanding insinuations to the contrary from men with factious and seditious views, that his Majesty's present administration have at no time entertained a design to propose to Parliament to lay any further taxes upon America, for the purpose of RAISING A REVENUE; and that it is at present their intention to propose, the next session of Parliament, to take off the duties upon glass, paper, and colors, upon consideration of such duties having been laid contrary to the true principles of commerce.

"These have always been, and still are, the sentiments of his Majesty's present servants, and by which their conduct in respect to America has been governed. And his Majesty relies upon your prudence and fidelity for such an explanation of his measures as may tend to remove the prejudices which have been excited by the misrepresentations of those who are enemies to the peace and prosperity of Great Britain and her colonies, and to reëstablish that mutual confidence and affection upon which the glory and safety of the British empire depend."

Here, Sir, is a canonical boot of ministerial scripture: the general epistle to the Americans. What does the gentleman say to it? Here a repeal is promised,—promised without condition,—and while your authority was actually resisted. I pass by the public promise of a peer relative to the repeal of taxes by this House. I pass by the use of the king's name in a matter of supply, that sacred and reserved right of the Commons. I conceal the ridiculous figure of Parliament hurling its thunders at the gigantic rebellion of America, and then, five days after, prostrate at the feet of those assemblies we affected to despise,—begging them, by the intervention of our ministerial sureties, to receive our submission, and heartily promising amendment. These might have been serious matters formerly; but we are grown wiser than our fathers. Passing, therefore, from the Constitutional consideration to the mere policy, does not this letter imply that the idea of taxing America for the purpose of revenue is an abominable project, when the ministry suppose none but factious men, and with seditious views, could charge them with it? does not this letter adopt and sanctify the American distinction of taxing for a revenue? does it not formally reject all future taxation on that principle? does it not state the ministerial rejection of such principle of taxation, not as the occasional, but the constant opinion of the king's servants? does it not say, (I care not how consistently,) but does it not say, that their conduct with regard to America has been always governed by this policy? It goes a great deal further. These excellent and trusty servants of the king, justly fearful lest they themselves should have lost all credit with the world, bring out the image of their gracious sovereign from the inmost and most sacred shrine, and they pawn him as a security for their promises:—"His Majesty relies on your prudence and fidelity for such an explanation of his measures." These sentiments of the minister and these measures of his Majesty can only relate to the principle and practice of taxing for a revenue; and accordingly Lord Botetourt, stating it as such, did, with great propriety, and in the exact spirit of his instructions, endeavor to remove the fears of the Virginian assembly lest the sentiments which it seems (unknown to the world) had always been those of the ministers, and by which their conduct in respect to America had been governed, should by some possible revolution, favorable to wicked American taxers, be hereafter counteracted. He addresses them in this manner:—

"It may possibly be objected, that, as his Majesty's present administration are not immortal, their successors may be inclined to attempt to undo what the present ministers shall have attempted to perform; and to that objection I can give but this answer: that it is my firm opinion, that the plan I have stated to you will certainly take place, and that it will never be departed from; and so determined am I forever to abide by it, that I will be content to be declared infamous, if I do not, to the last hour of my life, at all times, in all places, and upon all occasions, exert every power with which I either am or ever shall be legally invested, in order to obtain and maintain for the continent of America that satisfaction which I have been authorized to promise this day by the confidential servants of our gracious sovereign, who to my certain knowledge rates his honor so high that he would rather part with his crown than preserve it by deceit."
%[4] 
\footnote{A material point is omitted by Mr. Burke in this speech, viz. the manner in which the continent received this royal assurance. The assembly of Virginia, in their address in answer to Lord Botetourt's speech, express themselves thus:— 'We will not suffer our present hopes, arising from the pleasing prospect your Lordship hath so kindly opened and displayed to us, to be lashed by the bitter reflection that any future administration will entertain a wish to depart from that plan which affords the surest and most permanent foundation of public tranquillity and happiness. No, my Lord, we are sure our most gracious sovereign, under whatever changes may happen in his confidential servants, will remain immutable in the ways of truth and justice, and that he is incapable of deceiving his faithful subjects; and we esteem your Lordship's information not only as warranted, but even sanctified by the royal word.'  }

A glorious and true character! which (since we suffer his ministers with impunity to answer for his ideas of taxation) we ought to make it our business to enable his Majesty to preserve in all its lustre. Let him have character, since ours is no more! Let some part of government be kept in respect!

This epistle was not the letter of Lord Hillsborough solely, though he held the official pen. It was the letter of the noble lord upon the floor,
%[5] 
\footnote{Lord North.}
and of all the king's then ministers, who (with, I think, the exception of two only) are his ministers at this hour. The very first news that a British Parliament heard of what it was to do with the duties which it had given and granted to the king was by the publication of the votes of American assemblies. It was in America that your resolutions were pre-declared. It was from thence that we knew to a certainty how much exactly, and not a scruple more nor less, we were to repeal. We were unworthy to be let into the secret of our own conduct. The assemblies had confidential communications from his Majesty's confidential servants. We were nothing but instruments. Do you, after this, wonder that you have no weight and no respect in the colonies? After this are you surprised that Parliament is every day and everywhere losing (I feel it with sorrow, I utter it with reluctance) that reverential affection which so endearing a name of authority ought ever to carry with it? that you are obeyed solely from respect to the bayonet? and that this House, the ground and pillar of freedom, is itself held up only by the treacherous underpinning and clumsy buttresses of arbitrary power?

If this dignity, which is to stand in the place of just policy and common sense, had been consulted, there was a time for preserving it, and for reconciling it with any concession. If in the session of 1768, that session of idle terror and empty menaces, you had, as you were often pressed to do, repealed these taxes, then your strong operations would have come justified and enforced, in case your concessions had been returned by outrages. But, preposterously, you began with violence; and before terrors could have any effect, either good or bad, your ministers immediately begged pardon, and promised that repeal to the obstinate Americans which they had refused in an easy, good-natured, complying British Parliament. The assemblies, which had been publicly and avowedly dissolved for their contumacy, are called together to receive your submission. Your ministerial directors blustered like tragic tyrants here; and then went mumping with a sore leg in America, canting, and whining, and complaining of faction, which represented them as friends to a revenue from the colonies. I hope nobody in this House will hereafter have the impudence to defend American taxes in the name of ministry. The moment they do, with this letter of attorney in my hand, I will tell them, in the authorized terms, they are wretches "with factious and seditious views," "enemies to the peace and prosperity of the mother country and the colonies," and subverters "of the mutual affection and confidence on which the glory and safety of the British empire depend."

After this letter, the question is no more on propriety or dignity. They are gone already. The faith of your sovereign is pledged for the political principle. The general declaration in the letter goes to the whole of it. You must therefore either abandon the scheme of taxing, or you must send the ministers tarred and feathered to America, who dared to hold out the royal faith for a renunciation of all taxes for revenue. Them you must punish, or this faith you must preserve. The preservation of this faith is of more consequence than the duties on red lead, or white lead, or on broken glass, or atlas-ordinary, or demy-fine, or blue-royal, or bastard, or fools cap, which you have given up, or the three-pence on tea which you retained. The letter went stamped with the public authority of this kingdom. The instructions for the colony government go under no other sanction; and America cannot believe, and will not obey you, if you do not preserve this channel of communication sacred. You are now punishing the colonies for acting on distinctions held out by that very ministry which is here shining in riches, in favor, and in power, and urging the punishment of the very offence to which they had themselves been the tempters.

Sir, if reasons respecting simply your own commerce, which is your own convenience, were the sole grounds of the repeal of the five duties, why does Lord Hillsborough, in disclaiming in the name of the king and ministry their ever having had an intent to tax for revenue, mention it as the means "of reëstablishing the confidence and affection of the colonies?" Is it a way of soothing others, to assure them that you will take good care of yourself? The medium, the only medium, for regaining their affection and confidence is that you will take off something oppressive to their minds. Sir, the letter strongly enforces that idea: for though the repeal of the taxes is promised on commercial principles, yet the means of counteracting the "insinuations of men with factious and seditious views" is by a disclaimer of the intention of taxing for revenue, as a constant, invariable sentiment and rule of conduct in the government of America.

I remember that the noble lord on the floor, not in a former debate to be sure, (it would be disorderly to refer to it, I suppose I read it somewhere,) but the noble lord was pleased to say, that he did not conceive how it could enter into the head of man to impose such taxes as those of 1767: I mean those taxes which he voted for imposing, and voted for repealing,—as being taxes, contrary to all the principles of commerce, laid on British manufactures.

I dare say the noble lord is perfectly well read, because the duty of his particular office requires he should be so, in all our revenue laws, and in the policy which is to be collected out of them. Now, Sir, when he had read this act of American revenue, and a little recovered from his astonishment, I suppose he made one step retrograde (it is but one) and looked at the act which stands just before in the statute-book. The American revenue act is the forty-fifth chapter; the other to which I refer is the forty-fourth of the same session. These two acts are both to the same purpose: both revenue acts; both taxing out of the kingdom; and both taxing British manufactures exported. As the forty-fifth is an act for raising a revenue in America, the forty-fourth is an act for raising a revenue in the Isle of Man. The two acts perfectly agree in all respects, except one. In the act for taxing the Isle of Man the noble lord will find, not, as in the American act, four or fire articles, but almost the whole body of British manufactures, taxed from two and a half to fifteen per cent, and some articles, such as that of spirits, a great deal higher. You did not think it uncommercial to tax the whole mass of your manufactures, and, let me add, your agriculture too; for, I now recollect, British corn is there also taxed up to ten per cent, and this too in the very head-quarters, the very citadel of smuggling, the Isle of Man. Now will the noble lord condescend to tell me why he repealed the taxes on your manufactures sent out to America, and not the taxes on the manufactures exported to the Isle of Man? The principle was exactly the same, the objects charged infinitely more extensive, the duties without comparison higher. Why? Why, notwithstanding all his childish pretexts, because the taxes were quietly submitted to in the Isle of Man, and because they raised a flame in America. Your reasons were political, not commercial. The repeal was made, as Lord Hillsborough's letter well expresses it, to regain "the confidence and affection of the colonies, on which the glory and safety of the British empire depend." A wise and just motive, surely, if ever there was such. But the mischief and dishonor is, that you have not done what you had given the colonies just cause to expect, when your ministers disclaimed the idea of taxes for a revenue. There is nothing simple, nothing manly, nothing ingenuous, open, decisive, or steady, in the proceeding, with regard either to the continuance or the repeal of the taxes. The whole has an air of littleness and fraud. The article of tea is slurred over in the circular letter, as it were by accident: nothing is said of a resolution either to keep that tax or to give it up. There is no fair dealing in any part of the transaction.

If you mean to follow your true motive and your public faith, give up your tax on tea for raising a revenue, the principle of which has, in effect, been disclaimed in your name, and which produces you no advantage,—no, not a penny. Or, if you choose to go on with a poor pretence instead of a solid reason, and will still adhere to your cant of commerce, you have ten thousand times more strong commercial reasons for giving up this duty on tea than for abandoning the five others that you have already renounced.

The American consumption of teas is annually, I believe, worth 300,000l. at the least farthing. If you urge the American violence as a justification of your perseverance in enforcing this tax, you know that you can never answer this plain question,—Why did you repeal the others given in the same act, whilst the very same violence subsisted?—But you did not find the violence cease upon that concession.—No! because the concession was far short of satisfying the principle which Lord Hillsborough had abjured, or even the pretence on which the repeal of the other taxes was announced; and because, by enabling the East India Company to open a shop for defeating the American resolution not to pay that specific tax, you manifestly showed a hankering after the principle of the act which you formerly had renounced. Whatever road you take leads to a compliance with this motion. It opens to you at the end of every visto. Your commerce, your policy, your promises, your reasons, your pretences, your consistency, your inconsistency,—all jointly oblige you to this repeal.

But still it sticks in our throats, if we go so far, the Americans will go farther.—We do not know that. We ought, from experience, rather to presume the contrary. Do we not know for certain, that the Americans are going on as fast as possible, whilst we refuse to gratify them? Can they do more, or can they do worse, if we yield this point? I think this concession will rather fix a turnpike to prevent their further progress. It is impossible to answer for bodies of men. But I am sure the natural effect of fidelity, clemency, kindness in governors is peace, good-will, order, and esteem, on the part of the governed. I would certainly, at least, give these fair principles a fair trial; which, since the making of this act to this hour, they never have had.

Sir, the honorable gentleman having spoken what he thought necessary upon the narrow part of the subject, I have given him, I hope, a satisfactory answer. He next presses me, by a variety of direct challenges and oblique reflections, to say something on the historical part. I shall therefore, Sir, open myself fully on that important and delicate subject: not for the sake of telling you a long story, (which, I know, Mr. Speaker, you are not particularly fond of,) but for the sake of the weighty instruction that, I flatter myself, will necessarily result from it. It shall not be longer, if I can help it, than so serious a matter requires.

Permit me then, Sir, to lead your attention very far back,—back to the Act of Navigation, the cornerstone of the policy of this country with regard to its colonies. Sir, that policy was, from the beginning, purely commercial; and the commercial system was wholly restrictive. It was the system of a monopoly. No trade was let loose from that constraint, but merely to enable the colonists to dispose of what, in the course of your trade, you could not take,—or to enable them to dispose of such articles as we forced upon them, and for which, without some degree of liberty, they could not pay. Hence all your specific and detailed enumerations; hence the innumerable checks and counterchecks; hence that infinite variety of paper chains by which you bind together this complicated system of the colonies. This principle of commercial monopoly runs through no less than twenty-nine acts of Parliament, from the year 1660 to the unfortunate period of 1764.

In all those acts the system of commerce is established as that from whence alone you proposed to make the colonies contribute (I mean directly and by the operation of your superintending legislative power) to the strength of the empire. I venture to say, that, during that whole period, a Parliamentary revenue from thence was never once in contemplation. Accordingly, in all the number of laws passed with regard to the plantations, the words which distinguish revenue laws specifically as such were, I think, premeditately avoided. I do not say, Sir, that a form of words alters the nature of the law, or abridges the power of the lawgiver. It certainly does not. How ever, titles and formal preambles are not always idle words; and the lawyers frequently argue from them. I state these facts to show, not what was your right, but what has been your settled policy. Our revenue laws have usually a title, purporting their being grants; and the words "give and grant" usually precede the enacting parts. Although duties were imposed on America in acts of King Charles the Second, and in acts of King William, no one title of giving "an aid to his Majesty," or any other of the usual titles to revenue acts, was to be found in any of them till 1764; nor were the words "give and grant" in any preamble until the sixth of George the Second. However, the title of this act of George the Second, notwithstanding the words of donation, considers it merely as a regulation of trade; "An act for the better securing of the trade of his Majesty's sugar colonies in America." This act was made on a compromise of all, and at the express desire of a part, of the colonies themselves. It was therefore in some measure with their consent; and having a title directly purporting only a commercial regulation, and being in truth nothing more, the words were passed by, at a time when no jealousy was entertained, and things were little scrutinized. Even Governor Bernard, in his second printed letter, dated in 1763, gives it as his opinion, that "it was an act of prohibition, not of revenue." This is certainly true, that no act avowedly for the purpose of revenue, and with the ordinary title and recital taken together, is found in the statute-book until the year I have mentioned: that is, the year 1764. All before this period stood on commercial regulation and restraint. The scheme of a colony revenue by British authority appeared, therefore, to the Americans in the light of a great innovation. The words of Governor Bernard's ninth letter, written in November, 1765, state this idea very strongly. "It must," says he, "have been supposed such an innovation as a Parliamentary taxation would cause a great alarm, and meet with much opposition in most parts of America; it was quite new to the people, and had no visible bounds set to it." After stating the weakness of government there, he says, "Was this a time to introduce so great a novelty as a Parliamentary inland taxation in America?" Whatever the right might have been, this mode of using it was absolutely new in policy and practice.

Sir, they who are friends to the schemes of American revenue say, that the commercial restraint is full as hard a law for America to live under. I think so, too. I think it, if uncompensated, to be a condition of as rigorous servitude as men can be subject to. But America bore it from the fundamental Act of Navigation until 1764. Why? Because men do bear the inevitable constitution of their original nature with all its infirmities. The Act of Navigation attended the colonies from their infancy, grow with their growth, and strengthened with their strength They were confirmed in obedience to it even more by usage than by law. They scarcely had remembered a time when they were not subject to such restraint. Besides, they were indemnified for it by a pecuniary compensation. Their monopolist happened to be one of the richest men in the world. By his immense capital (primarily employed, not for their benefit, but his own) they were enabled to proceed with their fisheries, their agriculture, their shipbuilding, (and their trade, too, within the limits,) in such a manner as got far the start of the slow, languid operations of unassisted Nature. This capital was a hot-bed to them. Nothing in the history of mankind is like their progress. For my part, I never cast an eye on their flourishing commerce, and their cultivated and commodious life, but they seem to me rather ancient nations grown to perfection through a long series of fortunate events, and a train of successful industry, accumulating wealth in many centuries, than the colonies of yesterday,—than a set of miserable outcasts a few years ago, not so much sent as thrown out on the bleak and barren shore of a desolate wilderness three thousand miles from all civilized intercourse.

All this was done by England whilst England pursued trade and forgot revenue. You not only acquired commerce, but you actually created the very objects of trade in America; and by that creation you raised the trade of this kingdom at least fourfold. America had the compensation of your capital, which made her bear her servitude. She had another compensation, which you are now going to take away from her. She had, except the commercial restraint, every characteristic mark of a free people in all her internal concerns. She had the image of the British Constitution. She had the substance. She was taxed by her own representatives. She chose most of her own magistrates. She paid them all. She had in effect the sole disposal of her own internal government. This whole state of commercial servitude and civil liberty, taken together, is certainly not perfect freedom; but comparing it with the ordinary circumstances of human nature, it was an happy and a liberal condition.

I know, Sir, that great and not unsuccessful pains have been taken to inflame our minds by an outcry, in this House, and out of it, that in America the Act of Navigation neither is or never was obeyed. But if you take the colonies through, I affirm that its authority never was disputed,—that it was nowhere disputed for any length of time,—and, on the whole, that it was well observed. Wherever the act pressed hard, many individuals, indeed, evaded it. This is nothing. These scattered individuals never denied the law, and never obeyed it. Just as it happens, whenever the laws of trade, whenever the laws of revenue, press hard upon the people in England: in that case all your shores are full of contraband. Your right to give a monopoly to the East India Company, your right to lay immense duties on French brandy, are not disputed in England. You do not make this charge on any man. But you know that there is not a creek from Pentland Frith to the Isle of Wight in which they do not smuggle immense quantities of teas, East India goods, and brandies. I take it for granted that the authority of Governor Bernard in this point is indisputable. Speaking of these laws, as they regarded that part of America now in so unhappy a condition, he says, "I believe they are nowhere better supported than in this province: I do not pretend that it is entirely free from a breach of these laws, but that such a breach, if discovered, is justly punished." What more can you say of the obedience to any laws in any country? An obedience to these laws formed the acknowledgment, instituted by yourselves, for your superiority, and was the payment you originally imposed for your protection.

Whether you were right or wrong in establishing the colonies on the principles of commercial monopoly, rather than on that of revenue, is at this day a problem of mere speculation. You cannot have both by the same authority. To join together the restraints of an universal internal and external monopoly with an universal internal and external taxation is an unnatural union,—perfect, uncompensated slavery. You have long since decided for yourself and them; and you and they have prospered exceedingly under that decision.

This nation, Sir, never thought of departing from that choice until the period immediately on the close of the last war. Then a scheme of government, new in many things, seemed to have been adopted. I saw, or thought I saw, several symptoms of a great change, whilst I sat in your gallery, a good while before I had the honor of a seat in this House. At that period the necessity was established of keeping up no less than twenty new regiments, with twenty colonels capable of seats in this House. This scheme was adopted with very general applause from all sides, at the very time that, by your conquests in America, your danger from foreign attempts in that part of the world was much lessened, or indeed rather quite over. When this huge increase of military establishment was resolved on, a revenue was to be found to support so great a burden. Country gentlemen, the great patrons of economy, and the great resisters of a standing armed force, would not have entered with much alacrity into the vote for so large and so expensive an army, if they had been very sure that they were to continue to pay for it. But hopes of another kind were held out to them; and in particular, I well remember that Mr. Townshend, in a brilliant harangue on this subject, did dazzle them by playing before their eyes the image of a revenue to be raised in America.

Here began to dawn the first glimmerings of this new colony system. It appeared more distinctly afterwards, when it was devolved upon a person to whom, on other accounts, this country owes very great obligations. I do believe that he had a very serious desire to benefit the public. But with no small study of the detail, he did not seem to have his view, at least equally, carried to the total circuit of our affairs. He generally considered his objects in lights that were rather too detached. Whether the business of an American revenue was imposed upon him altogether,—whether it was entirely the result of his own speculation, or, what is more probable, that his own ideas rather coincided with the instructions he had received,—certain it is, that, with the best intentions in the world, he first brought this fatal scheme into form, and established it by Act of Parliament.

No man can believe, that, at this time of day, I mean to lean on the venerable memory of a great man, whose loss we deplore in common. Our little party differences have been long ago composed; and I have acted more with him, and certainly with more pleasure with him, than ever I acted against him. Undoubtedly Mr. Grenville was a first-rate figure in this country. With a masculine understanding, and a stout and resolute heart, he had an application undissipated and unwearied. He took public business, not as a duty which he was to fulfil, but as a pleasure he was to enjoy; and he seemed to have no delight out of this House, except in such things as some way related to the business that was to be done within it. If he was ambitious, I will say this for him, his ambition was of a noble and generous strain. It was to raise himself, not by the low, pimping politics of a court, but to win his way to power through the laborious gradations of public service, and to secure himself a well-earned rank in Parliament by a thorough knowledge of its constitution and a perfect practice in all its business.

Sir, if such a man fell into errors, it must be from defects not intrinsical; they must be rather sought in the particular habits of his life, which, though they do not alter the groundwork of character, yet tinge it with their own hue. He was bred in a profession. He was bred to the law, which is, in my opinion, one of the first and noblest of human sciences,—a science which does more to quicken and invigorate the understanding than all the other kinds of learning put together; but it is not apt, except in persons very happily born, to open and to liberalize the mind exactly in the same proportion. Passing from that study, he did not go very largely into the world, but plunged into business,—I mean into the business of office, and the limited and fixed methods and forms established there. Much knowledge is to be had, undoubtedly, in that line; and there is no knowledge which is not valuable. But it may be truly said, that men too much conversant in office are rarely minds of remarkable enlargement. Their habits of office are apt to give them a turn to think the substance of business not to be much more important than the forms in which it is conducted. These forms are adapted to ordinary occasions; and therefore persons who are nurtured in office do admirably well as long as things go on in their common order; but when the high-roads are broken up, and the waters out, when a new and troubled scene is opened, and the file affords no precedent, then it is that a greater knowledge of mankind, and a far more extensive comprehension of things is requisite, than ever office gave, or than office can ever give. Mr. Grenville thought better of the wisdom and power of human legislation than in truth it deserves. He conceived, and many conceived along with him, that the flourishing trade of this country was greatly owing to law and institution, and not quite so much to liberty; for but too many are apt to believe regulation to be commerce, and taxes to be revenue. Among regulations, that which stood first in reputation was his idol: I mean the Act of Navigation. He has often professed it to be so. The policy of that act is, I readily admit, in many respects well understood. But I do say, that, if the act be suffered to run the full length of its principle, and is not changed and modified according to the change of times and the fluctuation of circumstances, it must do great mischief, and frequently even defeat its own purpose.

After the war, and in the last years of it, the trade of America had increased far beyond the speculations of the most sanguine imaginations. It swelled out on every side. It filled all its proper channels to the brim. It overflowed with a rich redundance, and breaking its banks on the right and on the left, it spread out upon some places where it was indeed improper, upon others where it was only irregular. It is the nature of all greatness not to be exact; and great trade will always be attended with considerable abuses. The contraband will always keep pace in some measure with the fair trade. It should stand as a fundamental maxim, that no vulgar precaution ought to be employed in the cure of evils which are closely connected with the cause of our prosperity. Perhaps this great person turned his eyes somewhat less than was just towards the incredible increase of the fair trade, and looked with something of too exquisite a jealousy towards the contraband. He certainly felt a singular degree of anxiety on the subject, and even began to act from that passion earlier than is commonly imagined. For whilst he was First Lord of the Admiralty, though not strictly called upon in his official line, he presented a very strong memorial to the Lords of the Treasury, (my Lord Bute was then at the head of the board,) heavily complaining of the growth of the illicit commerce in America. Some mischief happened even at that time from this over-earnest zeal. Much greater happened afterwards, when it operated with greater power in the highest department of the finances. The bonds of the Act of Navigation were straitened so much that America was on the point of having no trade, either contraband or legitimate. They found, under the construction and execution then used, the act no longer tying, but actually strangling them. All this coming with new enumerations of commodities, with regulations which in a manner put a stop to the mutual coasting intercourse of the colonies, with the appointment of courts of admiralty under various improper circumstances, with a sudden extinction of the paper currencies, with a compulsory provision for the quartering of soldiers,—the people of America thought themselves proceeded against as delinquents, or, at best, as people under suspicion of delinquency, and in such a manner as they imagined their recent services in the war did not at all merit. Any of these innumerable regulations, perhaps, would not have alarmed alone; some might be thought reasonable; the multitude struck them with terror.

But the grand manoeuvre in that business of new regulating the colonies was the fifteenth act of the fourth of George the Third, which, besides containing several of the matters to which I have just alluded, opened a new principle. And here properly began the second period of the policy of this country with regard to the colonies, by which the scheme of a regular plantation Parliamentary revenue was adopted in theory and settled in practice: a revenue not substituted in the place of, but superadded to, a monopoly; which monopoly was enforced at the same time with additional strictness, and the execution put into military hands.

This act, Sir, had for the first time the title of "granting duties in the colonies and plantations of America," and for the first time it was asserted in the preamble "that it was just and necessary that a revenue should be raised there"; then came the technical words of "giving and granting." And thus a complete American revenue act was made in all the forms, and with a full avowal of the right, equity, policy, and even necessity, of taxing the colonies, without any formal consent of theirs. There are contained also in the preamble to that act these very remarkable words,—the Commons, \&c., "being desirous to make some provision in the present session of Parliament towards raising the said revenue." By these words it appeared to the colonies that this act was but a beginning of sorrows,—that every session was to produce something of the same kind,—that we were to go on, from day to day, in charging them with such taxes as we pleased, for such a military force as we should think proper. Had this plan been pursued, it was evident that the provincial assemblies, in which the Americans felt all their portion of importance, and beheld their sole image of freedom, were ipso facto annihilated. This ill prospect before them seemed to be boundless in extent and endless in duration. Sir, they were not mistaken. The ministry valued themselves when this act passed, and when they gave notice of the Stamp Act, that both of the duties came very short of their ideas of American taxation. Great was the applause of this measure here. In England we cried out for new taxes on America, whilst they cried out that they were nearly crushed with those which the war and their own grants had brought upon them.

Sir, it has been said in the debate, that, when the first American revenue act (the act in 1764, imposing the port-duties) passed, the Americans did not object to the principle. It is true they touched it but very tenderly. It was not a direct attack. They were, it is true, as yet novices,—as yet unaccustomed to direct attacks upon any of the rights of Parliament. The duties were port-duties, like those they had been accustomed to bear,—with this difference, that the title was not the same, the preamble not the same, and the spirit altogether unlike. But of what service is this observation to the cause of those that make it? It is a full refutation of the pretence for their present cruelty to America; for it shows, out of their own mouths, that our colonies were backward to enter into the present vexatious and ruinous controversy.

There is also another circulation abroad, (spread with a malignant intention, which I cannot attribute to those who say the same thing in this House,) that Mr. Grenville gave the colony agents an option for their assemblies to tax themselves, which they had refused. I find that much stress is laid on this, as a fact. However, it happens neither to be true nor possible. I will observe, first, that Mr. Grenville never thought fit to make this apology for himself in the innumerable debates that were had upon the subject. He might have proposed to the colony agents, that they should agree in some mode of taxation as the ground of an act of Parliament. But he never could have proposed that they should tax themselves on requisition, which is, the assertion of the day. Indeed, Mr. Grenville well knew that the colony agents could have no general powers to consent to it; and they had no time to consult their assemblies for particular powers, before he passed his first revenue act. If you compare dates, you will find it impossible. Burdened as the agents knew the colonies were at that time, they could not give the least hope of such grants. His own favorite governor was of opinion that the Americans were not then taxable objects.

"Nor was the time less favorable to the equity of such a taxation. I don't mean to dispute the reasonableness of America contributing to the charges of Great Britain, when she is able; nor, I believe, would the Americans themselves have disputed it at a proper time and season. But it should be considered, that the American governments themselves have, in the prosecution of the late war, contracted very large debts, which it will take some years to pay off, and in the mean time occasion very burdensome taxes for that purpose only. For instance, this government, which is as much beforehand as any, raises every year 37,500l. sterling for sinking their debt, and must continue it for four years longer at least before it will be clear."

These are the words of Governor Bernard's letter to a member of the old ministry, and which he has since printed.

Mr. Grenville could not have made this proposition to the agents for another reason. He was of opinion, which he has declared in this House an hundred times, that the colonies could not legally grant any revenue to the crown, and that infinite mischiefs would be the consequence of such a power. When Mr. Grenville had passed the first revenue act, and in the same session had made this House come to a resolution for laying a stamp-duty on America, between that time and the passing the Stamp Act into a law he told a considerable and most respectable merchant, a member of this House, whom I am truly sorry I do not now see in his place, when he represented against this proceeding, that, if the stamp-duty was disliked, he was willing to exchange it for any other equally productive,—but that, if he objected to the Americans being taxed by Parliament, he might save himself the trouble of the discussion, as he was determined on the measure. This is the fact, and, if you please, I will mention a very unquestionable authority for it.

Thus, Sir, I have disposed of this falsehood. But falsehood has a perennial spring. It is said that no conjecture could be made of the dislike of the colonies to the principle. This is as untrue as the other. After the resolution of the House, and before the passing of the Stamp Act, the colonies of Massachusetts Bay and New York did send remonstrances objecting to this mode of Parliamentary taxation. What was the consequence? They were suppressed, they were put under the table, notwithstanding an order of Council to the contrary, by the ministry which composed the very Council that had made the order; and thus the House proceeded to its business of taxing without the least regular knowledge of the objections which were made to it. But to give that House its due, it was not over-desirous to receive information or to hear remonstrance. On the 15th of February, 1765, whilst the Stamp Act was under deliberation, they refused with scorn even so much as to receive four petitions presented from so respectable colonies as Connecticut, Rhode Island, Virginia, and Carolina, besides one from the traders of Jamaica. As to the colonies, they had no alternative left to them but to disobey, or to pay the taxes imposed by that Parliament, which was not suffered, or did not suffer itself, even to hear them remonstrate upon the subject.

This was the state of the colonies before his Majesty thought fit to change his ministers. It stands upon no authority of mine. It is proved by uncontrovertible records. The honorable gentleman has desired some of us to lay our hands upon our hearts and answer to his queries upon the historical part of this consideration, and by his manner (as well as my eyes could discern it) he seemed to address himself to me.

Sir, I will answer him as clearly as I am able, and with great openness: I have nothing to conceal. In the year sixty-five, being in a very private station, far enough from any line of business, and not having the honor of a seat in this House, it was my fortune, unknowing and unknown to the then ministry, by the intervention of a common friend, to become connected with a very noble person, and at the head of the Treasury Department. It was, indeed, in a situation of little rank and no consequence, suitable to the mediocrity of my talents and pretensions,—but a situation near enough to enable me to see, as well as others, what was going on; and I did see in that noble person such sound principles, such an enlargement of mind, such clear and sagacious sense, and such unshaken fortitude, as have bound me, as well as others much better than me, by an inviolable attachment to him from that time forward. Sir, Lord Rockingham very early in that summer received a strong representation from many weighty English merchants and manufacturers, from governors of provinces and commanders of men-of-war, against almost the whole of the American commercial regulations,—and particularly with regard to the total ruin which was threatened to the Spanish trade. I believe, Sir, the noble lord soon saw his way in this business. But he did not rashly determine against acts which it might be supposed were the result of much deliberation. However, Sir, he scarcely began to open the ground, when the whole veteran body of office took the alarm. A violent outcry of all (except those who knew and felt the mischief) was raised against any alteration. On one hand, his attempt was a direct violation of treaties and public law; on the other, the Act of Navigation and all the corps of trade-laws were drawn up in array against it.

The first step the noble lord took was, to have the opinion of his excellent, learned, and ever-lamented friend, the late Mr. Yorke, then Attorney-General, on the point of law. When he knew that formally and officially which in substance he had known before, he immediately dispatched orders to redress the grievance. But I will say it for the then minister, he is of that constitution of mind, that I know he would have issued, on the same critical occasion, the very same orders, if the acts of trade had been, as they were not, directly against him, and would have cheerfully submitted to the equity of Parliament for his indemnity.

On the conclusion of this business of the Spanish trade, the news of the troubles on account of the Stamp Act arrived in England. It was not until the end of October that these accounts were received. No sooner had the sound of that mighty tempest reached us in England, than the whole of the then opposition, instead of feeling humbled by the unhappy issue of their measures, seemed to be infinitely elated, and cried out, that the ministry, from envy to the glory of their predecessors, were prepared to repeal the Stamp Act. Near nine years after, the honorable gentleman takes quite opposite ground, and now challenges me to put my hand to my heart and say whether the ministry had resolved on the repeal till a considerable time after the meeting of Parliament. Though I do not very well know what the honorable gentleman wishes to infer from the admission or from the denial of this fact on which he so earnestly adjures me, I do put my hand on my heart and assure him that they did not come to a resolution directly to repeal. They weighed this matter as its difficulty and importance required. They considered maturely among themselves. They consulted with all who could give advice or information. It was not determined until a little before the meeting of Parliament; but it was determined, and the main lines of their own plan marked out, before that meeting. Two questions arose. (I hope I am not going into a narrative troublesome to the House.)

[A cry of "Go on, go on!"]

The first of the two considerations was, whether the repeal should be total, or whether only partial,—taking out everything burdensome and productive, and reserving only an empty acknowledgment, such as a stamp on cards or dice. The other question was, on what principle the act should be repealed. On this head also two principles were started. One, that the legislative rights of this country with regard to America were not entire, but had certain restrictions and limitations. The other principle was, that taxes of this kind were contrary to the fundamental principles of commerce on which the colonies were founded, and contrary to every idea of political equity,—by which equity we are bound as much as possible to extend the spirit and benefit of the British Constitution to every part of the British dominions. The option, both of the measure and of the principle of repeal, was made before the session; and I wonder how any one can read the king's speech at the opening of that session, without seeing in that speech both the repeal and the Declaratory Act very sufficiently crayoned out. Those who cannot see this can see nothing.

Surely the honorable gentleman will not think that a great deal less time than was then employed ought to have been spent in deliberation, when he considers that the news of the troubles did not arrive till towards the end of October. The Parliament sat to fill the vacancies on the 14th day of December, and on business the 14th of the following January.

Sir, a partial repeal, or, as the bon-ton of the court then was, a modification, would have satisfied a timid, unsystematic, procrastinating ministry, as such a measure has since done such a ministry. A modification is the constant resource of weak, undeciding minds. To repeal by a denial of our right to tax in the preamble (and this, too, did not want advisers) would have cut, in the heroic style, the Gordian knot with a sword. Either measure would have cost no more than a day's debate. But when the total repeal was adopted, and adopted on principles of policy, of equity, and of commerce, this plan made it necessary to enter into many and difficult measures. It became necessary to open a very largo field of evidence commensurate to these extensive views. But then this labor did knights' service. It opened the eyes of several to the true state of the American affairs; it enlarged their ideas; it removed prejudices; and it conciliated the opinions and affections of men. The noble lord who then took the lead in administration, my honorable friend
%[6] 
\footnote{Mr. Dowdeswell.}
under me, and a right honorable gentleman
%[7] 
\footnote{General Conway.}
(if he will not reject his share, and it was a large one, of this business) exerted the most laudable industry in bringing before you the fullest, most impartial, and least garbled body of evidence that ever was produced to this House. I think the inquiry lasted in the committee for six weeks; and at its conclusion, this House, by an independent, noble, spirited, and unexpected majority, by a majority that will redeem all the acts ever done by majorities in Parliament, in the teeth of all the old mercenary Swiss of state, in despite of all the speculators and augurs of political events, in defiance of the whole embattled legion of veteran pensioners and practised instruments of a court, gave a total repeal to the Stamp Act, and (if it had been so permitted) a lasting peace to this whole empire.

I state, Sir, these particulars, because this act of spirit and fortitude has lately been, in the circulation of the season, and in some hazarded declamations in this House, attributed to timidity. If, Sir, the conduct of ministry, in proposing the repeal, had arisen from timidity with regard to themselves, it would have been greatly to be condemned. Interested timidity disgraces as much in the cabinet as personal timidity does in the field. But timidity with regard to the well-being of our country is heroic virtue. The noble lord who then conducted affairs, and his worthy colleagues, whilst they trembled at the prospect of such distresses as you have since brought upon yourselves, were not afraid steadily to look in the face that glaring and dazzling influence at which the eyes of eagles have blenched. He looked in the face one of the ablest, and, let me say, not the most scrupulous oppositions, that perhaps ever was in this House; and withstood it, unaided by even one of the usual supports of administration. He did this, when he repealed the Stamp Act. He looked in the face a person he had long respected and regarded, and whose aid was then particularly wanting: I mean Lord Chatham. He did this when he passed the Declaratory Act.

It is now given out, for the usual purposes, by the usual emissaries, that Lord Rockingham did not consent to the repeal of this act until he was bullied into it by Lord Chatham; and the reporters have gone so far as publicly to assert, in an hundred companies, that the honorable gentleman under the gallery,
%[8] 
\footnote{General Conway.}
who proposed the repeal in the American committee, had another set of resolutions in his pocket, directly the reverse of those he moved. These artifices of a desperate cause are at this time spread abroad, with incredible care, in every part of the town, from the highest to the lowest companies; as if the industry of the circulation were to make amends for the absurdity of the report.

Sir, whether the noble lord is of a complexion to be bullied by Lord Chatham, or by any man, I must submit to those who know him. I confess, when I look back to that time, I consider him as placed in one of the most trying situations in which, perhaps, any man ever stood. In the House of Peers there were very few of the ministry, out of the noble lord's own particular connection, (except Lord Egmont, who acted, as far as I could discern, an honorable and manly part,) that did not look to some other future arrangement, which warped his politics. There were in both Houses new and menacing appearances, that might very naturally drive any other than a most resolute minister from his measure or from his station. The household troops openly revolted. The allies of ministry (those, I mean, who supported some of their measures, but refused responsibility for any) endeavored to undermine their credit, and to take ground that must be fatal to the success of the very cause which they would be thought to countenance. The question of the repeal was brought on by ministry in the committee of this House in the very instant when it was known that more than one court negotiation was carrying on with the heads of the opposition. Everything, upon every side, was full of traps and mines. Earth below shook; heaven above menaced; all the elements of ministerial safety were dissolved. It was in the midst of this chaos of plots and counterplots, it was in the midst of this complicated warfare against public opposition and private treachery, that the firmness of that noble person was put to the proof. He never stirred from his ground: no, not an inch. He remained fixed and determined, in principle, in measure, and in conduct. He practised no managements. He secured no retreat. He sought no apology.

I will likewise do justice—I ought to do it—to the honorable gentleman who led us in this House.
%[9] 
\footnote{General Conway.}
Far from the duplicity wickedly charged on him, he acted his part with alacrity and resolution. We all felt inspired by the example he gave us, down even to myself, the weakest in that phalanx. I declare for one, I knew well enough (it could not be concealed from anybody) the true state of things; but, in my life, I never came with so much spirits into this House. It was a time for a man to act in. We had powerful enemies; but we had faithful and determined friends, and a glorious cause. We had a great battle to fight; but we had the means of fighting: not as now, when our arms are tied behind us. We did fight that day, and conquer.

I remember, Sir, with a melancholy pleasure, the situation of the honorable gentleman
%[10] 
\footnote{General Conway.}
who made the motion for the repeal: in that crisis, when the whole trading interest of this empire, crammed into your lobbies, with a trembling and anxious expectation, waited, almost to a winter's return of light, their fate from your resolutions. When at length you had determined in their favor, and your doors thrown open showed them the figure of their deliverer in the well-earned triumph of his important victory, from the whole of that grave multitude there arose an involuntary burst of gratitude and transport. They jumped upon him like children on a long absent father. They clung about him as captives about their redeemer. All England, all America, joined in his applause. Nor did he seem insensible to the best of all earthly rewards, the love and admiration of his fellow-citizens. Hope elevated and joy brightened his crest. I stood near him; and his face, to use the expression of the Scripture of the first martyr, "his face was as if it had been the face of an angel." I do not know how others feel; but if I had stood in that situation, I never would have exchanged it for all that kings in their profusion could bestow. I did hope that that day's danger and honor would have been a bond to hold us all together forever. But, alas! that, with other pleasing visions, is long since vanished.

Sir, this act of supreme magnanimity has been represented as if it had been a measure of an administration that, having no scheme of their own, took a middle line, pilfered a bit from one side and a bit from the other. Sir, they took no middle lines. They differed fundamentally from the schemes of both parties; but they preserved the objects of both. They preserved the authority of Great Britain; they preserved the equity of Great Britain. They made the Declaratory Act; they repealed the Stamp Act. They did both fully: because the Declaratory Act was without qualification; and the repeal of the Stamp Act total. This they did in the situation I have described.

Now, Sir, what will the adversary say to both these acts? If the principle of the Declaratory Act was not good, the principle we are contending for this day is monstrous. If the principle of the repeal was not good, why are we not at war for a real, substantial, effective revenue? If both were bad, why has this ministry incurred all the inconveniences of both and of all schemes? why have they enacted, repealed, enforced, yielded, and now attempt to enforce again?

Sir, I think I may as well now as at any other time speak to a certain matter of fact not wholly unrelated to the question under your consideration. We, who would persuade you to revert to the ancient policy of this kingdom, labor under the effect of this short current phrase, which the court leaders have given out to all their corps, in order to take away the credit of those who would prevent you from that frantic war you are going to wage upon your colonies. Their cant is this: "All the disturbances in America have been created by the repeal of the Stamp Act." I suppress for a moment my indignation at the falsehood, baseness, and absurdity of this most audacious assertion. Instead of remarking on the motives and character of those who have issued it for circulation, I will clearly lay before you the state of America, antecedently to that repeal, after the repeal, and since the renewal of the schemes of American taxation.

It is said, that the disturbances, if there were any before the repeal, were slight, and without difficulty or inconvenience might have been suppressed. For an answer to this assertion I will send you to the great author and patron of the Stamp Act, who, certainly meaning well to the authority of this country, and fully apprised of the state of that, made, before a repeal was so much as agitated in this House, the motion which is on your journals, and which, to save the clerk the trouble of turning to it, I will now read to you. It was for an amendment to the address of the 17th of December, 1765.

"To express our just resentment and indignation at the outrageous tumults and insurrections which have been excited and carried on in North America, and at the resistance given, by open and rebellious force, to the execution of the laws in that part of his Majesty's dominions; to assure his Majesty, that his faithful Commons, animated with the warmest duty and attachment to his royal person and government, ... will firmly and effectually support his Majesty in all such measures as shall be necessary for preserving and securing the legal dependence of the colonies upon this their mother country," \&c., \&c.

Here was certainly a disturbance preceding the repeal,—such a disturbance as Mr. Grenville thought necessary to qualify by the name of an insurrection, and the epithet of a rebellious force: terms much stronger than any by which those who then supported his motion have ever since thought proper to distinguish the subsequent disturbances in America. They were disturbances which seemed to him and his friends to justify as strong a promise of support as hath been usual to give in the beginning of a war with the most powerful and declared enemies. When the accounts of the American governors came before the House, they appeared stronger even than the warmth of public imagination had painted them: so much stronger, that the papers on your table bear me out in saying that all the late disturbances, which have been at one time the minister's motives for the repeal of five out of six of the new court taxes, and are now his pretences for refusing to repeal that sixth, did not amount—why do I compare them?—no, not to a tenth part of the tumults and violence which prevailed long before the repeal of that act.

Ministry cannot refuse the authority of the commander-in-chief, General Gage, who, in his letter of the 4th of November, from New York, thus represents the state of things:—

"It is difficult to say, from the highest to the lowest, who has not been accessory to this insurrection, either by writing, or mutual agreements to oppose the act, by what they are pleased to term all legal opposition to it. Nothing effectual has been proposed, either to prevent or quell the tumult. The rest of the provinces are in the same situation, as to a positive refusal to take the stamps, and threatening those who shall take them to plunder and murder them; and this affair stands in all the provinces, that, unless the act from its own nature enforce itself, nothing but a very considerable military force can do it."

It is remarkable, Sir, that the persons who formerly trumpeted forth the most loudly the violent resolutions of assemblies, the universal insurrections, the seizing and burning the stamped papers, the forcing stamp officers to resign their commissions under the gallows, the rifling and pulling down of the houses of magistrates, and the expulsion from their country of all who dared to write or speak a single word in defence of the powers of Parliament,—these very trumpeters are now the men that represent the whole as a mere trifle, and choose to date all the disturbances from the repeal of the Stamp Act, which put an end to them. Hear your officers abroad, and let them refute this shameless falsehood, who, in all their correspondence, state the disturbances as owing to their true causes, the discontent of the people from the taxes. You have this evidence in your own archives; and it will give you complete satisfaction, if you are not so far lost to all Parliamentary ideas of information as rather to credit the lie of the day than the records of your own House.

Sir, this vermin of court reporters, when they are forced into day upon one point, are sure to burrow in another: but they shall have no refuge; I will make them bolt out of all their holes. Conscious that they must be baffled, when they attribute a precedent disturbance to a subsequent measure, they take other ground, almost as absurd, but very common in modern practice, and very wicked; which is, to attribute the ill effect of ill-judged conduct to the arguments which had been used to dissuade us from it. They say, that the opposition made in Parliament to the Stamp Act, at the time of its passing, encouraged the Americans to their resistance. This has even formally appeared in print in a regular volume from an advocate of that faction,—a Dr. Tucker. This Dr. Tucker is already a dean, and his earnest labors in this vineyard will, I suppose, raise him to a bishopric. But this assertion, too, just like the rest, is false. In all the papers which have loaded your table, in all the vast crowd of verbal witnesses that appeared at your bar, witnesses which were indiscriminately produced from both sides of the House, not the least hint of such a cause of disturbance has ever appeared. As to the fact of a strenuous opposition to the Stamp Act, I sat as a stranger in your gallery when the act was under consideration. Far from anything inflammatory, I never heard a more languid debate in this House. No more than two or three gentlemen, as I remember, spoke against the act, and that with great reserve and remarkable temper. There was but one division in the whole progress of the bill; and the minority did not reach to more than 39 or 40. In the House of Lords I do not recollect that there was any debate or division at all. I am sure there was no protest. In fact, the affair passed with so very, very little noise, that in town they scarcely knew the nature of what you were doing. The opposition to the bill in England never could have done this mischief, because there scarcely ever was less of opposition to a bill of consequence.

Sir, the agents and distributors of falsehoods have, with their usual industry, circulated another lie, of the same nature with the former. It is this: that the disturbances arose from the account which had been received in America of the change in the ministry. No longer awed, it seems, with the spirit of the former rulers, they thought themselves a match for what our calumniators choose to qualify by the name of so feeble a ministry as succeeded. Feeble in one sense these men certainly may be called: for, with all their efforts, and they have made many, they have not been able to resist the distempered vigor and insane alacrity with which you are rushing to your ruin. But it does so happen, that the falsity of this circulation is (like the rest) demonstrated by indisputable dates and records.

So little was the change known in America, that the letters of your governors, giving an account of these disturbances long after they had arrived at their highest pitch, were all directed to the old ministry, and particularly to the Earl of Halifax, the Secretary of State corresponding with the colonies, without once in the smallest degree intimating the slightest suspicion of any ministerial revolution whatsoever. The ministry was not changed in England until the 10th day of July, 1765. On the 14th of the preceding June, Governor Fauquier, from Virginia, writes thus,—and writes thus to the Earl of Halifax:—"Government is set at defiance, not having strength enough in her hands to enforce obedience to the laws of the community.—The private distress, which every man feels, increases the general dissatisfaction at the duties laid by the Stamp Act, which breaks out and shows itself upon every trifling occasion." The general dissatisfaction had produced some time before, that is, on the 29th of May, several strong public resolves against the Stamp Act; and those resolves are assigned by Governor Bernard as the cause of the insurrections in Massachusetts Bay, in his letter of the 15th of August, still addressed to the Earl of Halifax; and he continued to address such accounts to that minister quite to the 7th of September of the same year. Similar accounts, and of as late a date, were sent from other governors, and all directed to Lord Halifax. Not one of these letters indicates the slightest idea of a change, either known or even apprehended.

Thus are blown away the insect race of courtly falsehoods! Thus perish the miserable inventions of the wretched runners for a wretched cause, which they have fly-blown into every weak and rotten part of the country, in vain hopes, that, when their maggots had taken wing, their importunate buzzing might sound something like the public voice!

Sir, I have troubled you sufficiently with the state of America before the repeal. Now I turn to the honorable gentleman who so stoutly challenges us to tell whether, after the repeal, the provinces were quiet. This is coming home to the point. Here I meet him directly, and answer most readily, They were quiet. And I, in my turn, challenge him to prove when, and where, and by whom, and in what numbers, and with what violence, the other laws of trade, as gentlemen assert, were violated in consequence of your concession, or that even your other revenue laws were attacked. But I quit the vantage-ground on which I stand, and where I might leave the burden of the proof upon him: I walk down upon the open plain, and undertake to show that they were not only quiet, but showed many unequivocal marks of acknowledgment and gratitude. And to give him every advantage, I select the obnoxious colony of Massachusetts Bay, which at this time (but without hearing her) is so heavily a culprit before Parliament: I will select their proceedings even under circumstances of no small irritation. For, a little imprudently, I must say, Governor Bernard mixed in the administration of the lenitive of the repeal no small acrimony arising from matters of a separate nature. Yet see, Sir, the effect of that lenitive, though mixed with these bitter ingredients,—and how this rugged people can express themselves on a measure of concession.

"If it is not now in our power," (say they, in their address to Governor Bernard,) "in so full a manner as will be expected, to show our respectful gratitude to the mother country, or to make a dutiful, affectionate return to the indulgence of the King and Parliament, it shall be no fault of ours; for this we intend, and hope shall be able fully to effect."

Would to God that this temper had been cultivated, managed, and set in action! Other effects than those which we have since felt would have resulted from it. On the requisition for compensation to those who had suffered from the violence of the populace, in the same address they say,—"The recommendation enjoined by Mr. Secretary Conway's letter, and in consequence thereof made to us, we shall embrace the first convenient opportunity to consider and act upon." They did consider; they did act upon it. They obeyed the requisition. I know the mode has been chicaned upon; but it was substantially obeyed, and much better obeyed than I fear the Parliamentary requisition of this session will be, though enforced by all your rigor and backed with all your power. In a word, the damages of popular fury were compensated by legislative gravity. Almost every other part of America in various ways demonstrated their gratitude. I am bold to say, that so sudden a calm recovered after so violent a storm is without parallel in history. To say that no other disturbance should happen from any other cause is folly. But as far as appearances went, by the judicious sacrifice of one law you procured an acquiescence in all that remained. After this experience, nobody shall persuade me, when an whole people are concerned, that acts of lenity are not means of conciliation.

I hope the honorable gentleman has received a fair and full answer to his question.

I have done with the third period of your policy,—that of your repeal, and the return of your ancient system, and your ancient tranquillity and concord. Sir, this period was not as long as it was happy. Another scene was opened, and other actors appeared on the stage. The state, in the condition I have described it, was delivered into the hands of Lord Chatham, a great and celebrated name,—a name that keeps the name of this country respectable in every other on the globe. It may be truly called

\begin{verse}
Clarum et venerabile nomen \\
Gentibus, et multum nostræ quod proderat urbi.
\end{verse}

Sir, the venerable age of this great man, his merited rank, his superior eloquence, his splendid qualities, his eminent services, the vast space he fills in the eye of mankind, and, more than all the rest, his fall from power, which, like death, canonizes and sanctifies a great character, will not suffer me to censure any part of his conduct. I am afraid to flatter him; I am sure I am not disposed to blame him. Let those who have betrayed him by their adulation insult him with their malevolence. But what I do not presume to censure I may have leave to lament. For a wise man, he seemed to me at that time to be governed too much by general maxims. I speak with the freedom of history, and I hope without offence. One or two of these maxims, flowing from an opinion not the most indulgent to our unhappy species, and surely a little too general, led him into measures that were greatly mischievous to himself, and for that reason, among others, perhaps fatal to his country,—measures, the effects of which, I am afraid, are forever incurable. He made an administration so checkered and speckled, he put together a piece of joinery so crossly indented and whimsically dovetailed, a cabinet so variously inlaid, such a piece of diversified mosaic, such a tessellated pavement without cement,—here a bit of black stone and there a bit of white, patriots and courtiers, king's friends and republicans, Whigs and Tories, treacherous friends and open enemies,—that it was, indeed, a very curious show, but utterly unsafe to touch and unsure to stand on. The colleagues whom he had assorted at the same boards stared at each other, and were obliged to ask,—"Sir, your name?"—"Sir, you have the advantage of me."—"Mr. Such-a-one."—"I beg a thousand pardons."—I venture to say, it did so happen that persons had a single office divided between them, who had never spoke to each other in their lives, until they found themselves, they knew not how, pigging together, heads and points, in the same truckle-bed.
%[11] 
\footnote{Supposed to allude to the Right Honorable Lord North, and George Cooke, Esq., who were made joint paymasters in the summer of 1766, on the removal of the Rockingham administration.}

Sir, in consequence of this arrangement, having put so much the larger part of his enemies and opposers into power, the confusion was such that his own principles could not possibly have any effect or influence in the conduct of affairs. If over he fell into a fit of the gout, or if any other cause withdrew him from public cares, principles directly the contrary were sure to predominate. When he had executed his plan, he had not an inch of ground to stand upon. When he had accomplished his scheme of administration, he was no longer a minister.

When his face was hid but for a moment, his whole system was on a wide sea without chart or compass. The gentlemen, his particular friends, who, with the names of various departments of ministry, were admitted to seem as if they acted a part under him, with a modesty that becomes all men, and with a confidence in him which was justified even in its extravagance by his superior abilities, had never in any instance presumed upon any opinion of their own. Deprived of his guiding influence, they were whirled about, the sport of every gust, and easily driven into any port; and as those who joined with them in manning the vessel were the most directly opposite to his opinions, measures, and character, and far the most artful and most powerful of the set, they easily prevailed, so as to seize upon the vacant, unoccupied, and derelict minds of his friends, and instantly they turned the vessel wholly out of the course of his policy. As if it were to insult as well as to betray him, even long before the close of the first session of his administration, when everything was publicly transacted, and with great parade, in his name, they made an act declaring it highly just and expedient to raise a revenue in America. For even then, Sir, even before this splendid orb was entirely set, and while the western horizon was in a blaze with his descending glory, on the opposite quarter of the heavens arose another luminary, and for his hour became lord of the ascendant.

This light, too, is passed and set forever. You understand, to be sure, that I speak of Charles Townshend, officially the reproducer of this fatal scheme, whom I cannot even now remember without some degree of sensibility. In truth, Sir, he was the delight and ornament of this House, and the charm of every private society which he honored with his presence. Perhaps there never arose in this country, nor in any country, a man of a more pointed and finished wit, and (where his passions were not concerned) of a more refined, exquisite, and penetrating judgment. If he had not so great a stock as some have had, who flourished formerly, of knowledge long treasured up, he knew, better by far than any man I ever was acquainted with, how to bring together within a short time all that was necessary to establish, to illustrate, and to decorate that side of the question he supported. He stated his matter skilfully and powerfully. He particularly excelled in a most luminous explanation and display of his subject. His style of argument was neither trite and vulgar, nor subtle and abstruse. He hit the House just between wind and water. And not being troubled with too anxious a zeal for any matter in question, he was never more tedious or more earnest than the preconceived opinions and present temper of his hearers required, to whom he was always in perfect unison. He conformed exactly to the temper of the House; and he seemed to guide, because he was always sure to follow it.

I beg pardon, Sir, if, when I speak of this and of other great men, I appear to digress in saying something of their characters. In this eventful history of the revolutions of America, the characters of such men are of much importance. Great men are the guideposts and landmarks in the state. The credit of such men at court or in the nation is the sole cause of all the public measures. It would be an invidious thing (most foreign, I trust, to what you think my disposition) to remark the errors into which the authority of great names has brought the nation, without doing justice at the same time to the great qualities whence that authority arose. The subject is instructive to those who wish to form themselves on whatever of excellence has gone before them. There are many young members in the House (such of late has been the rapid succession of public men) who never saw that prodigy, Charles Townshend, nor of course know what a ferment he was able to excite in everything by the violent ebullition of his mixed virtues and failings. For failings he had undoubtedly,—many of us remember them; we are this day considering the effect of them. But he had no failings which were not owing to a noble cause,—to an ardent, generous, perhaps an immoderate passion for fame: a passion which is the instinct of all great souls. He worshipped that goddess, wheresoever she appeared; but he paid his particular devotions to her in her favorite habitation, in her chosen temple, the House of Commons. Besides the characters of the individuals that compose our body, it is impossible, Mr. Speaker, not to observe that this House has a collective character of its own. That character, too, however imperfect, is not unamiable. Like all great public collections of men, you possess a marked love of virtue and an abhorrence of vice. But among vices there is none which the House abhors in the same degree with obstinacy. Obstinacy, Sir, is certainly a great vice; and in the changeful state of political affairs it is frequently the cause of great mischief. It happens, however, very unfortunately, that almost the whole line of the great and masculine virtues, constancy, gravity, magnanimity, fortitude, fidelity, and firmness, are closely allied to this disagreeable quality, of which you have so just an abhorrence; and, in their excess, all these virtues very easily fall into it. He who paid such a punctilious attention to all your feelings certainly took care not to shock them by that vice which is the most disgustful to you.

That fear of displeasing those who ought most to be pleased betrayed him sometimes into the other extreme. He had voted, and, in the year 1765, had been an advocate for the Stamp Act. Things and the disposition of men's minds were changed. In short, the Stamp Act began to be no favorite in this House. He therefore attended at the private meeting in which the resolutions moved by a right honorable gentleman were settled: resolutions leading to the repeal. The next day he voted for that repeal; and he would have spoken for it, too, if an illness (not, as was then given out, a political, but, to my knowledge, a very real illness) had not prevented it.

The very next session, as the fashion of this world passeth away, the repeal began to be in as bad an odor in this House as the Stamp Act had been in the session before. To conform to the temper which began to prevail, and to prevail mostly amongst those most in power, he declared, very early in the winter, that a revenue must be had out of America. Instantly he was tied down to his engagements by some, who had no objection to such experiments, when made at the cost of persons for whom they had no particular regard. The whole body of courtiers drove him onward. They always talked as if the king stood in a sort of humiliated state, until something of the kind should be done.

Here this extraordinary man, then Chancellor of the Exchequer, found himself in great straits. To please universally was the object of his life; but to tax and to please, no more than to love and to be wise, is not given to men. However, he attempted it. To render the tax palatable to the partisans of American revenue, he made a preamble stating the necessity of such a revenue. To close with the American distinction, this revenue was external or port-duty; but again, to soften it to the other party, it was a duty of supply. To gratify the colonists, it was laid on British manufactures; to satisfy the merchants of Britain, the duty was trivial, and (except that on tea, which touched only the devoted East India Company) on none of the grand objects of commerce. To counterwork the American contraband, the duty on tea was reduced from a shilling to three-pence; but to secure the favor of those who would tax America, the scene of collection was changed, and, with the rest, it was levied in the colonies. What need I say more? This fine-spun scheme had the usual fate of all exquisite policy. But the original plan of the duties, and the mode of executing that plan, both arose singly and solely from a love of our applause. He was truly the child of the House. He never thought, did, or said anything, but with a view to you. He every day adapted himself to your disposition, and adjusted himself before it as at a looking-glass.

He had observed (indeed, it could not escape him) that several persons, infinitely his inferiors in all respects, had formerly rendered themselves considerable in this House by one method alone. They were a race of men (I hope in God the species is extinct) who, when they rose in their place, no man living could divine, from any known adherence to parties, to opinions, or to principles, from any order or system in their politics, or from any sequel or connection in their ideas, what part they were going to take in any debate. It is astonishing how much this uncertainty, especially at critical times, called the attention of all parties on such men. All eyes were fixed on them, all ears open to hear them; each party gaped, and looked alternately for their vote, almost to the end of their speeches. While the House hung in this uncertainty, now the hear-hims rose from this side, now they rebellowed from the other; and that party to whom they fell at length from their tremulous and dancing balance always received them in a tempest of applause. The fortune of such men was a temptation too great to be resisted by one to whom a single whiff of incense withheld gave much greater pain than he received delight in the clouds of it which daily rose about him from the prodigal superstition of innumerable admirers. He was a candidate for contradictory honors; and his great aim was, to make those agree in admiration of him who never agreed in anything else.

Hence arose this unfortunate act, the subject of this day's debate: from a disposition which, after making an American revenue to please one, repealed it to please others, and again revived it in hopes of pleasing a third, and of catching something in the ideas of all.

This revenue act of 1767 formed the fourth period of American policy. How we have fared since then: what woful variety of schemes have been adopted; what enforcing, and what repealing; what bullying, and what submitting; what doing, and undoing; what straining, and what relaxing; what assemblies dissolved for not obeying, and called again without obedience; what troops sent out to quell resistance, and, on meeting that resistance, recalled; what shiftings, and changes, and jumblings of all kinds of men at home, which left no possibility of order, consistency, vigor, or even so much as a decent unity of color, in anyone public measure—It is a tedious, irksome task. My duty may call me to open it out some other time; on a former occasion
%[12] 
\footnote{Resolutions in May, 1770.}
I tried your temper on a part of it; for the present I shall forbear.

After all these changes and agitations, your immediate situation upon the question on your paper is at length brought to this. You have an act of Parliament stating that "it is expedient to raise a revenue in America." By a partial repeal you annihilated the greatest part of that revenue which this preamble declares to be so expedient. You have substituted no other in the place of it. A Secretary of State has disclaimed, in the king's name, all thoughts of such a substitution in future. The principle of this disclaimer goes to what has been left, as well as what has been repealed. The tax which lingers after its companions (under a preamble declaring an American revenue expedient, and for the sole purpose of supporting the theory of that preamble) militates with the assurance authentically conveyed to the colonies, and is an exhaustless source of jealousy and animosity. On this state, which I take to be a fair one,—not being able to discern any grounds of honor, advantage, peace, or power, for adhering, either to the act or to the preamble, I shall vote for the question which leads to the repeal of both.

If you do not fall in with this motion, then secure something to fight for, consistent in theory and valuable in practice. If you must employ your strength, employ it to uphold you in some honorable right or some profitable wrong. If you are apprehensive that the concession recommended to you, though proper, should be a means of drawing on you further, but unreasonable claims,—why, then employ your force in supporting that reasonable concession against those unreasonable demands. You will employ it with more grace, with better effect, and with great probable concurrence of all the quiet and rational people in the provinces, who are now united with and hurried away by the violent,—having, indeed, different dispositions, but a common interest. If you apprehend that on a concession you shall be pushed by metaphysical process to the extreme lines, and argued out of your whole authority, my advice is this: when you have recovered your old, your strong, your tenable position, then face about,—stop short,—do nothing more,—reason not at all,—oppose the ancient policy and practice of the empire as a rampart against the speculations of innovators on both sides of the question,—and you will stand on great, manly, and sure ground. On this solid basis fix your machines, and they will draw worlds towards you.

Tour ministers, in their own and his Majesty's name, have already adopted the American distinction of internal and external duties. It is a distinction, whatever merit it may have, that was originally moved by the Americans themselves; and I think they will acquiesce in it, if they are not pushed with too much logic and too little sense, in all the consequences: that is, if external taxation be understood, as they and you understand it, when you please, to be not a distinction of geography, but of policy; that it is a power for regulating trade, and not for supporting establishments. The distinction, which is as nothing with regard to right, is of most weighty consideration in practice. Recover your old ground, and your old tranquillity; try it; I am persuaded the Americans will compromise with you. When confidence is once restored, the odious and suspicious summum jus will perish of course. The spirit of practicability, of moderation, and mutual convenience will never call in geometrical exactness as the arbitrator of an amicable settlement. Consult and follow your experience. Let not the long story with which I have exercised your patience prove fruitless to your interests.

For my part, I should choose (if I could have my wish) that the proposition of the honorable gentleman
%[13] 
\footnote{Mr. Fuller.}
for the repeal could go to America without the attendance of the penal bills. Alone I could almost answer for its success. I cannot be certain of its reception in the bad company it may keep. In such heterogeneous assortments, the most innocent person will lose the effect of his innocency. Though you should send out this angel of peace, yet you are sending out a destroying angel too; and what would be the effect of the conflict of these two adverse spirits, or which would predominate in the end, is what I dare not say: whether the lenient measures would cause American passion to subside, or the severe would increase its fury,—all this is in the hand of Providence. Yet now, even now, I should confide in the prevailing virtue and efficacious operation of lenity, though working in darkness and in chaos, in the midst of all this unnatural and turbid combination: I should hope it might produce order and beauty in the end.

Let us, Sir, embrace some system or other before we end this session. Do you mean to tax America, and to draw a productive revenue from thence? If you do, speak out: name, fix, ascertain this revenue; settle its quantity; define its objects; provide for its collection; and then fight, when you have something to fight for. If you murder, rob; if you kill, take possession; and do not appear in the character of madmen as well as assassins, violent, vindictive, bloody, and tyrannical, without an object. But may better counsels guide you!

Again, and again, revert to your old principles,—seek peace and ensue it,—leave America, if she has taxable matter in her, to tax herself. I am not here going into the distinctions of rights, nor attempting to mark their boundaries. I do not enter into these metaphysical distinctions; I hate the very sound of them. Leave the Americans as they anciently stood, and these distinctions, born of our unhappy contest, will die along with it. They and we, and their and our ancestors, have been happy under that system. Let the memory of all actions in contradiction to that good old mode, on both sides, be extinguished forever. Be content to bind America by laws of trade: you have always done it. Let this be your reason for binding their trade. Do not burden them by taxes: you were not used to do so from the beginning. Let this be your reason for not taxing. These are the arguments of states and kingdoms. Leave the rest to the schools; for there only they may be discussed with safety. But if, intemperately, unwisely, fatally, you sophisticate and poison the very source of government, by urging subtle deductions, and consequences odious to those you govern, from the unlimited and illimitable nature of supreme sovereignty, you will teach them by these means to call that sovereignty itself in question. When you drive him hard, the boar will surely turn upon the hunters. If that sovereignty and their freedom cannot be reconciled, which will they take? They will cast your sovereignty in your face. Nobody will be argued into slavery. Sir, let the gentlemen on the other side call forth all their ability; let the best of them get up and tell me what one character of liberty the Americans have, and what one brand of slavery they are free from, if they are bound in their property and industry by all the restraints you can imagine on commerce, and at the same time are made pack-horses of every tax you choose to impose, without the least share in granting them. When they bear the burdens of unlimited monopoly, will you bring them to bear the burdens of unlimited revenue too? The Englishman in America will feel that this is slavery: that it is legal slavery will be no compensation either to his feelings or his understanding.

A noble lord,
%[14] 
\footnote{Lord Carmarthen.}
who spoke some time ago, is full of the fire of ingenuous youth; and when he has modelled the ideas of a lively imagination by further experience, he will be an ornament to his country in either House. He has said that the Americans are our children, and how can they revolt against their parent? He says, that, if they are not free in their present state, England is not free; because Manchester, and other considerable places, are not represented. So, then, because some towns in England are not represented, America is to have no representative at all. They are "our children"; but when children ask for bread, we are not to give a stone. Is it because the natural resistance of things, and the various mutations of time, hinders our government, or any scheme of government, from being any more than a sort of approximation to the right, is it therefore that the colonies are to recede from it infinitely? When this child of ours wishes to assimilate to its parent, and to reflect with a true filial resemblance the beauteous countenance of British liberty, are we to turn to them the shameful parts of our constitution? are we to give them our weakness for their strength, our opprobrium for their glory, and the slough of slavery, which we are not able to work off, to serve them for their freedom?

If this be the case, ask yourselves this question: Will they be content in such a state of slavery? If not, look to the consequences. Reflect how you are to govern a people who think they ought to be free, and think they are not. Your scheme yields no revenue; it yields nothing but discontent, disorder, disobedience: and such is the state of America, that, after wading up to your eyes in blood, you could only end just where you begun,—that is, to tax where no revenue is to be found, to ---- My voice fails me: my inclination, indeed, carries me no further; all is confusion beyond it.

Well, Sir, I have recovered a little, and before I sit down I must say something to another point with which gentlemen urge us. What is to become of the Declaratory Act, asserting the entireness of British legislative authority, if we abandon the practice of taxation?

For my part, I look upon the rights stated in that act exactly in the manner in which I viewed them on its very first proposition, and which I have often taken the liberty, with great humility, to lay before you. I look, I say, on the imperial rights of Great Britain, and the privileges which the colonists ought to enjoy under these rights, to be just the most reconcilable things in the world. The Parliament of Great Britain sits at the head of her extensive empire in two capacities. One as the local legislature of this island, providing for all things at home, immediately, and by no other instrument than the executive power. The other, and I think her nobler capacity, is what I call her imperial character; in which, as from the throne of heaven, she superintends all the several inferior legislatures, and guides and controls them all without annihilating any. As all these provincial legislatures are only coördinate to each other, they ought all to be subordinate to her; else they can neither preserve mutual peace, nor hope for mutual justice, nor effectually afford mutual assistance. It is necessary to coerce the negligent, to restrain the violent, and to aid the weak and deficient, by the overruling plenitude of her power. She is never to intrude into the place of the others, whilst they are equal to the common ends of their institution. But in order to enable Parliament to answer all these ends of provident and beneficent superintendence, her powers must be boundless. The gentlemen who think the powers of Parliament limited may please themselves to talk of requisitions. But suppose the requisitions are not obeyed? What! shall there be no reserved power in the empire, to supply a deficiency which may weaken, divide, and dissipate the whole? We are engaged in war,—the Secretary of State calls upon the colonies to contribute,—some would do it, I think most would cheerfully furnish whatever is demanded,—one or two, suppose, hang back, and, easing themselves, let the stress of the draft lie on the others,—surely it is proper that some authority might legally say, "Tax yourselves for the common Supply, or Parliament will do it for you." This backwardness was, as I am told, actually the case of Pennsylvania for some short time towards the beginning of the last war, owing to some internal dissensions in that colony. But whether the fact were so or otherwise, the case is equally to be provided for by a competent sovereign power. But then this ought to be no ordinary power, nor ever used in the first instance. This is what I meant, when I have said, at various times, that I consider the power of taxing in Parliament as an instrument of empire, and not as a means of supply.

Such, Sir, is my idea of the Constitution of the British Empire, as distinguished from the Constitution of Britain; and on these grounds I think subordination and liberty may be sufficiently reconciled through the whole,—whether to serve a refining speculatist or a factious demagogue I know not, but enough surely for the ease and happiness of man.

Sir, whilst we hold this happy course, we drew more from the colonies than all the impotent violence of despotism ever could extort from them. We did this abundantly in the last war; it has never been once denied; and what reason have we to imagine that the colonies would not have proceeded in supplying government as liberally, if you had not stepped in and hindered them from contributing, by interrupting the channel in which their liberality flowed with so strong a course,—by attempting to take, instead of being satisfied to receive? Sir William Temple says, that Holland has loaded itself with ten times the impositions which it revolted from Spain rather than submit to. He says true. Tyranny is a poor provider. It knows neither how to accumulate nor how to extract.

I charge, therefore, to this new and unfortunate system the loss not only of peace, of union, and of commerce, but even of revenue, which its friends are contending for. It is morally certain that we have lost at least a million of free grants since the peace. I think we have lost a great deal more; and that those who look for a revenue from the provinces never could have pursued, even in that light, a course more directly repugnant to their purposes.

Now, Sir, I trust I have shown, first on that narrow ground which the honorable gentleman measured, that you are like to lose nothing by complying with the motion, except what you have lost already. I have shown afterwards, that in time of peace you flourished in commerce, and, when war required it, had sufficient aid from the colonies, while you pursued your ancient policy; that you threw everything into confusion, when you made the Stamp Act; and that you restored everything to peace and order, when you repealed it. I have shown that the revival of the system of taxation has produced the very worst effects; and that the partial repeal has produced, not partial good, but universal evil. Let these considerations, founded on facts, not one of which can be denied, bring us back to our reason by the road of our experience.

I cannot, as I have said, answer for mixed measures: but surely this mixture of lenity would give the whole a better chance of success. When you once regain confidence, the way will be clear before you. Then you may enforce the Act of Navigation, when it ought to be enforced. You will yourselves open it, where it ought still further to be opened. Proceed in what you do, whatever you do, from policy, and not from rancor. Let us act like men, let us act like statesmen. Let us hold some sort of consistent conduct. It is agreed that a revenue is not to be had in America. If we lose the profit, let us get rid of the odium.

On this business of America, I confess I am serious, even to sadness. I have had but one opinion concerning it, since I sat, and before I sat in Parliament. The noble lord
%[15] 
\footnote{Lord North.}
will, as usual, probably, attribute the part taken by me and my friends in this business to a desire of getting his places. Let him enjoy this happy and original idea. If I deprived him of it, I should take away most of his wit, and all his argument. But I had rather bear the brunt of all his wit, and indeed blows much heavier, than stand answerable to God for embracing a system that tends to the destruction of some of the very best and fairest of His works. But I know the map of England as well as the noble lord, or as any other person; and I know that the way I take is not the road to preferment. My excellent and honorable friend under me on the floor
%[16] 
\footnote{Mr. Dowdeswell}
has trod that road with great toil for upwards of twenty years together. He is not yet arrived at the noble lord's destination. However, the tracks of my worthy friend are those I have ever wished to follow; because I know they lead to honor. Long may we tread the same road together, whoever may accompany us, or whoever may laugh at us on our journey! I honestly and solemnly declare, I have in all seasons adhered to the system of 1766 for no other reason than, that I think it laid deep in your truest interests,—and that, by limiting the exercise, it fixes on the firmest foundations a real, consistent, well-grounded authority in Parliament. Until you come back to that system, there will be no peace for England.


%%FOOTNOTES:
%[1] Charles Wolfran Cornwall, Esq., lately appointed one of the Lords of the Treasury.
%[2] Lord North, then Chancellor of the Exchequer.
%[3] Lord Hillsborough's Circular Letter to the Governors of the Colonies, concerning the repeal of some of the duties laid in the Act of 1767.
%[4] A material point is omitted by Mr. Burke in this speech, viz. the manner in which the continent received this royal assurance. The assembly of Virginia, in their address in answer to Lord Botetourt's speech, express themselves thus:—"We will not suffer our present hopes, arising from the pleasing prospect your Lordship hath so kindly opened and displayed to us, to be lashed by the bitter reflection that any future administration will entertain a wish to depart from that plan which affords the surest and most permanent foundation of public tranquillity and happiness. No, my Lord, we are sure our most gracious sovereign, under whatever changes may happen in his confidential servants, will remain immutable in the ways of truth and justice, and that he is incapable of deceiving his faithful subjects; and we esteem your Lordship's information not only as warranted, but even sanctified by the royal word."
%[5] Lord North.
%[6] Mr. Dowdeswell.
%[7] General Conway.
%[8] General Conway.
%[9] General Conway.
%[10] General Conway.
%[11] Supposed to allude to the Right Honorable Lord North, and George Cooke, Esq., who were made joint paymasters in the summer of 1766, on the removal of the Rockingham administration.
%[12] Resolutions in May, 1770.
%[13] Mr. Fuller.
%[14] Lord Carmarthen.
%[15] Lord North.
%[16] Mr. Dowdeswell


%%%%%%%%%%%%%%%%%%%%%%%%%%%%%%%%%%%%%%%%%%%%%%%%%%%%%%%%%%%%%%%%%%%%%%%
\chapter*[Speeches on Arrival at Bristol]{Speeches on Arrival at Bristol 
and at the Conclusion of the Poll
\\ \vspace{0.1cm}\large{October 13 and November 3, 1774}}
%\label{chap:vindication}
\addcontentsline{toc}{chapter}{SPEECHES ON ARRIVAL AT BRISTOL AND AT THE
CONCLUSION OF THE POLL, October 13 and November 3, 1774}

%%%%%%%%%%%%%%%%%%%%%%%%%%%%%%%%%%%%%%%%%
\begin{center}
  \textbf{\large EDITOR'S ADVERTISEMENT} \par 
\end{center}
\addcontentsline{toc}{section}{EDITOR'S ADVERTISEMENT}

We believe there is no need of an apology to the public for offering to them any genuine speeches of Mr. Burke: the two contained in this publication undoubtedly are so. The general approbation they met with (as we hear) from all parties at Bristol persuades us that a good edition of them will not be unacceptable in London; which we own to be the inducement, and we hope is a justification, of our offering it.

We do not presume to descant on the merit of these speeches; but as it is no less new than honorable to find a popular candidate, at a popular election, daring to avow his dissent to certain points that have been considered as very popular objects, and maintaining himself on the manly confidence of his own opinion, so we must say that it does great credit to the people of England, as it proves to the world, that, to insure their confidence, it is not necessary to flatter them, or to affect a subserviency to their passions or their prejudices.

It may be necessary to promise, that at the opening of the poll the candidates were Lord Clare, Mr. Brickdale, the two last members, and Mr. Cruger, a considerable merchant at Bristol. On the second day of the poll, Lord Clare declined; and a considerable body of gentlemen, who had wished that the city of Bristol should, at this critical season, be represented by some gentleman of tried abilities and known commercial knowledge, immediately put Mr. Burke in nomination. Some of them set off express for London to apprise that gentleman of this event; but he was gone to Malton, in Yorkshire. The spirit and active zeal of these gentlemen followed him to Malton. They arrived there just after Mr. Burke's election for that place, and invited him to Bristol.

Mr. Burke, as he tells us in his first speech, acquainted his constituents with the honorable offer that was made him, and, with their consent, he immediately set off for Bristol, on the Tuesday, at six in the evening; he arrived at Bristol at half past two in the afternoon, on Thursday, the 13th of October, being the sixth day of the poll.

He drove directly to the mayor's house, who not being at home, he proceeded to the Guildhall, where he ascended the hustings, and having saluted the electors, the sheriffs, and the two candidates, he reposed himself for a few minutes, and then addressed the electors in a speech which was received with great and universal applause and approbation.

%%%%%%%%%%%%%%%%%%%%%%%%%%%%%%%%%%%%%%%%%
\begin{center}
  \textbf{\large Speech at His Arrival at Bristol} \par 
\end{center}
\addcontentsline{toc}{section}{SPEECH AT HIS ARRIVAL AT BRISTOL}

Gentlemen,—I am come hither to solicit in person that favor which my friends have hitherto endeavored to procure for me, by the most obliging, and to me the most honorable exertions.

I have so high an opinion of the great trust which you have to confer on this occasion, and, by long experience, so just a diffidence in my abilities to fill it in a manner adequate even to my own ideas, that I should never have ventured of myself to intrude into that awful situation. But since I am called upon by the desire of several respectable fellow subjects, as I have done at other times, I give up my fears to their wishes. Whatever my other deficiencies may be, I do not know what it is to be wanting to my friends.

I am not fond of attempting to raise public expectations by great promises. At this time, there is much cause to consider, and very little to presume. We seem to be approaching to a great crisis in our affairs, which calls for the whole wisdom of the wisest among us, without being able to assure ourselves that any wisdom can preserve us from many and great inconveniences. You know I speak of our unhappy contest with America. I confess, it is a matter on which I look down as from a precipice. It is difficult in itself, and it is rendered more intricate by a great variety of plans of conduct. I do not mean to enter into them. I will not suspect a want of good intention in framing them. But however pure the intentions of their authors may have been, we all know that the event has been unfortunate. The means of recovering our affairs are not obvious. So many great questions of commerce, of finance, of constitution, and of policy are involved in this American deliberation, that I dare engage for nothing, but that I shall give it, without any predilection to former opinions, or any sinister bias whatsoever, the most honest and impartial consideration of which I am capable. The public has a full right to it; and this great city, a main pillar in the commercial interest of Great Britain, must totter on its base by the slightest mistake with regard to our American measures.

Thus much, however, I think it not amiss to lay before you,—that I am not, I hope, apt to take up or lay down my opinions lightly. I have held, and ever shall maintain, to the best of my power, unimpaired and undiminished, the just, wise, and necessary constitutional superiority of Great Britain. This is necessary for America as well as for us. I never mean to depart from it. Whatever may be lost by it, I avow it. The forfeiture even of your favor, if by such a declaration I could forfeit it, though the first object of my ambition, never will make me disguise my sentiments on this subject.

But—I have ever had a clear opinion, and have ever held a constant correspondent conduct, that this superiority is consistent with all the liberties a sober and spirited American ought to desire. I never mean to put any colonist, or any human creature, in a situation not becoming a free man. To reconcile British superiority with American liberty shall be my great object, as far as my little faculties extend. I am far from thinking that both, even yet, may not be preserved.

When I first devoted myself to the public service, I considered how I should render myself fit for it; and this I did by endeavoring to discover what it was that gave this country the rank it holds in the world. I found that our prosperity and dignity arose principally, if not solely, from two sources: our Constitution, and commerce. Both these I have spared no study to understand, and no endeavor to support.

The distinguishing part of our Constitution is its liberty. To preserve that liberty inviolate seems the particular duty and proper trust of a member of the House of Commons. But the liberty, the only liberty, I mean is a liberty connected with order: that not only exists along with order and virtue, but which cannot exist at all without them. It inheres in good and steady government, as in its substance and vital principle.

The other source of our power is commerce, of which you are so large a part, and which cannot exist, no more than your liberty, without a connection with many virtues. It has ever been a very particular and a very favorite object of my study, in its principles, and in its details. I think many here are acquainted with the truth of what I say. This I know,—that I have ever had my house open, and my poor services ready, for traders and manufacturers of every denomination. My favorite ambition is, to have those services acknowledged. I now appear before you to make trial, whether my earnest endeavors have been so wholly oppressed by the weakness of my abilities as to be rendered insignificant in the eyes of a great trading city; or whether you choose to give a weight to humble abilities, for the sake of the honest exertions with which they are accompanied. This is my trial to-day. My industry is not on trial. Of my industry I am sure, as far as my constitution of mind and body admitted.

When I was invited by many respectable merchants, freeholders, and freemen of this city to offer them my services, I had just received the honor of an election at another place, at a very great distance from this. I immediately opened the matter to those of my worthy constituents who were with me, and they unanimously advised me not to decline it. They told me that they had elected me with a view to the public service; and as great questions relative to our commerce and colonies were imminent that in such matters I might derive authority and support from the representation of this great commercial city: they desired me, therefore, to set off without delay, very well persuaded that I never could forget my obligations to them or to my friends, for the choice they had made of me. From that time to this instant I have not slept; and if I should have the honor of being freely chosen by you, I hope I shall be as far from slumbering or sleeping, when your service requires me to be awake, as I have been in coming to offer myself a candidate for your favor.


%%%%%%%%%%%%%%%%%%%%%%%%%%%%%%%%%%%%%%%%%
\begin{center}
  \textbf{{\large Speech to the Electors of Bristol, on His Being Declared
  by the Sheriffs Duly Elected One of the Representatives in Parliament
  for That City}, \\on Thursday, the 3rd of November, 1774}  \par 
\end{center}
\addcontentsline{toc}{section}{SPEECH TO THE ELECTORS OF BRISTOL}

Gentlemen,—I cannot avoid sympathizing strongly with the feelings of the gentleman who has received the same honor that you have conferred on me. If he, who was bred and passed his whole life amongst you,—if he, who, through the easy gradations of acquaintance, friendship, and esteem, has obtained the honor which seems of itself, naturally and almost insensibly, to meet with those who, by the even tenor of pleasing manners and social virtues, slide into the love and confidence of their fellow-citizens,—if he cannot speak but with great emotion on this subject, surrounded as he is on all sides with his old friends,—you will have the goodness to excuse me, if my real, unaffected embarrassment prevents me from expressing my gratitude to you as I ought.

I was brought hither under the disadvantage of being unknown, even by sight, to any of you. No previous canvass was made for me. I was put in nomination after the poll was opened. I did not appear until it was far advanced. If, under all these accumulated disadvantages, your good opinion has carried me to this happy point of success, you will pardon me, if I can only say to you collectively, as I said to you individually, simply and plainly, I thank you,—I am obliged to you,—I am not insensible of your kindness.

This is all that I am able to say for the inestimable favor you have conferred upon me. But I cannot be satisfied without saying a little more in defence of the right you have to confer such a favor. The person that appeared here as counsel for the candidate who so long and so earnestly solicited your votes thinks proper to deny that a very great part of you have any votes to give. He fixes a standard period of time in his own imagination, (not what the law defines, but merely what the convenience of his client suggests,) by which he would cut off at one stroke all those freedoms which are the dearest privileges of your corporation,—which the Common Law authorizes,—which your magistrates are compelled to grant,—which come duly authenticated into this court,—and are saved in the clearest words, and with the most religious care and tenderness, in that very act of Parliament which was made to regulate the elections by freemen, and to prevent all possible abuses in making them.

I do not intend to argue the matter here. My learned counsel has supported your cause with his usual ability; the worthy sheriffs have acted with their usual equity; and I have no doubt that the same equity which dictates the return will guide the final determination. I had the honor, in conjunction with many far wiser men, to contribute a very small assistance, but, however, some assistance, to the forming the judicature which is to try such questions. It would be unnatural in me to doubt the justice of that court, in the trial of my own cause, to which I have been so active to give jurisdiction over every other.

I assure the worthy freemen, and this corporation, that, if the gentleman perseveres in the intentions which his present warmth dictates to him, I will attend their cause with diligence, and I hope with effect. For, if I know anything of myself, it is not my own interest in it, but my full conviction, that induces me to tell you, I think there is not a shadow of doubt in the case.

I do not imagine that you find me rash in declaring myself, or very forward in troubling you. From the beginning to the end of the election, I have kept silence in all matters of discussion. I have never asked a question of a voter on the other side, or supported a doubtful vote on my own. I respected the abilities of my managers; I relied on the candor of the court. I think the worthy sheriffs will bear me witness that I have never once made an attempt to impose upon their reason, to surprise their justice, or to ruffle their temper. I stood on the hustings (except when I gave my thanks to those who favored me with their votes) less like a candidate than an unconcerned spectator of a public proceeding. But here the face of things is altered. Here is an attempt for a general massacre of suffrages,—an attempt, by a promiscuous carnage of friends and foes, to exterminate above two thousand votes, including seven hundred polled for the gentleman himself who now complains, and who would destroy the friends whom he has obtained, only because he cannot obtain as many of them as he wishes.

How he will be permitted, in another place, to stultify and disable himself, and to plead against his own acts, is another question. The law will decide it. I shall only speak of it as it concerns the propriety of public conduct in this city. I do not pretend to lay down rules of decorum for other gentlemen. They are best judges of the mode of proceeding that will recommend them to the favor of their fellow-citizens. But I confess I should look rather awkward, if I had been the very first to produce the new copies of freedom,—if I had persisted in producing them to the last,—if I had ransacked, with the most unremitting industry and the most penetrating research, the remotest corners of the kingdom to discover them,—if I were then, all at once, to turn short, and declare that I had been sporting all this while with the right of election, and that I had been drawing out a poll, upon no sort of rational grounds, which disturbed the peace of my fellow-citizens for a month together;—I really, for my part, should appear awkward under such circumstances.

It would be still more awkward in me, if I were gravely to look the sheriffs in the face, and to tell them they were not to determine my cause on my own principles, nor to make the return upon those votes upon which I had rested my election. Such would be my appearance to the court and magistrates.

But how should I appear to the voters themselves? If I had gone round to the citizens entitled to freedom, and squeezed them by the hand,—"Sir, I humbly beg your vote,—I shall be eternally thankful,—may I hope for the honor of your support?—Well!—come,—we shall see you at the Council-House."—If I were then to deliver them to my managers, pack them into tallies, vote them off in court, and when I heard from the bar,—"Such a one only! and such a one forever!—he's my man!"—"Thank you, good Sir,—Hah! my worthy friend! thank you kindly,—that's an honest fellow,—how is your good family?"—Whilst these words were hardly out of my mouth, if I should have wheeled round at once, and told them,—"Get you gone, you pack of worthless fellows! you have no votes,—you are usurpers! you are intruders on the rights of real freemen! I will have nothing to do with you! you ought never to have been produced at this election, and the sheriffs ought not to have admitted you to poll!"—

Gentlemen, I should make a strange figure, if my conduct had been of this sort. I am not so old an acquaintance of yours as the worthy gentleman. Indeed, I could not have ventured on such kind of freedoms with you. But I am bound, and I will endeavor, to have justice done to the rights of freemen,—even though I should at the same time be obliged to vindicate the former
%[17] 
\footnote{Mr. Brickdale opened his poll, it seems, with a tally of those very kind of freemen, and voted many hundreds of them.}
part of my antagonist's conduct against his own present inclinations.

I owe myself, in all things, to all the freemen of this city. My particular friends have a demand on mo that I should not deceive their expectations. Never was cause or man supported with more constancy, more activity, more spirit. I have been supported with a zeal, indeed, and heartiness in my friends, which (if their object had been at all proportioned to their endeavors) could never be sufficiently commended. They supported me upon the most liberal principles. They wished that the members for Bristol should be chosen for the city, and for their country at large, and not for themselves.

So far they are not disappointed. If I possess nothing else, I am sure I possess the temper that is fit for your service. I know nothing of Bristol, but by the favors I have received, and the virtues I have seen exerted in it.

I shall ever retain, what I now feel, the most perfect and grateful attachment to my friends,—and I have no enmities, no resentments. I never can consider fidelity to engagements and constancy in friendships but with the highest approbation, even when those noble qualities are employed against my own pretensions. The gentleman who is not so fortunate as I have been in this contest enjoys, in this respect, a consolation full of honor both to himself and to his friends. They have certainly left nothing undone for his service.

As for the trifling petulance which the rage of party stirs up in little minds, though it should show itself even in this court, it has not made the slightest impression on me. The highest flight of such clamorous birds is winged in an inferior region of the air. We hear them, and we look upon them, just as you, Gentlemen, when you enjoy the serene air on your lofty rocks, look down upon the gulls that skim the mud of your river, when it is exhausted of its tide.

I am sorry I cannot conclude without saying a word on a topic touched upon by my worthy colleague. I wish that topic had been passed by at a time when I have so little leisure to discuss it. But since he has thought proper to throw it out, I owe you a clear explanation of my poor sentiments on that subject.

He tells you that "the topic of instructions has occasioned much altercation and uneasiness in this city"; and he expresses himself (if I understand him rightly) in favor of the coercive authority of such instructions.

Certainly, Gentlemen, it ought to be the happiness and glory of a representative to live in the strictest union, the closest correspondence, and the most unreserved communication with his constituents. Their wishes ought to have great weight with him; their opinions high respect; their business unremitted attention. It is his duty to sacrifice his repose, his pleasure, his satisfactions, to theirs,—and above all, ever, and in all cases, to prefer their interest to his own.

But his unbiased opinion, his mature judgment, his enlightened conscience, he ought not to sacrifice to you, to any man, or to any set of men living. These he does not derive from your pleasure,—no, nor from the law and the Constitution. They are a trust from Providence, for the abuse of which he is deeply answerable. Your representative owes you, not his industry only, but his judgment; and he betrays, instead of serving you, if he sacrifices it to your opinion.

My worthy colleague says, his will ought to be subservient to yours. If that be all, the thing is innocent. If government were a matter of will upon any side, yours, without question, ought to be superior. But government and legislation are matters of reason and judgment, and not of inclination; and what sort of reason is that in which the determination precedes the discussion, in which one set of men deliberate and another decide, and where those who form the conclusion are perhaps three hundred miles distant from those who hear the arguments?

To deliver an opinion is the right of all men; that of constituents is a weighty and respectable opinion, which a representative ought always to rejoice to hear, and which he ought always most seriously to consider. But authoritative instructions, mandates issued, which the member is bound blindly and implicitly to obey, to vote, and to argue for, though contrary to the clearest conviction of his judgment and conscience,—these are things utterly unknown to the laws of this land, and which arise from a fundamental mistake of the whole order and tenor of our Constitution.

Parliament is not a congress of ambassadors from different and hostile interests, which interests each must maintain, as an agent and advocate, against other agents and advocates; but Parliament is a deliberative assembly of one nation, with one interest, that of the whole—where not local purposes, not local prejudices, ought to guide, but the general good, resulting from the general reason of the whole. You choose a member, indeed; but when you have chosen him, he is not member of Bristol, but he is a member of Parliament. If the local constituent should have an interest or should form an hasty opinion evidently opposite to the real good of the rest of the community, the member for that place ought to be as far as any other from any endeavor to give it effect. I beg pardon for saying so much on this subject; I have been unwillingly drawn into it; but I shall ever use a respectful frankness of communication with you. Your faithful friend, your devoted servant, I shall be to the end of my life: a flatterer you do not wish for. On this point of instructions, however, I think it scarcely possible we ever can have any sort of difference. Perhaps I may give you too much, rather than too little trouble.

From the first hour I was encouraged to court your favor, to this happy day of obtaining it, I have never promised you anything but humble and persevering endeavors to do my duty. The weight of that duty, I confess, makes me tremble; and whoever well considers what it is, of all things in the world, will fly from what has the least likeness to a positive and precipitate engagement. To be a good member of Parliament is, let me tell you, no easy task,—especially at this time, when there is so strong a disposition to run into the perilous extremes of servile compliance or wild popularity. To unite circumspection with vigor is absolutely necessary, but it is extremely difficult. We are now members for a rich commercial city; this city, however, is but a part of a rich commercial nation, the interests of which are various, multiform, and intricate. We are members for that great nation, which, however, is itself but part of a great empire, extended by our virtue and our fortune to the farthest limits of the East and of the West. All these wide-spread interests must be considered,—must be compared,—must be reconciled, if possible. We are members for a free country; and surely we all know that the machine of a free constitution is no simple thing, but as intricate and as delicate as it is valuable. We are members in a great and ancient monarchy; and we must preserve religiously the true, legal rights of the sovereign, which form the keystone that binds together the noble and well-constructed arch of our empire and our Constitution. A constitution made up of balanced powers must ever be a critical thing. As such I mean to touch that part of it which comes within my reach. I know my inability, and I wish for support from every quarter. In particular I shall aim at the friendship, and shall cultivate the best correspondence, of the worthy colleague you have given me.

I trouble you no farther than once more to thank you all: you, Gentlemen, for your favors; the candidates, for their temperate and polite behavior; and the sheriffs, for a conduct which may give a model for all who are in public stations.

%%FOOTNOTES:
%[17] Mr. Brickdale opened his poll, it seems, with a tally of those very kind of freemen, and voted many hundreds of them.


%%%%%%%%%%%%%%%%%%%%%%%%%%%%%%%%%%%%%%%%%%%%%%%%%%%%%%%%%%%%%%%%%%%%%%%
\chapter*[Speech on Conciliation with America]{Speech on Moving Resolutions
for Conciliation with America
\\ \vspace{0.1cm}\large{March 22, 1775}}
%\label{chap:vindication}
\addcontentsline{toc}{chapter}{SPEECH ON MOVING RESOLUTIONS FOR CONCILIATION
WITH AMERICA}

I hope, Sir, that, notwithstanding the austerity of the Chair, your good-nature will incline you to some degree of indulgence towards human frailty. You will not think it unnatural, that those who have an object depending, which strongly engages their hopes and fears, should be somewhat inclined to superstition. As I came into the House, full of anxiety about the event of my motion, I found, to my infinite surprise, that the grand penal bill by which we had passed sentence on the trade and sustenance of America is to be returned to us from the other House.
%[18] 
\footnote{The act to restrain the trade and commerce of the provinces of Massachusetts Bay and New Hampshire, and colonies of Connecticut and Rhode Island and Providence Plantation, in North America, to Great Britain, Ireland, and the British Islands in the West Indies; and to prohibit such provinces and colonies from carrying on any fishery on the banks of Newfoundland, and other places therein mentioned, under certain conditions and limitations.}
I do confess, I could not help looking on this event as a fortunate omen. I look upon it as a sort of Providential favor, by which we are put once more in possession of our deliberative capacity, upon a business so very questionable in its nature, so very uncertain in its issue. By the return of this bill, which seemed to have taken its flight forever, we are at this very instant nearly as free to choose a plan for our American government as we were on the first day of the session. If, Sir, we incline to the side of conciliation, we are not at all embarrassed (unless we please to make ourselves so) by any incongruous mixture of coercion and restraint. We are therefore called upon, as it were by a superior warning voice, again to attend to America,—to attend to the whole of it together,—and to review the subject with an unusual degree of care and calmness.

Surely it is an awful subject,—or there is none so on this side of the grave. When I first had the honor of a seat in this House, the affairs of that continent pressed themselves upon us as the most important and most delicate object of Parliamentary attention. My little share in this great deliberation oppressed me. I found myself a partaker in a very high trust; and having no sort of reason to rely on the strength of my natural abilities for the proper execution of that trust, I was obliged to take more than common pains to instruct myself in everything which relates to our colonies. I was not less under the necessity of forming some fixed ideas concerning the general policy of the British empire. Something of this sort seemed to be indispensable, in order, amidst so vast a fluctuation of passions and opinions, to concentre my thoughts, to ballast my conduct, to preserve me from being blown about by every wind of fashionable doctrine. I really did not think it safe or manly to have fresh principles to seek upon every fresh mail which should arrive from America.

At that period I had the fortune to find myself in perfect concurrence with a large majority in this House. Bowing under that high authority, and penetrated with the sharpness and strength of that early impression, I have continued ever since, without the least deviation, in my original sentiments. Whether this be owing to an obstinate perseverance in error, or to a religious adherence to what appears to me truth and reason, it is in your equity to judge.

Sir, Parliament, having an enlarged view of objects, made, during this interval, more frequent changes in their sentiments and their conduct than could be justified in a particular person upon the contracted scale of private information. But though I do not hazard anything approaching to a censure on the motives of former Parliaments to all those alterations, one fact is undoubted,—that under them the state of America has been kept in continual agitation. Everything administered as remedy to the public complaint, if it did not produce, was at least followed by, an heightening of the distemper, until, by a variety of experiments, that important country has been brought into her present situation,—a situation which I will not miscall, which I dare not name, which I scarcely know how to comprehend in the terms of any description.

In this posture, Sir, things stood at the beginning of the session. About that time, a worthy member,
%[19] 
\footnote{Mr. Rose Fuller.}
of great Parliamentary experience, who in the year 1766 filled the chair of the American Committee with much ability, took me aside, and, lamenting the present aspect of our politics, told me, things were come to such a pass that our former methods of proceeding in the House would be no longer tolerated,—that the public tribunal (never too indulgent to a long and unsuccessful opposition) would now scrutinize our conduct with unusual severity,—that the very vicissitudes and shiftings of ministerial measures, instead of convicting their authors of inconstancy and want of system, would be taken as an occasion of charging us with a predetermined discontent which nothing could satisfy, whilst we accused every measure of vigor as cruel and every proposal of lenity as weak and irresolute. The public, he said, would not have patience to see us play the game out with our adversaries; we must produce our hand: it would be expected that those who for many years had been active in such affairs should show that they had formed some clear and decided idea of the principles of colony government, and were capable of drawing out something like a platform of the ground which might be laid for future and permanent tranquillity.

I felt the truth of what my honorable friend represented; but I felt my situation, too. His application might have been made with far greater propriety to many other gentlemen. No man was, indeed, ever better disposed, or worse qualified, for such an undertaking, than myself. Though I gave so far into his opinion, that I immediately threw my thoughts into a sort of Parliamentary form, I was by no means equally ready to produce them. It generally argues some degree of natural impotence of mind, or some want of knowledge of the world, to hazard plans of government, except from a seat of authority. Propositions are made, not only ineffectually, but somewhat disreputably, when the minds of men are not properly disposed for their reception; and for my part, I am not ambitious of ridicule, not absolutely a candidate for disgrace.

Besides, Sir, to speak the plain truth, I have in general no very exalted opinion of the virtue of paper government, nor of any polities in which the plan is to be wholly separated from the execution. But when I saw that anger and violence prevailed every day more and more, and that things were hastening towards an incurable alienation of our colonies, I confess my caution gave way. I felt this as one of those few moments in which decorum yields to an higher duty. Public calamity is a mighty leveller; and there are occasions when any, even the slightest, chance of doing good must be laid hold on, even by the most inconsiderable person.

To restore order and repose to an empire so great and so distracted as ours is, merely in the attempt, an undertaking that would ennoble the flights of the highest genius, and obtain pardon for the efforts of the meanest understanding. Struggling a good while with these thoughts, by degrees I felt myself more firm. I derived, at length, some confidence from what in other circumstances usually produces timidity. I grew less anxious, even from the idea of my own insignificance. For, judging of what you are by what you ought to be, I persuaded myself that you would not reject a reasonable proposition because it had nothing but its reason to recommend it. On the other hand, being totally destitute of all shadow of influence, natural or adventitious, I was very sure, that, if my proposition were futile or dangerous, if it were weakly conceived or improperly timed, there was nothing exterior to it of power to awe, dazzle, or delude you. You will see it just as it is, and you will treat it just as it deserves.

The proposition is peace. Not peace through the medium of war; not peace to be hunted through the labyrinth of intricate and endless negotiations; not peace to arise out of universal discord, fomented from principle, in all parts of the empire; not peace to depend on the juridical determination of perplexing questions, or the precise marking the shadowy boundaries of a complex government. It is simple peace, sought in its natural course and in its ordinary haunts. It is peace sought in the spirit of peace, and laid in principles purely pacific. I propose, by removing the ground of the difference, and by restoring the former unsuspecting confidence of the colonies in the mother country, to give permanent satisfaction to your people,—and (far from a scheme of ruling by discord) to reconcile them to each other in the same act and by the bond of the very same interest which reconciles them to British government.

My idea is nothing more. Refined policy ever has been the parent of confusion,—and ever will be so, as long as the world endures. Plain good intention, which is as easily discovered at the first view as fraud is surely detected at last, is, let me say, of no mean force in the government of mankind. Genuine simplicity of heart is an healing and cementing principle. My plan, therefore, being formed upon the most simple grounds imaginable, may disappoint some people, when they hear it. It has nothing to recommend it to the pruriency of curious ears. There is nothing at all new and captivating in it. It has nothing of the splendor of the project which has been lately laid upon your table by the noble lord in the blue riband.
%[20] 
\footnote{'That when the governor, council, and assembly, or general court, of any of his Majesty's provinces or colonies in America shall propose to make provision, according to the condition, circumstances, and situation of such province or colony, for contributing their proportion to the common defence, (such proportion to be raised under the authority of the general court or general assembly of such province or colony, and disposable by Parliament,) and shall engage to make provision, also for the support of the civil government and the administration, of justice in such province or colony, it will be proper, if such proposal shall be approved by his Majesty and the two Houses of Parliament, and for so long as such provision shall be made accordingly, to forbear, in respect of such province or colony, to levy any duty, tax, or assessment, or to impose any farther duty, tax, or assessment, except only such duties as it may be expedient to continue to levy or to impose for the regulation of commerce: the net produce of the duties last mentioned to be carried to the account of such province or colony respectively.'—Resolution moved by Lord North in the Committee, and agreed to by the House, 27th February, 1775.}
It does not propose to fill your lobby with squabbling colony agents, who will require the interposition of your mace at every instant to keep the peace amongst them. It does not institute a magnificent auction of finance, where captivated provinces come to general ransom by bidding against each other, until you knock down the hammer, and determine a proportion of payments beyond all the powers of algebra to equalize and settle.

The plan which I shall presume to suggest derives, however, one great advantage from the proposition and registry of that noble lord's project. The idea of conciliation is admissible. First, the House, in accepting the resolution moved by the noble lord, has admitted, notwithstanding the menacing front of our address, notwithstanding our heavy bill of pains and penalties, that we do not think ourselves precluded from all ideas of free grace and bounty.

The House has gone farther: it has declared conciliation admissible previous to any submission on the part of America. It has even shot a good deal beyond that mark, and has admitted that the complaints of our former mode of exerting the right of taxation were not wholly unfounded. That right thus exerted is allowed to have had something reprehensible in it,—something unwise, or something grievous; since, in the midst of our heat and resentment, we, of ourselves, have proposed a capital alteration, and, in order to get rid of what seemed so very exceptionable, have instituted a mode that is altogether new,—one that is, indeed, wholly alien from all the ancient methods and forms of Parliament.

The principle of this proceeding is large enough for my purpose. The means proposed by the noble lord for carrying his ideas into execution, I think, indeed, are very indifferently suited to the end; and this I shall endeavor to show you before I sit down. But, for the present, I take my ground on the admitted principle. I mean to give peace. Peace implies reconciliation; and where there has been a material dispute, reconciliation does in a manner always imply concession on the one part or on the other. In this state of things I make no difficulty in affirming that the proposal ought to originate from us. Great and acknowledged force is not impaired, either in effect or in opinion, by an unwillingness to exert itself. The superior power may offer peace with honor and with safety. Such an offer from such a power will be attributed to magnanimity. But the concessions of the weak are the concessions of fear. When such a one is disarmed, he is wholly at the mercy of his superior; and he loses forever that time and those chances which, as they happen to all men, are the strength and resources of all inferior power.

The capital leading questions on which you must this day decide are these two: First, whether you ought to concede; and secondly, what your concession ought to be. On the first of these questions we have gained (as I have just taken the liberty of observing to you) some ground. But I am sensible that a good deal more is still to be done. Indeed, Sir, to enable us to determine both on the one and the other of these great questions with a firm and precise judgment, I think it may be necessary to consider distinctly the true nature and the peculiar circumstances of the object which we have before us: because, after all our struggle, whether we will or not, we must govern America according to that nature and to those circumstances, and not according to our own imaginations, not according to abstract ideas of right, by no means according to mere general theories of government, the resort to which appears to me, in our present situation, no better than arrant trifling. I shall therefore endeavor, with your leave, to lay before you some of the most material of these circumstances in as full and as clear a manner as I am able to state them.

The first thing that we have to consider with regard to the nature of the object is the number of people in the colonies. I have taken for some years a good deal of pains on that point. I can by no calculation justify myself in placing the number below two millions of inhabitants of our own European blood and color,—besides at least 500,000 others, who form no inconsiderable part of the strength and opulence of the whole. This, Sir, is, I believe, about the true number. There is no occasion to exaggerate, where plain truth is of so much weight and importance. But whether I put the present numbers too high or too low is a matter of little moment. Such is the strength with which population shoots in that part of the world, that, state the numbers as high as we will, whilst the dispute continues, the exaggeration ends. Whilst we are discussing any given magnitude, they are grown to it. Whilst we spend our time in deliberating on the mode of governing two millions, we shall find we have millions more to manage. Your children do not grow faster from infancy to manhood than they spread from families to communities, and from villages to nations.

I put this consideration of the present and the growing numbers in the front of our deliberation, because, Sir, this consideration will make it evident to a blunter discernment than yours, that no partial, narrow, contracted, pinched, occasional system will be at all suitable to such an object. It will show you that it is not to be considered as one of those minima which are out of the eye and consideration of the law,—not a paltry excrescence of the state,—not a mean dependant, who may be neglected with little damage and provoked with little danger. It will prove that some degree of care and caution is required in the handling such an object; it will show that you ought not, in reason, to trifle with so large a mass of the interests and feelings of the human race. You could at no time do so without guilt; and be assured you will not be able to do it long with impunity.

But the population of this country, the great and growing population, though a very important consideration, will lose much of its weight, if not combined with other circumstances. The commerce of your colonies is out of all proportion beyond the numbers of the people. This ground of their commerce, indeed, has been trod some days ago, and with great ability, by a distinguished person,
%[21] 
\footnote{Mr. Glover.}
at your bar. This gentleman, after thirty-five years,—it is so long since he first appeared at the same place to plead for the commerce of Great Britain,—has come again before you to plead the same cause, without any other effect of time than that to the fire of imagination and extent of erudition, which even then marked him as one of the first literary characters of his age, he has added a consummate knowledge in the commercial interest of his country, formed by a long course of enlightened and discriminating experience.

Sir, I should be inexcusable in coming after such a person with any detail, if a great part of the members who now fill the House had not the misfortune to be absent when he appeared at your bar. Besides, Sir, I propose to take the matter at periods of time somewhat different from his. There is, if I mistake not, a point of view from whence, if you will look at this subject, it is impossible that it should not make an impression upon you.

I have in my hand two accounts: one a comparative state of the export trade of England to its colonies, as it stood in the year 1704, and as it stood in the year 1772; the other a state of the export trade of this country to its colonies alone, as it stood in 1772, compared with the whole trade of England to all parts of the world (the colonies included) in the year 1704. They are from good vouchers: the latter period from the accounts on your table; the earlier from an original manuscript of Davenant, who first established the Inspector-General's office, which has been ever since his time so abundant a source of Parliamentary information.

The export trade to the colonies consists of three great branches: the African, which, terminating almost wholly in the colonies, must be put to the account of their commerce; the West Indian; and the North American. All these are so interwoven, that the attempt to separate them would tear to pieces the contexture of the whole, and, if not entirely destroy, would very much depreciate, the value of all the parts. I therefore consider these three denominations to be, what in effect they are, one trade.

The trade to the colonies, taken on the export side, at the beginning of this century, that is, in the year 1704, stood thus:—

\begin{center}
\begin{tabular}{l r}
Exports to North America and the West Indies & £ 483,265 \\
To Africa & 86,665 \\
	      & ———  \\
          & £ 569,930
\end{tabular}
\end{center}

In the year 1772, which I take as a middle year between the highest and lowest of those lately laid on your table, the account was as follows:—

\begin{center}
\begin{tabular}{l r}
To North America and the West Indies & £ 4,791,734 \\
To Africa	& 866,398 \\
To which if you add the export trade from Scotland, & \\
which had in 1704 no existence & 364,000 \\
	      & ———  \\
          & £ 6,024,171
\end{tabular}
\end{center}

From five hundred and odd thousand, it has grown to six millions. It has increased no less than twelve-fold. This is the state of the colony trade, as compared with itself at these two periods, within this century;—and this is matter for meditation. But this is not all. Examine my second account. See how the export trade to the colonies alone in 1772 stood in the other point of view, that is, as compared to the whole trade of England in 1704.

\begin{center}
\begin{tabular}{l r}
The whole export trade of England, including that & \\
to the colonies, in 1704 & £6,509,000 \\
Export to the colonies alone, in 1772 & 6,024,000 \\
	       & ———  \\
Difference & £485,000
\end{tabular}
\end{center}

The trade with America alone is now within less than 500,000l. of being equal to what this great commercial nation, England, carried on at the beginning of this century with the whole world! If I had taken the largest year of those on your table, it would rather have exceeded. But, it will be said, is not this American trade an unnatural protuberance, that has drawn the juices from the rest of the body? The reverse. It is the very food that has nourished every other part into its present magnitude. Our general trade has been greatly augmented, and augmented more or less in almost every part to which it ever extended, but with this material difference: that of the six millions which in the beginning of the century constituted the whole mass of our export commerce the colony trade was but one twelfth part; it is now (as a part of sixteen millions) considerably more than a third of the whole. This is the relative proportion of the importance of the colonies at these two periods: and all reasoning concerning our mode of treating them must have this proportion as its basis, or it is a reasoning weak, rotten, and sophistical.

Mr. Speaker, I cannot prevail on myself to hurry over this great consideration. It is good for us to be here. We stand where we have an immense view of what is, and what is past. Clouds indeed, and darkness, rest upon the future. Let us, however, before we descend from this noble eminence, reflect that this growth of our national prosperity has happened within the short period of the life of man. It has happened within sixty-eight years. There are those alive whose memory might touch the two extremities. For instance, my Lord Bathurst might remember all the stages of the progress. He was in 1704 of an age at least to be made to comprehend such things. He was then old enough acta parentum jam legere, et quæ sit poterit cognoscere virtus. Suppose, Sir, that the angel of this auspicious youth, foreseeing the many virtues which made him one of the most amiable, as he is one of the most fortunate men of his age, had opened to him in vision, that, when, in the fourth generation, the third prince of the House of Brunswick had sat twelve years on the throne of that nation which (by the happy issue of moderate and healing councils) was to be made Great Britain, he should see his son, Lord Chancellor of England, turn back the current of hereditary dignity to its fountain, and raise him to an higher rank of peerage, whilst he enriched the family with a new one,—if, amidst these bright and happy scenes of domestic honor and prosperity, that angel should have drawn up the curtain, and unfolded the rising glories of his country, and whilst he was gazing with admiration on the then commercial grandeur of England, the genius should point out to him a little speck, scarce visible in the mass of the national interest, a small seminal principle rather than a formed body, and should tell him,—"Young man, there is America,—which at this day serves for little more than to amuse you with stories of savage men and uncouth manners, yet shall, before you taste of death, show itself equal to the whole of that commerce which now attracts the envy of the world. Whatever England has been growing to by a progressive increase of improvement, brought in by varieties of people, by succession of civilizing conquests and civilizing settlements in a series of seventeen hundred years, you shall see as much added to her by America in the course of a single life!" If this state of his country had been foretold to him, would it not require all the sanguine credulity of youth, and all the fervid glow of enthusiasm, to make him believe it? Fortunate man, he has lived to see it! Fortunate indeed, if he lives to see nothing that shall vary the prospect, and cloud the setting of his day!

Excuse me, Sir, if, turning from such thoughts, I resume this comparative view once more. You have seen it on a large scale; look at it on a small one. I will point out to your attention a particular instance of it in the single province of Pennsylvania. In the year 1704, that province called for 11,459l. in value of your commodities, native and foreign. This was the whole. What did it demand in 1772! Why, nearly fifty times as much; for in that year the export to Pennsylvania was 507,909l., nearly equal to the export to all the colonies together in the first period.

I choose, Sir, to enter into these minute and particular details; because generalities, which in all other cases are apt to heighten and raise the subject, have here a tendency to sink it. When we speak of the commerce with our colonies, fiction lags after truth, invention is unfruitful, and imagination cold and barren.

So far, Sir, as to the importance of the object in the view of its commerce, as concerned in the exports from England. If I were to detail the imports, I could show how many enjoyments they procure which deceive the burden of life, how many materials which invigorate the springs of national industry and extend and animate every part of our foreign and domestic commerce. This would be a curious subject indeed,—but I must prescribe bounds to myself in a matter so vast and various.

I pass, therefore, to the colonies in another point of view,—their agriculture. This they have prosecuted with such a spirit, that, besides feeding plentifully their own growing multitude, their annual export of grain, comprehending rice, has some years ago exceeded a million in value. Of their last harvest, I am persuaded, they will export much more. At the beginning of the century some of these colonies imported corn from the mother country. For some time past the Old World has been fed from the New. The scarcity which you have felt would have been a desolating famine, if this child of your old age, with a true filial piety, with a Roman charity, had not put the full breast of its youthful exuberance to the mouth of its exhausted parent.

As to the wealth which the colonies have drawn from the sea by their fisheries, you had all that matter fully opened at your bar. You surely thought those acquisitions of value, for they seemed even to excite your envy; and yet the spirit by which that enterprising employment has been exercised ought rather, in my opinion, to have raised your esteem and admiration. And pray, Sir, what in the world is equal to it? Pass by the other parts, and look at the manner in which the people of New England have of late carried on the whale-fishery. Whilst we follow them among the tumbling mountains of ice, and behold them penetrating into the deepest frozen recesses of Hudson's Bay and Davis's Straits, whilst we are looking for them beneath the arctic circle, we hear that they have pierced into the opposite region of polar cold, that they are at the antipodes, and engaged under the frozen serpent of the South. Falkland Island, which seemed too remote and romantic an object for the grasp of national ambition, is but a stage and resting-place in the progress of their victorious industry. Nor is the equinoctial heat more discouraging to them than the accumulated winter of both the poles. We know, that, whilst some of them draw the line and strike the harpoon on the coast of Africa, others run the longitude, and pursue their gigantic game along the coast of Brazil. No sea but what is vexed by their fisheries. No climate that is not witness to their toils. Neither the perseverance of Holland, nor the activity of France, nor the dexterous and firm sagacity of English enterprise, ever carried this most perilous mode of hardy industry to the extent to which it has been pushed by this recent people,—a people who are still, as it were, but in the gristle, and not yet hardened into the bone of manhood. When I contemplate these things,—when I know that the colonies in general owe little or nothing to any care of ours, and that they are not squeezed into this happy form by the constraints of watchful and suspicious government, but that, through a wise and salutary neglect, a generous nature has been suffered to take her own way to perfection,—when I reflect upon these effects, when I see how profitable they have been to us, I feel all the pride of power sink, and all presumption in the wisdom of human contrivances melt and die away within me,—my rigor relents,—I pardon something to the spirit of liberty.

I am sensible, Sir, that all which I have asserted in my detail is admitted in the gross, but that quite a different conclusion is drawn from it. America, gentlemen say, is a noble object,—it is an object well worth fighting for. Certainly it is, if fighting a people be the best way of gaining them. Gentlemen in this respect will be led to their choice of means by their complexions and their habits. Those who understand the military art will of course have some predilection for it. Those who wield the thunder of the state may have more confidence in the efficacy of arms. But I confess, possibly for want of this knowledge, my opinion is much more in favor of prudent management than of force,—considering force not as an odious, but a feeble instrument, for preserving a people so numerous, so active, so growing, so spirited as this, in a profitable and subordinate connection with us.

First, Sir, permit me to observe, that the use of force alone is but temporary. It may subdue for a moment; but it does not remove the necessity of subduing again: and a nation is not governed which is perpetually to be conquered.

My next objection is its uncertainty. Terror is not always the effect of force, and an armament is not a victory. If you do not succeed, you are without resource: for, conciliation failing, force remains; but, force failing, no further hope of reconciliation is left. Power and authority are sometimes bought by kindness; but they can never be begged as alms by an impoverished and defeated violence.

A further objection to force is, that you impair the object by your very endeavors to preserve it. The thing you fought for is not the thing which you recover, but depreciated, sunk, wasted, and consumed in the contest. Nothing less will content me than whole America. I do not choose to consume its strength along with our own; because in all parts it is the British strength that I consume. I do not choose to be caught by a foreign enemy at the end of this exhausting conflict, and still less in the midst of it. I may escape, but I can make no insurance against such an event. Let me add, that I do not choose wholly to break the American spirit; because it is the spirit that has made the country.

Lastly, we have no sort of experience in favor of force as an instrument in the rule of our colonies. Their growth and their utility has been owing to methods altogether different. Our ancient indulgence has been said to be pursued to a fault. It may be so; but we know, if feeling is evidence, that our fault was more tolerable than our attempt to mend it, and our sin far more salutary than our penitence.

These, Sir, are my reasons for not entertaining that high opinion of untried force by which many gentlemen, for whose sentiments in other particulars I have great respect, seem to be so greatly captivated. But there is still behind a third consideration concerning this object, which serves to determine my opinion on the sort of policy which ought to be pursued in the management of America, even more than its population and its commerce: I mean its temper and character.

In this character of the Americans a love of freedom is the predominating feature which marks and distinguishes the whole: and as an ardent is always a jealous affection, your colonies become suspicious, restive, and untractable, whenever they see the least attempt to wrest from them by force, or shuffle from them by chicane, what they think the only advantage worth living for. This fierce spirit of liberty is stronger in the English colonies, probably, than in any other people of the earth, and this from a great variety of powerful causes; which, to understand the true temper of their minds, and the direction which this spirit takes, it will not be amiss to lay open somewhat more largely.

First, the people of the colonies are descendants of Englishmen. England, Sir, is a nation which still, I hope, respects, and formerly adored, her freedom. The colonists emigrated from you when this part of your character was most predominant; and they took this bias and direction the moment they parted from your hands. They are therefore not only devoted to liberty, but to liberty according to English ideas and on English principles. Abstract liberty, like other mere abstractions, is not to be found. Liberty inheres in some sensible object; and every nation has formed to itself some favorite point, which by way of eminence becomes the criterion of their happiness. It happened, you know, Sir, that the great contests for freedom in this country were from the earliest times chiefly upon the question of taxing. Most of the contests in the ancient commonwealths turned primarily on the right of election of magistrates, or on the balance among the several orders of the state. The question of money was not with them so immediate. But in England it was otherwise. On this point of taxes the ablest pens and most eloquent tongues have been exercised, the greatest spirits have acted and suffered. In order to give the fullest satisfaction concerning the importance of this point, it was not only necessary for those who in argument defended the excellence of the English Constitution to insist on this privilege of granting money as a dry point of fact, and to prove that the right had been acknowledged in ancient parchments and blind usages to reside in a certain body called an House of Commons: they went much further: they attempted to prove, and they succeeded, that in theory it ought to be so, from the particular nature of a House of Commons, as an immediate representative of the people, whether the old records had delivered this oracle or not. They took infinite pains to inculcate, as a fundamental principle, that in all monarchies the people must in effect themselves, mediately or immediately, possess the power of granting their own money, or no shadow of liberty could subsist. The colonies draw from you, as with their life-blood, these ideas and principles. Their love of liberty, as with you, fixed and attached on this specific point of taxing. Liberty might be safe or might be endangered in twenty other particulars without their being much pleased or alarmed. Here they felt its pulse; and as they found that beat, they thought themselves sick or sound. I do not say whether they were right or wrong in applying your general arguments to their own case. It is not easy, indeed, to make a monopoly of theorems and corollaries. The fact is, that they did thus apply those general arguments; and your mode of governing them, whether through lenity or indolence, through wisdom or mistake, confirmed them in the imagination, that they, as well as you, had an interest in these common principles.

They were further confirmed in this pleasing error by the form of their provincial legislative assemblies. Their governments are popular in an high degree: some are merely popular; in all, the popular representative is the most weighty; and this share of the people in their ordinary government never fails to inspire them with lofty sentiments, and with a strong aversion from whatever tends to deprive them of their chief importance.

If anything were wanting to this necessary operation of the form of government, religion would have given it a complete effect. Religion, always a principle of energy, in this new people is no way worn out or impaired; and their mode of professing it is also one main cause of this free spirit. The people are Protestants, and of that kind which is the most adverse to all implicit submission of mind and opinion. This is a persuasion not only favorable to liberty, but built upon it. I do not think, Sir, that the reason of this averseness in the dissenting churches from all that looks like absolute government is so much to be sought in their religious tenets as in their history. Every one knows that the Roman Catholic religion is at least coeval with most of the governments where it prevails, that it has generally gone hand in hand with them, and received great favor and every kind of support from authority. The Church of England, too, was formed from her cradle under the nursing care of regular government. But the dissenting interests have sprung up in direct opposition to all the ordinary powers of the world, and could justify that opposition only on a strong claim to natural liberty. Their very existence depended on the powerful and unremitted assertion of that claim. All Protestantism, even the most cold and passive, is a sort of dissent. But the religion most prevalent in our northern colonies is a refinement on the principle of resistance: it is the dissidence of dissent, and the protestantism of the Protestant religion. This religion, under a variety of denominations agreeing in nothing but in the communion of the spirit of liberty, is predominant in most of the northern provinces, where the Church of England, notwithstanding its legal rights, is in reality no more than a sort of private sect, not composing, most probably, the tenth of the people. The colonists left England when this spirit was high, and in the emigrants was the highest of all; and even that stream of foreigners which has been constantly flowing into these colonies has, for the greatest part, been composed of dissenters from the establishments of their several countries, and have brought with them a temper and character far from alien to that of the people with whom they mixed.

Sir, I can perceive, by their manner, that some gentlemen object to the latitude of this description, because in the southern colonies the Church of England forms a large body, and has a regular establishment. It is certainly true. There is, however, a circumstance attending these colonies, which, in my opinion, fully counterbalances this difference, and makes the spirit of liberty still more high and haughty than in those to the northward. It is, that in Virginia and the Carolinas they have a vast multitude of slaves. Where this is the case in any part of the world, those who are free are by far the most proud and jealous of their freedom. Freedom is to them not only an enjoyment, but a kind of rank and privilege. Not seeing there, that freedom, as in countries where it is a common blessing, and as broad and general as the air, may be united with much abject toil, with great misery, with all the exterior of servitude, liberty looks, amongst them, like something that is more noble and liberal. I do not mean, Sir, to commend the superior morality of this sentiment, which has at least as much pride as virtue in it; but I cannot alter the nature of man. The fact is so; and these people of the southern colonies are much more strongly, and with an higher and more stubborn spirit, attached to liberty, than those to the northward. Such were all the ancient commonwealths; such were our Gothic ancestors; such in our days were the Poles; and such will be all masters of slaves, who are not slaves themselves. In such a people, the haughtiness of domination combines with the spirit of freedom, fortifies it, and renders it invincible.

Permit me, Sir, to add another circumstance in our colonies, which contributes no mean part towards the growth and effect of this untractable spirit: I mean their education. In no country, perhaps, in the world is the law so general a study. The profession itself is numerous and powerful, and in most provinces it takes the lead. The greater number of the deputies sent to the Congress were lawyers. But all who read, and most do read, endeavor to obtain some smattering in that science. I have been told by an eminent bookseller, that in no branch of his business, after tracts of popular devotion, were so many books as those on the law exported to the plantations. The colonists have now fallen into the way of printing them for their own use. I hear that they have sold nearly as many of Blackstone's "Commentaries" in America as in England. General Gage marks out this disposition very particularly in a letter on your table. He states, that all the people in his government are lawyers, or smatterers in law,—and that in Boston they have been enabled, by successful chicane, wholly to evade many parts of one of your capital penal constitutions. The smartness of debate will say, that this knowledge ought to teach them more clearly the rights of legislature, their obligations to obedience, and the penalties of rebellion. All this is mighty well. But my honorable and learned friend
%[22] 
\footnote{The Attorney-General.}
on the floor, who condescends to mark what I say for animadversion, will disdain that ground. He has heard, as well as I, that, when great honors and great emoluments do not win over this knowledge to the service of the state, it is a formidable adversary to government. If the spirit be not tamed and broken by these happy methods, it is stubborn and litigious. Abeunt studia in mores. This study renders men acute, inquisitive, dexterous, prompt in attack, ready in defence, full of resources. In other countries, the people, more simple, and of a less mercurial cast, judge of an ill principle in government only by an actual grievance; here they anticipate the evil, and judge of the pressure of the grievance by the badness of the principle. They augur misgovernment at a distance, and snuff the approach of tyranny in every tainted breeze.

The last cause of this disobedient spirit in the colonies is hardly less powerful than the rest, as it is not merely moral, but laid deep in the natural constitution of things. Three thousand miles of ocean lie between you and them. No contrivance can prevent the effect of this distance in weakening government. Seas roll, and months pass, between the order and the execution; and the want of a speedy explanation of a single point is enough to defeat an whole system. You have, indeed, winged ministers of vengeance, who carry your bolts in their pounces to the remotest verge of the sea: but there a power steps in, that limits the arrogance of raging passions and furious elements, and says, "So far shalt thou go, and no farther." Who are you, that should fret and rage, and bite the chains of Nature? Nothing worse happens to you than does to all nations who have extensive empire; and it happens in all the forms into which empire can be thrown. In large bodies, the circulation of power must be less vigorous at the extremities. Nature has said it. The Turk cannot govern Egypt, and Arabia, and Kurdistan, as he governs Thrace; nor has he the same dominion in Crimea and Algiers which he has at Brusa and Smyrna. Despotism itself is obliged to truck and huckster. The Sultan gets such obedience as he can. He governs with a loose rein, that he may govern at all; and the whole of the force and vigor of his authority in his centre is derived from a prudent relaxation in all his borders. Spain, in her provinces, is perhaps not so well obeyed as you are in yours. She complies, too; she submits; she watches times. This is the immutable condition, the eternal law, of extensive and detached empire.

Then, Sir, from these six capital sources, of descent, of form of government, of religion in the northern provinces, of manners in the southern, of education, of the remoteness of situation from the first mover of government,—from all these causes a fierce spirit of liberty has grown up. It has grown with the growth of the people in your colonies, and increased with the increase of their wealth: a spirit, that, unhappily meeting with an exercise of power in England, which, however lawful, is not reconcilable to any ideas of liberty, much less with theirs, has kindled this flame that is ready to consume us.

I do not mean to commend either the spirit in this excess, or the moral causes which produce it. Perhaps a more smooth and accommodating spirit of freedom in them would be more acceptable to us. Perhaps ideas of liberty might be desired more reconcilable with an arbitrary and boundless authority. Perhaps we might wish the colonists to be persuaded that their liberty is more secure when held in trust for them by us (as their guardians during a perpetual minority) than with any part of it in their own hands. But the question is not, whether their spirit deserves praise or blame,—what, in the name of God, shall we do with it? You have before you the object, such as it is,—with all its glories, with all its imperfections on its head. You see the magnitude, the importance, the temper, the habits, the disorders. By all these considerations we are strongly urged to determine something concerning it. We are called upon to fix some rule and line for our future conduct, which may give a little stability to our politics, and prevent the return of such unhappy deliberations as the present. Every such return will bring the matter before us in a still more untractable form. For what astonishing and incredible things have we not seen already! What monsters have not been generated from this unnatural contention! Whilst every principle of authority and resistance has been pushed, upon both sides, as far as it would go, there is nothing so solid and certain, either in reasoning or in practice, that has not been shaken. Until very lately, all authority in America seemed to be nothing but an emanation from yours. Even the popular part of the colony constitution derived all its activity, and its first vital movement, from the pleasure of the crown. We thought, Sir, that the utmost which the discontented colonists could do was to disturb authority; we never dreamt they could of themselves supply it, knowing in general what an operose business it is to establish a government absolutely new. But having, for our purposes in this contention, resolved that none but an obedient assembly should sit, the humors of the people there, finding all passage through the legal channel stopped, with great violence broke out another way. Some provinces have tried their experiment, as we have tried ours; and theirs has succeeded. They have formed a government sufficient for its purposes, without the bustle of a revolution, or the troublesome formality of an election. Evident necessity and tacit consent have done the business in an instant. So well they have done it, that Lord Dunmore (the account is among the fragments on your table) tells you that the new institution is infinitely better obeyed than the ancient government ever was in its most fortunate periods. Obedience is what makes government, and not the names by which it is called: not the name of Governor, as formerly, or Committee, as at present. This new government has originated directly from the people, and was not transmitted through any of the ordinary artificial media of a positive constitution. It was not a manufacture ready formed, and transmitted to them in that condition from England. The evil arising from hence is this: that the colonists having once found the possibility of enjoying the advantages of order in the midst of a struggle for liberty, such struggles will not henceforward seem so terrible to the settled and sober part of mankind as they had appeared before the trial.

Pursuing the same plan of punishing by the denial of the exercise of government to still greater lengths, we wholly abrogated the ancient government of Massachusetts. We were confident that the first feeling, if not the very prospect of anarchy, would instantly enforce a complete submission. The experiment was tried. A new, strange, unexpected face of things appeared. Anarchy is found tolerable. A vast province has now subsisted, and subsisted in a considerable degree of health and vigor, for near a twelvemonth, without governor, without public council, without judges, without executive magistrates. How long it will continue in this state, or what may arise out of this unheard-of situation, how can the wisest of us conjecture? Our late experience has taught us that many of those fundamental principles formerly believed infallible are either not of the importance they were imagined to be, or that we have not at all adverted to some other far more important and far more powerful principles which entirely overrule those we had considered as omnipotent. I am much against any further experiments which tend to put to the proof any more of these allowed opinions which contribute so much to the public tranquillity. In effect, we suffer as much at home by this loosening of all ties, and this concussion of all established opinions, as we do abroad. For, in order to prove that the Americans have no right to their liberties, we are every day endeavoring to subvert the maxims which preserve the whole spirit of our own. To prove that the Americans ought not to be free, we are obliged to depreciate the value of freedom itself; and we never seem to gain a paltry advantage over them in debate, without attacking some of those principles, or deriding some of those feelings, for which our ancestors have shed their blood.

But, Sir, in wishing to put an end to pernicious experiments, I do not mean to preclude the fullest inquiry. Far from it. Far from deciding on a sudden or partial view, I would patiently go round and round the subject, and survey it minutely in every possible aspect. Sir, if I were capable of engaging you to an equal attention, I would state, that, as far as I am capable of discerning, there are but three ways of proceeding relative to this stubborn spirit which prevails in your colonies and disturbs your government. These are,—to change that spirit, as inconvenient, by removing the causes,—to prosecute it, as criminal,—or to comply with it, as necessary. I would not be guilty of an imperfect enumeration; I can think of but these three. Another has, indeed, been started,—that of giving up the colonies; but it met so slight a reception that I do not think myself obliged to dwell a great while upon it. It is nothing but a little sally of anger, like the frowardness of peevish children, who, when they cannot get all they would have, are resolved to take nothing.

The first of these plans—to change the spirit, as inconvenient, by removing the causes—I think is the most like a systematic proceeding. It is radical in its principle; but it is attended with great difficulties: some of them little short, as I conceive, of impossibilities. This will appear by examining into the plans which have been proposed.

As the growing population of the colonies is evidently one cause of their resistance, it was last session mentioned in both Houses, by men of weight, and received not without applause, that, in order to check this evil, it would be proper for the crown to make no further grants of land. But to this scheme there are two objections. The first, that there is already so much unsettled land in private hands as to afford room for an immense future population, although the crown not only withheld its grants, but annihilated its soil. If this be the case, then the only effect of this avarice of desolation, this hoarding of a royal wilderness, would be to raise the value of the possessions in the hands of the great private monopolists, without any adequate check to the growing and alarming mischief of population.

But if you stopped your grants, what would be the consequence? The people would occupy without grants. They have already so occupied in many places. You cannot station garrisons in every part of these deserts. If you drive the people from one place, they will carry on their annual tillage, and remove with their flocks and herds to another. Many of the people in the back settlements are already little attached to particular situations. Already they have topped the Appalachian mountains. From thence they behold before them an immense plain, one vast, rich, level meadow: a square of five hundred miles. Over this they would wander without a possibility of restraint; they would change their manners with the habits of their life; would soon forget a government by which they were disowned; would become hordes of English Tartars, and, pouring down upon your unfortified frontiers a fierce and irresistible cavalry, become masters of your governors and your counsellors, your collectors and comptrollers, and of all the slaves that adhered to them. Such would, and, in no long time, must be, the effect of attempting to forbid as a crime, and to suppress as an evil, the command and blessing of Providence, "Increase and multiply." Such would be the happy result of an endeavor to keep as a lair of wild beasts that earth which God by an express charter has given to the children of men. Far different, and surely much wiser, has been our policy hitherto. Hitherto we have invited our people, by every kind of bounty, to fixed establishments. We have invited the husbandman to look to authority for his title. We have taught him piously to believe in the mysterious virtue of wax and parchment. We have thrown each tract of land, as it was peopled, into districts, that the ruling power should never be wholly out of sight. We have settled all we could; and we have carefully attended every settlement with government.

Adhering, Sir, as I do, to this policy, as well as for the reasons I have just given, I think this new project of hedging in population to be neither prudent nor practicable.

To impoverish the colonies in general, and in particular to arrest the noble course of their marine enterprises, would be a more easy task. I freely confess it. We have shown a disposition to a system of this kind,—a disposition even to continue the restraint after the offence,—looking on ourselves as rivals to our colonies, and persuaded that of course we must gain all that they shall lose. Much mischief we may certainly do. The power inadequate to all other things is often more than sufficient for this. I do not look on the direct and immediate power of the colonies to resist our violence as very formidable. In this, however, I may be mistaken. But when I consider that we have colonies for no purpose but to be serviceable to us, it seems to my poor understanding a little preposterous to make them unserviceable, in order to keep them obedient. It is, in truth, nothing more than the old, and, as I thought, exploded problem of tyranny, which proposes to beggar its subjects into submission. But remember, when you have completed your system of impoverishment, that Nature still proceeds in her ordinary course; that discontent will increase with misery; and that there are critical moments in the fortune of all states, when they who are too weak to contribute to your prosperity may be strong enough to complete your ruin. Spoliatis arma supersunt.

The temper and character which prevail in our colonies are, I am afraid, unalterable by any human art. We cannot, I fear, falsify the pedigree of this fierce people, and persuade them that they are not sprung from a nation in whose veins the blood of freedom circulates. The language in which they would hear you tell them this tale would detect the imposition; your speech would betray you. An Englishman is the unfittest person on earth to argue another Englishman into slavery.

I think it is nearly as little in our power to change their republican religion as their free descent, or to substitute the Roman Catholic as a penalty, or the Church of England as an improvement. The mode of inquisition and dragooning is going out of fashion in the Old World, and I should not confide much to their efficacy in the New. The education of the Americans is also on the same unalterable bottom with their religion. You cannot persuade them to burn their books of curious science, to banish their lawyers from their courts of law, or to quench the lights of their assemblies by refusing to choose those persons who are best read in their privileges. It would be no less impracticable to think of wholly annihilating the popular assemblies in which these lawyers sit. The army, by which we must govern in their place, would be far more chargeable to us, not quite so effectual, and perhaps, in the end, full as difficult to be kept in obedience.

With regard to the high aristocratic spirit of Virginia and the southern colonies, it has been proposed, I know, to reduce it by declaring a general enfranchisement of their slaves. This project has had its advocates and panegyrists; yet I never could argue myself into any opinion of it. Slaves are often much attached to their masters. A general wild offer of liberty would not always be accepted. History furnishes few instances of it. It is sometimes as hard to persuade slaves to be free as it is to compel freemen to be slaves; and in this auspicious scheme we should have both these pleasing tasks on our hands at once. But when we talk of enfranchisement, do we not perceive that the American master may enfranchise, too, and arm servile hands in defence of freedom?—a measure to which other people have had recourse more than once, and not without success, in a desperate situation of their affairs.

Slaves as these unfortunate black people are, and dull as all men are from slavery, must they not a little suspect the offer of freedom from that very nation which has sold them to their present masters,—from that nation, one of whose causes of quarrel with those masters is their refusal to deal any more in that inhuman traffic? An offer of freedom from England would come rather oddly, shipped to them in an African vessel, which is refused an entry into the ports of Virginia or Carolina, with a cargo of three hundred Angola negroes. It would be curious to see the Guinea captain attempting at the same instant to publish his proclamation of liberty and to advertise his sale of slaves.

But let us suppose all these moral difficulties got over. The ocean remains. You cannot pump this dry; and as long as it continues in its present bed, so long all the causes which weaken authority by distance will continue.

\begin{verse}
Ye Gods! annihilate but space and time, \\
And make two lovers happy,
\end{verse}

was a pious and passionate prayer,—but just as reasonable as many of the serious wishes of very grave and solemn politicians.

If, then, Sir, it seems almost desperate to think of any alterative course for changing the moral causes (and not quite easy to remove the natural) which produce prejudices irreconcilable to the late exercise of our authority, but that the spirit infallibly will continue, and, continuing, will produce such effects as now embarrass us,—the second mode under consideration is, to prosecute that spirit in its overt acts, as criminal.

At this proposition I must pause a moment. The thing seems a great deal too big for my ideas of jurisprudence. It should seem, to my way of conceiving such matters, that there is a very wide difference, in reason and policy, between the mode of proceeding on the irregular conduct of scattered individuals, or even of bands of men, who disturb order within the state, and the civil dissensions which may, from time to time, on great questions, agitate the several communities which compose a great empire. It looks to me to be narrow and pedantic to apply the ordinary ideas of criminal justice to this great public contest. I do not know the method of drawing up an indictment against an whole people. I cannot insult and ridicule the feelings of millions of my fellow-creatures as Sir Edward Coke insulted one excellent individual (Sir Walter Raleigh) at the bar. I am not ripe to pass sentence on the gravest public bodies, intrusted with magistracies of great authority and dignity, and charged with the safety of their fellow-citizens, upon the very same title that I am. I really think that for wise men this is not judicious, for sober men not decent, for minds tinctured with humanity not mild and merciful.

Perhaps, Sir, I am mistaken in my idea of an empire, as distinguished from a single state or kingdom. But my idea of it is this: that an empire is the aggregate of many states under one common head, whether this head be a monarch or a presiding republic. It does, in such constitutions, frequently happen (and nothing but the dismal, cold, dead uniformity of servitude can prevent its happening) that the subordinate parts have many local privileges and immunities. Between these privileges and the supreme common authority the line may be extremely nice. Of course disputes, often, too, very bitter disputes, and much ill blood, will arise. But though every privilege is an exemption (in the case) from the ordinary exercise of the supreme authority, it is no denial of it. The claim of a privilege seems rather, ex vi termini, to imply a superior power: for to talk of the privileges of a state or of a person who has no superior is hardly any better than speaking nonsense. Now in such unfortunate quarrels among the component parts of a great political union of communities, I can scarcely conceive anything more completely imprudent than for the head of the empire to insist, that if any privilege is pleaded against his will or his acts, that his whole authority is denied,—instantly to proclaim rebellion, to beat to arms, and to put the offending provinces under the ban. Will not this, Sir, very soon teach the provinces to make no distinctions on their part? Will it not teach them that the government against which a claim of liberty is tantamount to high treason is a government to which submission is equivalent to slavery? It may not always be quite convenient to impress dependent communities with such an idea.

We are, indeed, in all disputes with the colonies, by the necessity of things, the judge. It is true, Sir. But I confess that the character of judge in my own cause is a thing that frightens me. Instead of filling me with pride, I am exceedingly humbled by it. I cannot proceed with a stern, assured judicial confidence, until I find myself in something more like a judicial character. I must have these hesitations as long as I am compelled to recollect, that, in my little reading upon such contests as these, the sense of mankind has at least as often decided against the superior as the subordinate power. Sir, let me add, too, that the opinion of my having some abstract right in my favor would not put me much at my ease in passing sentence, unless I could be sure that there were no rights which, in their exercise under certain circumstances, were not the most odious of all wrongs and the most vexatious of all injustice. Sir, these considerations have great weight with me, when I find things so circumstanced that I see the same party at once a civil litigant against me in a point of right and a culprit before me, while I sit as criminal judge on acts of his whose moral quality is to be decided upon the merits of that very litigation. Men are every now and then put, by the complexity of human affairs, into strange situations; but justice is the same, let the judge be in what situation he will.

There is, Sir, also a circumstance which convinces me that this mode of criminal proceeding is not (at least in the present stage of our contest) altogether expedient,—which is nothing less than the conduct of those very persons who have seemed to adopt that mode, by lately declaring a rebellion in Massachusetts Bay, as they had formerly addressed to have traitors brought hither, under an act of Henry the Eighth, for trial. For, though rebellion is declared, it is not proceeded against as such; nor have any steps been taken towards the apprehension or conviction of any individual offender, either on our late or our former address; but modes of public coercion have been adopted, and such as have much more resemblance to a sort of qualified hostility towards an independent power than the punishment of rebellious subjects. All this seems rather inconsistent; but it shows how difficult it is to apply these juridical ideas to our present case.

In this situation, let us seriously and coolly ponder. What is it we have got by all our menaces, which have been many and ferocious? What advantage have we derived from the penal laws we have passed, and which, for the time, have been severe and numerous? What advances have we made towards our object, by the sending of a force, which, by land and sea, is no contemptible strength? Has the disorder abated? Nothing less.—When I see things in this situation, after such confident hopes, bold promises, and active exertions, I cannot, for my life, avoid a suspicion that the plan itself is not correctly right.

If, then, the removal of the causes of this spirit of American liberty be, for the greater part, or rather entirely, impracticable,—if the ideas of criminal process be inapplicable, or, if applicable, are in the highest degree inexpedient, what way yet remains? No way is open, but the third and last,—to comply with the American spirit as necessary, or, if you please, to submit, to it as a necessary evil.

If we adopt this mode, if we mean to conciliate and concede, let us see of what nature the concession ought to be. To ascertain the nature of our concession, we must look at their complaint. The colonies complain that they have not the characteristic mark and seal of British freedom. They complain that they are taxed in a Parliament in which they are not represented. If you mean to satisfy them at all, you must satisfy them with regard to this complaint. If you mean to please any people, you must give them the boon which they ask,—not what you may think better for them, but of a kind totally different. Such an act may be a wise regulation, but it is no concession; whereas our present theme is the mode of giving satisfaction.

Sir, I think you must perceive that I am resolved this day to have nothing at all to do with the question of the right of taxation. Some gentlemen startle,—but it is true: I put it totally out of the question. It is less than nothing in my consideration. I do not indeed wonder, nor will you, Sir, that gentlemen of profound learning are fond of displaying it on this profound subject. But my consideration is narrow, confined, and wholly limited to the policy of the question. I do not examine whether the giving away a man's money be a power excepted and reserved out of the general trust of government, and how far all mankind, in all forms of polity, are entitled to an exercise of that right by the charter of Nature,—or whether, on the contrary, a right of taxation is necessarily involved in the general principle of legislation, and inseparable from the ordinary supreme power. These are deep questions, where great names militate against each other, where reason is perplexed, and an appeal to authorities only thickens the confusion: for high and reverend authorities lift up their heads on both sides, and there is no sure footing in the middle. This point is the great Serbonian bog, betwixt Damiata and Mount Casius old, where armies whole have sunk. I do not intend to be overwhelmed in that bog, though in such respectable company. The question with me is, not whether you have a right to render your people miserable, but whether it is not your interest to make them happy. It is not what a lawyer tells me I may do, but what humanity, reason, and justice tell me I ought to do. Is a politic act the worse for being a generous one? Is no concession proper, but that which is made from your want of right to keep what you grant? Or does it lessen the grace or dignity of relaxing in the exercise of an odious claim, because you have your evidence-room full of titles, and your magazines stuffed with arms to enforce them? What signify all those titles and all those arms? Of what avail are they, when the reason of the thing tells me that the assertion of my title is the loss of my suit, and that I could do nothing but wound myself by the use of my own weapons?

Such is steadfastly my opinion of the absolute necessity of keeping up the concord of this empire by a unity of spirit, though in a diversity of operations, that, if I were sure the colonists had, at their leaving this country, sealed a regular compact of servitude, that they had solemnly abjured all the rights of citizens, that they had made a vow to renounce all ideas of liberty for them and their posterity to all generations, yet I should hold myself obliged to conform to the temper I found universally prevalent in my own day, and to govern two million of men, impatient of servitude, on the principles of freedom. I am not determining a point of law; I am restoring tranquillity: and the general character and situation of a people must determine what sort of government is fitted for them. That point nothing else can or ought to determine.

My idea, therefore, without considering whether we yield as matter of right or grant as matter of favor, is, to admit the people of our colonies into an interest in the Constitution, and, by recording that admission in the journals of Parliament, to give them as strong an assurance as the nature of the thing will admit that we mean forever to adhere to that solemn declaration of systematic indulgence.

Some years ago, the repeal of a revenue act, upon its understood principle, might have served to show that we intended an unconditional abatement of the exercise of a taxing power. Such a measure was then sufficient to remove all suspicion and to give perfect content. But unfortunate events since that time may make something further necessary,—and not more necessary for the satisfaction of the colonies than for the dignity and consistency of our own future proceedings.

I have taken a very incorrect measure of the disposition of the House, if this proposal in itself would be received with dislike. I think, Sir, we have few American financiers. But our misfortune is, we are too acute, we are too exquisite in our conjectures of the future, for men oppressed with such great and present evils. The more moderate among the opposers of Parliamentary concession freely confess that they hope no good from taxation; but they apprehend the colonists have further views, and if this point were conceded, they would instantly attack the trade laws. These gentlemen are convinced that this was the intention from the beginning, and the quarrel of the Americans with taxation was no more than a cloak and cover to this design. Such has been the language even of a gentleman
%[23] 
\footnote{Mr. Rice.}
of real moderation, and of a natural temper well adjusted to fair and equal government. I am, however, Sir, not a little surprised at this kind of discourse, whenever I hear it; and I am the more surprised on account of the arguments which I constantly find in company with it, and which are often urged from the same mouths and on the same day.

For instance, when we allege that it is against reason to tax a people under so many restraints in trade as the Americans, the noble lord
%[24] 
\footnote{Lord North.}
in the blue riband shall tell you that the restraints on trade are futile and useless, of no advantage to us, and of no burden to those on whom they are imposed,—that the trade to America is not secured by the Acts of Navigation, but by the natural and irresistible advantage of a commercial preference.

Such is the merit of the trade laws in this posture of the debate. But when strong internal circumstances are urged against the taxes,—when the scheme is dissected,—when experience and the nature of things are brought to prove, and do prove, the utter impossibility of obtaining an effective revenue from the colonies,—when these things are pressed, or rather press themselves, so as to drive the advocates of colony taxes to a clear admission of the futility of the scheme,—then, Sir, the sleeping trade laws revive from their trance, and this useless taxation is to be kept sacred, not for its own sake, but as a counter-guard and security of the laws of trade.

Then, Sir, you keep up revenue laws which are mischievous in order to preserve trade laws that are useless. Such is the wisdom of our plan in both its members. They are separately given up as of no value; and yet one is always to be defended for the sake of the other. But I cannot agree with the noble lord, nor with the pamphlet from whence he seems to have borrowed these ideas concerning the inutility of the trade laws. For, without idolizing them, I am sure they are still, in many ways, of great use to us; and in former times they have been of the greatest. They do confine, and they do greatly narrow, the market for the Americans. But my perfect conviction of this does not help me in the least to discern how the revenue laws form any security whatsoever to the commercial regulations,—or that these commercial regulations are the true ground of the quarrel,—or that the giving way, in any one instance, of authority is to lose all that may remain unconceded.

One fact is clear and indisputable: the public and avowed origin of this quarrel was on taxation. This quarrel has, indeed, brought on new disputes on new questions, but certainly the least bitter, and the fewest of all, on the trade laws. To judge which of the two be the real, radical cause of quarrel, we have to see whether the commercial dispute did, in order of time, precede the dispute on taxation. There is not a shadow of evidence for it. Next, to enable us to judge whether at this moment a dislike to the trade laws be the real cause of quarrel, it is absolutely necessary to put the taxes out of the question by a repeal. See how the Americans act in this position, and then you will be able to discern correctly what is the true object of the controversy, or whether any controversy at all will remain. Unless you consent to remove this cause of difference, it is impossible, with decency, to assert that the dispute is not upon what it is avowed to be. And I would, Sir, recommend to your serious consideration, whether it be prudent to form a rule for punishing people, not on their own acts, but on your conjectures. Surely it is preposterous, at the very best. It is not justifying your anger by their misconduct, but it is converting your ill-will into their delinquency.

But the colonies will go further.—Alas! alas! when will this speculating against fact and reason end? What will quiet these panic fears which we entertain of the hostile effect of a conciliatory conduct? Is it true that no case can exist in which it is proper for the sovereign to accede to the desires of his discontented subjects? Is there anything peculiar in this case, to make a rule for itself? Is all authority of course lost, when it is not pushed to the extreme? Is it a certain maxim, that, the fewer causes of dissatisfaction are left by government, the more the subject will be inclined to resist and rebel?

All these objections being in fact no more than suspicions, conjectures, divinations, formed in defiance of fact and experience, they did not, Sir, discourage me from entertaining the idea of a conciliatory concession, founded on the principles which I have just stated.

In forming a plan for this purpose, I endeavored to put myself in that frame of mind which was the most natural and the most reasonable, and which was certainly the most probable means of securing me from all error. I set out with a perfect distrust of my own abilities, a total renunciation of every speculation of my own, and with a profound reverence for the wisdom of our ancestors, who have left us the inheritance of so happy a Constitution and so flourishing an empire, and, what is a thousand times more valuable, the treasury of the maxims and principles which formed the one and obtained the other.

During the reigns of the kings of Spain of the Austrian family, whenever they were at a loss in the Spanish councils, it was common for their statesmen to say that they ought to consult the genius of Philip the Second. The genius of Philip the Second might mislead them; and the issue of their affairs showed that they had not chosen the most perfect standard. But, Sir, I am sure that I shall not be misled, when, in a case of constitutional difficulty, I consult the genius of the English Constitution. Consulting at that oracle, (it was with all due humility and piety,) I found four capital examples in a similar case before me: those of Ireland, Wales, Chester, and Durham.

Ireland, before the English conquest, though never governed by a despotic power, had no Parliament. How far the English Parliament itself was at that time modelled according to the present form is disputed among antiquarians. But we have all the reason in the world to be assured, that a form of Parliament, such as England then enjoyed, she instantly communicated to Ireland; and we are equally sure that almost every successive improvement in constitutional liberty, as fast as it was made here, was transmitted thither. The feudal baronage, and the feudal knighthood, the roots of our primitive Constitution, were early transplanted into that soil, and grew and flourished there. Magna Charta, if it did not give us originally the House of Commons, gave us at least an House of Commons of weight and consequence. But your ancestors did not churlishly sit down alone to the feast of Magna Charta. Ireland was made immediately a partaker. This benefit of English laws and liberties, I confess, was not at first extended to all Ireland. Mark the consequence. English authority and English liberty had exactly the same boundaries. Your standard could never be advanced an inch before your privileges. Sir John Davies shows beyond a doubt, that the refusal of a general communication of these rights was the true cause why Ireland was five hundred years in subduing; and after the vain projects of a military government, attempted in the reign of Queen Elizabeth, it was soon discovered that nothing could make that country English, in civility and allegiance, but your laws and your forms of legislature. It was not English arms, but the English Constitution, that conquered Ireland. From that time, Ireland has ever had a general Parliament, as she had before a partial Parliament. You changed the people, you altered the religion, but you never touched the form or the vital substance of free government in that kingdom. You deposed kings; you restored them; you altered the succession to theirs, as well as to your own crown; but you never altered their Constitution, the principle of which was respected by usurpation, restored with the restoration of monarchy, and established, I trust, forever by the glorious Revolution. This has made Ireland the great and flourishing kingdom that it is, and, from a disgrace and a burden intolerable to this nation, has rendered her a principal part of our strength and ornament. This country cannot be said to have ever formally taxed her. The irregular things done in the confusion of mighty troubles, and on the hinge of great revolutions, even if all were done that is said to have been done, form no example. If they have any effect in argument, they make an exception to prove the rule. None of your own liberties could stand a moment, if the casual deviations from them, at such times, were suffered to be used as proofs of their nullity. By the lucrative amount of such casual breaches in the Constitution, judge what the stated and fixed rule of supply has been in that kingdom. Your Irish pensioners would starve, if they had no other fund to live on than taxes granted by English authority. Turn your eyes to those popular grants from whence all your great supplies are come, and learn to respect that only source of public wealth in the British empire.

My next example is Wales. This country was said to be reduced by Henry the Third. It was said more truly to be so by Edward the First. But though then conquered, it was not looked upon as any part of the realm of England. Its old Constitution, whatever that might have been, was destroyed; and no good one was substituted in its place. The care of that tract was put into the hands of Lords Marchers: a form of government of a very singular kind; a strange, heterogeneous monster, something between hostility and government: perhaps it has a sort of resemblance, according to the modes of those times, to that of commander-in-chief at present, to whom all civil power is granted as secondary. The manners of the Welsh nation followed the genius of the government: the people were ferocious, restive, savage, and uncultivated; sometimes composed, never pacified. Wales, within itself, was in perpetual disorder; and it kept the frontier of England in perpetual alarm. Benefits from it to the state there were none. Wales was only known to England by incursion and invasion.

Sir, during that state of things, Parliament was not idle. They attempted to subdue the fierce spirit of the Welsh by all sorts of rigorous laws. They prohibited by statute the sending all sorts of arms into Wales, as you prohibit by proclamation (with something more of doubt on the legality) the sending arms to America. They disarmed the Welsh by statute, as you attempted (but still with more question on the legality) to disarm New England by an instruction. They made an act to drag offenders from Wales into England for trial, as you have done (but with more hardship) with regard to America. By another act, where one of the parties was an Englishman, they ordained that his trial should be always by English. They made acts to restrain trade, as you do; and they prevented the Welsh from the use of fairs and markets, as you do the Americans from fisheries and foreign ports. In short, when the statute-book was not quite so much swelled as it is now, you find no less than fifteen acts of penal regulation on the subject of Wales.

Here we rub our hands,—A fine body of precedents for the authority of Parliament and the use of it!—I admit it fully; and pray add likewise to these precedents, that all the while Wales rid this kingdom like an incubus; that it was an unprofitable and oppressive burden; and that an Englishman travelling in that country could not go six yards from the highroad without being murdered.

The march of the human mind is slow. Sir, it was not until after two hundred years discovered, that, by an eternal law, Providence had decreed vexation to violence, and poverty to rapine. Your ancestors did, however, at length open their eyes to the ill husbandry of injustice. They found that the tyranny of a free people could of all tyrannies the least be endured, and that laws made against an whole nation were not the most effectual methods for securing its obedience. Accordingly, in the twenty-seventh year of Henry the Eighth the course was entirely altered. With a preamble stating the entire and perfect rights of the crown of England, it gave to the Welsh all the rights and privileges of English subjects. A political order was established; the military power gave way to the civil; the marches were turned into counties. But that a nation should have a right to English liberties, and yet no share at all in the fundamental security of these liberties,—the grant of their own property,—seemed a thing so incongruous, that eight years after, that is, in the thirty-fifth of that reign, a complete and not ill-proportioned representation by counties and boroughs was bestowed upon Wales by act of Parliament. From that moment, as by a charm, the tumults subsided; obedience was restored; peace, order, and civilization followed in the train of liberty. When the day-star of the English Constitution had arisen in their hearts, all was harmony within and without:—

\begin{verse}
Simul alba nautis \\
Stella refulsit, \\
Defluit saxis agitatus humor, \\
Concidunt venti, fugiuntque nubes, \\
Et minax (quod sic voluere) ponto \\
Unda recumbit.
\end{verse}

The very same year the County Palatine of Chester received the same relief from its oppressions, and the same remedy to its disorders. Before this time Chester was little less distempered than Wales. The inhabitants, without rights themselves, were the fittest to destroy the rights of others; and from thence Richard the Second drew the standing army of archers with which for a time he oppressed England. The people of Chester applied to Parliament in a petition penned as I shall read to you.

"To the king our sovereign lord, in most humble wise shown unto your most excellent Majesty, the inhabitants of your Grace's County Palatine of Chester: That where the said County Palatine of Chester is and hath been alway hitherto exempt, excluded, and separated out and from your high court of Parliament, to have any knights and burgesses within the said court; by reason whereof the said inhabitants have hitherto sustained manifold disherisons, losses, and damages, as well in their lands, goods, and bodies, as in the good, civil, and politic governance and maintenance of the common wealth of their said country: And forasmuch as the said inhabitants have always hitherto been bound by the acts and statutes made and ordained by your said Highness, and your most noble progenitors, by authority of the said court, as far forth as other counties, cities, and boroughs have been, that have had their knights and burgesses within your said court of Parliament, and yet have had neither knight no burgess there for the said County Palatine; the said inhabitants, for lack thereof, have been oftentimes touched and grieved with acts and statutes made within the said court, as well derogatory unto the most ancient jurisdictions, liberties, and privileges of your said County Palatine, as prejudicial unto the common wealth, quietness, rest, and peace of your Grace's most bounden subjects inhabiting within the same."

What did Parliament with this audacious address?—Reject it as a libel? Treat it as an affront to government? Spurn it as a derogation from the rights of legislature? Did they toss it over the table? Did they burn it by the hands of the common hangman?—They took the petition of grievance, all rugged as it was, without softening or temperament, unpurged of the original bitterness and indignation of complaint; they made it the very preamble to their act of redress, and consecrated its principle to all ages in the sanctuary of legislation.

Here is my third example. It was attended with the success of the two former. Chester, civilized as well as Wales, has demonstrated that freedom, and not servitude, is the cure of anarchy; as religion, and not atheism, is the true remedy for superstition. Sir, this pattern of Chester was followed in the reign of Charles the Second with regard to the County Palatine of Durham, which is my fourth example. This county had long lain out of the pale of free legislation. So scrupulously was the example of Chester followed, that the style of the preamble is nearly the came with that of the Chester act; and, without affecting the abstract extent of the authority of Parliament, it recognizes the equity of not suffering any considerable district, in which the British subjects may act as a body, to be taxed without their own voice in the grant.

Now if the doctrines of policy contained in these preambles, and the force of these examples in the acts of Parliament, avail anything, what can be said against applying them with regard to America? Are not the people of America as much Englishmen as the Welsh? The preamble of the act of Henry the Eighth says, the Welsh speak a language no way resembling that of his Majesty's English subjects. Are the Americana not as numerous? If we may trust the learned and accurate Judge Barrington's account of North Wales, and take that as a standard to measure the rest, there is no comparison. The people cannot amount to above 200,000: not a tenth part of the number in the colonies. Is America in rebellion? Wales was hardly ever free from it. Have you attempted to govern America by penal statutes? You made fifteen for Wales. But your legislative authority is perfect with regard to America: was it less perfect in Wales, Chester, and Durham? But America is virtually represented. What! does the electric force of virtual representation more easily pass over the Atlantic than pervade Wales, which lies in your neighborhood? or than Chester and Durham, surrounded by abundance of representation that is actual and palpable? But, Sir, your ancestors thought this sort of virtual representation, however ample, to be totally insufficient for the freedom of the inhabitants of territories that are so near, and comparatively so inconsiderable. How, then, can I think it sufficient for those which are infinitely greater, and infinitely more remote?

You will now, Sir, perhaps imagine that I am on the point of proposing to you a scheme for a representation of the colonies in Parliament. Perhaps I might be inclined to entertain some such thought; but a great flood stops me in my course. Opposuit Natura. I cannot remove the eternal barriers of the creation. The thing, in that mode, I do not know to be possible. As I meddle with no theory, I do not absolutely assert the impracticability of such a representation; but I do not see my way to it; and those who have been more confident have not been more successful. However, the arm of public benevolence is not shortened; and there are often several means to the same end. What Nature has disjoined in one way wisdom may unite in another. When we cannot give the benefit as we would wish, let us not refuse it altogether. If we cannot give the principal, let us find a substitute. But how? where? what substitute?

Fortunately, I am not obliged, for the ways and means of this substitute, to tax my own unproductive invention. I am not even obliged to go to the rich treasury of the fertile framers of imaginary commonwealths: not to the Republic of Plato, not to the Utopia of More, not to the Oceana of Harrington. It is before me,—it is at my feet,—

\begin{verse}
And the rude swain \\
Treads daily on it with his clouted shoon.
\end{verse}

I only wish you to recognize, for the theory, the ancient constitutional policy of this kingdom with regard to representation, as that policy has been declared in acts of Parliament,—and as to the practice, to return to that mode which an uniform experience has marked out to you as best, and in which you walked with security, advantage, and honor, until the year 1763.

My resolutions, therefore, mean to establish the equity and justice of a taxation of America by grant, and not by imposition; to mark the legal competency of the colony assemblies for the support of their government in peace, and for public aids in time of war; to acknowledge that this legal competency has had a dutiful and beneficial exercise, and that experience has shown the benefit of their grants, and the futility of Parliamentary taxation, as a method of supply.

These solid truths compose six fundamental propositions. There are three more resolutions corollary to these. If you admit the first set, you can hardly reject the others. But if you admit the first, I shall be far from solicitous whether you accept or refuse the last. I think these six massive pillars will be of strength sufficient to support the temple of British concord. I have no more doubt than I entertain of my existence, that, if you admitted these, you would command an immediate peace, and, with but tolerable future management, a lasting obedience in America. I am not arrogant in this confident assurance. The propositions are all mere matters of fact; and if they are such facts as draw irresistible conclusions even in the stating, this is the power of truth, and not any management of mine.

Sir, I shall open the whole plan to you together, with such observations on the motions as may tend to illustrate them, where they may want explanation.

The first is a resolution,—"That the colonies and plantations of Great Britain in North America, consisting of fourteen separate governments, and containing two millions and upwards of free inhabitants, have not had the liberty and privilege of electing and sending any knights and burgesses, or others, to represent them in the high court of Parliament."

This is a plain matter of fact, necessary to be laid down, and (excepting the description) it is laid down in the language of the Constitution; it is taken nearly verbatim from acts of Parliament.

The second is like unto the first,—"That the said colonies and plantations have been made liable to, and bounden by, several subsidies, payments, rates, and taxes, given and granted by Parliament, though the said colonies and plantations have not their knights and burgesses in the said high court of Parliament, of their own election, to represent the condition of their country; by lack whereof they have been oftentimes touched and grieved by subsidies, given, granted, and assented to, in the said court, in a manner prejudicial to the common wealth, quietness, rest, and peace of the subjects inhabiting within the same."

Is this description too hot or too cold, too strong or too weak? Does it arrogate too much to the supreme legislature? Does it lean too much to the claims of the people? If it runs into any of these errors, the fault is not mine. It is the language of your own ancient acts of Parliament.

\begin{verse}
Non meus hic sermo, sed quæ præcepit Ofellus \\
Rusticus, abnormis sapiens.
\end{verse}

It is the genuine produce of the ancient, rustic, manly, home-bred sense of this country. I did not dare to rub off a particle of the venerable rust that rather adorns and preserves than destroys the metal. It would be a profanation to touch with a tool the stones which construct the sacred altar of peace. I would not violate with modern polish the ingenuous and noble roughness of these truly constitutional materials. Above all things, I was resolved not to be guilty of tampering,—the odious vice of restless and unstable minds. I put my foot in the tracks of our forefathers, where I can neither wander nor stumble. Determining to fix articles of peace, I was resolved not to be wise beyond what was written; I was resolved to use nothing else than the form of sound words, to let others abound in their own sense, and carefully to abstain from all expressions of my own. What the law has said, I say. In all things else I am silent. I have no organ but for her words. This, if it be not ingenious, I am sure is safe.

There are, indeed, words expressive of grievance in this second resolution, which those who are resolved always to be in the right will deny to contain matter of fact, as applied to the present case; although Parliament thought them true with regard to the Counties of Chester and Durham. They will deny that the Americans were ever "touched and grieved" with the taxes. If they consider nothing in taxes but their weight as pecuniary impositions, there might be some pretence for this denial. But men may be sorely touched and deeply grieved in their privileges, as well as in their purses. Men may lose little in property by the act which takes away all their freedom. When a man is robbed of a trifle on the highway, it is not the twopence lost that constitutes the capital outrage. This is not confined to privileges. Even ancient indulgences withdrawn, without offence on the part of those who enjoyed such favors, operate as grievances. But were the Americans, then, not touched and grieved by the taxes, in some measure, merely as taxes? If so, why were they almost all either wholly repealed or exceedingly reduced? Were they not touched and grieved even by the regulating duties of the sixth of George the Second? Else why were the duties first reduced to one third in 1764, and afterwards to a third of that third in the year 1766? Were they not touched and grieved by the Stamp Act? I shall say they were, until that tax is revived. Were they not touched and grieved by the duties of 1767, which were likewise repealed, and which Lord Hillsborough tells you (for the ministry) were laid contrary to the true principle of commerce? Is not the assurance given by that noble person to the colonies of a resolution to lay no more taxes on them an admission that taxes would touch and grieve them? Is not the resolution of the noble lord in the blue riband, now standing on your journals, the strongest of all proofs that Parliamentary subsidies really touched and grieved them? Else why all these changes, modifications, repeals, assurances, and resolutions?

The next proposition is,—"That, from the distance of the said colonies, and from other circumstances, no method hath hitherto been devised for procuring a representation in Parliament for the said colonies."

This is an assertion of a fact. I go no further on the paper; though, in my private judgment, an useful representation is impossible; I am sure it is not desired by them, nor ought it, perhaps, by us: but I abstain from opinions.

The fourth resolution is,—"That each of the said colonies hath within itself a body, chosen, in part or in the whole, by the freemen, freeholders, or other free inhabitants thereof, commonly called the General Assembly, or General Court, with powers legally to raise, levy, and assess, according to the several usages of such colonies, duties and taxes towards defraying all sorts of public services."

This competence in the colony assemblies is certain. It is proved by the whole tenor of their acts of supply in all the assemblies, in which the constant style of granting is, "An aid to his Majesty"; and acts granting to the crown have regularly, for near a century, passed the public offices without dispute. Those who have been pleased paradoxically to deny this right, holding that none but the British Parliament can grant to the crown, are wished to look to what is done, not only in the colonies, but in Ireland, in one uniform, unbroken tenor, every session. Sir, I am surprised that this doctrine should come from Rome of the law servants of the crown. I say, that, if the crown could be responsible, his Majesty,—but certainly the ministers, and even these law officers themselves, through whose hands the acts pass biennially in Ireland, or annually in the colonies, are in an habitual course of committing impeachable offences. What habitual offenders have been all Presidents of the Council, all Secretaries of State, all First Lords of Trade, all Attorneys and all Solicitors General! However, they are safe, as no one impeaches them; and there is no ground of charge against them, except in their own unfounded theories.

The fifth resolution is also a resolution of fact,—"That the said general assemblies, general courts, or other bodies legally qualified as aforesaid, have at sundry times freely granted several large subsidies and public aids for his Majesty's service, according to their abilities, when required thereto by letter from one of his Majesty's principal Secretaries of State; and that their right to grant the same, and their cheerfulness and sufficiency in the said grants, have been at sundry times acknowledged by Parliament."

To say nothing of their great expenses in the Indian wars, and not to take their exertion in foreign ones, so high as the supplies in the year 1695, not to go back to their public contributions in the year 1710, I shall begin to travel only where the journals give me light,—resolving to deal in nothing but fact authenticated by Parliamentary record, and to build myself wholly on that solid basis.

On the 4th of April, 1748,
%[25] 
\footnote{Journals of the House, Vol. XXV.}
a committee of this House came to the following resolution:—

"Resolved, That it is the opinion of this committee, that it is just and reasonable, that the several provinces and colonies of Massachusetts Bay, New Hampshire, Connecticut, and Rhode Island be reimbursed the expenses they have been at in taking and securing to the crown of Great Britain the island of Caps Breton and its dependencies."

These expenses were immense for such colonies. They were above 200,000l. sterling: money first raised and advanced on their public credit.

On the 28th of January, 1756,
%[26] 
\footnote{Journals of the House, Vol. XXVII.}
a message from the king came to us, to this effect:—"His Majesty, being sensible of the zeal and vigor with which his faithful subjects of certain colonies in North America have exerted themselves in defence of his Majesty's just rights and possessions, recommends it to this House to take the same into their consideration, and to enable his Majesty to give them such assistance as may be proper reward and encouragement."

On the 3d of February, 1756,
%[27] 
\footnote{Ibid.}
the House came to a suitable resolution, expressed in words nearly the same as those of the message; but with the further addition, that the money then voted was as an encouragement to the colonies to exert themselves with vigor. It will not be necessary to go through all the testimonies which your own records have given to the truth of my resolutions. I will only refer you to the places in the journals:—

Vol. XXVII—16th and 19th May, 1757.

Vol. XXVIII.—June 1st, 1758,—April 26th and 30th, 1759,—March 26th and 31st, and April 28th, 1760,—Jan. 9th and 20th, 1761.

Vol. XXIX.—Jan. 22d and 26th, 1762,—March 14th and 17th, 1763.


Sir, here is the repeated acknowledgment of Parliament, that the colonies not only gave, but gave to satiety. This nation has formally acknowledged two things: first, that the colonies had gone beyond their abilities, Parliament having thought it necessary to reimburse them; secondly, that they had acted legally and laudably in their grants of money, and their maintenance of troops, since the compensation is expressly given as reward and encouragement. Reward is not bestowed for acts that are unlawful; and encouragement is not held out to things that deserve reprehension. My resolution, therefore, does nothing more than collect into one proposition what is scattered through your journals. I give you nothing but your own; and you cannot refuse in the gross what you have so often acknowledged in detail. The admission of this, which will be so honorable to them and to you, will, indeed, be mortal to all the miserable stories by which the passions of the misguided people have been engaged in an unhappy system. The people heard, indeed, from the beginning of these disputes, one thing continually dinned in their ears: that reason and justice demanded, that the Americans, who paid no taxes, should be compelled to contribute. How did that fact, of their paying nothing, stand, when the taxing system began? When Mr. Grenville began to form his system of American revenue, he stated in this House that the colonies were then in debt two million six hundred thousand pounds sterling money, and was of opinion they would discharge that debt in four years. On this state, those untaxed people were actually subject to the payment of taxes to the amount of six hundred and fifty thousand a year. In fact, however, Mr. Grenville was mistaken. The funds given for sinking the debt did not prove quite so ample as both the colonies and he expected. The calculation was too sanguine: the reduction was not completed till some years after, and at different times in different colonies. However, the taxes after the war continued too great to bear any addition, with prudence or propriety; and when the burdens imposed in consequence of former requisitions were discharged, our tone became too high to resort again to requisition. No colony, since that time, ever has had any requisition whatsoever made to it.

We see the sense of the crown, and the sense of Parliament, on the productive nature of a revenue by grant. Now search the same journals for the produce of the revenue by imposition. Where is it?—let us know the volume and the page. What is the gross, what is the net produce? To what service is it applied? How have you appropriated its surplus?—What! can none of the many skilful index-makers that we are now employing find any trace of it?—Well, let them and that rest together.—But are the journals, which say nothing of the revenue, as silent on the discontent?—Oh, no! a child may find it. It is the melancholy burden and blot of every page.

I think, then, I am, from those journals, justified in the sixth and last resolution, which is,—"That it hath been found by experience, that the manner of granting the said supplies and aids by the said general assemblies hath been more agreeable to the inhabitants of the said colonies, and more beneficial and conducive to the public service, than the mode of giving and granting aids and subsidies in Parliament, to be raised and paid in the said colonies."

This makes the whole of the fundamental part of the plan. The conclusion is irresistible. You cannot say that you were driven by any necessity to an exercise of the utmost rights of legislature. You cannot assert that you took on yourselves the task of imposing colony taxes, from the want of another legal body that is competent to the purpose of supplying the exigencies of the state without wounding the prejudices of the people. Neither is it true, that the body so qualified, and having that competence, had neglected the duty.

The question now, on all this accumulated matter, is,—Whether you will choose to abide by a profitable experience or a mischievous theory? whether you choose to build on imagination or fact? whether you prefer enjoyment or hope? satisfaction in your subjects, or discontent?

If these propositions are accepted, everything which has been made to enforce a contrary system must, I take it for granted, fall along with it. On that ground, I have drawn the following resolution, which, when it comes to be moved, will naturally be divided in a proper manner:—"That it may be proper to repeal an act, made in the seventh year of the reign of his present Majesty, intituled, 'An act for granting certain duties in the British colonies and plantations in America; for allowing a drawback of the duties of customs, upon the exportation from this kingdom, of coffee and cocoa-nuts, of the produce of the said colonies or plantations; for discontinuing the drawbacks payable on China earthen ware exported to America; and for more effectually preventing the clandestine running of goods in the said colonies and plantations.'—And also, that it may be proper to repeal an act, made in the fourteenth year of the reign of his present Majesty, intituled, 'An act to discontinue, in such manner and for such time as are therein mentioned, the landing and discharging, lading or shipping, of goods, wares, and merchandise, at the town and within the harbor of Boston, in the province of Massachusetts Bay, in North America.'—And also, that it may be proper to repeal an act, made in the fourteenth year of the reign of his present Majesty, intituled, 'An act for the impartial administration of justice, in the cases of persons questioned for any acts done by them, in the execution of the law, or for the suppression of riots and tumults, in the province of the Massachusetts Bay, in New England.'—And also, that it may be proper to repeal an act, made in the fourteenth year of the reign of his present Majesty, intituled,' An act for the better regulating the government of the province of the Massachusetts Bay, in New England.'—And also, that it may be proper to explain and amend an act, made in the thirty-fifth year of the reign of King Henry the Eighth, intituled, 'An act for the trial of treasons committed out of the king's dominions.'"

I wish, Sir, to repeal the Boston Port Bill, because (independently of the dangerous precedent of suspending the rights of the subject during the king's pleasure) it was passed, as I apprehend, with less regularity, and on more partial principles, than it ought. The corporation of Boston was not heard before it was condemned. Other towns, full as guilty as she was, have not had their ports blocked up. Even the Restraining Bill of the present session does not go to the length of the Boston Port Act. The same ideas of prudence, which induced you not to extend equal punishment to equal guilt, even when you were punishing, induce me, who mean not to chastise, but to reconcile, to be satisfied with the punishment already partially inflicted.

Ideas of prudence and accommodation to circumstances prevent you from taking away the charters of Connecticut and Rhode Island, as you have taken away that of Massachusetts Colony, though the crown has far less power in the two former provinces than it enjoyed in the latter, and though the abuses have bean full as great and as flagrant in the exempted as in the punished. The same reasons of prudence and accommodation have weight with me in restoring the charter of Massachusetts Bay. Besides, Sir, the act which changes the charter of Massachusetts is in many particulars so exceptionable, that, if I did not wish absolutely to repeal, I would by all means desire to alter it; as several of its provisions tend to the subversion of all public and private justice. Such, among others, is the power in the governor to change the sheriff at his pleasure, and to make a new returning officer for every special cause. It is shameful to behold such a regulation standing among English laws.

The act for bringing persons accused of committing murder under the orders of government to England for trial is but temporary. That act has calculated the probable duration of our quarrel with the colonies, and is accommodated to that supposed duration. I would hasten the happy moment of reconciliation, and therefore must, on my principle, get rid of that most justly obnoxious act.

The act of Henry the Eighth for the trial of treasons I do not mean to take away, but to confine it to its proper bounds and original intention: to make it expressly for trial of treasons (and the greatest treasons may be committed) in places where the jurisdiction of the crown does not extend.

Having guarded the privileges of local legislature, I would next secure to the colonies a fair and unbiased judicature; for which purpose, Sir, I propose the following resolution:—"That, from the time when the general assembly, or general court, of any colony or plantation in North America shall have appointed, by act of assembly duly confirmed, a settled salary to the offices of the chief justice and other judges of the superior courts, it may be proper that the said chief justice and other judges of the superior courts of such colony shall hold his and their office and offices during their good behavior, and shall not be removed therefrom, but when the said removal shall be adjudged by his Majesty in council, upon a hearing on complaint from the general assembly, or on a complaint from the governor, or the council, or the house of representatives, severally, of the colony in which the said chief justice and other judges have exercised the said offices."

The next resolution relates to the courts of admiralty. It is this:—"That it may be proper to regulate the courts of admiralty or vice-admiralty, authorized by the 15th chapter of the 4th George the Third, in such a manner as to make the same more commodious to those who sue or are sued in the said courts, and to provide for the more decent maintenance of the judges of the same."

These courts I do not wish to take away: they are in themselves proper establishments. This court is one of the capital securities of the Act of Navigation. The extent of its jurisdiction, indeed, has been increased; but this is altogether as proper, and is, indeed, on many accounts, more eligible, where new powers were wanted, than a court absolutely new. But courts incommodiously situated, in effect, deny justice; and a court partaking in the fruits of its own condemnation is a robber. The Congress complain, and complain justly, of this grievance.
%[28]
\footnote{The Solicitor-General informed Mr. B., when the resolutions were separately moved, that the grievance of the judges partaking of the profits of the seizure had been redressed by office; accordingly the resolution was amended.}

These are the three consequential propositions. I have thought of two or three more; but they come rather too near detail, and to the province of executive government, which I wish Parliament always to superintend, never to assume. If the first six are granted, congruity will carry the latter three. If not, the things that remain unrepealed will be, I hope, rather unseemly incumbrances on the building than very materially detrimental to its strength and stability.

Here, Sir, I should close, but that I plainly perceive some objections remain, which I ought, if possible, to remove. The first will be, that, in resorting to the doctrine of our ancestors, as contained in the preamble to the Chester act, I prove too much: that the grievance from a want of representation, stated in that preamble, goes to the whole of legislation as well as to taxation; and that the colonies, grounding themselves upon that doctrine, will apply it to all parts of legislative authority.

To this objection, with all possible deference and humility, and wishing as little as any man living to impair the smallest particle of our supreme authority, I answer, that the words are the words of Parliament, and not mine; and that all false and inconclusive inferences drawn from them are not mine; for I heartily disclaim any such inference. I have chosen the words of an act of Parliament, which Mr. Grenville, surely a tolerably zealous and very judicious advocate for the sovereignty of Parliament, formerly moved to have read at your table in confirmation of his tenets. It is true that Lord Chatham considered these preambles as declaring strongly in favor of his opinions. He was a no less powerful advocate for the privileges of the Americans. Ought I not from hence to presume that these preambles are as favorable as possible to both, when properly understood: favorable both to the rights of Parliament, and to the privilege of the dependencies of this crown? But, Sir, the object of grievance in my resolution I have not taken from the Chester, but from the Durham act, which confines the hardship of want of representation to the case of subsidies, and which therefore falls in exactly with the case of the colonies. But whether the unrepresented counties were de jure or de facto bound the preambles do not accurately distinguish; nor, indeed, was it necessary: for, whether de jure or de facto, the legislature thought the exercise of the power of taxing, as of right, or as of fact without right, equally a grievance, and equally oppressive.

I do not know that the colonies have, in any general way, or in any cool hour, gone much beyond the demand of immunity in relation to taxes. It is not fair to judge of the temper or dispositions of any man or any set of men, when they are composed and at rest, from their conduct or their expressions in a state of disturbance and irritation. It is, besides, a very great mistake to imagine that mankind follow up practically any speculative principle, either of government or of freedom, as far as it will go in argument and logical illation. We Englishmen stop very short of the principles upon which we support any given part of our Constitution, or even the whole of it together. I could easily, if I had not already tired you, give you very striking and convincing instances of it. This is nothing but what is natural and proper. All government, indeed every human benefit and enjoyment, every virtue and every prudent act, is founded on compromise and barter. We balance inconveniences; we give and take; we remit some rights, that we may enjoy others; and we choose rather to be happy citizens than subtle disputants. As we must give away some natural liberty, to enjoy civil advantages, so we must sacrifice some civil liberties, for the advantages to be derived from the communion and fellowship of a great empire. But, in all fair dealings, the thing bought must bear some proportion to the purchase paid. None will barter away the immediate jewel of his soul. Though a great house is apt to make slaves haughty, yet it is purchasing a part of the artificial importance of a great empire too dear, to pay for it all essential rights, and all the intrinsic dignity of human nature. None of us who would not risk his life rather than fall under a government purely arbitrary. But although there are some amongst us who think our Constitution wants many improvements to make it a complete system of liberty, perhaps none who are of that opinion would think it right to aim at such improvement by disturbing his country and risking everything that is dear to him. In every arduous enterprise, we consider what we are to lose, as well as what we are to gain; and the more and better stake of liberty every people possess, the less they will hazard in a vain attempt to make it more. These are the cords of man. Man acts from adequate motives relative to his interest, and not on metaphysical speculations. Aristotle, the great master of reasoning, cautions us, and with great weight and propriety, against this species of delusive geometrical accuracy in moral arguments, as the most fallacious of all sophistry.

The Americans will have no interest contrary to the grandeur and glory of England, when they are not oppressed by the weight of it; and they will rather be inclined to respect the acts of a superintending legislature, when they see them the acts of that power which is itself the security, not the rival, of their secondary importance. In this assurance my mind most perfectly acquiesces, and I confess I feel not the least alarm from the discontents which are to arise from putting people at their ease; nor do I apprehend the destruction of this empire from giving, by an act of free grace and indulgence, to two millions of my fellow-citizens some share of those rights upon which I have always been taught to value myself.

It is said, indeed, that this power of granting, vested in American assemblies, would dissolve the unity of the empire,—which was preserved entire, although Wales, and Chester, and Durham were added to it. Truly, Mr. Speaker, I do not know what this unity means; nor has it ever been heard of, that I know, in the constitutional policy of this country. The very idea of subordination of parts excludes this notion of simple and undivided unity. England is the head; but she is not the head and the members too. Ireland has ever had from the beginning a separate, but not an independent legislature, which, far from distracting, promoted the union of the whole. Everything was sweetly and harmoniously disposed through both islands for the conservation of English dominion and the communication of English liberties. I do not see that the same principles might not be carried into twenty islands, and with the same good effect. This is my model with regard to America, as far as the internal circumstances of the two countries are the same. I know no other unity of this empire than I can draw from its example during these periods, when it seemed to my poor understanding more united than it is now, or than it is likely to be by the present methods.

But since I speak of these methods, I recollect, Mr. Speaker, almost too late, that I promised, before I finished, to say something of the proposition of the noble lord
%[29] 
\footnote{Lord North.}
on the floor, which has been so lately received, and stands on your journals. I must be deeply concerned, whenever it is my misfortune to continue a difference with the majority of this House. But as the reasons for that difference are my apology for thus troubling you, suffer me to state them in a very few words. I shall compress them into as small a body as I possibly can, having already debated that matter at large, when the question was before the committee.

First, then, I cannot admit that proposition of a ransom by auction,—because it is a mere project. It is a thing new, unheard of, supported by no experience, justified by no analogy, without example of our ancestors, or root in the Constitution. It is neither regular Parliamentary taxation nor colony grant. Experimentum in corpore vili is a good rule, which will ever make me adverse to any trial of experiments on what is certainly the most valuable of all subjects, the peace of this empire.

Secondly, it is an experiment which must be fatal in the end to our Constitution. For what is it but a scheme for taxing the colonies in the antechamber of the noble lord and his successors? To settle the quotas and proportions in this House is clearly impossible. You, Sir, may flatter yourself you shall sit a state auctioneer, with your hammer in your hand, and knock down to each colony as it bids. But to settle (on the plan laid down by the noble lord) the true proportional payment for four or five and twenty governments, according to the absolute and the relative wealth of each, and according to the British proportion of wealth and burden, is a wild and chimerical notion. This new taxation must therefore come in by the back-door of the Constitution. Each quota must be brought to this House ready formed. You can neither add nor alter. You must register it. You can do nothing further. For on what grounds can you deliberate either before or after the proposition? You cannot hear the counsel for all these provinces, quarrelling each on its own quantity of payment, and its proportion to others. If you should attempt it, the Committee of Provincial Ways and Means, or by whatever other name it will delight to be called, must swallow up all the time of Parliament.

Thirdly, it does not give satisfaction to the complaint of the colonies. They complain that they are taxed without their consent. You answer, that you will fix the sum at which they shall be taxed. That is, you give them the very grievance for the remedy. You tell them, indeed, that you will leave the mode to themselves. I really beg pardon; it gives me pain to mention it; but you must be sensible that you will not perform this part of the compact. For suppose the colonies were to lay the duties which furnished their contingent upon the importation of your manufactures; you know you would never suffer such a tax to be laid. You know, too, that you would not suffer many other modes of taxation. So that, when you come to explain yourself, it will be found that you will neither leave to themselves the quantum nor the mode, nor indeed anything. The whole is delusion, from one end to the other.

Fourthly, this method of ransom by auction, unless it be universally accepted, will plunge you into great and inextricable difficulties. In what year of our Lord are the proportions of payments to be settled? To say nothing of the impossibility that colony agents should have general powers of taxing the colonies at their discretion, consider, I implore you, that the communication by special messages and orders between these agents and their constituents on each variation of the case, when the parties come to contend together, and to dispute on their relative proportions, will be a matter of delay, perplexity, and confusion, that never can have an end.

If all the colonies do not appear at the outcry, what is the condition of those assemblies who offer, by themselves or their agents, to tax themselves up to your ideas of their proportion? The refractory colonies, who refuse all composition, will remain taxed only to your old impositions, which, however grievous in principle, are trifling as to production. The obedient colonies in this scheme are heavily taxed; the refractory remain unburdened. What will you do? Will you lay new and heavier taxes by Parliament on the disobedient? Pray consider in what way you can do it. You are perfectly convinced, that, in the way of taxing, you can do nothing but at the ports. Now suppose it is Virginia that refuses to appear at your auction, while Maryland and North Carolina bid handsomely for their ransom, and are taxed to your quota, how will you put these colonies on a par? Will you tax the tobacco of Virginia? If you do, you give its death-wound to your English revenue at home, and to one of the very greatest articles of your own foreign trade. If you tax the import of that rebellious colony, what do you tax but your own manufactures, or the goods of some other obedient and already well-taxed colony? Who has said one word on this labyrinth of detail, which bewilders you more and more as you enter into it? Who has presented, who can present, you with a clew to lead you out of it? I think, Sir, it is impossible that you should not recollect that the colony bounds are so implicated in one another (you know it by your other experiments in the bill for prohibiting the New England fishery) that you can lay no possible restraints on almost any of them which may not be presently eluded, if you do not confound the innocent with the guilty, and burden those whom upon every principle you ought to exonerate. He must be grossly ignorant of America, who thinks, that, without falling into this confusion of all rules of equity and policy, you can restrain any single colony, especially Virginia and Maryland, the central, and most important of them all.

Let it also be considered, that either in the present confusion you settle a permanent contingent, which will and must be trifling, and then you have no effectual revenue,—or you change the quota at every exigency, and then on every new repartition you will have a new quarrel.

Reflect besides, that, when you have fixed a quota for every colony, you have not provided for prompt and punctual payment. Suppose one, two, five, ten years' arrears. You cannot issue a Treasury extent against the failing colony. You must make new Boston port bills, new restraining laws, new acts for dragging men to England for trial. You must send out new fleets, new armies. All is to begin again. From this day forward the empire is never to know an hour's tranquillity. An intestine fire will be kept alive in the bowels of the colonies, which one time or other must consume this whole empire. I allow, indeed, that the Empire of Germany raises her revenue and her troops by quotas and contingents; but the revenue of the Empire and the army of the Empire is the worst revenue and the worst army in the world.

Instead of a standing revenue, you will therefore have a perpetual quarrel. Indeed, the noble lord who proposed this project of a ransom by auction seemed himself to be of that opinion. His project was rather designed for breaking the union of the colonies than for establishing a revenue. He confessed he apprehended that his proposal would not be to their taste. I say, this scheme of disunion seems to be at the bottom of the project; for I will not suspect that the noble lord meant nothing but merely to delude the nation by an airy phantom which he never intended to realize. But whatever his views may be, as I propose the peace and union of the colonies as the very foundation of my plan, it cannot accord with one whose foundation is perpetual discord.

Compare the two. This I offer to give you is plain and simple: the other full of perplexed and intricate mazes. This is mild: that harsh. This is found by experience effectual for its purposes: the other is a new project. This is universal: the other calculated for certain colonies only. This is immediate in its conciliatory operation: the other remote, contingent, full of hazard. Mine is what becomes the dignity of a ruling people: gratuitous, unconditional, and not held out as matter of bargain and sale. I have done my duty in proposing it to you. I have, indeed, tired you by a long discourse; but this is the misfortune of those to whose influence nothing will be conceded, and who must win every inch of their ground by argument. You have heard me with goodness. May you decide with wisdom! For my part, I feel my mind greatly disburdened by what I have done to-day. I have been the less fearful of trying your patience, because on this subject I mean to spare it altogether in future. I have this comfort,—that, in every stage of the American affairs, I have steadily opposed the measures that have produced the confusion, and may bring on the destruction, of this empire. I now go so far as to risk a proposal of my own. If I cannot give peace to my country, I give it to my conscience.

But what (says the financier) is peace to us without money? Your plan gives us no revenue.—No! But it does: for it secures to the subject the power of REFUSAL,—the first of all revenues. Experience is a cheat, and fact a liar, if this power in the subject, of proportioning his grant, or of not granting at all, has not been found the richest mine of revenue ever discovered by the skill or by the fortune of man. It does not, indeed, vote you £152,750: 11: 2-3/4ths, nor any other paltry limited sum; but it gives the strong-box itself, the fund, the bank, from whence only revenues can arise amongst a people sensible of freedom: Posita luditur arca. Cannot you in England, cannot you at this time of day, cannot you, an House of Commons, trust to the principle which has raised so mighty a revenue, and accumulated a debt of near 140 millions in this country? Is this principle to be true in England and false everywhere else? Is it not true in Ireland? Has it not hitherto been true in the colonies? Why should you presume, that, in any country, a body duly constituted for any function will neglect to perform its duty, and abdicate its trust? Such a presumption would go against all government in all modes. But, in truth, this dread of penury of supply from a free assembly has no foundation in Nature. For first, observe, that, besides the desire which all men have naturally of supporting the honor of their own government, that sense of dignity, and that security to property, which ever attends freedom, has a tendency to increase the stock of the free community. Most may be taken where most is accumulated. And what is the soil or climate where experience has not uniformly proved that the voluntary flow of heaped-up plenty, bursting from the weight of its own rich luxuriance, has ever run with a more copious stream of revenue than could be squeezed from the dry husks of oppressed indigence by the straining of all the politic machinery in the world?

Next, we know that parties must ever exist in a free country. We know, too, that the emulations of such parties, their contradictions, their reciprocal necessities, their hopes, and their fears, must send them all in their turns to him that holds the balance of the state. The parties are the gamesters; but government keeps the table, and is sure to be the winner in the end. When this game is played, I really think it is more to be feared that the people will be exhausted than that government will not be supplied. Whereas whatever is got by acts of absolute power ill obeyed because odious, or by contracts ill kept because constrained, will be narrow, feeble, uncertain, and precarious.

\begin{verse}
Ease would retract \\
Vows made in pain, as violent and void.
\end{verse}

I, for one, protest against compounding our demands: I declare against compounding, for a poor limited sum, the immense, ever-growing, eternal debt which is due to generous government from protected freedom. And so may I speed in the great object I propose to you, as I think it would not only be an act of injustice, but would be the worst economy in the world, to compel the colonies to a sum certain, either in the way of ransom, or in the way of compulsory compact.

But to clear up my ideas on this subject,—a revenue from America transmitted hither. Do not delude yourselves: you can never receive it,—no, not a shilling. We have experience that from remote countries it is not to be expected. If, when you attempted to extract revenue from Bengal, you were obliged to return in loan what you had taken in imposition, what can you expect from North America? For, certainly, if ever there was a country qualified to produce wealth, it is India; or an institution fit for the transmission, it is the East India Company. America has none of these aptitudes. If America gives you taxable objects on which you lay your duties here, and gives you at the same time a surplus by a foreign sale of her commodities to pay the duties on these objects which you tax at home, she has performed her part to the British revenue. But with regard to her own internal establishments, she may, I doubt not she will, contribute in moderation. I say in moderation; for she ought not to be permitted to exhaust herself. She ought to be reserved to a war; the weight of which, with the enemies that we are most likely to have, must be considerable in her quarter of the globe. There she may serve you, and serve you essentially.

For that service, for all service, whether of revenue, trade, or empire, my trust is in her interest in the British Constitution. My hold of the colonies is in the close affection which grows from common names, from kindred blood, from similar privileges, and equal protection. These are ties which, though light as air, are as strong as links of iron. Let the colonies always keep the idea of their civil rights associated with your government,—they will cling and grapple to you, and no force under heaven will be of power to tear them from their allegiance. But let it be once understood that your government may be one thing and their privileges another, that these two things may exist without any mutual relation,—the cement is gone, the cohesion is loosened, and everything hastens to decay and dissolution. As long as you have the wisdom to keep the sovereign authority of this country as the sanctuary of liberty, the sacred temple consecrated to our common faith, wherever the chosen race and sons of England worship freedom, they will turn their faces towards you. The more they multiply, the more friends you will have; the more ardently they love liberty, the more perfect will be their obedience. Slavery they can have anywhere. It is a weed that grows in every soil. They may have it from Spain, they may have it from Prussia. But, until you become lost to all feeling of your true interest and your natural dignity, freedom they can have from none but you. This is the commodity of price, of which you have the monopoly. This is the true Act of Navigation, which binds to you the commerce of the colonies, and through them secures to you the wealth of the world. Deny them this participation of freedom, and you break that sole bond which originally made, and must still preserve, the unity of the empire. Do not entertain so weak an imagination as that your registers and your bonds, your affidavits and your sufferances, your cockets and your clearances, are what form the great securities of your commerce. Do not dream that your letters of office, and your instructions, and your suspending clauses are the things that hold together the great contexture of this mysterious whole. These things do not make your government. Dead instruments, passive tools as they are, it is the spirit of the English communion that gives all their life and efficacy to them. It is the spirit of the English Constitution, which, infused through the mighty mass, pervades, feeds, unites, invigorates, vivifies every part of the empire, even down to the minutest member.

Is it not the same virtue which does everything for us here in England? Do you imagine, then, that it is the Land-Tax Act which raises your revenue? that it is the annual vote in the Committee of Supply which gives you your army? or that it is the Mutiny Bill which inspires it with bravery and discipline? No! surely, no! It is the love of the people; it is their attachment to their government, from the sense of the deep stake they have in such a glorious institution, which gives you your army and your navy, and infuses into both that liberal obedience without which your army would be a base rabble and your navy nothing but rotten timber.

All this, I know well enough, will sound wild and chimerical to the profane herd of those vulgar and mechanical politicians who have no place among us: a sort of people who think that nothing exists but what is gross and material,—and who, therefore, far from being qualified to be directors of the great movement of empire, are not fit to turn a wheel in the machine. But to men truly initiated and rightly taught, these ruling and master principles, which in the opinion of such men as I have mentioned have no substantial existence, are in truth everything, and all in all. Magnanimity in politics is not seldom the truest wisdom; and a great empire and little minds go ill together. If we are conscious of our situation, and glow with zeal to fill our place as becomes our station and ourselves, we ought to auspicate all our public proceedings on America with the old warning of the Church, Sursum corda! We ought to elevate our minds to the greatness of that trust to which the order of Providence has called us. By adverting to the dignity of this high calling our ancestors have turned a savage wilderness into a glorious empire, and have made the most extensive and the only honorable conquests, not by destroying, but by promoting the wealth, the number, the happiness of the human race. Let us get an American revenue as we have got an American empire. English privileges have made it all that it is; English privileges alone will make it all it can be.

In full confidence of this unalterable truth, I now (quod felix faustumque sit!) lay the first stone of the Temple of Peace; and I move you,—

"That the colonies and plantations of Great Britain in North America, consisting of fourteen separate governments, and containing two millions and upwards of free inhabitants, have not had the liberty and privilege of electing and sending any knights and burgesses, or others, to represent them in the high court of Parliament."

Upon this resolution the previous question was put and carried: for the previous question, 270; against it, 78.

\PRLsep

As the propositions were opened separately in the body of the speech, the reader perhaps may wish to see the whole of them together, in the form in which they were moved for.

"MOVED,

"That the colonies and plantations of Great Britain in North America, consisting of fourteen separate governments, and containing two millions and upwards of free inhabitants, have not had the liberty and privilege of electing and sending any knights and burgesses, or others, to represent them in the high court of Parliament."

"That the said colonies and plantations have been made liable to, and bounden by, several subsidies, payments, rates, and taxes, given and granted by Parliament, though the said colonies and plantations have not their knights and burgesses in the said high court of Parliament, of their own election, to represent the condition of their country; 
\textit{by lack whereof they have been oftentimes touched and grieved by subsidies, given, granted, and amended to, in the said, court, in a manner prejudicial to the common wealth, quietness, rest, and peace of the subjects inhabiting within the same.}"

"That, from the distance of the said colonies, and from other circumstances, no method hath hitherto been devised for procuring a representation in Parliament for the said colonies."

"That each of the said colonies hath within itself a body, chosen, in part or in the whole, by the freemen, freeholders, or other free inhabitants thereof, commonly called the General Assembly, or General Court, with powers legally to raise, levy, and assess, according to the several usages of such colonies, duties and taxes towards defraying all sorts of public services."
%[30]
\footnote{The first four motions and the last had the previous question put on them. The others were negatived.

The words in Italics were, by an amendment that was carried, left out of the motion; which will appear in the journals, though it is not the practice to insert such amendments in the votes.
}

"That the said general assemblies, general courts, or other bodies legally qualified as aforesaid, have at sundry times freely granted several large subsidies and public aids for his Majesty's service, according to their abilities, when required thereto by letter from one of his Majesty's principal Secretaries of State; and that their right to grant the same, and their cheerfulness and sufficiency in the said grants, have been at sundry times acknowledged by Parliament."

"That it hath been found by experience, that the manner of granting the said supplies and aids by the said general assemblies hath been more agreeable to the inhabitants of the said colonies, and more beneficial and conducive to the public service, than the mode of giving and granting aids and subsidies in Parliament, to be raised and paid in the said colonies."

"That it may be proper to repeal an act, made in the seventh year of the reign of his present Majesty, intituled, 'An act for granting certain duties in the British colonies and plantations in America; for allowing a drawback of the duties of customs, upon the exportation from this kingdom, of coffee and cocoa-nuts, of the produce of the said colonies or plantations; for discontinuing the drawbacks payable on China earthen ware exported to America; and for more effectually preventing the clandestine running of goods in the said colonies and plantations.'"

"That it may be proper to repeal an act, made in the fourteenth year of the reign of his present Majesty, intituled, 'An act to discontinue, in such manner and for such time as are therein mentioned, the landing and discharging, lading or chipping, of goods, wares, and merchandise, at the town and within the harbor of Boston, in the province of Massachusetts Bay, in North America.'"

"That it may be proper to repeal an act, made in the fourteenth year of the reign of his present Majesty, intituled, 'An act for the impartial administration of justice, in the cases of persons questioned for any acts done by them, in the execution of the law, or for the suppression of riots and tumults, in the province of the Massachusetts Bay, in New England.'"

"That it may be proper to repeal an act, made in the fourteenth year of the reign of his present Majesty, intituled, 'An act for the better regulating the government of the province of the Massachusetts Bay, in New England.'"

"That it may be proper to explain and amend an act, made in the thirty-fifth year of the reign of King Henry the Eighth, intituled, 'An act for the trial of treasons committed out of the king's dominions.'"

"That, from the time when the general assembly, or general court, of any colony or plantation in North America, shall have appointed, by act of assembly duly confirmed, a settled salary to the offices of the chief justice and other judges of the superior courts, it may be proper that the said chief justice and other judges of the superior courts of such colony shall hold his and their office and offices during their good behavior, and shall not be removed therefrom, but when the said removal shall be adjudged by his Majesty in council, upon a hearing on complaint from the general assembly, or on a complaint from the governor, or the council, or the house of representatives, severally, of the colony in which the said chief justice and other judges have exercised the said offices."

"That it may be proper to regulate the courts of admiralty or vice-admiralty, authorized by the 15th chapter of the 4th George the Third, in such a manner as to make the same more commodious to those who sue or are sued in the said courts; and to provide for the mere decent maintenance of the judges of the same."

%FOOTNOTES:
%[18] The act to restrain the trade and commerce of the provinces of Massachusetts Bay and New Hampshire, and colonies of Connecticut and Rhode Island and Providence Plantation, in North America, to Great Britain, Ireland, and the British Islands in the West Indies; and to prohibit such provinces and colonies from carrying on any fishery on the banks of Newfoundland, and other places therein mentioned, under certain conditions and limitations.

%[19] Mr. Rose Fuller.

%[20] "That when the governor, council, and assembly, or general court, of any of his Majesty's provinces or colonies in America shall propose to make provision, according to the condition, circumstances, and situation of such province or colony, for contributing their proportion to the common defence, (such proportion to be raised under the authority of the general court or general assembly of such province or colony, and disposable by Parliament,) and shall engage to make provision, also for the support of the civil government and the administration, of justice in such province or colony, it will be proper, if such proposal shall be approved by his Majesty and the two Houses of Parliament, and for so long as such provision shall be made accordingly, to forbear, in respect of such province or colony, to levy any duty, tax, or assessment, or to impose any farther duty, tax, or assessment, except only such duties as it may be expedient to continue to levy or to impose for the regulation of commerce: the net produce of the duties last mentioned to be carried to the account of such province or colony respectively."—Resolution moved by Lord North in the Committee, and agreed to by the House, 27th February, 1775.

%[21] Mr. Glover.

%[22] The Attorney-General.

%[23] Mr. Rice.

%[24] Lord North.

%[25] Journals of the House, Vol. XXV.

%[26] Journals of the House, Vol. XXVII.

%[27] Ibid.

%[28] The Solicitor-General informed Mr. B., when the resolutions were separately moved, that the grievance of the judges partaking of the profits of the seizure had been redressed by office; accordingly the resolution was amended.

%[29] Lord North.

%[30] The first four motions and the last had the previous question put on them. The others were negatived.

%The words in Italics were, by an amendment that was carried, left out of the motion; which will appear in the journals, though it is not the practice to insert such amendments in the votes.

%%%%%%%%%%%%%%%%%%%%%%%%%%%%%%%%%%%%%%%%%%%%%%%%%%%%%%%%%%%%%%%%%%%%%%%
\chapter*[Letter to the Sheriffs of Bristol]{
Letter to John Farr and John Harris, Esqrs., Sheriffs of the City of Bristol,
on the Affairs of America,
\\ \vspace{0.1cm}\large{April 3, 1777}}
%\label{chap:vindication}
\addcontentsline{toc}{chapter}{LETTER TO THE SHERIFFS OF THE CITY OF BRISTOL,
ON THE AFFAIRS OF AMERICA, April 3, 1777}

Gentlemen,—I have the honor of sending you the two last acts which have been passed with regard to the troubles in America. These acts are similar to all the rest which have been made on the same subject. They operate by the same principle, and they are derived from the very same policy. I think they complete the number of this sort of statutes to nine. It affords no matter for very pleasing reflection to observe that our subjects diminish as our laws increase.

If I have the misfortune of differing with some of my fellow-citizens on this great and arduous subject, it is no small consolation to me that I do not differ from you. With you I am perfectly united. We are heartily agreed in our detestation of a civil war. We have ever expressed the most unqualified disapprobation of all the steps which have led to it, and of all those which tend to prolong it. And I have no doubt that we feel exactly the same emotions of grief and shame on all its miserable consequences, whether they appear, on the one side or the other, in the shape of victories or defeats, of captures made from the English on the continent or from the English in these islands, of legislative regulations which subvert the liberties of our brethren or which undermine our own.

Of the first of these statutes (that for the letter of marque) I shall say little. Exceptionable as it may be, and as I think it is in some particulars, it seems the natural, perhaps necessary, result of the measures we have taken and the situation we are in. The other (for a partial suspension of the Habeas Corpus) appears to me of a much deeper malignity. During its progress through the House of Commons, it has been amended, so as to express, more distinctly than at first it did, the avowed sentiments of those who framed it; and the main ground of my exception to it is, because it does express, and does carry into execution, purposes which appear to me so contradictory to all the principles, not only of the constitutional policy of Great Britain, but even of that species of hostile justice which no asperity of war wholly extinguishes in the minds of a civilized people.

It seems to have in view two capital objects: the first, to enable administration to confine, as long as it shall think proper, those whom that act is pleased to qualify by the name of pirates. Those so qualified I understand to be the commanders and mariners of such privateers and ships of war belonging to the colonies as in the course of this unhappy contest may fall into the hands of the crown. They are therefore to be detained in prison, under the criminal description of piracy, to a future trial and ignominious punishment, whenever circumstances shall make it convenient to execute vengeance on them, under the color of that odious and infamous offence.

To this first purpose of the law I have no small dislike, because the act does not (as all laws and all equitable transactions ought to do) fairly describe its object. The persons who make a naval war upon us, in consequence of the present troubles, may be rebels; but to call and treat them as pirates is confounding not only the natural distinction of things, but the order of crimes,—which, whether by putting them from a higher part of the scale to the lower or from the lower to the higher, is never done without dangerously disordering the whole frame of jurisprudence. Though piracy may be, in the eye of the law, a less offence than treason, yet, as both are, in effect, punished with the same death, the same forfeiture, and the same corruption of blood, I never would take from any fellow-creature whatever any sort of advantage which he may derive to his safety from the pity of mankind, or to his reputation from their general feelings, by degrading his offence, when I cannot soften his punishment. The general sense of mankind tells me that those offences which may possibly arise from mistaken virtue are not in the class of infamous actions. Lord Coke, the oracle of the English law, conforms to that general sense, where he says that "those things which are of the highest criminality may be of the least disgrace." The act prepares a sort of masked proceeding, not honorable to the justice of the kingdom, and by no means necessary for its safety. I cannot enter into it. If Lord Balmerino, in the last rebellion, had driven off the cattle of twenty clans, I should have thought it would have been a scandalous and low juggle, utterly unworthy of the manliness of an English judicature, to have tried him for felony as a stealer of cows.

Besides, I must honestly tell you that I could not vote for, or countenance in any way, a statute which stigmatizes with the crime of piracy these men whom an act of Parliament had previously put out of the protection of the law. When the legislature of this kingdom had ordered all their ships and goods, for the mere new-created offence of exercising trade, to be divided as a spoil among the seamen, of the navy,—to consider the necessary reprisal of an unhappy, proscribed, interdicted people, as the crime of piracy, would have appeared, in any other legislature than ours, a strain of the most insulting and most unnatural cruelty and injustice. I assure you I never remember to have heard of anything like it in any time or country.

The second professed purpose of the act is to detain in England for trial those who shall commit high treason in America.

That you may be enabled to enter into the true spirit of the present law, it is necessary, Gentlemen, to apprise you that there is an act, made so long ago as in the reign of Henry the Eighth, before the existence or thought of any English colonies in America, for the trial in this kingdom of treasons committed out of the realm. In the year 1769 Parliament thought proper to acquaint the crown with their construction of that act in a formal address, wherein they entreated his Majesty to cause persons charged with high treason in America to be brought into this kingdom for trial. By this act of Henry the Eighth, so construed and so applied, almost all that is substantial and beneficial in a trial by jury is taken away from the subject in the colonies. This is, however, saying too little; for to try a man under that act is, in effect, to condemn him unheard. A person is brought hither in the dungeon of a ship's hold; thence he is vomited into a dungeon on land, loaded with irons, unfurnished with money, unsupported by friends, three thousand miles from all means of calling upon or confronting evidence, where no one local circumstance that tends to detect perjury can possibly be judged of;—such a person may be executed according to form, but he can never be tried according to justice.

I therefore could never reconcile myself to the bill I send you, which is expressly provided to remove all inconveniences from the establishment of a mode of trial which has ever appeared to me most unjust and most unconstitutional. Far from removing the difficulties which impede the execution of so mischievous a project, I would heap new difficulties upon it, if it were in my power. All the ancient, honest, juridical principles and institutions of England are so many clogs to check and retard the headlong course of violence and oppression. They were invented for this one good purpose, that what was not just should not be convenient. Convinced of this, I would leave things as I found them. The old, cool-headed, general law is as good as any deviation dictated by present heat.

I could see no fair, justifiable expedience pleaded to favor this new suspension of the liberty of the subject. If the English in the colonies can support the independency to which they have been unfortunately driven, I suppose nobody has such a fanatical zeal for the criminal justice of Henry the Eighth that he will contend for executions which must be retaliated tenfold on his own friends, or who has conceived so strange an idea of English dignity as to think the defeats in America compensated by the triumphs at Tyburn. If, on the contrary, the colonies are reduced to the obedience of the crown, there must be, under that authority, tribunals in the country itself fully competent to administer justice on all offenders. But if there are not, and that we must suppose a thing so humiliating to our government as that all this vast continent should unanimously concur in thinking that no ill fortune can convert resistance to the royal authority into a criminal act, we may call the effect of our victory peace, or obedience, or what we will, but the war is not ended; the hostile mind continues in full vigor, and it continues under a worse form. If your peace be nothing more than a sullen pause from arms, if their quiet be nothing but the meditation of revenge, where smitten pride smarting from its wounds festers into new rancor, neither the act of Henry the Eighth nor its handmaid of this reign will answer any wise end of policy or justice. For, if the bloody fields which they saw and felt are not sufficient to subdue the reason of America, (to use the expressive phrase of a great lord in office,) it is not the judicial slaughter which is made in another hemisphere against their universal sense of justice that will ever reconcile them to the British government.

I take it for granted, Gentlemen, that we sympathize in a proper horror of all punishment further than as it serves for an example. To whom, then does the example of an execution in England for this American rebellion apply? Remember, you are told every day, that the present is a contest between the two countries, and that we in England are at war for our own dignity against our rebellious children. Is this true? If it be, it is surely among such rebellious children that examples for disobedience should be made, to be in any degree instructive: for who ever thought of teaching parents their duty by an example from the punishment of an undutiful son? As well might the execution of a fugitive negro in the plantations be considered as a lesson to teach masters humanity to their slaves. Such executions may, indeed, satiate our revenge; they may harden our hearts, and puff us up with pride and arrogance. Alas! this is not instruction.

If anything can be drawn from such examples by a parity of the case, it is to show how deep their crime and how heavy their punishment will be, who shall at any time dare to resist a distant power actually disposing of their property without their voice or consent to the disposition, and overturning their franchises without charge or hearing. God forbid that England should ever read this lesson written in the blood of any of her offspring!

War is at present carried on between the king's natural and foreign troops, on one side, and the English in America, on the other, upon the usual footing of other wars; and accordingly an exchange of prisoners has been regularly made from the beginning. If, notwithstanding this hitherto equal procedure, upon some prospect of ending the war with success (which, however, may be delusive) administration prepares to act against those as traitors who remain in their hands at the end of the troubles, in my opinion we shall exhibit to the world as indecent a piece of injustice as ever civil fury has produced. If the prisoners who have been exchanged have not by that exchange been virtually pardoned, the cartel (whether avowed or understood) is a cruel fraud; for you have received the life of a man, and you ought to return a life for it, or there is no parity or fairness in the transaction.

If, on the other hand, we admit that they who are actually exchanged are pardoned, but contend that you may justly reserve for vengeance those who remain unexchanged, then this unpleasant and unhandsome consequence will follow: that you judge of the delinquency of men merely by the time of their guilt, and not by the heinousness of it; and you make fortune and accidents, and not the moral qualities of human action, the rule of your justice.

These strange incongruities must ever perplex those who confound the unhappiness of civil dissension with the crime of treason. Whenever a rebellion really and truly exists, which is as easily known in fact as it is difficult to define in words, government has not entered into such military conventions, but has ever declined all intermediate treaty which should put rebels in possession of the law of nations with regard to war. Commanders would receive no benefits at their hands, because they could make no return for them. Who has ever heard of capitulation, and parole of honor, and exchange of prisoners in the late rebellions in this kingdom? The answer to all demands of that sort was, "We can engage for nothing; you are at the king's pleasure." We ought to remember, that, if our present enemies be in reality and truth rebels, the king's generals have no right to release them upon any conditions whatsoever; and they are themselves answerable to the law, and as much in want of a pardon, for doing so, as the rebels whom they release.

Lawyers, I know, cannot make the distinction for which I contend; because they have their strict rule to go by. But legislators ought to do what lawyers cannot; for they have no other rules to bind them but the great principles of reason and equity and the general sense of mankind. These they are bound to obey and follow, and rather to enlarge and enlighten law by the liberality of legislative reason than to fetter and bind their higher capacity by the narrow constructions of subordinate, artificial justice. If we had adverted to this, we never could consider the convulsions of a great empire, not disturbed by a little disseminated faction, but divided by whole communities and provinces, and entire legal representatives of a people, as fit matter of discussion under a commission of Oyer and Terminer. It is as opposite to reason and prudence as it is to humanity and justice.

This act, proceeding on these principles, that is, preparing to end the present troubles by a trial of one sort of hostility under the name of piracy, and of another by the name of treason, and executing the act of Henry the Eighth according to a new and unconstitutional interpretation, I have thought evil and dangerous, even though the instruments of effecting such purposes had been merely of a neutral quality.

But it really appears to me that the means which this act employs are at least as exceptionable as the end. Permit me to open myself a little upon this subject; because it is of importance to me, when I am obliged to submit to the power without acquiescing in the reason of an act of legislature, that I should justify my dissent by such arguments as may be supposed to have weight with a sober man.

The main operative regulation of the act is to suspend the Common Law and the statute Habeas Corpus (the sole securities either for liberty or justice) with regard to all those who have been out of the realm, or on the high seas, within a given time. The rest of the people, as I understand, are to continue as they stood before.

I confess, Gentlemen, that this appears to me as bad in the principle, and far worse in its consequence, than an universal suspension of the Habeas Corpus Act; and the limiting qualification, instead of taking out the sting, does in my humble opinion sharpen and envenom it to a greater degree. Liberty, if I understand it at all, is a general principle, and the clear right of all the subjects within the realm, or of none. Partial freedom seems to me a most invidious mode of slavery. But, unfortunately, it is the kind of slavery the most easily admitted in times of civil discord: for parties are but too apt to forget their own future safety in their desire of sacrificing their enemies. People without much difficulty admit the entrance of that injustice of which they are not to be the immediate victims. In times of high proceeding it is never the faction of the predominant power that is in danger: for no tyranny chastises its own instruments. It is the obnoxious and the suspected who want the protection of law; and there is nothing to bridle the partial violence of state factions but this,—"that, whenever an act is made for a cessation of law and justice, the whole people should be universally subjected to the same suspension of their franchises." The alarm of such a proceeding would then be universal. It would operate as a sort of call of the nation. It would become every man's immediate and instant concern to be made very sensible of the absolute necessity of this total eclipse of liberty. They would more carefully advert to every renewal, and more powerfully resist it. These great determined measures are not commonly so dangerous to freedom. They are marked with too strong lines to slide into use. No plea, nor pretence, of inconvenience or evil example (which must in their nature be daily and ordinary incidents) can be admitted as a reason for such mighty operations. But the true danger is when liberty is nibbled away, for expedients, and by parts. The Habeas Corpus Act supposes, contrary to the genius of most other laws, that the lawful magistrate may see particular men with a malignant eye, and it provides for that identical case. But when men, in particular descriptions, marked out by the magistrate himself, are delivered over by Parliament to this possible malignity, it is not the Habeas Corpus that is occasionally suspended, but its spirit that is mistaken, and its principle that is subverted. Indeed, nothing is security to any individual but the common interest of all.

This act, therefore, has this distinguished evil in it, that it is the first partial suspension of the Habeas Corpus that has been made. The precedent, which is always of very great importance, is now established. For the first time a distinction is made among the people within this realm. Before this act, every man putting his foot on English ground, every stranger owing only a local and temporary allegiance, even negro slaves who had been sold in the colonies and under an act of Parliament, became as free as every other man who breathed the same air with them. Now a line is drawn, which may be advanced further and further at pleasure, on the same argument of mere expedience on which it was first described. There is no equality among us; we are not fellow-citizens, if the mariner who lands on the quay does not rest on as firm legal ground as the merchant who sits in his counting-house. Other laws may injure the community; this dissolves it. As things now stand, every man in the West Indies, every one inhabitant of three unoffending provinces on the continent, every person coming from the East Indies, every gentleman who has travelled for his health or education, every mariner who has navigated the seas, is, for no other offence, under a temporary proscription. Let any of these facts (now become presumptions of guilt) be proved against him, and the bare suspicion of the crown puts him out of the law. It is even by no means clear to me whether the negative proof does not lie upon the person apprehended on suspicion, to the subversion of all justice.

I have not debated against this bill in its progress through the House; because it would have been vain to oppose, and impossible to correct it. It is some time since I have been clearly convinced, that, in the present state of things, all opposition to any measures proposed by ministers, where the name of America appears, is vain and frivolous. You may be sure that I do not speak of my opposition, which in all circumstances must be so, but that of men of the greatest wisdom and authority in the nation. Everything proposed against America is supposed of course to be in favor of Great Britain. Good and ill success are equally admitted as reasons for persevering in the present methods. Several very prudent and very well-intentioned persons were of opinion, that, during the prevalence of such dispositions, all struggle rather inflamed than lessened the distemper of the public counsels. Finding such resistance to be considered as factious by most within doors and by very many without, I cannot conscientiously support what is against my opinion, nor prudently contend with what I know is irresistible. Preserving my principles unshaken, I reserve my activity for rational endeavors; and I hope that my past conduct has given sufficient evidence, that, if I am a single day from my place, it is not owing to indolence or love of dissipation. The slightest hope of doing good is sufficient to recall me to what I quitted with regret In declining for some time my usual strict attendance, I do not in the least condemn the spirit of those gentlemen who, with a just confidence in their abilities, (in which I claim a sort of share from my love and admiration of them,) were of opinion that their exertions in this desperate case might be of some service. They thought that by contracting the sphere of its application they might lessen the malignity of an evil principle. Perhaps they were in the right. But when my opinion was so very clearly to the contrary, for the reasons I have just stated, I am sure my attendance would have been ridiculous.

I must add, in further explanation of my conduct, that, far from softening the features of such a principle, and thereby removing any part of the popular odium or natural terrors attending it, I should be sorry that anything framed in contradiction to the spirit of our Constitution did not instantly produce, in fact, the grossest of the evils with which it was pregnant in its nature. It is by lying dormant a long time, or being at first very rarely exercised, that arbitrary power steals upon a people. On the next unconstitutional act, all the fashionable world will be ready to say, "Your prophecies are ridiculous, your fears are vain, you see how little of the mischiefs which you formerly foreboded are come to pass." Thus, by degrees, that artful softening of all arbitrary power, the alleged infrequency or narrow extent of its operation, will be received as a sort of aphorism,—and Mr. Hume will not be singular in telling us, that the felicity of mankind is no more disturbed by it than by earthquakes or thunder, or the other more unusual accidents of Nature.

The act of which I speak is among the fruits of the American war,—a war in my humble opinion productive of many mischiefs, of a kind which distinguish it from all others. Not only our policy is deranged, and our empire distracted, but our laws and our legislative spirit appear to have been totally perverted by it. We have made war on our colonies, not by arms only, but by laws. As hostility and law are not very concordant ideas, every step we have taken in this business has been made by trampling on some maxim of justice or some capital principle of wise government. What precedents were established, and what principles overturned, (I will not say of English privilege, but of general justice,) in the Boston Port, the Massachusetts Charter, the Military Bill, and all that long array of hostile acts of Parliament by which the war with America has been begun and supported! Had the principles of any of these acts been first exerted on English ground, they would probably have expired as soon as they touched it. But by being removed from our persons, they have rooted in our laws, and the latest posterity will taste the fruits of them.

Nor is it the worst effect of this unnatural contention, that our laws are corrupted. Whilst manners remain entire, they will correct the vices of law, and soften it at length to their own temper. But we have to lament that in most of the late proceedings we see very few traces of that generosity, humanity, and dignity of mind, which formerly characterized this nation. War suspends the rules of moral obligation, and what is long suspended is in danger of being totally abrogated. Civil wars strike deepest of all into the manners of the people. They vitiate their politics; they corrupt their morals; they pervert even the natural taste and relish of equity and justice. By teaching us to consider our fellow-citizens in an hostile light, the whole body of our nation becomes gradually less dear to us. The very names of affection and kindred, which were the bond of charity whilst we agreed, become new incentives to hatred and rage when the communion of our country is dissolved. We may flatter ourselves that we shall not fall into this misfortune. But we have no charter of exemption, that I know of, from the ordinary frailties of our nature.

What but that blindness of heart which arises from the frenzy of civil contention could have made any persons conceive the present situation of the British affairs as an object of triumph to themselves or of congratulation to their sovereign? Nothing surely could be more lamentable to those who remember the flourishing days of this kingdom than to see the insane joy of several unhappy people, amidst the sad spectacle which our affairs and conduct exhibit to the scorn of Europe. We behold (and it seems some people rejoice in beholding) our native land, which used to sit the envied arbiter of all her neighbors, reduced to a servile dependence on their mercy,—acquiescing in assurances of friendship which she does not trust,—complaining of hostilities which she dares not resent,—deficient to her allies, lofty to her subjects, and submissive to her enemies,—whilst the liberal government of this free nation is supported by the hireling sword of German boors and vassals, and three millions of the subjects of Great Britain are seeking for protection to English privileges in the arms of France!

These circumstances appear to me more like shocking prodigies than natural changes in human affairs. Men of firmer minds may see them without staggering or astonishment. Some may think them matters of congratulation and complimentary addresses; but I trust your candor will be so indulgent to my weakness as not to have the worse opinion of me for my declining to participate in this joy, and my rejecting all share whatsoever in such a triumph. I am too old, too stiff in my inveterate partialities, to be ready at all the fashionable evolutions of opinion. I scarcely know how to adapt my mind to the feelings with which the Court Gazettes mean to impress the people. It is not instantly that I can be brought to rejoice, when I hear of the slaughter and captivity of long lists of those names which have been familiar to my ears from my infancy, and to rejoice that they have fallen under the sword of strangers, whose barbarous appellations I scarcely know how to pronounce. The glory acquired at the White Plains by Colonel Rahl has no charms for me, and I fairly acknowledge that I have not yet learned to delight in finding Fort Kniphausen in the heart of the British dominions.

It might be some consolation for the loss of our old regards, if our reason were enlightened in proportion as our honest prejudices are removed. Wanting feelings for the honor of our country, we might then in cold blood be brought to think a little of our interests as individual citizens and our private conscience as moral agents.

Indeed, our affairs are in a bad condition. I do assure those gentlemen who have prayed for war, and obtained the blessing they have sought, that they are at this instant in very great straits. The abused wealth of this country continues a little longer to feed its distemper. As yet they, and their German allies of twenty hireling states, have contended only with the unprepared strength of our own infant colonies. But America is not subdued. Not one unattacked village which was originally adverse throughout that vast continent has yet submitted from love or terror. You have the ground you encamp on, and you have no more. The cantonments of your troops and your dominions are exactly of the same extent. You spread devastation, but you do not enlarge the sphere of authority.

The events of this war are of so much greater magnitude than those who either wished or feared it ever looked for, that this alone ought to fill every considerate mind with anxiety and diffidence. Wise men often tremble at the very things which fill the thoughtless with security. For many reasons I do not choose to expose to public view all the particulars of the state in which you stood with regard to foreign powers during the whole course of the last year. Whether you are yet wholly out of danger from them is more than I know, or than your rulers can divine. But even if I were certain of my safety, I could not easily forgive those who had brought me into the most dreadful perils, because by accidents, unforeseen by them or me, I have escaped.

Believe me, Gentlemen, the way still before you is intricate, dark, and full of perplexed and treacherous mazes. Those who think they have the clew may lead us out of this labyrinth. We may trust them as amply as we think proper; but as they have most certainly a call for all the reason which their stock can furnish, why should we think it proper to disturb its operation by inflaming their passions? I may be unable to lend an helping hand to those who direct the state; but I should be ashamed to make myself one of a noisy multitude to halloo and hearten them into doubtful and dangerous courses. A conscientious man would be cautious how he dealt in blood. He would feel some apprehension at being called to a tremendous account for engaging in so deep a play without any sort of knowledge of the game. It is no excuse for presumptuous ignorance, that it is directed by insolent passion. The poorest being that crawls on earth, contending to save itself from injustice and oppression, is an object respectable in the eyes of God and man. But I cannot conceive any existence under heaven (which in the depths of its wisdom tolerates all sorts of things) that is more truly odious and disgusting than an impotent, helpless creature, without civil wisdom or military skill, without a consciousness of any other qualification for power but his servility to it, bloated with pride and arrogance, calling for battles which he is not to fight, contending for a violent dominion which he can never exercise, and satisfied to be himself mean and miserable, in order to render others contemptible and wretched.

If you and I find our talents not of the great and ruling kind, our conduct, at least, is conformable to our faculties. No man's life pays the forfeit of our rashness. No desolate widow weeps tears of blood over our ignorance. Scrupulous and sober in a well-grounded distrust of ourselves, we would keep in the port of peace and security; and perhaps in recommending to others something of the same diffidence, we should show ourselves more charitable to their welfare than injurious to their abilities.

There are many circumstances in the zeal shown for civil war which seem to discover but little of real magnanimity. The addressers offer their own persons, and they are satisfied with hiring Germans. They promise their private fortunes, and they mortgage their country. They have all the merit of volunteers, without risk of person or charge of contribution; and when the unfeeling arm of a foreign soldiery pours out their kindred blood like water, they exult and triumph as if they themselves had performed some notable exploit. I am really ashamed of the fashionable language which has been held for some time past, which, to say the best of it, is full of levity. You know that I allude to the general cry against the cowardice of the Americans, as if we despised them for not making the king's soldiery purchase the advantage they have obtained at a dearer rate. It is not, Gentlemen, it is not to respect the dispensations of Providence, nor to provide any decent retreat in the mutability of human affairs. It leaves no medium between insolent victory and infamous defeat. It tends to alienate our minds further and further from our natural regards, and to make an eternal rent and schism in the British nation. Those who do not wish for such a separation would not dissolve that cement of reciprocal esteem and regard which can alone bind together the parts of this great fabric. It ought to be our wish, as it is our duty, not only to forbear this style of outrage ourselves, but to make every one as sensible as we can of the impropriety and unworthiness of the tempers which give rise to it, and which designing men are laboring with such malignant industry to diffuse amongst us. It is our business to counteract them, if possible,—if possible, to awake our natural regards, and to revive the old partiality to the English name. Without something of this kind I do not see how it is ever practicable really to reconcile with those whose affection, after all, must be the surest hold of our government, and which is a thousand times more worth to us than the mercenary zeal of all the circles of Germany.

I can well conceive a country completely overrun, and miserably wasted, without approaching in the least to settlement. In my apprehension, as long as English government is attempted to be supported over Englishmen by the sword alone, things will thus continue. I anticipate in my mind the moment of the final triumph of foreign military force. When that hour arrives, (for it may arrive,) then it is that all this mass of weakness and violence will appear in its full light. If we should be expelled from America, the delusion of the partisans of military government might still continue. They might still feed their imaginations with the possible good consequences which might have attended success. Nobody could prove the contrary by facts. But in case the sword should do all that the sword can do, the success of their arms and the defeat of their policy will be one and the same thing. You will never see any revenue from America. Some increase of the means of corruption, without ease of the public burdens, is the very best that can happen. Is it for this that we are at war,—and in such a war?

As to the difficulties of laying once more the foundations of that government which, for the sake of conquering what was our own, has been voluntarily and wantonly pulled down by a court faction here, I tremble to look at them. Has any of these gentlemen who are so eager to govern all mankind shown himself possessed of the first qualification towards government, some knowledge of the object, and of the difficulties which occur in the task they have undertaken?

I assure you, that, on the most prosperous issue of your arms, you will not be where you stood when you called in war to supply the defects of your political establishment. Nor would any disorder or disobedience to government which could arise from the most abject concession on our part ever equal those which will be felt after the most triumphant violence. You have got all the intermediate evils of war into the bargain.

I think I know America,—if I do not, my ignorance is incurable, for I have spared no pains to understand it,—and I do most solemnly assure those of my constituents who put any sort of confidence in my industry and integrity, that everything that has been done there has arisen from a total misconception of the object: that our means of originally holding America, that our means of reconciling with it after quarrel, of recovering it after separation, of keeping it after victory, did depend, and must depend, in their several stages and periods, upon a total renunciation of that unconditional submission which has taken such possession of the minds of violent men. The whole of those maxims upon which we have made and continued this war must be abandoned. Nothing, indeed, (for I would not deceive you,) can place us in our former situation. That hope must be laid aside. But there is a difference between bad and the worst of all. Terms relative to the cause of the war ought to be offered by the authority of Parliament. An arrangement at home promising some security for them ought to be made. By doing this, without the least impairing of our strength, we add to the credit of our moderation, which, in itself, is always strength more or less.

I know many have been taught to think that moderation in a case like this is a sort of treason,—and that all arguments for it are sufficiently answered by railing at rebels and rebellion, and by charging all the present or future miseries which we may suffer on the resistance of our brethren. But I would wish them, in this grave matter, and if peace is not wholly removed from their hearts, to consider seriously, first, that to criminate and recriminate never yet was the road to reconciliation, in any difference amongst men. In the next place, it would be right to reflect that the American English (whom they may abuse, if they think it honorable to revile the absent) can, as things now stand, neither be provoked at our railing or bettered by our instruction. All communication is cut off between us. But this we know with certainty, that, though we cannot reclaim them, we may reform ourselves. If measures of peace are necessary, they must begin somewhere; and a conciliatory temper must precede and prepare every plan of reconciliation. Nor do I conceive that we suffer anything by thus regulating our own minds. We are not disarmed by being disencumbered of our passions. Declaiming on rebellion never added a bayonet or a charge of powder to your military force; but I am afraid that it has been the means of taking up many muskets against you.

This outrageous language, which has been encouraged and kept alive by every art, has already done incredible mischief. For a long time, even amidst the desolations of war, and the insults of hostile laws daily accumulated on one another, the American leaders seem to have had the greatest difficulty in bringing up their people to a declaration of total independence. But the Court Gazette accomplished what the abettors of independence had attempted in vain. When that disingenuous compilation and strange medley of railing and flattery was adduced as a proof of the united sentiments of the people of Great Britain, there was a great change throughout all America. The tide of popular affection, which had still set towards the parent country, began immediately to turn, and to flow with great rapidity in a contrary course. Par from concealing these wild declarations of enmity, the author of the celebrated pamphlet which prepared the minds of the people for independence insists largely on the multitude and the spirit of these addresses; and he draws an argument from them, which, if the fact were as he supposes, must be irresistible. For I never knew a writer on the theory of government so partial to authority as not to allow that the hostile mind of the rulers to their people did fully justify a change of government; nor can any reason whatever be given why one people should voluntarily yield any degree of preëminence to another but on a supposition of great affection and benevolence towards them. Unfortunately, your rulers, trusting to other things, took no notice of this great principle of connection. From the beginning of this affair, they have done all they could to alienate your minds from your own kindred; and if they could excite hatred enough in one of the parties towards the other, they seemed to be of opinion that they had gone half the way towards reconciling the quarrel.

I know it is said, that your kindness is only alienated on account of their resistance, and therefore, if the colonies surrender at discretion, all sort of regard, and even much indulgence, is meant towards them in future. But can those who are partisans for continuing a war to enforce such a surrender be responsible (after all that has passed) for such a future use of a power that is bound by no compacts and restrained by no terror? Will they tell us what they call indulgences? Do they not at this instant call the present war and all its horrors a lenient and merciful proceeding?

No conqueror that I ever heard of has professed to make a cruel, harsh, and insolent use of his conquest. No! The man of the most declared pride scarcely dares to trust his own heart with this dreadful secret of ambition. But it will appear in its time; and no man who professes to reduce another to the insolent mercy of a foreign arm ever had any sort of good-will towards him. The profession of kindness, with that sword in his hand, and that demand of surrender, is one of the most provoking acts of his hostility. I shall be told that all this is lenient as against rebellious adversaries. But are the leaders of their faction more lenient to those who submit? Lord Howe and General Howe have powers, under an act of Parliament, to restore to the king's peace and to free trade any men or district which shall submit. Is this done? We have been over and over informed by the authorized gazette, that the city of New York and the countries of Staten and Long Island have submitted voluntarily and cheerfully, and that many are very full of zeal to the cause of administration. Were they instantly restored to trade? Are they yet restored to it? Is not the benignity of two commissioners, naturally most humane and generous men, some way fettered by instructions, equally against their dispositions and the spirit of Parliamentary faith, when Mr. Tryon, vaunting of the fidelity of the city in which he is governor, is obliged to apply to ministry for leave to protect the King's loyal subjects, and to grant to them, not the disputed rights and privileges of freedom, but the common rights of men, by the name of graces? Why do not the commissioners restore them on the spot? Were they not named as commissioners for that express purpose? But we see well enough to what the whole leads. The trade of America is to be dealt out in private indulgences and grants,—that is, in jobs to recompense the incendiaries of war. They will be informed of the proper time in which to send out their merchandise. From a national, the American trade is to be turned into a personal monopoly, and one set of merchants are to be rewarded for the pretended zeal of which another set are the dupes; and thus, between craft and credulity, the voice of reason is stifled, and all the misconduct, all the calamities of the war are covered and continued.

If I had not lived long enough to be little surprised at anything, I should have been in some degree astonished at the continued rage of several gentlemen, who, not satisfied with carrying fire and sword into America, are animated nearly with the same fury against those neighbors of theirs whose only crime it is, that they have charitably and humanely wished them to entertain more reasonable sentiments, and not always to sacrifice their interest to their passion. All this rage against unresisting dissent convinces me, that, at bottom, they are far from satisfied they are in the right. For what is it they would have? A war? They certainly have at this moment the blessing of something that is very like one; and if the war they enjoy at present be not sufficiently hot and extensive, they may shortly have it as warm and as spreading as their hearts can desire. Is it the force of the kingdom they call for? They have it already; and if they choose to fight their battles in their own person, nobody prevents their setting sail to America in the next transports. Do they think that the service is stinted for want of liberal supplies? Indeed they complain without reason. The table of the House of Commons will glut them, let their appetite for expense be never so keen. And I assure them further, that those who think with them in the House of Commons are full as easy in the control as they are liberal in the vote of these expenses. If this be not supply or confidence sufficient, let them open their own private purse-strings, and give, from what is left to them, as largely and with as little care as they think proper.

Tolerated in their passions, let them learn not to persecute the moderation of their fellow-citizens. If all the world joined them in a full cry against rebellion, and were as hotly inflamed against the whole theory and enjoyment of freedom as those who are the most factious for servitude, it could not, in my opinion, answer any one end whatsoever in this contest. The leaders of this war could not hire (to gratify their friends) one German more than they do, or inspire him with less feeling for the persons or less value for the privileges of their revolted brethren. If we all adopted their sentiments to a man, their allies, the savage Indians, could not be more ferocious than they are: they could not murder one more helpless woman or child, or with more exquisite refinements of cruelty torment to death one more of their English flesh and blood, than they do already. The public money is given to purchase this alliance;—and they have their bargain.

They are continually boasting of unanimity, or calling for it. But before this unanimity can be matter either of wish or congratulation, we ought to be pretty sure that we are engaged in a rational pursuit. Frenzy does not become a slighter distemper on account of the number of those who may be infected with it. Delusion and weakness produce not one mischief the less because they are universal. I declare that I cannot discern the least advantage which could accrue to us, if we were able to persuade our colonies that they had not a single friend in Great Britain. On the contrary, if the affections and opinions of mankind be not exploded as principles of connection, I conceive it would be happy for us, if they were taught to believe that there was even a formed American party in England, to whom they could always look for support. Happy would it be for us, if, in all tempers, they might turn their eyes to the parent state, so that their very turbulence and sedition should find vent in no other place than this! I believe there is not a man (except those who prefer the interest of some paltry faction to the very being of their country) who would not wish that the Americans should from time to time carry many points, and even some of them not quite reasonable, by the aid of any denomination of men here, rather than they should be driven to seek for protection against the fury of foreign mercenaries and the waste of savages in the arms of France.

When any community is subordinately connected with another, the great danger of the connection is the extreme pride and self-complacency of the superior, which in all matters of controversy will probably decide in its own favor. It is a powerful corrective to such a very rational cause of fear, if the inferior body can be made to believe that the party inclination or political views of several in the principal state will induce them in some degree to counteract this blind and tyrannical partiality. There is no danger that any one acquiring consideration or power in the presiding state should carry this leaning to the inferior too far. The fault of human nature is not of that sort. Power, in whatever hands, is rarely guilty of too strict limitations on itself. But one great advantage to the support of authority attends such an amicable and protecting connection: that those who have conferred favors obtain influence, and from the foresight of future events can persuade men who have received obligations sometimes to return them. Thus, by the mediation of those healing principles, (call them good or evil,) troublesome discussions are brought to some sort of adjustment, and every hot controversy is not a civil war.

But, if the colonies (to bring the general matter home to us) could see that in Great Britain the mass of the people is melted into its government, and that every dispute with the ministry must of necessity be always a quarrel with the nation, they can stand no longer in the equal and friendly relation of fellow-citizens to the subjects of this kingdom. Humble as this relation may appear to some, when it is once broken, a strong tie is dissolved. Other sort of connections will be sought. For there are very few in the world who will not prefer an useful ally to an insolent master.

Such discord has been the effect of the unanimity into which so many have of late been seduced or bullied, or into the appearance of which they have sunk through mere despair. They have been told that their dissent from violent measures is an encouragement to rebellion. Men of great presumption and little knowledge will hold a language which is contradicted by the whole course of history. General rebellions and revolts of an whole people never were encouraged, now or at any time. They are always provoked. But if this unheard-of doctrine of the encouragement of rebellion were true, if it were true that an assurance of the friendship of numbers in this country towards the colonies could become an encouragement to them to break off all connection with it, what is the inference? Does anybody seriously maintain, that, charged with my share of the public councils, I am obliged not to resist projects which I think mischievous, lest men who suffer should be encouraged to resist? The very tendency of such projects to produce rebellion is one of the chief reasons against them. Shall that reason not be given? Is it, then, a rule, that no man in this nation shall open his mouth in favor of the colonies, shall defend their rights, or complain of their sufferings,—or when war finally breaks out, no man shall express his desires of peace? Has this been the law of our past, or is it to be the terms of our future connection? Even looking no further than ourselves, can it be true loyalty to any government, or true patriotism towards any country, to degrade their solemn councils into servile drawing-rooms, to flatter their pride and passions rather than to enlighten their reason, and to prevent them from being cautioned against violence lest others should be encouraged to resistance? By such acquiescence great kings and mighty nations have been undone; and if any are at this day in a perilous situation from rejecting truth and listening to flattery, it would rather become them to reform the errors under which they suffer than to reproach those who forewarned them of their danger.

But the rebels looked for assistance from this country.—They did so, in the beginning of this controversy, most certainly; and they sought it by earnest supplications to government, which dignity rejected, and by a suspension of commerce, which the wealth of this nation enabled you to despise. When they found that neither prayers nor menaces had any sort of weight, but that a firm resolution was taken to reduce them to unconditional obedience by a military force, they came to the last extremity. Despairing of us, they trusted in themselves. Not strong enough themselves, they sought succor in France. In proportion as all encouragement here lessened, their distance from this country increased. The encouragement is over; the alienation is complete.

In order to produce this favorite unanimity in delusion, and to prevent all possibility of a return to our ancient happy concord, arguments for our continuance in this course are drawn from the wretched situation itself into which we have been betrayed. It is said, that, being at war with the colonies, whatever our sentiments might have been before, all ties between us are now dissolved, and all the policy we have left is to strengthen the hands of government to reduce them. On the principle of this argument, the more mischiefs we suffer from any administration, the more our trust in it is to be confirmed. Let them but once get us into a war, and then their power is safe, and an act of oblivion passed for all their misconduct.

But is it really true that government is always to be strengthened with the instruments of war, but never furnished with the means of peace? In former times, ministers, I allow, have been sometimes driven by the popular voice to assert by arms the national honor against foreign powers. But the wisdom of the nation has been far more clear, when those ministers have been compelled to consult its interests by treaty. We all know that the sense of the nation obliged the court of Charles the Second to abandon the Dutch war: a war, next to the present, the most impolitic which we ever carried on. The good people of England considered Holland as a sort of dependency on this kingdom; they dreaded to drive it to the protection or subject it to the power of France by their own inconsiderate hostility. They paid but little respect to the court jargon of that day; nor were they inflamed by the pretended rivalship of the Dutch in trade,—by the massacre at Amboyna, acted on the stage to provoke the public vengeance,—nor by declamations against the ingratitude of the United Provinces for the benefits England had conferred upon them in their infant state. They were not moved from their evident interest by all these arts; nor was it enough to tell them, they were at war, that they must go through with it, and that the cause of the dispute was lost in the consequences. The people of England were then, as they are now, called upon to make government strong. They thought it a great deal better to make it wise and honest.

When I was amongst my constituents at the last summer assizes, I remember that men of all descriptions did then express a very strong desire for peace, and no slight hopes of attaining it from the commission sent out by my Lord Howe. And it is not a little remarkable, that, in proportion as every person showed a zeal for the court measures, he was then earnest in circulating an opinion of the extent of the supposed powers of that commission. When I told them that Lord Howe had no powers to treat, or to promise satisfaction on any point whatsoever of the controversy, I was hardly credited,—so strong and general was the desire of terminating this war by the method of accommodation. As far as I could discover, this was the temper then prevalent through the kingdom. The king's forces, it must be observed, had at that time been obliged to evacuate Boston. The superiority of the former campaign rested wholly with the colonists. If such powers of treaty were to be wished whilst success was very doubtful, how came they to be less so, since his Majesty's arms have been crowned with many considerable advantages? Have these successes induced us to alter our mind, as thinking the season of victory not the time for treating with honor or advantage? Whatever changes have happened in the national character, it can scarcely be our wish that terms of accommodation never should be proposed to our enemy, except when they must be attributed solely to our fears. It has happened, let me say unfortunately, that we read of his Majesty's commission for making peace, and his troops evacuating his last town in the Thirteen Colonies, at the same hour and in the same gazette. It was still more unfortunate that no commission went to America to settle the troubles there, until several months after an act had been passed to put the colonies out of the protection of this government, and to divide their trading property, without a possibility of restitution, as spoil among the seamen of the navy. The most abject submission on the part of the colonies could not redeem them. There was no man on that whole continent, or within three thousand miles of it, qualified by law to follow allegiance with protection or submission with pardon. A proceeding of this kind has no example in history. Independency, and independency with an enmity, (which, putting ourselves out of the question, would be called natural and much provoked,) was the inevitable consequence. How this came to pass the nation may be one day in an humor to inquire.

All the attempts made this session to give fuller powers of peace to the commanders in America were stifled by the fatal confidence of victory and the wild hopes of unconditional submission. There was a moment favorable to the king's arms, when, if any powers of concession had existed on the other side of the Atlantic, even after all our errors, peace in all probability might have been restored. But calamity is unhappily the usual season of reflection; and the pride of men will not often suffer reason to have any scope, until it can be no longer of service.

I have always wished, that as the dispute had its apparent origin from things done in Parliament, and as the acts passed there had provoked the war, that the foundations of peace should be laid in Parliament also. I have been astonished to find that those whose zeal for the dignity of our body was so hot as to light up the flames of civil war should even publicly declare that these delicate points ought to be wholly left to the crown. Poorly as I may be thought affected to the authority of Parliament, I shall never admit that our constitutional rights can ever become a matter of ministerial negotiation.

I am charged with being an American. If warm affection towards those over whom I claim any share of authority be a crime, I am guilty of this charge. But I do assure you, (and they who know me publicly and privately will bear witness to me,) that, if ever one man lived more zealous than another for the supremacy of Parliament and the rights of this imperial crown, it was myself. Many others, indeed, might be more knowing in the extent of the foundation of these rights. I do not pretend to be an antiquary, a lawyer, or qualified for the chair of professor in metaphysics. I never ventured to put your solid interests upon speculative grounds. My having constantly declined to do so has been attributed to my incapacity for such disquisitions; and I am inclined to believe it is partly the cause. I never shall be ashamed to confess, that, where I am ignorant, I am diffident. I am, indeed, not very solicitous to clear myself of this imputed incapacity; because men even less conversant than I am in this kind of subtleties, and placed in stations to which I ought not to aspire, have, by the mere force of civil discretion, often conducted the affairs of great nations with distinguished felicity and glory.

When I first came into a public trust, I found your Parliament in possession of an unlimited legislative power over the colonies. I could not open the statute-book without seeing the actual exercise of it, more or less, in all cases whatsoever. This possession passed with me for a title. It does so in all human affairs. No man examines into the defects of his title to his paternal estate or to his established government. Indeed, common sense taught me that a legislative authority not actually limited by the express terms of its foundation, or by its own subsequent acts, cannot have its powers parcelled out by argumentative distinctions, so as to enable us to say that here they can and there they cannot bind. Nobody was so obliging as to produce to me any record of such distinctions, by compact or otherwise, either at the successive formation of the several colonies or during the existence of any of them. If any gentlemen were able to see how one power could be given up (merely on abstract reasoning) without giving up the rest, I can only say that they saw further than I could. Nor did I ever presume to condemn any one for being clear-sighted when I was blind. I praise their penetration and learning, and hope that their practice has been correspondent to their theory.

I had, indeed, very earnest wishes to keep the whole body of this authority perfect and entire as I found it,—and to keep it so, not for our advantage solely, but principally for the sake of those on whose account all just authority exists: I mean the people to be governed. For I thought I saw that many cases might well happen in which the exercise of every power comprehended in the broadest idea of legislature might become, in its time and circumstances, not a little expedient for the peace and union of the colonies amongst themselves, as well as for their perfect harmony with Great Britain. Thinking so, (perhaps erroneously, but being honestly of that opinion,) I was at the same time very sure that the authority of which I was so jealous could not, under the actual circumstances of our plantations, be at all preserved in any of its members, but by the greatest reserve in its application, particularly in those delicate points in which the feelings of mankind are the most irritable. They who thought otherwise have found a few more difficulties in their work than (I hope) they were thoroughly aware of, when they undertook the present business. I must beg leave to observe, that it is not only the invidious branch of taxation that will be resisted, but that no other given part of legislative rights can be exercised, without regard to the general opinion of those who are to be governed. That general opinion is the vehicle and organ of legislative omnipotence. Without this, it may be a theory to entertain the mind, but it is nothing in the direction of affairs. The completeness of the legislative authority of Parliament over this kingdom is not questioned; and yet many things indubitably included in the abstract idea of that power, and which carry no absolute injustice in themselves, yet being contrary to the opinions and feelings of the people, can as little be exercised as if Parliament in that case had been possessed of no right at all. I see no abstract reason, which can be given, why the same power which made and repealed the High Commission Court and the Star-Chamber might not revive them again; and these courts, warned by their former fate, might possibly exercise their powers with some degree of justice. But the madness would be as unquestionable as the competence of that Parliament which should attempt such things. If anything can be supposed out of the power of human legislature, it is religion; I admit, however, that the established religion of this country has been three or four times altered by act of Parliament, and therefore that a statute binds even in that case. But we may very safely affirm, that, notwithstanding this apparent omnipotence, it would be now found as impossible for King and Parliament to alter the established religion of this country as it was to King James alone, when he attempted to make such an alteration without a Parliament. In effect, to follow, not to force, the public inclination,—to give a direction, a form, a technical dress, and a specific sanction, to the general sense of the community, is the true end of legislature.

It is so with regard to the exercise of all the powers which our Constitution knows in any of its parts, and indeed to the substantial existence of any of the parts themselves. The king's negative to bills is one of the most indisputed of the royal prerogatives; and it extends to all cases whatsoever. I am far from certain, that if several laws, which I know, had fallen under the stroke of that sceptre, that the public would have had a very heavy loss. But it is not the propriety of the exercise which is in question. The exercise itself is wisely forborne. Its repose may be the preservation of its existence; and its existence may be the means of saying the Constitution itself, on an occasion worthy of bringing it forth.

As the disputants whose accurate and logical reasonings have brought us into our present condition think it absurd that powers or members of any constitution should exist, rarely, if ever, to be exercised, I hope I shall be excused in mentioning another instance that is material. We know that the Convocation of the Clergy had formerly been called, and sat with nearly as much regularity to business as Parliament itself. It is now called for form only. It sits for the purpose of making some polite ecclesiastical compliments to the king, and, when that grace is said, retires and is heard of no more. It is, however, a part of the Constitution, and may be called out into act and energy, whenever there is occasion, and whenever those who conjure up that spirit will choose to abide the consequences. It is wise to permit its legal existence: it is much wiser to continue it a legal existence only. So truly has prudence (constituted as the god of this lower world) the entire dominion over every exercise of power committed into its hands! And yet I have lived to see prudence and conformity to circumstances wholly set at nought in our late controversies, and treated as if they were the most contemptible and irrational of all things. I have heard it an hundred times very gravely alleged, that, in order to keep power in wind, it was necessary, by preference, to exert it in those very points in which it was most likely to be resisted and the least likely to be productive of any advantage.

These were the considerations, Gentlemen, which led me early to think, that, in the comprehensive dominion which the Divine Providence had put into our hands, instead of troubling our understandings with speculations concerning the unity of empire and the identity or distinction of legislative powers, and inflaming our passions with the heat and pride of controversy, it was our duty, in all soberness, to conform our government to the character and circumstances of the several people who composed this mighty and strangely diversified mass. I never was wild enough to conceive that one method would serve for the whole, that the natives of Hindostan and those of Virginia could be ordered in the same manner, or that the Cutchery court and the grand jury of Salem could be regulated on a similar plan. I was persuaded that government was a practical thing, made for the happiness of mankind, and not to furnish out a spectacle of uniformity to gratify the schemes of visionary politicians. Our business was to rule, not to wrangle; and it would have been a poor compensation that we had triumphed in a dispute, whilst we lost an empire.

If there be one fact in the world perfectly clear, it is this,—"that the disposition of the people of America is wholly averse to any other than a free government"; and this is indication enough to any honest statesman how he ought to adapt whatever power he finds in his hands to their case. If any ask me what a free government is, I answer, that, for any practical purpose, it is what the people think so,—and that they, and not I, are the natural, lawful, and competent judges of this matter. If they practically allow me a greater degree of authority over them than is consistent with any correct ideas of perfect freedom, I ought to thank them for so great a trust, and not to endeavor to prove from thence that they have reasoned amiss, and that, having gone so far, by analogy they must hereafter have no enjoyment but by my pleasure.

If we had seen this done by any others, we should have concluded them far gone in madness. It is melancholy, as well as ridiculous, to observe the kind of reasoning with which the public has been amused, in order to divert our minds from the common sense of our American policy. There are people who have split and anatomized the doctrine of free government, as if it were an abstract question concerning metaphysical liberty and necessity, and not a matter of moral prudence and natural feeling. They have disputed whether liberty be a positive or a negative idea; whether it does not consist in being governed by laws, without considering what are the laws, or who are the makers; whether man has any rights by Nature; and whether all the property he enjoys be not the alms of his government, and his life itself their favor and indulgence. Others, corrupting religion as these have perverted philosophy, contend that Christians are redeemed into captivity, and the blood of the Saviour of mankind has been shed to make them the slaves of a few proud and insolent sinners. These shocking extremes provoking to extremes of another kind, speculations are let loose as destructive to all authority as the former are to all freedom; and every government is called tyranny and usurpation which is not formed on their fancies. In this manner the stirrers-up of this contention, not satisfied with distracting our dependencies and filling them with blood and slaughter, are corrupting our understandings: they are endeavoring to tear up, along with practical liberty, all the foundations of human society, all equity and justice, religion and order.

Civil freedom, Gentlemen, is not, as many have endeavored to persuade you, a thing that lies hid in the depth of abstruse science. It is a blessing and a benefit, not an abstract speculation; and all the just reasoning that can be upon it is of so coarse a texture as perfectly to suit the ordinary capacities of those who are to enjoy, and of those who are to defend it. Far from any resemblance to those propositions in geometry and metaphysics which admit no medium, but must be true or false in all their latitude, social and civil freedom, like all other things in common life, are variously mixed and modified, enjoyed in very different degrees, and shaped into an infinite diversity of forms, according to the temper and circumstances of every community. The extreme of liberty (which is its abstract perfection, but its real fault) obtains nowhere, nor ought to obtain anywhere; because extremes, as we all know, in every point which relates either to our duties or satisfactions in life, are destructive both to virtue and enjoyment. Liberty, too, must be limited in order to be possessed. The degree of restraint it is impossible in any case to settle precisely. But it ought to be the constant aim of every wise public counsel to find out by cautious experiments, and rational, cool endeavors, with how little, not how much, of this restraint the community can subsist: for liberty is a good to be improved, and not an evil to be lessened. It is not only a private blessing of the first order, but the vital spring and energy of the state itself, which has just so much life and vigor as there is liberty in it. But whether liberty be advantageous or not, (for I know it is a fashion to decry the very principle,) none will dispute that peace is a blessing; and peace must, in the course of human affairs, be frequently bought by some indulgence and toleration at least to liberty: for, as the Sabbath (though of divine institution) was made for man, not man for the Sabbath, government, which can claim no higher origin or authority, in its exercise at least, ought to conform to the exigencies of the time, and the temper and character of the people with whom it is concerned, and not always to attempt violently to bend the people to their theories of subjection. The bulk of mankind, on their part, are not excessively curious concerning any theories whilst they are really happy; and one sure symptom of an ill-conducted state is the propensity of the people to resort to them.

But when subjects, by a long course of such ill conduct, are once thoroughly inflamed, and the state itself violently distempered, the people must have some satisfaction to their feelings more solid than a sophistical speculation on law and government. Such was our situation: and such a satisfaction was necessary to prevent recourse to arms; it was necessary towards laying them down; it will be necessary to prevent the taking them up again and again. Of what nature this satisfaction ought to be I wish it had been the disposition of Parliament seriously to consider. It was certainly a deliberation that called for the exertion of all their wisdom.

I am, and ever have been, deeply sensible of the difficulty of reconciling the strong presiding power, that is so useful towards the conservation of a vast, disconnected, infinitely diversified empire, with that liberty and safety of the provinces which they must enjoy, (in opinion and practice at least,) or they will not be provinces at all. I know, and have long felt, the difficulty of reconciling the unwieldy haughtiness of a great ruling nation, habituated to command, pampered by enormous wealth, and confident from a long course of prosperity and victory, to the high spirit of free dependencies, animated with the first glow and activity of juvenile heat, and assuming to themselves, as their birthright, some part of that very pride which oppresses them. They who perceive no difficulty in reconciling these tempers (which, however, to make peace, must some way or other be reconciled) are much above my capacity, or much below the magnitude of the business. Of one thing I am perfectly clear: that it is not by deciding the suit, but by compromising the difference, that peace can be restored or kept. They who would put an end to such quarrels by declaring roundly in favor of the whole demands of either party have mistaken, in my humble opinion, the office of a mediator.

The war is now of full two years' standing: the controversy of many more. In different periods of the dispute, different methods of reconciliation were to be pursued. I mean to trouble you with a short state of things at the most important of these periods, in order to give you a more distinct idea of our policy with regard to this most delicate of all objects. The colonies were from the beginning subject to the legislature of Great Britain on principles which they never examined; and we permitted to them many local privileges, without asking how they agreed with that legislative authority. Modes of administration were formed in an insensible and very unsystematic manner. But they gradually adapted themselves to the varying condition of things. What was first a single kingdom stretched into an empire; and an imperial superintendence, of some kind or other, became necessary. Parliament, from a mere representative of the people, and a guardian of popular privileges for its own immediate constituents, grew into a mighty sovereign. Instead of being a control on the crown on its own behalf, it communicated a sort of strength to the royal authority, which was wanted for the conservation of a new object, but which could not be safely trusted to the crown alone. On the other hand, the colonies, advancing by equal steps, and governed by the same necessity, had formed within themselves, either by royal instruction or royal charter, assemblies so exceedingly resembling a parliament, in all their forms, functions, and powers, that it was impossible they should not imbibe some opinion of a similar authority.

At the first designation of these assemblies, they were probably not intended for anything more (nor perhaps did they think themselves much higher) than the municipal corporations within this island, to which some at present love to compare them. But nothing in progression can rest on its original plan. We may as well think of rocking a grown man in the cradle of an infant. Therefore, as the colonies prospered and increased to a numerous and mighty people, spreading over a very great tract of the globe, it was natural that they should attribute to assemblies so respectable in their formal constitution some part of the dignity of the great nations which they represented. No longer tied to by-laws, these assemblies made acts of all sorts and in all cases whatsoever. They levied money, not for parochial purposes, but upon regular grants to the crown, following all the rules and principles of a parliament, to which they approached every day more and more nearly. Those who think themselves wiser than Providence and stronger than the course of Nature may complain of all this variation, on the one side or the other, as their several humors and prejudices may lead them. But things could not be otherwise; and English colonies must be had on these terms, or not had at all. In the mean time neither party felt any inconvenience from this double legislature, to which they had been formed by imperceptible habits, and old custom, the great support of all the governments in the world. Though these two legislatures were sometimes found perhaps performing the very same functions, they did not very grossly or systematically clash. In all likelihood this arose from mere neglect, possibly from the natural operation of things, which, left to themselves, generally fall into their proper order. But whatever was the cause, it is certain that a regular revenue, by the authority of Parliament, for the support of civil and military establishments, seems not to have been thought of until the colonies were too proud to submit, too strong to be forced, too enlightened not to see all the consequences which must arise from such a system.

If ever this scheme of taxation was to be pushed against the inclinations of the people, it was evident that discussions must arise, which would let loose all the elements that composed this double constitution, would show how much each of their members had departed from its original principles, and would discover contradictions in each legislature, as well to its own first principles as to its relation to the other, very difficult, if not absolutely impossible, to be reconciled.

Therefore, at the first fatal opening of this contest, the wisest course seemed to be to put an end as soon as possible to the immediate causes of the dispute, and to quiet a discussion, not easily settled upon clear principles, and arising from claims which pride would permit neither party to abandon, by resorting as nearly as possible to the old, successful course. A mere repeal of the obnoxious tax, with a declaration of the legislative authority of this kingdom, was then fully sufficient to procure peace to both sides. Man is a creature of habit, and, the first breach being of very short continuance, the colonies fell back exactly into their ancient state. The Congress has used an expression with regard to this pacification which appears to me truly significant. After the repeal of the Stamp Act, "the colonies fell," says this assembly, "into their ancient state of unsuspecting confidence in the mother country." This unsuspecting confidence is the true centre of gravity amongst mankind, about which all the parts are at rest. It is this unsuspecting confidence that removes all difficulties, and reconciles all the contradictions which occur in the complexity of all ancient puzzled political establishments. Happy are the rulers which have the secret of preserving it!

The whole empire has reason to remember with eternal gratitude the wisdom and temper of that man and his excellent associates, who, to recover this confidence, formed a plan of pacification in 1766. That plan, being built upon the nature of man, and the circumstances and habits of the two countries, and not on any visionary speculations, perfectly answered its end, as long as it was thought proper to adhere to it. Without giving a rude shock to the dignity (well or ill understood) of this Parliament, they gave perfect content to our dependencies. Had it not been for the mediatorial spirit and talents of that great man between such clashing pretensions and passions, we should then have rushed headlong (I know what I say) into the calamities of that civil war in which, by departing from his system, we are at length involved; and we should have been precipitated into that war at a time when circumstances both at home and abroad were far, very far, more unfavorable unto us than they were at the breaking out of the present troubles.

I had the happiness of giving my first votes in Parliament for that pacification. I was one of those almost unanimous members who, in the necessary concessions of Parliament, would as much as possible have preserved its authority and respected its honor. I could not at once tear from my heart prejudices which were dear to me, and which bore a resemblance to virtue. I had then, and I have still, my partialities. What Parliament gave up I wished to be given as of grace and favor and affection, and not as a restitution of stolen goods. High dignity relented as it was soothed; and a benignity from old acknowledged greatness had its full effect on our dependencies. Our unlimited declaration of legislative authority produced not a single murmur. If this undefined power has become odious since that time, and full of horror to the colonies, it is because the unsuspicious confidence is lost, and the parental affection, in the bosom of whose boundless authority they reposed their privileges, is become estranged and hostile.

It will be asked, if such was then my opinion of the mode of pacification, how I came to be the very person who moved, not only for a repeal of all the late coercive statutes, but for mutilating, by a positive law, the entireness of the legislative power of Parliament, and cutting off from it the whole right of taxation. I answer, Because a different state of things requires a different conduct. When the dispute had gone to these last extremities, (which no man labored more to prevent than I did,) the concessions which had satisfied in the beginning could satisfy no longer; because the violation of tacit faith required explicit security. The same cause which has introduced all formal compacts and covenants among men made it necessary: I mean, habits of soreness, jealousy, and distrust. I parted with it as with a limb, but as a limb to save the body: and I would have parted with more, if more had been necessary; anything rather than a fruitless, hopeless, unnatural civil war. This mode of yielding would, it is said, give way to independency without a war. I am persuaded, from the nature of things, and from every information, that it would have had a directly contrary effect. But if it had this effect, I confess that I should prefer independency without war to independency with it; and I have so much trust in the inclinations and prejudices of mankind, and so little in anything else, that I should expect ten times more benefit to this kingdom from the affection of America, though under a separate establishment, than from her perfect submission to the crown and Parliament, accompanied with her terror, disgust, and abhorrence. Bodies tied together by so unnatural a bond of union as mutual hatred are only connected to their ruin.

One hundred and ten respectable members of Parliament voted for that concession. Many not present when the motion was made were of the sentiments of those who voted. I knew it would then have made peace. I am not without hopes that it would do so at present, if it were adopted. No benefit, no revenue, could be lost by it; something might possibly be gained by its consequences. For be fully assured, that, of all the phantoms that ever deluded the fond hopes of a credulous world, a Parliamentary revenue in the colonies is the most perfectly chimerical. Your breaking them to any subjection, far from relieving your burdens, (the pretext for this war,) will never pay that military force which will be kept up to the destruction of their liberties and yours. I risk nothing in this prophecy.

Gentlemen, you have my opinions on the present state of public affairs. Mean as they may be in themselves, your partiality has made them of some importance. Without troubling myself to inquire whether I am under a formal obligation to it, I have a pleasure in accounting for my conduct to my constituents. I feel warmly on this subject, and I express myself as I feel. If I presume to blame any public proceeding, I cannot be supposed to be personal. Would to God I could be suspected of it! My fault might be greater, but the public calamity would be less extensive. If my conduct has not been able to make any impression on the warm part of that ancient and powerful party with whose support I was not honored at my election, on my side, my respect, regard, and duty to them is not at all lessened. I owe the gentlemen who compose it my most humble service in everything. I hope that whenever any of them were pleased to command me, that they found me perfectly equal in my obedience. But flattery and friendship are very different things; and to mislead is not to serve them. I cannot purchase the favor of any man by concealing from him what I think his ruin.

By the favor of my fellow-citizens, I am the representative of an honest, well-ordered, virtuous city,—of a people who preserve more of the original English simplicity and purity of manners than perhaps any other. You possess among you several men and magistrates of large and cultivated understandings, fit for any employment in any sphere. I do, to the best of my power, act so as to make myself worthy of so honorable a choice. If I were ready, on any call of my own vanity or interest, or to answer any election purpose, to forsake principles (whatever they are) which I had formed at a mature age, on full reflection, and which had been confirmed by long experience, I should forfeit the only thing which makes you pardon so many errors and imperfections in me.

Not that I think it fit for any one to rely too much on his own understanding, or to be filled with a presumption not becoming a Christian man in his own personal stability and rectitude. I hope I am far from that vain confidence which almost always fails in trial. I know my weakness in all respects, as much at least as any enemy I have; and I attempt to take security against it. The only method which has ever been found effectual to preserve any man against the corruption of nature and example is an habit of life and communication of councils with the most virtuous and public-spirited men of the age you live in. Such a society cannot be kept without advantage, or deserted without shame. For this rule of conduct I may be called in reproach a party man; but I am little affected with such aspersions. In the way which they call party I worship the Constitution of your fathers; and I shall never blush for my political company. All reverence to honor, all idea of what it is, will be lost out of the world, before it can be imputed as a fault to any man, that he has been closely connected with those incomparable persons, living and dead, with whom for eleven years I have constantly thought and acted. If I have wandered out of the paths of rectitude into those of interested faction, it was in company with the Saviles, the Dowdeswells, the Wentworths, the Bentincks; with the Lenoxes, the Manchesters, the Keppels, the Saunderses; with the temperate, permanent, hereditary virtue of the whole house of Cavendish: names, among which, some have extended your fame and empire in arms, and all have fought the battle of your liberties in fields not less glorious. These, and many more like these, grafting public principles on private honor, have redeemed the present age, and would have adorned the most splendid period in your history. Where could any man, conscious of his own inability to act alone, and willing to act as he ought to do, have arranged himself better? If any one thinks this kind of society to be taken up as the best method of gratifying low personal pride or ambitious interest, he is mistaken, and knows nothing of the world.

Preferring this connection, I do not mean to detract in the slightest degree from others. There are some of those whom I admire at something of a greater distance, with whom I have had the happiness also perfectly to agree, in almost all the particulars in which I have differed with some successive administrations; and they are such as it never can be reputable to any government to reckon among its enemies.

I hope there are none of you corrupted with the doctrine taught by wicked men for the worst purposes, and received by the malignant credulity of envy and ignorance, which is, that the men who act upon the public stage are all alike, all equally corrupt, all influenced by no other views than the sordid lure of salary and pension. The thing I know by experience to be false. Never expecting to find perfection in men, and not looking for divine attributes in created beings, in my commerce with my contemporaries I have found much human virtue. I have seen not a little public spirit, a real subordination of interest to duty, and a decent and regulated sensibility to honest fame and reputation. The age unquestionably produces (whether in a greater or less number than former times I know not) daring profligates and insidious hypocrites. What then? Am I not to avail myself of whatever good is to be found in the world, because of the mixture of evil that will always be in it? The smallness of the quantity in currency only heightens the value. They who raise suspicions on the good on account of the behavior of ill men are of the party of the latter. The common cant is no justification for taking this party. I have been deceived, say they, by Titius and Mævius; I have been the dupe of this pretender or of that mountebank; and I can trust appearances no longer. But my credulity and want of discernment cannot, as I conceive, amount to a fair presumption against any man's integrity. A conscientious person would rather doubt his own judgment than condemn his species. He would say, "I have observed without attention, or judged upon erroneous maxims; I trusted to profession, when I ought to have attended to conduct." Such a man will grow wise, not malignant, by his acquaintance with the world. But he that accuses all mankind of corruption ought to remember that he is sure to convict only one. In truth, I should much rather admit those whom at any time I have disrelished the most to be patterns of perfection than seek a consolation to my own unworthiness in a general communion of depravity with all about me.

That this ill-natured doctrine should be preached by the missionaries of a court I do not wonder. It answers their purpose. But that it should be heard among those who pretend to be strong assertors of liberty is not only surprising, but hardly natural. This moral levelling is a servile principle. It leads to practical passive obedience far better than all the doctrines which the pliant accommodation of theology to power has ever produced. It cuts up by the roots, not only all idea of forcible resistance, but even of civil opposition. It disposes men to an abject submission, not by opinion, which may be shaken by argument or altered by passion, but by the strong ties of public and private interest. For, if all men who act in a public situation are equally selfish, corrupt, and venal, what reason can be given for desiring any sort of change, which, besides the evils which must attend all changes, can be productive of no possible advantage? The active men in the state are true samples of the mass. If they are universally depraved, the commonwealth itself is not sound. We may amuse ourselves with talking as much as we please of the virtue of middle or humble life; that is, we may place our confidence in the virtue of those who have never been tried. But if the persons who are continually emerging out of that sphere be no better than those whom birth has placed above it, what hopes are there in the remainder of the body which is to furnish the perpetual succession of the state? All who have ever written on government are unanimous, that among a people generally corrupt liberty cannot long exist. And, indeed, how is it possible, when those who are to make the laws, to guard, to enforce, or to obey them, are, by a tacit confederacy of manners, indisposed to the spirit of all generous and noble institutions?

I am aware that the age is not what we all wish. But I am sure that the only means of checking its precipitate degeneracy is heartily to concur with whatever is the best in our time, and to have some more correct standard of judging what that best is than the transient and uncertain favor of a court. If once we are able to find, and can prevail on ourselves to strengthen an union of such men, whatever accidentally becomes indisposed to ill-exercised power, even by the ordinary operation of human passions, must join with that society, and cannot long be joined without in some degree assimilating to it. Virtue will catch as well as vice by contact; and the public stock of honest, manly principle will daily accumulate. We are not too nicely to scrutinize motives as long as action is irreproachable. It is enough (and for a worthy man perhaps too much) to deal out its infamy to convicted guilt and declared apostasy.

This, Gentlemen, has been from the beginning the rule of my conduct; and I mean to continue it, as long as such a body as I have described can by any possibility be kept together; for I should think it the most dreadful of all offences, not only towards the present generation, but to all the future, if I were to do anything which could make the minutest breach in this great conservatory of free principles. Those who perhaps have the same intentions, but are separated by some little political animosities, will, I hope, discern at last how little conducive it is to any rational purpose to lower its reputation. For my part, Gentlemen, from much experience, from no little thinking, and from comparing a great variety of things, I am thoroughly persuaded that the last hopes of preserving the spirit of the English Constitution, or of reuniting the dissipated members of the English race upon a common plan of tranquillity and liberty, does entirely depend on their firm and lasting union, and above all on their keeping themselves from that despair which is so very apt to fall on those whom a violence of character and a mixture of ambitious views do not support through a long, painful, and unsuccessful struggle.

There never, Gentlemen, was a period in which the steadfastness of some men has been put to so sore a trial. It is not very difficult for well-formed minds to abandon their interest; but the separation of fame and virtue is an harsh divorce. Liberty is in danger of being made unpopular to Englishmen. Contending for an imaginary power, we begin to acquire the spirit of domination, and to lose the relish of honest equality. The principles of our forefathers become suspected to us, because we see them animating the present opposition of our children. The faults which grow out of the luxuriance of freedom appear much more shocking to us than the base vices which are generated from the rankness of servitude. Accordingly, the least resistance to power appears more inexcusable in our eyes than the greatest abuses of authority. All dread of a standing military force is looked upon as a superstitious panic. All shame of calling in foreigners and savages in a civil contest is worn off. We grow indifferent to the consequences inevitable to ourselves from the plan of ruling half the empire by a mercenary sword. We are taught to believe that a desire of domineering over our countrymen is love to our country, that those who hate civil war abet rebellion, and that the amiable and conciliatory virtues of lenity, moderation, and tenderness to the privileges of those who depend on this kingdom are a sort of treason to the state.

It is impossible that we should remain long in a situation which breeds such notions and dispositions without some great alteration in the national character. Those ingenuous and feeling minds who are so fortified against all other things, and so unarmed to whatever approaches in the shape of disgrace, finding these principles, which they considered as sure means of honor, to be grown into disrepute, will retire disheartened and disgusted. Those of a more robust make, the bold, able, ambitious men, who pay some of their court to power through the people, and substitute the voice of transient opinion in the place of true glory, will give into the general mode; and those superior understandings which ought to correct vulgar prejudice will confirm and aggravate its errors. Many things have been long operating towards a gradual change in our principles; but this American war has done more in a very few years than all the other causes could have effected in a century. It is therefore not on its own separate account, but because of its attendant circumstances, that I consider its continuance, or its ending in any way but that of an honorable and liberal accommodation, as the greatest evils which can befall us. For that reason I have troubled you with this long letter. For that reason I entreat you, again and again, neither to be persuaded, shamed, or frighted out of the principles that have hitherto led so many of you to abhor the war, its cause, and its consequences. Let us not be amongst the first who renounce the maxims of our forefathers.

\vspace{0.3cm}
\hspace{1in}I have the honor to be,

\hspace{2.5in}Gentlemen,

Your most obedient and faithful humble servant,

\hfill EDMUND BURKE.

BEACONSFIELD, April 3, 1777.

\vspace{0.3cm}
P.S. You may communicate this letter in any manner you think proper to my constituents.


%%%%%%%%%%%%%%%%%%%%%%%%%%%%%%%%%%%%%%%%%%%%%%%%%%%%%%%%%%%%%%%%%%%%%%%
\chapter*[Two Letters to Gentlemen of Bristol]{
Two Letters to Gentlemen in the City of Bristol,
on the Bills Depending in Parliament Relative to the Trade of Ireland
\\ \vspace{0.1cm}\large{April 23 and May 2, 1777}}
%\label{chap:vindication}
\addcontentsline{toc}{chapter}{
TWO LETTERS TO GENTLEMEN IN THE CITY OF BRISTOL,
ON THE BILLS DEPENDING IN PARLIAMENT RELATIVE TO THE TRADE OF IRELAND,
April 23 and May 2, 1778}

%%%%%%%%%%%%%%%%%%%%%%%%%%%%%%%%%%%%%%%%%
\begin{center}
  \textbf{\large 
     TO SAMUEL SPAN, ESQ., MASTER OF THE SOCIETY OF MERCHANTS ADVENTURERS OF BRISTOL.
  } \par 
\end{center}

Sir,—I am honored with your letter of the 13th, in answer to mine, which accompanied the resolutions of the House relative to the trade of Ireland.

You will be so good as to present my best respects to the Society, and to assure them that it was altogether unnecessary to remind me of the interest of the constituents. I have never regarded anything else since I had a seat in Parliament. Having frequently and maturely considered that interest, and stated it to myself in almost every point of view, I am persuaded, that, under the present circumstances, I cannot more effectually pursue it than by giving all the support in my power to the propositions which I lately transmitted to the Hall.

The fault I find in the scheme is, that it falls extremely short of that liberality in the commercial system which I trust will one day be adopted. If I had not considered the present resolutions merely as preparatory to better things, and as a means of showing, experimentally, that justice to others is not always folly to ourselves, I should have contented myself with receiving them in a cold and silent acquiescence. Separately considered, they are matters of no very great importance. But they aim, however imperfectly, at a right principle. I submit to the restraint to appease prejudice; I accept the enlargement, so far as it goes, as the result of reason and of sound policy.

We cannot be insensible of the calamities which have been brought upon this nation by an obstinate adherence to narrow and restrictive plans of government. I confess, I cannot prevail on myself to take them up precisely at a time when the most decisive experience has taught the rest of the world to lay them down. The propositions in question did not originate from me, or from my particular friends. But when things are so right in themselves, I hold it my duty not to inquire from what hands they come. I opposed the American measures upon the very same principle on which I support those that relate to Ireland. I was convinced that the evils which have arisen from the adoption of the former would be infinitely aggravated by the rejection of the latter.

Perhaps gentlemen are not yet fully aware of the situation of their country, and what its exigencies absolutely require. I find that we are still disposed to talk at our ease, and as if all things were to be regulated by our good pleasure. I should consider it as a fatal symptom, if, in our present distressed and adverse circumstances, we should persist in the errors which are natural only to prosperity. One cannot, indeed, sufficiently lament the continuance of that spirit of delusion, by which, for a long time past, we have thought fit to measure our necessities by our inclinations. Moderation, prudence, and equity are far more suitable to our condition than loftiness, and confidence, and rigor. We are threatened by enemies of no small magnitude, whom, if we think fit, we may despise, as we have despised others; but they are enemies who can only cease to be truly formidable by our entertaining a due respect for their power. Our danger will not be lessened by our shutting our eyes to it; nor will our force abroad be increased by rendering ourselves feeble and divided at home.

There is a dreadful schism in the British nation. Since we are not able to reunite the empire, it is our business to give all possible vigor and soundness to those parts of it which are still content to be governed by our councils. Sir, it is proper to inform you that our measures must be healing. Such a degree of strength must be communicated to all the members of the state as may enable them to defend themselves, and to coöperate in the defence of the whole. Their temper, too, must be managed, and their good affections cultivated. They may then be disposed to bear the load with cheerfulness, as a contribution towards what may be called with truth and propriety, and not by an empty form of words, a common cause. Too little dependence cannot be had, at this time of day, on names and prejudices. The eyes of mankind are opened, and communities must be held together by an evident and solid interest. God forbid that our conduct should demonstrate to the world that Great Britain can in no instance whatsoever be brought to a sense of rational and equitable policy but by coercion and force of arms!

I wish you to recollect with what powers of concession, relatively to commerce, as well as to legislation, his Majesty's commissioners to the United Colonies have sailed from England within this week. Whether these powers are sufficient for their purposes it is not now my business to examine. But we all know that our resolutions in favor of Ireland are trifling and insignificant, when compared with the concessions to the Americans. At such a juncture, I would implore every man, who retains the least spark of regard to the yet remaining honor and security of this country, not to compel others to an imitation of their conduct, or by passion and violence to force them to seek in the territories of the separation that freedom and those advantages which they are not to look for whilst they remain under the wings of their ancient government.

After all, what are the matters we dispute with so much warmth? Do we in these resolutions bestow anything upon Ireland? Not a shilling. We only consent to leave to them, in two or three instances, the use of the natural faculties which God has given to them, and to all mankind. Is Ireland united to the crown of Great Britain for no other purpose than that we should counteract the bounty of Providence in her favor? and in proportion as that bounty has been liberal, that we are to regard it as an evil, which is to be met with in every sort of corrective? To say that Ireland interferes with us, and therefore must be checked, is, in my opinion, a very mistaken, and a very dangerous principle. I must beg leave to repeat, what I took the liberty of suggesting to you in my last letter, that Ireland is a country in the same climate and of the same natural qualities and productions with this, and has consequently no other means of growing wealthy in herself, or, in other words, of being useful to us, but by doing the very same things which we do for the same purposes. I hope that in Great Britain we shall always pursue, without exception, every means of prosperity, and, of course, that Ireland will interfere with us in something or other: for either, in order to limit her, we must restrain ourselves, or we must fall into that shocking conclusion, that we are to keep our yet remaining dependency under a general and indiscriminate restraint for the mere purpose of oppression. Indeed, Sir, England and Ireland may flourish together. The world is large enough for us both. Let it be our care not to make ourselves too little for it.

I know it is said, that the people of Ireland do not pay the same taxes, and therefore ought not in equity to enjoy the same benefits with this. I had hopes that the unhappy phantom of a compulsory equal taxation had haunted us long enough. I do assure you, that, until it is entirely banished from our imaginations, (where alone it has, or can have, any existence,) we shall never cease to do ourselves the most substantial injuries. To that argument of equal taxation I can only say, that Ireland pays as many taxes as those who are the best judges of her powers are of opinion she can bear. To bear more, she must have more ability; and, in the order of Nature, the advantage must precede the charge. This disposition of things being the law of God, neither you nor I can alter it. So that, if you will have more help from Ireland, you must previously supply her with more means. I believe it will be found, that, if men are suffered freely to cultivate their natural advantages, a virtual equality of contribution will come in its own time, and will flow by an easy descent through its own proper and natural channels. An attempt to disturb that course, and to force Nature, will only bring on universal discontent, distress, and confusion.

You tell me, Sir, that you prefer an union with Ireland to the little regulations which are proposed in Parliament. This union is a great question of state, to which, when it comes properly before me in my Parliamentary capacity, I shall give an honest and unprejudiced consideration. However, it is a settled rule with me, to make the most of my actual situation, and not to refuse to do a proper thing because there is something else more proper which I am not able to do. This union is a business of difficulty, and, on the principles of your letter, a business impracticable. Until it can be matured into a feasible and desirable scheme, I wish to have as close an union of interest and affection with Ireland as I can have; and that, I am sure, is a far better thing than any nominal union of government.

France, and indeed most extensive empires, which by various designs and fortunes have grown into one great mass, contain many provinces that are very different from each other in privileges and modes of government; and they raise their supplies in different ways, in different proportions, and under different authorities: yet none of them are for this reason curtailed of their natural rights; but they carry on trade and manufactures with perfect equality. In some way or other the true balance is found; and all of them are properly poised and harmonized. How much have you lost by the participation of Scotland in all your commerce? The external trade of England has more than doubled since that period; and I believe your internal (which is the most advantageous) has been augmented at least fourfold. Such virtue there is in liberality of sentiment, that you have grown richer even by the partnership of poverty.

If you think that this participation was a loss, commercially considered, but that it has been compensated by the share which Scotland has taken in defraying the public charge, I believe you have not very carefully looked at the public accounts. Ireland, Sir, pays a great deal more than Scotland, and is perhaps as much and as effectually united to England as Scotland is. But if Scotland, instead of paying little, had paid nothing at all, we should be gainers, not losers, by acquiring the hearty coöperation of an active, intelligent people towards the increase of the common stock, instead of our being employed in watching and counteracting them, and their being employed in watching and counteracting us, with the peevish and churlish jealousy of rivals and enemies on both sides.

I am sure, Sir, that the commercial experience of the merchants of Bristol will soon disabuse them of the prejudice, that they can trade no longer, if countries more lightly taxed are permitted to deal in the same commodities at the same markets. You know, that, in fact, you trade very largely where you are met by the goods of all nations. You even pay high duties on the import of your goods, and afterwards undersell nations less taxed, at their own markets, and where goods of the same kind are not charged at all. If it were otherwise, you could trade very little. You know that the price of all sorts of manufacture is not a great deal enhanced (except to the domestic consumer) by any taxes paid in this country. This I might very easily prove.

The same consideration will relieve you from the apprehension you express with relation to sugars, and the difference of the duties paid here and in Ireland. Those duties affect the interior consumer only, and for obvious reasons, relative to the interest of revenue itself, they must be proportioned to his ability of payment; but in all cases in which sugar can be an object of commerce, and therefore (in this view) of rivalship, you are sensible that you are at least on a par with Ireland. As to your apprehensions concerning the more advantageous situation of Ireland for some branches of commerce, (for it is so but for some,) I trust you will not find them more serious. Milford Haven, which is at your door, may serve to show you that the mere advantage of ports, is not the thing which shifts the seat of commerce from one part of the world to the other. If I thought you inclined to take up this matter on local considerations, I should state to you, that I do not know any part of the kingdom so well situated for an advantageous commerce with Ireland as Bristol, and that none would be so likely to profit of its prosperity as our city. But your profit and theirs must concur. Beggary and bankruptcy are not the circumstances which invite to an intercourse with that or with any country; and I believe it will be found invariably true, that the superfluities of a rich nation furnish a better object of trade than the necessities of a poor one. It is the interest of the commercial world that wealth should be found everywhere.

The true ground of fear, in my opinion, is this: that Ireland, from the vicious system of its internal polity, will be a long time before it can derive any benefit from the liberty now granted, or from any thing else. But, as I do not vote advantages in hopes that they may not be enjoyed, I will not lay any stress upon this consideration. I rather wish that the Parliament of Ireland may, in its own wisdom, remove these impediments, and put their country in a condition to avail itself of its natural advantages. If they do not, the fault is with them, and not with us.

I have written this long letter in order to give all possible satisfaction to my constituents with regard to the part I have taken in this affair. It gave me inexpressible concern to find that my conduct had been a cause of uneasiness to any of them. Next to my honor and conscience, I have nothing so near and dear to me as their approbation. However, I had much rather run the risk of displeasing than of injuring them,—if I am driven to make such an option. You obligingly lament that you are not to have me for your advocate; but if I had been capable of acting as an advocate in opposition to a plan so perfectly consonant to my known principles, and to the opinions I had publicly declared on an hundred occasions, I should only disgrace myself, without supporting, with the smallest degree of credit or effect, the cause you wished me to undertake. I should have lost the only thing which can make such abilities as mine of any use to the world now or hereafter: I mean that authority which is derived from an opinion that a member speaks the language of truth and sincerity, and that he is not ready to take up or lay down a great political system for the convenience of the hour, that he is in Parliament to support his opinion of the public good, and does not form his opinion in order to get into Parliament, or to continue in it. It is in a great measure for your sake that I wish to preserve this character. Without it, I am sure, I should be ill able to discharge, by any service, the smallest part of that debt of gratitude and affection which I owe you for the great and honorable trust you have reposed in me.

I am, with the highest regard and esteem, Sir,

\hspace{1.5in} Your most obedient and humble servant,

\hfill E.B.

BEACONSFIELD, 23rd April, 1778.

%%%%%%%%%%%%%%%%%%%%%%%%%%%%%%%%%%%%%%%%%
\begin{center}
  \textbf{\large 
    COPY OF A LETTER TO MESSRS. ******* ****** AND CO., BRISTOL.
  } \par 
\end{center}

Gentlemen,—

It gives me the most sensible concern to find that my vote on the resolutions relative to the trade of Ireland has not been fortunate enough to meet with your approbation. I have explained at large the grounds of my conduct on that occasion in my letters to the Merchants' Hall; but my very sincere regard and esteem for you will not permit me to let the matter pass without an explanation which is particular to yourselves, and which I hope will prove satisfactory to you.

You tell me that the conduct of your late member is not much wondered at; but you seem to be at a loss to account for mine; and you lament that I have taken so decided a part against my constituents.

This is rather an heavy imputation. Does it, then, really appear to you that the propositions to which you refer are, on the face of them, so manifestly wrong, and so certainly injurious to the trade and manufactures of Great Britain, and particularly to yours, that no man could think of proposing or supporting them, except from resentment to you, or from some other oblique motive? If you suppose your late member, or if you suppose me, to act upon other reasons than we choose to avow, to what do you attribute the conduct of the other members, who in the beginning almost unanimously adopted those resolutions? To what do you attribute the strong part taken by the ministers, and, along with the ministers, by several of their most declared opponents? This does not indicate a ministerial job, a party design, or a provincial or local purpose. It is, therefore, not so absolutely clear that the measure is wrong, or likely to be injurious to the true interests of any place or any person.

The reason, Gentlemen, for taking this step, at this time, is but too obvious and too urgent. I cannot imagine that you forget the great war which has been carried on with so little success (and, as I thought, with so little policy) in America, or that you are not aware of the other great wars which are impending. Ireland has been called upon to repel the attacks of enemies of no small power, brought upon her by councils in which she has had no share. The very purpose and declared object of that original war, which has brought other wars and other enemies on Ireland, was not very flattering to her dignity, her interest, or to the very principle of her liberty. Yet she submitted patiently to the evils she suffered from an attempt to subdue to your obedience countries whose very commerce was not open to her. America was to be conquered in order that Ireland should not trade thither; whilst the miserable trade which she is permitted to carry on to other places has been torn to pieces in the struggle. In this situation, are we neither to suffer her to have any real interest in our quarrel, or to be flattered with the hope of any future means of bearing the burdens which she is to incur in defending herself against enemies which we have brought upon her?

I cannot set my face against such arguments. Is it quite fair to suppose that I have no other motive for yielding to them but a desire of acting against my constituents? It is for you, and for your interest, as a dear, cherished, and respected part of a valuable whole, that I have taken my share in this question. You do not, you cannot, suffer by it. If honesty be true policy with regard to the transient interest of individuals, it is much more certainly so with regard to the permanent interests of communities. I know that it is but too natural for us to see our own certain ruin in the possible prosperity of other people. It is hard to persuade us that everything which is got by another is not taken from ourselves. But it is fit that We should get the better of these suggestions, which come from what is not the best and soundest part of our nature, and that we should form to ourselves a way of thinking, more rational, more just, and more religious. Trade is not a limited thing: as if the objects of mutual demand and consumption could not stretch beyond the bounds of our jealousies. God has given the earth to the children of men, and He has undoubtedly, in giving it to them, given them what is abundantly sufficient for all their exigencies: not a scanty, but a most liberal, provision for them all. The Author of our nature has written it strongly in that nature, and has promulgated the same law in His written word, that man shall eat his bread by his labor; and I am persuaded that no man, and no combination of men, for their own ideas of their particular profit, can, without great impiety, undertake to say that he shall not do so,—that they have no sort of right either to prevent the labor or to withhold the bread. Ireland having received no compensation, directly or indirectly, for any restraints on their trade, ought not, in justice or common honesty, to be made subject to such restraints. I do not mean to impeach the right of the Parliament of Great Britain to make laws for the trade of Ireland: I only speak of what laws it is right for Parliament to make.

It is nothing to an oppressed people, to say that in part they are protected at our charge. The military force which shall be kept up in order to cramp the natural faculties of a people, and to prevent their arrival to their utmost prosperity, is the instrument of their servitude, not the means of their protection. To protect men is to forward, and not to restrain, their improvement. Else, what is it more than to avow to them, and to the world, that you guard them from others only to make them a prey to yourself? This fundamental nature of protection does not belong to free, but to all governments, and is as valid in Turkey as in Great Britain. No government ought to own that it exists for the purpose of checking the prosperity of its people, or that there is such a principle involved in its policy.

Under the impression of these sentiments, (and not as wanting every attention to my constituents which affection and gratitude could inspire,) I voted for these bills which give you so much trouble. I voted for them, not as doing complete justice to Ireland, but as being something less unjust than the general prohibition which has hitherto prevailed. I hear some discourse as if, in one or two paltry duties on materials, Ireland had a preference, and that those who set themselves against this act of scanty justice assert that they are only contending for an equality. What equality? Do they forget that the whole woollen manufacture of Ireland, the most extensive and profitable of any, and the natural staple of that kingdom, has been in a manner so destroyed by restrictive laws of ours, and (at our persuasion, and on our promises) by restrictive laws of their own, that in a few years, it is probable, they will not be able to wear a coat of their own fabric? Is this equality? Do gentlemen forget that the understood faith upon which they were persuaded to such an unnatural act has not been kept,—but a linen-manufacture has been set up, and highly encouraged, against them? Is this equality? Do they forget the state of the trade of Ireland in beer, so great an article of consumption, and which now stands in so mischievous a position with regard to their revenue, their manufacture, and their agriculture? Do they find any equality in all this? Yet, if the least step is taken towards doing them common justice in the slightest articles for the most limited markets, a cry is raised, as if we were going to be ruined by partiality to Ireland.

Gentlemen, I know that the deficiency in these arguments is made up (not by you, but by others) by the usual resource on such occasions, the confidence in military force and superior power. But that ground of confidence, which at no time was perfectly just, or the avowal of it tolerably decent, is at this time very unseasonable. Late experience has shown that it cannot be altogether relied upon; and many, if not all, of our present difficulties have arisen from putting our trust in what may very possibly fail, and, if it should fail, leaves those who are hurt by such a reliance without pity. Whereas honesty and justice, reason and equity, go a very great way in securing prosperity to those who use them, and, in case of failure, secure the best retreat and the most honorable consolations.

It is very unfortunate that we should consider those as rivals, whom we ought to regard as fellow-laborers in a common cause. Ireland has never made a single step in its progress towards prosperity, by which you have not had a share, and perhaps the greatest share, in the benefit. That progress has been chiefly owing to her own natural advantages, and her own efforts, which, after a long time, and by slow degrees, have prevailed in some measure over the mischievous systems which have been adopted. Far enough she is still from having arrived even at an ordinary state of perfection; and if our jealousies were to be converted into politics as systematically as some would have them, the trade of Ireland would vanish out of the system of commerce. But, believe me, if Ireland is beneficial to you, it is so not from the parts in which it is restrained, but from those in which it is left free, though not left unrivalled. The greater its freedom, the greater must be your advantage. If you should lose in one way, you will gain in twenty.

Whilst I remain under this unalterable and powerful conviction, you will not wonder at the decided part I take. It is my custom so to do, when I see my way clearly before me, and when I know that I am not misled by any passion or any personal interest, which in this case I am very sure I am not. I find that disagreeable things are circulated among my constituents; and I wish my sentiments, which form my justification, may be equally general with the circulation against me. I have the honor to be, with the greatest regard and esteem, Gentlemen,

\hspace{1.5in} Your most obedient and humble servant,

\hfill E.B.

\noindent
Westminster, May 2, 1778.

I send the bills.




%%%%%%%%%%%%%%%%%%%%%%%%%%%%%%%%%%%%%%%%%%%%%%%%%%%%%%%%%%%%%%%%%%%%%%%
\chapter*[Speech on The Plan for Economical Reform]{
Speech on Presenting to the House of Commons
\\(on the 11th February, 1780)
\\A Plan for the Better Security of the Independence of Parliament, 
and the Economical Reformation of the Civil and Other Establishments}
%\label{chap:vindication}
\addcontentsline{toc}{chapter}{
SPEECH ON PRESENTING TO THE HOUSE OF COMMONS
A PLAN FOR THE BETTER SECURITY OF THE INDEPENDENCE OF PARLIAMENT, 
AND THE ECONOMICAL REFORMATION OF THE CIVIL AND OTHER ESTABLISHMENTS,
February 11, 1780}

Mr. Speaker,—I rise, in acquittal of my engagement to the House, in obedience to the strong and just requisition of my constituents, and, I am persuaded, in conformity to the unanimous wishes of the whole nation, to submit to the wisdom of Parliament "A Plan of Reform in the Constitution of Several Parts of the Public Economy."

I have endeavored that this plan should include, in its execution, a considerable reduction of improper expense; that it should effect a conversion of unprofitable titles into a productive estate; that it should lead to, and indeed almost compel, a provident administration of such sums of public money as must remain under discretionary trusts; that it should render the incurring debts on the civil establishment (which must ultimately affect national strength and national credit) so very difficult as to become next to impracticable.

But what, I confess, was uppermost with me, what I bent the whole force of my mind to, was the reduction of that corrupt influence which is itself the perennial spring of all prodigality and of all disorder,—which loads us more than millions of debt,—which takes away vigor from our arms, wisdom from our councils, and every shadow of authority and credit from the most venerable parts of our Constitution.

Sir, I assure you very solemnly, and with a very clear conscience, that nothing in the world has led me to such an undertaking but my zeal for the honor of this House, and the settled, habitual, systematic affection I bear to the cause and to the principles of government.

I enter perfectly into the nature and consequences of my attempt, and I advance to it with a tremor that shakes me to the inmost fibre of my frame. I feel that I engage in a business, in itself most ungracious, totally wide of the course of prudent conduct, and, I really think, the most completely adverse that can be imagined to the natural turn and temper of my own mind. I know that all parsimony is of a quality approaching to unkindness, and that (on some person or other) every reform must operate as a sort of punishment. Indeed, the whole class of the severe and restrictive virtues are at a market almost too high for humanity. What is worse, there are very few of those virtues which are not capable of being imitated, and even outdone in many of their most striking effects, by the worst of vices. Malignity and envy will carve much more deeply, and finish much more sharply, in the work of retrenchment, than frugality and providence. I do not, therefore, wonder that gentlemen have kept away from such a task, as well from good-nature as from prudence. Private feeling might, indeed, be overborne by legislative reason; and a man of a long-sighted and a strong-nerved humanity might bring himself not so much to consider from whom he takes a superfluous enjoyment as for whom in the end he may preserve the absolute necessaries of life.

But it is much more easy to reconcile this measure in humanity than to bring it to any agreement with prudence. I do not mean that little, selfish, pitiful, bastard thing which sometimes goes by the name of a family in which it is not legitimate and to which it is a disgrace;—I mean even that public and enlarged prudence, which, apprehensive of being disabled from rendering acceptable services to the world, withholds itself from those that are invidious. Gentlemen who are, with me, verging towards the decline of life, and are apt to form their ideas of kings from kings of former times, might dread the anger of a reigning prince;—they who are more provident of the future, or by being young are more interested in it, might tremble at the resentment of the successor; they might see a long, dull, dreary, unvaried visto of despair and exclusion, for half a century, before them. This is no pleasant prospect at the outset of a political journey.

Besides this, Sir, the private enemies to be made in all attempts of this kind are innumerable; and their enmity will be the more bitter, and the more dangerous too, because a sense of dignity will oblige them to conceal the cause of their resentment. Very few men of great families and extensive connections but will feel the smart of a cutting reform, in some close relation, some bosom friend, some pleasant acquaintance, some dear, protected dependant. Emolument is taken from some; patronage from others; objects of pursuit from all. Men forced into an involuntary independence will abhor the authors of a blessing which in their eyes has so very near a resemblance to a curse. When officers are removed, and the offices remain, you may set the gratitude of some against the anger of others, you may oppose the friends you oblige against the enemies you provoke. But services of the present sort create no attachments. The individual good felt in a public benefit is comparatively so small, comes round through such an involved labyrinth of intricate and tedious revolutions, whilst a present personal detriment is so heavy, where it falls, and so instant in its operation, that the cold commendation of a public advantage never was and never will be a match for the quick sensibility of a private loss; and you may depend upon it, Sir, that, when many people have an interest in railing, sooner or later, they will bring a considerable degree of unpopularity upon any measure. So that, for the present at least, the reformation will operate against the reformers; and revenge (as against them at the least) will produce all the effects of corruption.

This, Sir, is almost always the case, where the plan has complete success. But how stands the matter in the mere attempt? Nothing, you know, is more common than for men to wish, and call loudly too, for a reformation, who, when it arrives, do by no means like the severity of its aspect. Reformation is one of those pieces which must be put at some distance in order to please. Its greatest favorers love it better in the abstract than in the substance. When any old prejudice of their own, or any interest that they value, is touched, they become scrupulous, they become captious; and every man has his separate exception. Some pluck out the black hairs, some the gray; one point must be given up to one, another point must be yielded to another; nothing is suffered to prevail upon its own principle; the whole is so frittered down and disjointed, that scarcely a trace of the original scheme remains. Thus, between the resistance of power, and the unsystematical process of popularity, the undertaker and the undertaking are both exposed, and the poor reformer is hissed off the stags both by friends and foes.

Observe, Sir, that the apology for my undertaking (an apology which, though long, is no longer than necessary) is not grounded on my want of the fullest sense of the difficult and invidious nature of the task I undertake. I risk odium, if I succeed, and contempt, if I fail. My excuse must rest in mine and your conviction of the absolute, urgent necessity there is that something of the kind should be done. If there is any sacrifice to be made, either of estimation or of fortune, the smallest is the best. Commanders-in-chief are not to be put upon the forlorn hope. But, indeed, it is necessary that the attempt should be made. It is necessary from our own political circumstances; it is necessary from the operations of the enemy; it is necessary from the demands of the people, whose desires, when they do not militate with the stable and eternal rules of justice and reason, (rules which are above us and above them,) ought to be as a law to a House of Commons.

As to our circumstances, I do not mean to aggravate the difficulties of them by the strength of any coloring whatsoever. On the contrary, I observe, and observe with pleasure, that our affairs rather wear a more promising aspect than they did on the opening of this session. We have had some leading successes. But those who rate them at the highest (higher a great deal, indeed, than I dare to do) are of opinion, that, upon the ground of such advantages, we cannot at this time hope to make any treaty of peace which would not be ruinous and completely disgraceful. In such an anxious state of things, if dawnings of success serve to animate our diligence, they are good; if they tend to increase our presumption, they are worse than defeats. The state of our affairs shall, then, be as promising as any one may choose to conceive it: it is, however, but promising. We must recollect, that, with but half of our natural strength, we are at war against confederated powers who have singly threatened us with ruin; we must recollect, that, whilst we are left naked on one side, our other flank is uncovered by any alliance; that, whilst we are weighing and balancing our successes against our losses, we are accumulating debt to the amount of at least fourteen millions in the year. That loss is certain.

I have no wish to deny that our successes are as brilliant as any one chooses to make them; our resources, too, may, for me, be as unfathomable as they are represented. Indeed, they are just whatever the people possess and will submit to pay. Taxing is an easy business. Any projector can contrive new impositions; any bungler can add to the old. But is it altogether wise to have no other bounds to your impositions than the patience of those who are to bear them?

All I claim upon the subject of your resources is this: that they are not likely to be increased by wasting them. I think I shall be permitted to assume that a system of frugality will not lessen your riches, whatever they may be. I believe it will not be hotly disputed, that those resources which lie heavy on the subject ought not to be objects of preference,—that they ought not to be the very first choice, to an honest representative of the people.

This is all, Sir, that I shall say upon our circumstances and our resources: I mean to say a little more on the operations of the enemy, because this matter seems to me very natural in our present deliberation. When I look to the other side of the water, I cannot help recollecting what Pyrrhus said, on reconnoitring the Roman camp:—"These barbarians have nothing barbarous in their discipline." When I look, as I have pretty carefully looked, into the proceedings of the French king, I am sorry to say it, I see nothing of the character and genius of arbitrary finance, none of the bold frauds of bankrupt power, none of the wild struggles and plunges of despotism in distress,—no lopping off from the capital of debt, no suspension of interest, no robbery under the name of loan, no raising the value, no debasing the substance of the coin. I see neither Louis the Fourteenth nor Louis the Fifteenth. On the contrary, I behold, with astonishment, rising before me, by the very hands of arbitrary power, and in the very midst of war and confusion, a regular, methodical system of public credit; I behold a fabric laid on the natural and solid foundations of trust and confidence among men, and rising, by fair gradations, order over order, according to the just rules of symmetry and art. What a reverse of things! Principle, method, regularity, economy, frugality, justice to individuals, and care of the people are the resources with which France makes war upon Great Britain. God avert the omen! But if we should see any genius in war and politics arise in France to second what is done in the bureau!—I turn my eyes from the consequences.

The noble lord in the blue ribbon, last year, treated all this with contempt. He never could conceive it possible that the French minister of finance could go through that year with a loan of but seventeen hundred thousand pounds, and that he should be able to fund that loan without any tax. The second year, however, opens the very same scene. A small loan, a loan of no more than two millions five hundred thousand pounds, is to carry our enemies through the service of this year also. No tax is raised to fund that debt; no tax is raised for the current services. I am credibly informed that there is no anticipation whatsoever. Compensations
%[31] 
\footnote{This term comprehends various retributions made to persons whose offices are taken away, or who in any other way suffer by the new arrangements that are made.}
are correctly made. Old debts continue to be sunk as in the time of profound peace. Even payments which their treasury had been authorized to suspend during the time of war are not suspended.

A general reform, executed through every department of the revenue, creates an annual income of more than half a million, whilst it facilitates and simplifies all the functions of administration. The king's household—at the remotest avenues to which all reformation has been hitherto stopped, that household which has been the stronghold of prodigality, the virgin fortress which was never before attacked—has been not only not defended, but it has, even in the forms, been surrendered by the king to the economy of his minister. No capitulation; no reserve. Economy has entered in triumph into the public splendor of the monarch, into his private amusements, into the appointments of his nearest and highest relations. Economy and public spirit have made a beneficent and an honest spoil: they have plundered from extravagance and luxury, for the use of substantial service, a revenue of near four hundred thousand pounds. The reform of the finances, joined to this reform of the court, gives to the public nine hundred thousand pounds a year, and upwards.

The minister who does these things is a great man; but the king who desires that they should be done is a far greater. We must do justice to our enemies: these are the acts of a patriot king. I am not in dread of the vast armies of France; I am not in dread of the gallant spirit of its brave and numerous nobility; I am not alarmed even at the great navy which has been so miraculously created. All these things Louis the Fourteenth had before. With all these things, the French monarchy has more than once fallen prostrate at the feet of the public faith of Great Britain. It was the want of public credit which disabled France from recovering after her defeats, or recovering even from her victories and triumphs. It was a prodigal court, it was an ill-ordered revenue, that sapped the foundations of all her greatness. Credit cannot exist under the arm of necessity. Necessity strikes at credit, I allow, with a heavier and quicker blow under an arbitrary monarchy than under a limited and balanced government; but still necessity and credit are natural enemies, and cannot be long reconciled in any situation. From necessity and corruption, a free state may lose the spirit of that complex constitution which is the foundation of confidence. On the other hand, I am far from being sure that a monarchy, when once it is properly regulated, may not for a long time furnish a foundation for credit upon the solidity of its maxims, though it affords no ground of trust in its institutions. I am afraid I see in England, and in France, something like a beginning of both these things. I wish I may be found in a mistake.

This very short and very imperfect state of what is now going on in France (the last circumstances of which I received in about eight days after the registry of the edict
%[32]
\footnote{Edict registered 29th January, 1780.}
) I do not, Sir, lay before you for any invidious purpose. It is in order to excite in us the spirit of a noble emulation. Let the nations make war upon each other, (since we must make war,) not with a low and vulgar malignity, but by a competition of virtues. This is the only way by which both parties can gain by war. The French have imitated us: let us, through them, imitate ourselves,—ourselves in our better and happier days. If public frugality, under whatever men, or in whatever mode of government, is national strength, it is a strength which our enemies are in possession of before us.

Sir, I am well aware that the state and the result of the French economy which I have laid before you are even now lightly treated by some who ought never to speak but from information. Pains have not been spared to represent them as impositions on the public. Let me tell you, Sir, that the creation of a navy, and a two years' war without taxing, are a very singular species of imposture. But be it so. For what end does Necker carry on this delusion? Is it to lower the estimation of the crown he serves, and to render his own administration contemptible? No! No! He is conscious that the sense of mankind is so clear and decided in favor of economy, and of the weight and value of its resources, that he turns himself to every species of fraud and artifice to obtain the mere reputation of it. Men do not affect a conduct that tends to their discredit. Let us, then, get the better of Monsieur Necker in his own way; let us do in reality what he does only in pretence; let us turn his French tinsel into English gold. Is, then, the mere opinion and appearance of frugality and good management of such use to France, and is the substance to be so mischievous to England? Is the very constitution of Nature so altered by a sea of twenty miles, that economy should give power on the Continent, and that profusion should give it here? For God's sake, let not this be the only fashion of France which we refuse to copy!

To the last kind of necessity, the desires of the people, I have but a very few words to say. The ministers seem to contest this point, and affect to doubt whether the people do really desire a plan of economy in the civil government. Sir, this is too ridiculous. It is impossible that they should not desire it. It is impossible that a prodigality which draws its resources from their indigence should be pleasing to them. Little factions of pensioners, and their dependants, may talk another language. But the voice of Nature is against them, and it will be heard. The people of England will not, they cannot, take it kindly, that representatives should refuse to their constituents what an absolute sovereign voluntarily offers to his subjects. The expression of the petitions is, that, "before any new burdens are laid upon this country, effectual measures be taken by this House to inquire into and correct the gross abuses in the expenditure of public money."

This has been treated by the noble lord in the blue ribbon as a wild, factious language. It happens, however, that the people, in their address to us, use, almost word for word, the same terms as the king of France uses in addressing himself to his people; and it differs only as it falls short of the French king's idea of what is due to his subjects. "To convince," says he, "our faithful subjects of the desire we entertain not to recur to new impositions, until we have first exhausted all the resources which order and economy can possibly supply," \&c., \&c.

These desires of the people of England, which come far short of the voluntary concessions of the king of France, are moderate indeed. They only contend that we should interweave some economy with the taxes with which we have chosen to begin the war. They request, not that you should rely upon economy exclusively, but that you should give it rank and precedence, in the order of the ways and means of this single session.

But if it were possible that the desires of our constituents, desires which are at once so natural and so very much tempered and subdued, should have no weight with an House of Commons which has its eye elsewhere, I would turn my eyes to the very quarter to which theirs are directed. I would reason this matter with the House on the mere policy of the question; and I would undertake to prove that an early dereliction of abuse is the direct interest of government,—of government taken abstractedly from its duties, and considered merely as a system intending its own conservation.

If there is any one eminent criterion which above all the rest distinguishes a wise government from an administration weak and improvident, it is this: "well to know the best time and manner of yielding what it is impossible to keep." There have been, Sir, and there are, many who choose to chicane with their situation rather than be instructed by it. Those gentlemen argue against every desire of reformation upon the principles of a criminal prosecution. It is enough for them to justify their adherence to a pernicious system, that it is not of their contrivance,—that it is an inheritance of absurdity, derived to them from their ancestors,—that they can make out a long and unbroken pedigree of mismanagers that have gone before them. They are proud of the antiquity of their house; and they defend their errors as if they were defending their inheritance, afraid of derogating from their nobility, and carefully avoiding a sort of blot in their scutcheon, which they think would degrade them forever.

It was thus that the unfortunate Charles the First defended himself on the practice of the Stuart who went before him, and of all the Tudors. His partisans might have gone to the Plantagenets. They might have found bad examples enough, both abroad and at home, that could have shown an ancient and illustrious descent. But there is a time when men will not suffer bad things because their ancestors have suffered worse. There is a time when the hoary head of inveterate abuse will neither draw reverence nor obtain protection. If the noble lord in the blue ribbon pleads, "Not guilty," to the charges brought against the present system of public economy, it is not possible to give a fair verdict by which he will not stand acquitted. But pleading is not our present business. His plea or his traverse may be allowed as an answer to a charge, when a charge is made. But if he puts himself in the way to obstruct reformation, then the faults of his office instantly become his own. Instead of a public officer in an abusive department, whose province is an object to be regulated, he becomes a criminal who is to be punished. I do most seriously put it to administration to consider the wisdom of a timely reform. Early reformations are amicable arrangements with a friend in power; late reformations are terms imposed upon a conquered enemy: early reformations are made in cool blood; late reformations are made under a state of inflammation. In that state of things the people behold in government nothing that is respectable. They see the abuse, and they will see nothing else. They fall into the temper of a furious populace provoked at the disorder of a house of ill-fame; they never attempt to correct or regulate; they go to work by the shortest way: they abate the nuisance, they pull down the house.

This is my opinion with regard to the true interest of government. But as it is the interest of government that reformation should be early, it is the interest of the people that it should be temperate. It is their interest, because a temperate reform is permanent, and because it has a principle of growth. Whenever we improve, it is right to leave room for a further improvement. It is right to consider, to look about us, to examine the effect of what we have done. Then we can proceed with confidence, because we can proceed with intelligence. Whereas in hot reformations, in what men more zealous than considerate call making clear work, the whole is generally so crude, so harsh, so indigested, mixed with so much imprudence and so much injustice, so contrary to the whole course of human nature and human institutions, that the very people who are most eager for it are among the first to grow disgusted at what they have done. Then some part of the abdicated grievance is recalled from its exile in order to become a corrective of the correction. Then the abuse assumes all the credit and popularity of a reform. The very idea of purity and disinterestedness in politics falls into disrepute, and is considered as a vision of hot and inexperienced men; and thus disorders become incurable, not by the virulence of their own quality, but by the unapt and violent nature of the remedies. A great part, therefore, of my idea of reform is meant to operate gradually: some benefits will come at a nearer, some at a more remote period. We must no more make haste to be rich by parsimony than by intemperate acquisition.

In my opinion, it is our duty, when we have the desires of the people before us, to pursue them, not in the spirit of literal obedience, which may militate with their very principle,—much less to treat them with a peevish and contentious litigation, as if we were adverse parties in a suit. It would, Sir, be most dishonorable for a faithful representative of the Commons to take advantage of any inartificial expression of the people's wishes, in order to frustrate their attainment of what they have an undoubted right to expect. We are under infinite obligations to our constituents, who have raised us to so distinguished a trust, and have imparted such a degree of sanctity to common characters. We ought to walk before them with purity, plainness, and integrity of heart,—with filial love, and not with slavish fear, which is always a low and tricking thing. For my own part, in what I have meditated upon that subject, I cannot, indeed, take upon me to say I have the honor to follow the sense of the people. The truth is, I met it on the way, while I was pursuing their interest according to my own ideas. I am happy beyond expression to find that my intentions have so far coincided with theirs, that I have not had, cause to be in the least scrupulous to sign their petition, conceiving it to express my own opinions, as nearly as general terms can express the object of particular arrangements.

I am therefore satisfied to act as a fair mediator between government and the people, endeavoring to form a plan which should have both an early and a temperate operation. I mean, that it should be substantial, that it should be systematic, that it should rather strike at the first cause of prodigality and corrupt influence than attempt to follow them in all their effects.

It was to fulfil the first of these objects (the proposal of something substantial) that I found myself obliged, at the outset, to reject a plan proposed by an honorable and attentive member of Parliament,
%[33] 
\footnote{Thomas Gilbert, Esq., member for Lichfield.}
with very good intentions on his part, about a year or two ago. Sir, the plan I speak of was the tax of twenty-five per cent moved upon places and pensions during the continuance of the American war. Nothing, Sir, could have met my ideas more than such a tax, if it was considered as a practical satire on that war, and as a penalty upon those who led us into it; but in any other view it appeared to me very liable to objections. I considered the scheme as neither substantial, nor permanent, nor systematical, nor likely to be a corrective of evil influence. I have always thought employments a very proper subject of regulation, but a very ill-chosen subject for a tax. An equal tax upon property is reasonable; because the object is of the same quality throughout. The species is the same; it differs only in its quantity. But a tax upon salaries is totally of a different nature; there can be no equality, and consequently no justice, in taxing them by the hundred in the gross.

We have, Sir, on our establishment several offices which perform real service: we have also places that provide large rewards for no service at all. We have stations which are made for the public decorum, made for preserving the grace and majesty of a great people: we have likewise expensive formalities, which tend rather to the disgrace than the ornament of the state and the court. This, Sir, is the real condition of our establishments. To fall with the same severity on objects so perfectly dissimilar is the very reverse of a reformation,—I mean a reformation framed, as all serious things ought to be, in number, weight, and measure.—Suppose, for instance, that two men receive a salary of 800l. a year each. In the office of one there is nothing at all to be done; in the other, the occupier is oppressed by its duties. Strike off twenty-five per cent from these two offices, you take from one man 200l. which in justice he ought to have, and you give in effect to the other 600l. which he ought not to receive. The public robs the former, and the latter robs the public; and this mode of mutual robbery is the only way in which the office and the public can make up their accounts.

But the balance, in settling the account of this double injustice, is much against the state. The result is short. You purchase a saving of two hundred pounds by a profusion of six. Besides, Sir, whilst you leave a supply of unsecured money behind, wholly at the discretion of ministers, they make up the tax to such places as they wish to favor, or in such new places as they may choose to create. Thus the civil list becomes oppressed with debt; and the public is obliged to repay, and to repay with an heavy interest, what it has taken by an injudicious tax. Such has been the effect of the taxes hitherto laid on pensions and employments, and it is no encouragement to recur again to the same expedient.

In effect, such a scheme is not calculated to produce, but to prevent reformation. It holds out a shadow of present gain to a greedy and necessitous public, to divert their attention from those abuses which in reality are the great causes of their wants. It is a composition to stay inquiry; it is a fine paid by mismanagement for the renewal of its lease; what is worse, it is a fine paid by industry and merit for an indemnity to the idle and the worthless. But I shall say no more upon this topic, because (whatever may be given out to the contrary) I know that the noble lord in the blue ribbon perfectly agrees with me in these sentiments.

After all that I have said on this subject, I am so sensible that it is our duty to try everything which may contribute to the relief of the nation, that I do not attempt wholly to reprobate the idea even of a tax. Whenever, Sir, the incumbrance of useless office (which lies no less a dead weight upon the service of the state than upon its revenues) shall be removed,—when the remaining offices shall be classed according to the just proportion of their rewards and services, so as to admit the application of an equal rule to their taxation,—when the discretionary power over the civil list cash shall be so regulated that a minister shall no longer have the means of repaying with a private what is taken by a public hand,—if, after all these preliminary regulations, it should be thought that a tax on places is an object worthy of the public attention, I shall be very ready to lend my hand to a reduction of their emoluments.

Having thus, Sir, not so much absolutely rejected as postponed the plan of a taxation of office, my next business was to find something which might be really substantial and effectual. I am quite clear, that, if we do not go to the very origin and first ruling cause of grievances, we do nothing. What does it signify to turn abuses out of one door, if we are to let them in at another? What does it signify to promote economy upon a measure, and to suffer it to be subverted in the principle? Our ministers are far from being wholly to blame for the present ill order which prevails. Whilst institutions directly repugnant to good management are suffered to remain, no effectual or lasting reform can be introduced.

I therefore thought it necessary, as soon as I conceived thoughts of submitting to you some plan of reform, to take a comprehensive view of the state of this country,—to make a sort of survey of its jurisdictions, its estates, and its establishments. Something in every one of them seemed to me to stand in the way of all economy in their administration, and prevented every possibility of methodizing the system. But being, as I ought to be, doubtful of myself, I was resolved not to proceed in an arbitrary manner in any particular which tended to change the settled state of things, or in any degree to affect the fortune or situation, the interest or the importance, of any individual. By an arbitrary proceeding I mean one conducted by the private opinions, tastes, or feelings of the man who attempts to regulate. These private measures are not standards of the exchequer, nor balances of the sanctuary. General principles cannot be debauched or corrupted by interest or caprice; and by those principles I was resolved to work.

Sir, before I proceed further, I will lay these principles fairly before you, that afterwards you may be in a condition to judge whether every object of regulation, as I propose it, comes fairly under its rule. This will exceedingly shorten all discussion between us, if we are perfectly in earnest in establishing a system of good management. I therefore lay down to myself seven fundamental rules: they might, indeed, be reduced to two or three simple maxims; but they would be too general, and their application to the several heads of the business before us would not be so distinct and visible. I conceive, then,

First, That all jurisdictions which furnish more matter of expense, more temptation to oppression, or more means and instruments of corrupt influence, than advantage to justice or political administration, ought to be abolished.

Secondly, That all public estates which are more subservient to the purposes of vexing, overawing, and influencing those who hold under them, and to the expense of perception and management, than of benefit to the revenue, ought, upon every principle both of revenue and of freedom, to be disposed of.

Thirdly, That all offices which bring more charge than proportional advantage to the state, that all offices which may be engrafted on others, uniting and simplifying their duties, ought, in the first case, to be taken away, and, in the second, to be consolidated.

Fourthly, That all such offices ought to be abolished as obstruct the prospect of the general superintendent of finance, which destroy his superintendency, which disable him from foreseeing and providing for charges as they may occur, from preventing expense in its origin, checking it in its progress, or securing its application to its proper purposes. A minister, under whom expenses can be made without his knowledge, can never say what it is that he can spend, or what it is that he can save.

Fifthly, That it is proper to establish an invariable order in all payments, which will prevent partiality, which will give preference to services, not according to the importunity of the demandant, but the rank and order of their utility or their justice.

Sixthly, That it is right to reduce every establishment and every part of an establishment (as nearly as possible) to certainty, the life of all order and good management.

Seventhly, That all subordinate treasuries, as the nurseries of mismanagement, and as naturally drawing to themselves as much money as they can, keeping it as long as they can, and accounting for it as late as they can, ought to be dissolved. They have a tendency to perplex and distract the public accounts, and to excite a suspicion of government even beyond the extent of their abuse.

Under the authority and with the guidance of those principles I proceed,—wishing that nothing in any establishment may be changed, where I am not able to make a strong, direct, and solid application of those principles, or of some one of them. An economical constitution is a necessary basis for an economical administration.

First, with regard to the sovereign jurisdictions, I must observe, Sir, that whoever takes a view of this kingdom in a cursory manner will imagine that he beholds a solid, compacted, uniform system of monarchy, in which all inferior jurisdictions are but as rays diverging from one centre. But on examining it more nearly, you find much eccentricity and confusion. It is not a monarchy in strictness. But, as in the Saxon times this country was an heptarchy, it is now a strange sort of pentarchy. It is divided into five several distinct principalities, besides the supreme. There is, indeed, this difference from the Saxon times,—that, as in the itinerant exhibitions of the stage, for want of a complete company, they are obliged to throw a variety of parts on their chief performer, so our sovereign condescends himself to act not only the principal, but all the subordinate parts in the play. He condescends to dissipate the royal character, and to trifle with those light, subordinate, lacquered sceptres in those hands that sustain the ball representing the world, or which wield the trident that commands the ocean. Cross a brook, and you lose the King of England; but you have some comfort in coming again under his Majesty, though "shorn of his beams," and no more than Prince of Wales. Go to the north, and you find him dwindled to a Duke of Lancaster; turn to the west of that north, and he pops upon you in the humble character of Earl of Chester. Travel a few miles on, the Earl of Chester disappears, and the king surprises you again as Count Palatine of Lancaster. If you travel beyond Mount Edgecombe, you find him ones more in his incognito, and he is Duke of Cornwall. So that, quite fatigued and satiated with this dull variety, you are infinitely refreshed when you return to the sphere of his proper splendor, and behold your amiable sovereign in his true, simple, undisguised, native character of Majesty.

In every one of these five principalities, duchies, palatinates, there is a regular establishment of considerable expense and most domineering influence. As his Majesty submits to appear in this state of subordination to himself, his loyal peers and faithful commons attend his royal transformations, and are not so nice as to refuse to nibble at those crumbs of emoluments which console their petty metamorphoses. Thus every one of those principalities has the apparatus of a kingdom for the jurisdiction over a few private estates, and the formality and charge of the Exchequer of Great Britain for collecting the rents of a country squire. Cornwall is the best of them; but when you compare the charge with the receipt, you will find that it furnishes no exception to the general rule. The Duchy and County Palatine of Lancaster do not yield, as I have reason to believe, on an average of twenty years, four thousand pounds a year clear to the crown. As to Wales, and the County Palatine of Chester, I have my doubts whether their productive exchequer yields any returns at all. Yet one may say, that this revenue is more faithfully applied to its purposes than any of the rest; as it exists for the sole purpose of multiplying offices and extending influence.

An attempt was lately made to improve this branch of local influence, and to transfer it to the fund of general corruption. I have on the seat behind me the constitution of Mr. John Probert, a knight-errant dubbed by the noble lord in the blue ribbon, and sent to search for revenues and adventures upon the mountains of Wales. The commission is remarkable, and the event not less so. The commission sets forth, that, "upon a report of the deputy-auditor" (for there is a deputy-auditor) "of the Principality of Wales, it appeared that his Majesty's land revenues in the said principality are greatly diminished";—and "that upon a report of the surveyor-general of his Majesty's land revenues, upon a memorial of the auditor of his Majesty's revenues, within the said principality, that his mines and forests have produced very little profit either to the public revenue or to individuals";—and therefore they appoint Mr. Probert, with a pension of three hundred pounds a year from the said principality, to try whether he can make anything more of that very little which is stated to be so greatly diminished. "A beggarly account of empty boxes." And yet, Sir, you will remark, that this diminution from littleness (which serves only to prove the infinite divisibility of matter) was not for want of the tender and officious care (as we see) of surveyors general and surveyors particular, of auditors and deputy-auditors,—not for want of memorials, and remonstrances, and reports, and commissions, and constitutions, and inquisitions, and pensions.

Probert, thus armed, and accoutred,—and paid,—proceeded on his adventure; but he was no sooner arrived on the confines of Wales than all Wales was in arms to meet him. That nation is brave and full of spirit. Since the invasion of King Edward, and the massacre of the bards, there never was such a tumult and alarm and uproar through the region of Prestatyn. Snowdon shook to its base; Cader-Idris was loosened from its foundations. The fury of litigious war blew her horn on the mountains. The rocks poured down their goatherds, and the deep caverns vomited out their miners. Everything above ground and everything under ground was in arms.

In short, Sir, to alight from my Welsh Pegasus, and to come to level ground, the Preux Chevalier Probert went to look for revenue, like his masters upon other occasions, and, like his masters, he found rebellion. But we were grown cautious by experience. A civil war of paper might end in a more serious war; for now remonstrance met remonstrance, and memorial was opposed to memorial. The wise Britons thought it more reasonable that the poor, wasted, decrepit revenue of the principality should die a natural than a violent death. In truth, Sir, the attempt was no less an affront upon the understanding of that respectable people than it was an attack on their property. They chose rather that their ancient, moss-grown castles should moulder into decay, under the silent touches of time, and the slow formality of an oblivious and drowsy exchequer, than that they should be battered down all at once by the lively efforts of a pensioned engineer. As it is the fortune of the noble lord to whom the auspices of this campaign belonged frequently to provoke resistance, so it is his rule and nature to yield to that resistance in all cases whatsoever. He was true to himself on this occasion. He submitted with spirit to the spirited remonstrances of the Welsh. Mr. Probert gave up his adventure, and keeps his pension; and so ends "the famous history of the revenue adventures of the bold Baron North and the good Knight Probert upon the mountains of Venodotia."

In such a state is the exchequer of Wales at present, that, upon the report of the Treasury itself, its little revenue is greatly diminished; and we see, by the whole of this strange transaction, that an attempt to improve it produces resistance, the resistance produces submission, and the whole ends in pension.
%[34]
\footnote{ Here Lord North shook his head, and told those who sat near him that Mr. Probert's pension was to depend on his success. It may be so. Mr. Probert's pension was, however, no essential part of the question; nor did Mr. B. care whether he still possessed it or not. His point was, to show the ridicule of attempting an improvement of the Welsh revenue under its present establishment.}


It is nearly the same with the revenues of the Duchy of Lancaster. To do nothing with them is extinction; to improve them is oppression. Indeed, the whole of the estates which support these minor principalities is made up, not of revenues, and rents, and profitable fines, but of claims, of pretensions, of vexations, of litigations. They are exchequers of unfrequent receipt and constant charge: a system of finances not fit for an economist who would be rich, not fit for a prince who would govern his subjects with equity and justice.

It is not only between prince and subject that these mock jurisdictions and mimic revenues produce great mischief. They excite among the people a spirit of informing and delating, a spirit of supplanting and undermining one another: so that many, in such circumstances, conceive it advantageous to them rather to continue subject to vexation themselves than to give up the means and chance of vexing others. It is exceedingly common for men to contract their love to their country into an attachment to its petty subdivisions; and they sometimes even cling to their provincial abuses, as if they were franchises and local privileges. Accordingly, in places where there is much of this kind of estate, persons will be always found who would rather trust to their talents in recommending themselves to power for the renewal of their interests, than to incumber their purses, though never so lightly, in order to transmit independence to their posterity. It is a great mistake, that the desire of securing property is universal among mankind. Gaming is a principle inherent in human nature. It belongs to us all. I would therefore break those tables; I would furnish no evil occupation for that spirit. I would make every man look everywhere, except to the intrigue of a court, for the improvement of his circumstances or the security of his fortune. I have in my eye a very strong case in the Duchy of Lancaster (which lately occupied Westminster Hall and the House of Lords) as my voucher for many of these reflections.
%[35]
\footnote{ Case of Richard Lee, Esq., appellant, against George Venables Lord Vernon, respondent, in the year 1775.}


For what plausible reason are these principalities suffered to exist? When a government is rendered complex, (which in itself is no desirable thing,) it ought to be for some political end which cannot be answered otherwise. Subdivisions in government are only admissible in favor of the dignity of inferior princes and high nobility, or for the support of an aristocratic confederacy under some head, or for the conservation of the franchises of the people in some privileged province. For the two former of these ends, such are the subdivisions in favor of the electoral and other princes in the Empire; for the latter of these purposes are the jurisdictions of the Imperial cities and the Hanse towns. For the latter of these ends are also the countries of the States (Pays d'États) and certain cities and orders in France. These are all regulations with an object, and some of them with a very good object. But how are the principles of any of these subdivisions applicable in the case before us?

Do they answer any purpose to the king? The Principality of Wales was given by patent to Edward the Black Prince on the ground on which it has since stood. Lord Coke sagaciously observes upon it, "That in the charter of creating the Black Prince Edward Prince of Wales there is a great mystery: for less than an estate of inheritance so great a prince could not have, and an absolute estate of inheritance in so great a principality as Wales (this principality being so dear to him) he should not have; and therefore it was made sibi et heredibus suis regibus Angliæ, that by his decease, or attaining to the crown, it might be extinguished in the crown."

For the sake of this foolish mystery, of what a great prince could not have less and should not have so much, of a principality which was too dear to be given and too great to be kept,—and for no other cause that ever I could find,—this form and shadow of a principality, without any substance, has been maintained. That you may judge in this instance (and it serves for the rest) of the difference between a great and a little economy, you will please to recollect, Sir, that Wales may be about the tenth part of England in size and population, and certainly not a hundredth part in opulence. Twelve judges perform the whole of the business, both of the stationary and the itinerant justice of this kingdom; but for Wales there are eight judges. There is in Wales an exchequer, as well as in all the duchies, according to the very best and most authentic absurdity of form. There are in all of them a hundred more difficult trifles and laborious fooleries, which serve no other purpose than to keep alive corrupt hope and servile dependence.

These principalities are so far from contributing to the ease of the king, to his wealth, or his dignity, that they render both his supreme and his subordinate authority perfectly ridiculous. It was but the other day, that that pert, factious fellow, the Duke of Lancaster, presumed to fly in the face of his liege lord, our gracious sovereign, and, associating with a parcel of lawyers as factious as himself, to the destruction of all law and order, and in committees leading directly to rebellion, presumed to go to law with the king. The object is neither your business nor mine. Which of the parties got the better I really forget. I think it was (as it ought to be) the king. The material point is, that the suit cost about fifteen thousand pounds. But as the Duke of Lancaster is but a sort of Duke Humphrey, and not worth a groat, our sovereign was obliged to pay the costs of both. Indeed, this art of converting a great monarch into a little prince, this royal masquerading, is a very dangerous and expensive amusement, and one of the king's menus plaisirs, which ought to be reformed. This duchy, which is not worth four thousand pounds a year at best to revenue, is worth forty or fifty thousand to influence.

The Duchy of Lancaster and the County Palatine of Lancaster answered, I admit, some purpose in their original creation. They tended to make a subject imitate a prince. When Henry the Fourth from that stair ascended the throne, high-minded as he was, he was not willing to kick away the ladder. To prevent that principality from being extinguished in the crown, he severed it by act of Parliament. He had a motive, such as it was: he thought his title to the crown unsound, and his possession insecure. He therefore managed a retreat in his duchy, which Lord Coke calls (I do not know why) "par multis regnis." He flattered himself that it was practicable to make a projecting point half way down, to break his fall from the precipice of royalty; as if it were possible for one who had lost a kingdom to keep anything else. However, it is evident that he thought so. When Henry the Fifth united, by act of Parliament, the estates of his mother to the duchy, he had the same predilection with his father to the root of his family honors, and the same policy in enlarging the sphere of a possible retreat from the slippery royalty of the two great crowns he held. All this was changed by Edward the Fourth. He had no such family partialities, and his policy was the reverse of that of Henry the Fourth and Henry the Fifth. He accordingly again united the Duchy of Lancaster to the crown. But when Henry the Seventh, who chose to consider himself as of the House of Lancaster, came to the throne, he brought with him the old pretensions and the old politics of that house. A new act of Parliament, a second time, dissevered the Duchy of Lancaster from the crown; and in that line tilings continued until the subversion of the monarchy, when principalities and powers fell along with the throne. The Duchy of Lancaster must have been extinguished, if Cromwell, who began to form ideas of aggrandizing his house and raising the several branches of it, had not caused the duchy to be again separated from the commonwealth, by an act of the Parliament of those times.

What partiality, what objects of the politics of the House of Lancaster, or of Cromwell, has his present Majesty, or his Majesty's family? What power have they within any of these principalities, which they have not within their kingdom? In what manner is the dignity of the nobility concerned in these principalities? What rights have the subject there, which they have not at least equally in every other part of the nation? These distinctions exist for no good end to the king, to the nobility, or to the people. They ought not to exist at all. If the crown (contrary to its nature, but most conformably to the whole tenor of the advice that has been lately given) should so far forget its dignity as to contend that these jurisdictions and revenues are estates of private property, I am rather for acting as if that groundless claim were of some weight than for giving up that essential part of the reform. I would value the clear income, and give a clear annuity to the crown, taken on the medium produce for twenty years.

If the crown has any favorite name or title, if the subject has any matter of local accommodation within any of these jurisdictions, it is meant to preserve them,—and to improve them, if any improvement can be suggested. As to the crown reversions or titles upon the property of the people there, it is proposed to convert them from a snare to their independence into a relief from their burdens. I propose, therefore, to unite all the five principalities to the crown, and to its ordinary jurisdiction,—to abolish all those offices that produce an useless and chargeable separation from the body of the people,—to compensate those who do not hold their offices (if any such there are) at the pleasure of the crown,—to extinguish vexatious titles by an act of short limitation,—to sell those unprofitable estates which support useless jurisdictions,—and to turn the tenant-right into a fee, on such moderate terms as will be better for the state than its present right, and which it is impossible for any rational tenant to refuse.

As to the duchies, their judicial economy may be provided for without charge. They have only to fall of course into the common county administration. A commission more or less, made or omitted, settles the matter fully. As to Wales, it has been proposed to add a judge to the several courts of Westminster Hall; and it has been considered as an improvement in itself. For my part, I cannot pretend to speak upon it with clearness or with decision; but certainly this arrangement would be more than sufficient for Wales. My original thought was, to suppress five of the eight judges; and to leave the chief-justice of Chester, with the two senior judges; and, to facilitate the business, to throw the twelve counties into six districts, holding the sessions alternately in the counties of which each district shall be composed. But on this I shall be more clear, when I come to the particular bill.

Sir, the House will now see, whether, in praying for judgment against the minor principalities, I do not act in conformity to the laws that I had laid to myself: of getting rid of every jurisdiction more subservient to oppression and expense than to any end of justice or honest policy; of abolishing offices more expensive than useful; of combining duties improperly separated; of changing revenues more vexatious than productive into ready money; of suppressing offices which stand in the way of economy; and of cutting off lurking subordinate treasuries. Dispute the rules, controvert the application, or give your hands to this salutary measure.

Most of the same rules will be found applicable to my second object,—the landed estate of the crown. A landed estate is certainly the very worst which the crown can possess. All minute and dispersed possessions, possessions that are often of indeterminate value, and which require a continued personal attendance, are of a nature more proper for private management than public administration. They are fitter for the care of a frugal land-steward than of an office in the state. Whatever they may possibly have been in other times or in other countries, they are not of magnitude enough with us to occupy a public department, nor to provide for a public object. They are already given up to Parliament, and the gift is not of great value. Common prudence dictates, even in the management of private affairs, that all dispersed and chargeable estates should be sacrificed to the relief of estates more compact and better circumstanced.

If it be objected, that these lands at present would sell at a low market, this is answered by showing that money is at a high price. The one balances the other. Lands sell at the current rate; and nothing can sell for more. But be the price what it may, a great object is always answered, whenever any property is transferred from hands that are not fit for that property to those that are. The buyer and seller must mutually profit by such a bargain; and, what rarely happens in matters of revenue, the relief of the subject will go hand in hand with the profit of the Exchequer.

As to the forest lands, in which the crown has (where they are not granted or prescriptively held) the dominion of the soil, and the vert and venison, that is to say, the timber and the game, and in which the people have a variety of rights, in common of herbage, and other commons, according to the usage of the several forests,—I propose to have those rights of the crown valued as manorial rights are valued on an inclosure, and a defined portion of land to be given for them, which land is to be sold for the public benefit.

As to the timber, I propose a survey of the whole. What is useless for the naval purposes of the kingdom I would condemn and dispose of for the security of what may be useful, and to inclose such other parts as may be most fit to furnish a perpetual supply,—wholly extinguishing, for a very obvious reason, all right of venison in those parts.

The forest rights which extend over the lands and possessions of others, being of no profit to the crown, and a grievance, as far as it goes, to the subject,—these I propose to extinguish without charge to the proprietors. The several commons are to be allotted and compensated for, upon ideas which I shall hereafter explain. They are nearly the same with the principles upon which you have acted in private inclosures. I shall never quit precedents, where I find them applicable. For those regulations and compensations, and for every other part of the detail, you will be so indulgent as to give me credit for the present.

The revenue to be obtained from the sale of the forest lands and rights will not be so considerable, I believe, as many people have imagined; and I conceive it would be unwise to screw it up to the utmost, or even to suffer bidders to enhance, according to their eagerness, the purchase of objects wherein the expense of that purchase may weaken the capital to be employed in their cultivation. This, I am well aware, might give room for partiality in the disposal. In my opinion it would be the lesser evil of the two. But I really conceive that a rule of fair preference might be established, which would take away all sort of unjust and corrupt partiality. The principal revenue which I propose to draw from these uncultivated wastes is to spring from the improvement and population of the kingdom,—which never can happen without producing an improvement more advantageous to the revenues of the crown than the rents of the best landed estate which it can hold. I believe, Sir, it will hardly be necessary for me to add, that in this sale I naturally except all the houses, gardens, and parks belonging to the crown, and such one forest as shall be chosen by his Majesty as best accommodated to his pleasures.

By means of this part of the reform will fall the expensive office of surveyor-general, with all the influence that attends it. By this will fall two chief-justices in Eyre, with all their train of dependants. You need be under no apprehension, Sir, that your office is to be touched in its emoluments. They are yours by law; and they are but a moderate part of the compensation which is given to you for the ability with which you execute an office of quite another sort of importance: it is far from overpaying your diligence, or more than sufficient for sustaining the high rank you stand in as the first gentleman of England. As to the duties of your chief-justiceship, they are very different from those for which you have received the office. Your dignity is too high for a jurisdiction over wild beasts, and your learning and talents too valuable to be wasted as chief-justice of a desert. I cannot reconcile it to myself, that you, Sir, should be stuck up as a useless piece of antiquity.

I have now disposed of the unprofitable landed estates of the crown, and thrown them into the mass of private property; by which they will come, through the course of circulation, and through the political secretions of the state, into our better understood and better ordered revenues.

I come next to the great supreme body of the civil government itself. I approach it with that awe and reverence with which a young physician approaches to the cure of the disorders of his parent. Disorders, Sir, and infirmities, there are,—such disorders, that all attempts towards method, prudence, and frugality will be perfectly vain, whilst a system of confusion remains, which is not only alien, but adverse to all economy; a system which is not only prodigal in its very essence, but causes everything else which belongs to it to be prodigally conducted.

It is impossible, Sir, for any person to be an economist, where no order in payments is established; it is impossible for a man to be an economist, who is not able to take a comparative view of his means and of his expenses for the year which lies before him; it is impossible for a man to be an economist, under whom various officers in their several departments may spend—even just what they please,—and often with an emulation of expense, as contributing to the importance, if not profit of their several departments. Thus much is certain: that neither the present nor any other First Lord of the Treasury has been ever able to take a survey, or to make even a tolerable guess, of the expenses of government for any one year, so as to enable him with the least degree of certainty, or even probability, to bring his affairs within compass. Whatever scheme may be formed upon them must be made on a calculation of chances. As things are circumstanced, the First Lord of the Treasury cannot make an estimate. I am sure I serve the king, and I am sure I assist administration, by putting economy at least in their power. We must class services; we must (as far as their nature admits) appropriate funds; or everything, however reformed, will fall again into the old confusion.

Coming upon this ground of the civil list, the first thing in dignity and charge that attracts our notice is the royal household. This establishment, in my opinion, is exceedingly abusive in its constitution. It is formed upon manners and customs that have long since expired. In the first place, it is formed, in many respects, upon feudal principles. In the feudal times, it was not uncommon, even among subjects, for the lowest offices to be held by considerable persons,—persons as unfit by their incapacity as improper from their rank to occupy such employments. They were held by patent, sometimes for life, and sometimes by inheritance. If my memory does not deceive me, a person of no slight consideration held the office of patent hereditary cook to an Earl of Warwick: the Earl of Warwick's soups, I fear, were not the better for the dignity of his kitchen. I think it was an Earl of Gloucester who officiated as steward of the household to the Archbishops of Canterbury. Instances of the same kind may in some degree be found in the Northumberland house-book, and other family records. There was some reason in ancient necessities for these ancient customs. Protection was wanted; and the domestic tie, though not the highest, was the closest.

The king's household has not only several strong traces of this feudality, but it is formed also upon the principles of a body corporate: it has its own magistrates, courts, and by-laws. This might be necessary in the ancient times, in order to have a government within itself, capable of regulating the vast and often unruly multitude which composed and attended it. This was the origin of the ancient court called the Green Cloth,—composed of the marshal, treasurer, and other great officers of the household, with certain clerks. The rich subjects of the kingdom, who had formerly the same establishments, (only on a reduced scale,) have since altered their economy, and turned the course of their expense from the maintenance of vast establishments within their walls to the employment of a great variety of independent trades abroad. Their influence is lessened; but a mode of accommodation and a style of splendor suited to the manners of the times has been increased. Royalty itself has insensibly followed, and the royal household has been carried away by the resistless tide of manners, but with this very material difference: private men have got rid of the establishments along with the reasons of them; whereas the royal household has lost all that was stately and venerable in the antique manners, without retrenching anything of the cumbrous charge of a Gothic establishment. It is shrunk into the polished littleness of modern elegance and personal accommodation; it has evaporated from the gross concrete into an essence and rectified spirit of expense, where you have tuns of ancient pomp in a vial of modern luxury.

But when the reason of old establishments is gone, it is absurd to preserve nothing but the burden of them. This is superstitiously to embalm a carcass not worth an ounce of the gums that are used to preserve it. It is to burn precious oils in the tomb; it is to offer meat and drink to the dead: not so much an honor to the deceased as a disgrace to the survivors. Our palaces are vast inhospitable halls. There the bleak winds, there "Boreas, and Eurus, and Caurus, and Argestes loud," howling through the vacant lobbies, and clattering the doors of deserted guardrooms, appall the imagination, and conjure up the grim spectres of departed tyrants,—the Saxon, the Norman, and the Dane,—the stern Edwards and fierce Henrys,—who stalk from desolation to desolation, through the dreary vacuity and melancholy succession of chill and comfortless chambers. When this tumult subsides, a dead and still more frightful silence would reign in this desert, if every now and then the tacking of hammers did not announce that those constant attendants upon all courts in all ages, jobs, were still alive,—for whose sake alone it is that any trace of ancient grandeur is suffered to remain. These palaces are a true emblem of some governments: the inhabitants are decayed, but the governors and magistrates still flourish. They put me in mind of Old Sarum, where the representatives, more in number than the constituents, only serve to inform us that this was once a place of trade, and sounding with "the busy hum of men," though now you can only trace the streets by the color of the corn, and its sole manufacture is in members of Parliament.

These old establishments were formed also on a third principle, still more adverse to the living economy of the age. They were formed, Sir, on the principle of purveyance and receipt in kind. In former days, when the household was vast, and the supply scanty and precarious, the royal purveyors, sallying forth from under the Gothic portcullis to purchase provision with power and prerogative instead of money, brought home the plunder of an hundred markets, and all that could be seized from a flying and hiding country, and deposited their spoil in an hundred caverns, with each its keeper. There, every commodity, received in its rawest condition, went through all the process which fitted it for use. This inconvenient receipt produced an economy suited only to itself. It multiplied offices beyond all measure,—buttery, pantry, and all that rabble of places, which, though profitable to the holders, and expensive to the state, are almost too mean to mention.

All this might be, and I believe was, necessary at first; for it is remarkable, that purveyance, after its regulation had been the subject of a long line of statutes, (not fewer, I think, than twenty-six,) was wholly taken away by the 12th of Charles the Second; yet in the next year of the same reign it was found necessary to revive it by a special act of Parliament, for the sake of the king's journeys. This, Sir, is curious, and what would hardly he expected in so reduced a court as that of Charles the Second and in so improved a country as England might then be thought. But so it was. In our time, one well-filled and well-covered stage-coach requires more accommodation than a royal progress, and every district, at an hour's warning, can supply an army.

I do not say, Sir, that all these establishments, whose principle is gone, have been systematically kept up for influence solely: neglect had its share. But this I am sure of: that a consideration of influence has hindered any one from attempting to pull them down. For the purposes of influence, and for those purposes only, are retained half at least of the household establishments. No revenue, no, not a royal revenue, can exist under the accumulated charge of ancient establishment, modern luxury, and Parliamentary political corruption.

If, therefore, we aim at regulating this household, the question will be, whether we ought to economize by detail or by principle. The example we have had of the success of an attempt to economize by detail, and under establishments adverse to the attempt, may tend to decide this question.

At the beginning of his Majesty's reign, Lord Talbot came to the administration of a great department in the household. I believe no man ever entered into his Majesty's service, or into the service of any prince, with a more clear integrity, or with more zeal and affection for the interest of his master, and, I must add, with abilities for a still higher service. Economy was then announced as a maxim of the reign. This noble lord, therefore, made several attempts towards a reform. In the year 1777, when the king's civil list debts came last to be paid, he explained very fully the success of his undertaking. He told the House of Lords that he had attempted to reduce the charges of the king's tables and his kitchen. The thing, Sir, was not below him. He knew that there is nothing interesting in the concerns of men whom we love and honor, that is beneath our attention. "Love," says one of our old poets, "esteems no office mean,"—and with still more spirit, "Entire affection scorneth nicer hands." Frugality, Sir, is founded on the principle, that all riches have limits. A royal household, grown enormous, even in the meanest departments, may weaken and perhaps destroy all energy in the highest offices of the state. The gorging a royal kitchen may stint and famish the negotiations of a kingdom. Therefore the object was worthy of his, was worthy of any man's attention.

In consequence of this noble lord's resolution, (as he told the other House,) he reduced several tables, and put the persons entitled to them upon board wages, much to their own satisfaction. But, unluckily, subsequent duties requiring constant attendance, it was not possible to prevent their being fed where they were employed: and thus this first step towards economy doubled the expense.

There was another disaster far more doleful than this. I shall state it, as the cause of that misfortune lies at the bottom of almost all our prodigality. Lord Talbot attempted to reform the kitchen; but such, as he well observed, is the consequence of having duty done by one person whilst another enjoys the emoluments, that he found himself frustrated in all his designs. On that rock his whole adventure split, his whole scheme of economy was dashed to pieces. His department became more expensive than ever; the civil list debt accumulated. Why? It was truly from a cause which, though perfectly adequate to the effect, one would not have instantly guessed. It was because the turnspit in the king's kitchen was a member of Parliament!
%[36]
\footnote{ Vide Lord Talbot's speech in Almon's Parliamentary Register. Vol VII. p. 79, of the Proceedings of the Lords.}
 The king's domestic servants were all undone, his tradesmen remained unpaid and became bankrupt,—because the turnspit of the king's kitchen was a member of Parliament. His Majesty's slumbers were interrupted, his pillow was stuffed with thorns, and his peace of mind entirely broken,—because the king's turnspit was a member of Parliament. The judges were unpaid, the justice of the kingdom bent and gave way, the foreign ministers remained inactive and unprovided, the system of Europe was dissolved, the chain of our alliances was broken, all the wheels of government at home and abroad were stopped,—because the king's turnspit was a member of Parliament.

Such, Sir, was the situation of affairs, and such the cause of that situation, when his Majesty came a second time to Parliament to desire the payment of those debts which the employment of its members in various offices, visible and invisible, had occasioned. I believe that a like fate will attend every attempt at economy by detail, under similar, circumstances, and in every department. A complex, operose office of account and control is, in itself, and even if members of Parliament had nothing to do with it, the most prodigal of all things. The most audacious robberies or the most subtle frauds would never venture upon such a waste as an over-careful detailed guard against them will infallibly produce. In our establishments, we frequently see an office of account of an hundred pounds a year expense, and another office of an equal expense to control that office, and the whole upon a matter that is not worth twenty shillings.

To avoid, therefore, this minute care, which produces the consequences of the most extensive neglect, and to oblige members of Parliament to attend to public cares, and not to the servile offices of domestic management, I propose, Sir, to economize by principle: that is, I propose to put affairs into that train which experience points out as the most effectual, from the nature of things, and from the constitution of the human mind. In all dealings, where it is possible, the principles of radical economy prescribe three things: first, undertaking by the great; secondly, engaging with persons of skill in the subject-matter; thirdly, engaging with those who shall have an immediate and direct interest in the proper execution of the business.

To avoid frittering and crumbling down the attention by a blind, unsystematic observance of every trifle, it has ever been found the best way to do all things which are great in the total amount and minute in the component parts by a general contrast. The principles of trade have so pervaded every species of dealing, from the highest to the lowest objects, all transactions are got so much into system, that we may, at a moment's warning, and to a farthing value, be informed at what rate any service may be supplied. No dealing is exempt from the possibility of fraud. But by a contract on a matter certain you have this advantage: you are sure to know the utmost extent of the fraud to which you are subject. By a contract with a person in his own trade you are sure you shall not suffer by want of skill. By a short contract you are sure of making it the interest of the contractor to exert that skill for the satisfaction of his employers.

I mean to derogate nothing from the diligence or integrity of the present, or of any former board of Green Cloth. But what skill can members of Parliament obtain in that low kind of province? What pleasure can they have in the execution of that kind of duty? And if they should neglect it, how does it affect their interest, when we know that it is their vote in Parliament, and not their diligence in cookery or catering, that recommends them to their office, or keeps them in it?

I therefore propose that the king's tables (to whatever number of tables, or covers to each, he shall think proper to command) should be classed by the steward of the household, and should be contracted for, according to their rank, by the head or cover; that the estimate and circumstance of the contract should be carried to the Treasury to be approved; and that its faithful and satisfactory performance should be reported there previous to any payment; that there, and there only, should the payment be made. I propose that men should be contracted with only in their proper trade; and that no member of Parliament should be capable of such contract. By this plan, almost all the infinite offices under the lord steward may be spared,—to the extreme simplification, and to the far better execution, of every one of his functions. The king of Prussia is so served. He is a great and eminent (though, indeed, a very rare) instance of the possibility of uniting, in a mind of vigor and compass, an attention to minute objects with the largest views and the most complicated plans. His tables are served by contract, and by the head. Let me say, that no prince can be ashamed to imitate the king of Prussia, and particularly to learn in his school, when the problem is, "The best manner of reconciling the state of a court with the support of war." Other courts, I understand, have followed his with effect, and to their satisfaction.

The same clew of principle leads us through the labyrinth of the other departments. What, Sir, is there in the office of the great wardrobe (which has the care of the king's furniture) that may not be executed by the lord chamberlain himself? He has an honorable appointment; he has time sufficient to attend to the duty; and he has the vice-chamberlain to assist him. Why should not he deal also by contract for all things belonging to this office, and carry his estimates first, and his report of the execution in its proper time, for payment, directly to the Board of Treasury itself? By a simple operation, (containing in it a treble control,) the expenses of a department which for naked walls, or walls hung with cobwebs, has in a few years cost the crown 150,000l., may at length hope for regulation. But, Sir, the office and its business are at variance. As it stands, it serves, not to furnish the palace with its hangings, but the Parliament with its dependent members.

To what end, Sir, does the office of removing wardrobe serve at all? Why should a jewel office exist for the sole purpose of taxing the king's gifts of plate? Its object falls naturally within the chamberlain's province, and ought to be under his care and inspection without any fee. Why should an office of the robes exist, when that of groom, of the stole is a sinecure, and that this is a proper object of his department?

All these incumbrances, which are themselves nuisances, produce other incumbrances and other nuisances. For the payment of these useless establishments there are no less than three useless treasurers: two to hold a purse, and one to play with a stick. The treasurer of the household is a mere name. The cofferer and the treasurer of the chamber receive and pay great sums, which it is not at all necessary they should either receive or pay. All the proper officers, servants, and tradesmen may be enrolled in their several departments, and paid in proper classes and times with great simplicity and order, at the Exchequer, and by direction from the Treasury.

The Board of Works, which in the seven years preceding 1777 has cost towards 400,000l.,
%[37]
\footnote{ More exactly, 378,616l. 10 s. 1¾ d.}
 and (if I recollect rightly) has not cost less in proportion from the beginning of the reign, is under the very same description of all the other ill-contrived establishments, and calls for the very same reform. We are to seek for the visible signs of all this expense. For all this expense, we do not see a building of the size and importance of a pigeon-house. Buckingham House was reprised by a bargain with the public for one hundred thousand pounds; and the small house at Windsor has been, if I mistake not, undertaken since that account was brought before us. The good works of that Board of Works are as carefully concealed as other good works ought to be: they are perfectly invisible. But though it is the perfection of charity to be concealed, it is, Sir, the property and glory of magnificence to appear and stand forward to the eye.

That board, which ought to be a concern of builders and such like, and of none else, is turned into a junto of members of Parliament. That office, too, has a treasury and a paymaster of its own; and lest the arduous affairs of that important exchequer should be too fatiguing, that paymaster has a deputy to partake his profits and relieve his cares. I do not believe, that, either now or in former times, the chief managers of that board have made any profit of its abuse. It is, however, no good reason that an abusive establishment should subsist, because it is of as little private as of public advantage. But this establishment has the grand radical fault, the original sin, that pervades and perverts all our establishments: the apparatus is not fitted to the object, nor the workmen to the work. Expenses are incurred on the private opinion of an inferior establishment, without consulting the principal, who can alone determine the proportion which it ought to bear to the other establishments of the state, in the order of their relative importance.

I propose, therefore, along with the rest, to pull down this whole ill-contrived scaffolding, which obstructs, rather than forwards, our public works; to take away its treasury; to put the whole into the hands of a real builder, who shall not be a member of Parliament; and to oblige him, by a previous estimate and final payment, to appear twice at the Treasury before the public can be loaded. The king's gardens are to come under a similar regulation.

The Mint, though not a department of the household, has the same vices. It is a great expense to the nation, chiefly for the sake of members of Parliament. It has its officers of parade and dignity. It has its treasury, too. It is a sort of corporate body, and formerly was a body of great importance,—as much so, on the then scale of things, and the then order of business, as the Bank is at this day. It was the great centre of money transactions and remittances for our own and for other nations, until King Charles the First, among other arbitrary projects dictated by despotic necessity, made it withhold the money that lay there for remittance. That blow (and happily, too) the Mint never recovered. Now it is no bank, no remittance-shop. The Mint, Sir, is a manufacture, and it is nothing else; and it ought to be undertaken upon the principles of a manufacture,—that is, for the best and cheapest execution, by a contract upon proper securities and under proper regulations.

The artillery is a far greater object; it is a military concern; but having an affinity and kindred in its defects with the establishments I am now speaking of, I think it best to speak of it along with them. It is, I conceive, an establishment not well suited to its martial, though exceedingly well calculated for its Parliamentary purposes. Here there is a treasury, as in all the other inferior departments of government. Here the military is subordinate to the civil, and the naval confounded with the land service. The object, indeed, is much the same in both. But, when the detail is examined, it will be found that they had better be separated. For a reform of this office, I propose to restore things to what (all considerations taken together) is their natural order: to restore them to their just proportion, and to their just distribution. I propose, in this military concern, to render the civil subordinate to the military; and this will annihilate the greatest part of the expense, and all the influence belonging to the office. I propose to send the military branch to the army, and the naval to the Admiralty; and I intend to perfect and accomplish the whole detail (where it becomes too minute and complicated for legislature, and requires exact, official, military, and mechanical knowledge) by a commission of competent officers in both departments. I propose to execute by contract what by contract can be executed, and to bring, as much as possible, all estimates to be previously approved and finally to be paid by the Treasury.

Thus, by following the course of Nature, and not the purposes of politics, or the accumulated patchwork of occasional accommodation, this vast, expensive department may be methodized, its service proportioned to its necessities, and its payments subjected to the inspection of the superior minister of finance, who is to judge of it on the result of the total collective exigencies of the state. This last is a reigning principle through my whole plan; and it is a principle which I hope may hereafter be applied to other plans.

By these regulations taken together, besides the three subordinate treasuries in the lesser principalities, five other subordinate treasuries are suppressed. There is taken away the whole establishment of detail in the household: the treasurer; the comptroller (for a comptroller is hardly necessary where there is no treasurer); the cofferer of the household; the treasurer of the chamber; the master of the household; the whole board of green cloth;—and a vast number of subordinate offices in the department of the steward of the household,—the whole establishment of the great wardrobe,—the removing wardrobe,—the jewel office,—the robes,—the Board of Works,—almost the whole charge of the civil branch of the Board of Ordnance, are taken away. All these arrangements together will be found to relieve the nation from a vast weight of influence, without distressing, but rather by forwarding every public service. When something of this kind is done, then the public may begin to breathe. Under other governments, a question of expense is only a question of economy, and it is nothing more: with us, in every question of expense there is always a mixture of constitutional considerations.

It is, Sir, because I wish to keep this business of subordinate treasuries as much as I can together, that I brought the ordnance office before you, though it is properly a military department. For the same reason I will now trouble you with my thoughts and propositions upon two of the greatest under-treasuries: I mean the office of paymaster of the land forces, or treasurer of the army, and that of the treasurer of the navy. The former of these has long been a great object of public suspicion and uneasiness. Envy, too, has had its share in the obloquy which is cast upon this office. But I am sure that it has no share at all in the reflections I shall make upon it, or in the reformations that I shall propose. I do not grudge to the honorable gentleman who at present holds the office any of the effects of his talents, his merit, or his fortune. He is respectable in all these particulars. I follow the constitution of the office without persecuting its holder. It is necessary in all matters of public complaint, where men frequently feel right and argue wrong, to separate prejudice from reason, and to be very sure, in attempting the redress of a grievance, that we hit upon its real seat and its true nature. Where there is an abuse in office, the first thing that occurs in heat is to censure the officer. Our natural disposition leads all our inquiries rather to persons than to things. But this prejudice is to be corrected by maturer thinking.

Sir, the profits of the pay office (as an office) are not too great, in my opinion, for its duties, and for the rank of the person who has generally held it. He has been generally a person of the highest rank,—that is to say, a person of eminence and consideration in this House. The great and the invidious profits of the pay office are from the bank that is held in it. According to the present course of the office, and according to the present mode of accounting there, this bank must necessarily exist somewhere. Money is a productive thing; and when the usual time of its demand can be tolerably calculated, it may with prudence be safely laid out to the profit of the holder. It is on this calculation that the business of banking proceeds. But no profit can be derived from the use of money which does not make it the interest of the holder to delay his account. The process of the Exchequer colludes with this interest. Is this collusion from its want of rigor and strictness and great regularity of form? The reverse is true. They have in the Exchequer brought rigor and formalism to their ultimate perfection. The process against accountants is so rigorous, and in a manner so unjust, that correctives must from time to time be applied to it. These correctives being discretionary, upon the case, and generally remitted by the Barons to the Lords of the Treasury, as the test judges of the reasons for respite, hearings are had, delays are produced, and thus the extreme of rigor in office (as usual in all human affairs) leads to the extreme of laxity. What with the interested delay of the officer, the ill-conceived exactness of the court, the applications for dispensations from that exactness, the revival of rigorous process after the expiration of the time, and the new rigors producing new applications and new enlargements of time, such delays happen in the public accounts that they can scarcely ever be closed.

Besides, Sir, they have a rule in the Exchequer, which, I believe, they have founded upon a very ancient statute, that of the 51st of Henry the Third, by which it is provided, that, "when a sheriff or bailiff hath begun his account, none other shall be received to account, until he that was first appointed hath clearly accounted, and that the sum has been received."
%[38]
\footnote{ Et quaunt viscount ou baillif eit comence de acompter, nul autre ne seit resceu de aconter tanque le primer qe soit assis eit peraccompte, et qe la somme soit resceu.—Stat. 5. Ann Dom. 1266.}
 Whether this clause of that statute be the ground of that absurd practice I am not quite able to ascertain. But it has very generally prevailed, though I am told that of late they have began to relax from it. In consequence of forms adverse to substantial account, we have a long succession of paymasters and their representatives who have never been admitted to account, although perfectly ready to do so.

As the extent of our wars has scattered the accountants under the paymaster into every part of the globe, the grand and sure paymaster, Death, in all his shapes, calls these accountants to another reckoning. Death, indeed, domineers over everything but the forms of the Exchequer. Over these he has no power. They are impassive and immortal. The audit of the Exchequer, more severe than the audit to which the accountants are gone, demands proofs which in the nature of things are difficult, sometimes impossible, to be had. In this respect, too, rigor, as usual, defeats itself. Then the Exchequer never gives a particular receipt, or clears a man of his account as far as it goes. A final acquittance (or a quietus, as they term it) is scarcely ever to be obtained. Terrors and ghosts of unlaid accountants haunt the houses of their children from generation to generation. Families, in the course of succession, fall into minorities; the inheritance comes into the hands of females; and very perplexed affairs are often delivered over into the hands of negligent guardians and faithless stewards. So that the demand remains, when the advantage of the money is gone,—if ever any advantage at all has been made of it. This is a cause of infinite distress to families, and becomes a source of influence to an extent that can scarcely be imagined, but by those who have taken some pains to trace it. The mildness of government, in the employment of useless and dangerous powers, furnishes no reason for their continuance.

As things stand, can you in justice (except perhaps in that over-perfect kind of justice which has obtained by its merits the title of the opposite vice
%[39]
\footnote{ Summum jus summa injuria.}
) insist that any man should, by the course of his office, keep a bank from whence he is to derive no advantage? that a man should be subject to demands below and be in a manner refused an acquittance above, that he should transmit an original sin and inheritance of vexation to his posterity, without a power of compensating himself in some way or other for so perilous a situation? We know, that, if the paymaster should deny himself the advantages of his bank, the public, as things stand, is not the richer for it by a single shilling. This I thought it necessary to say as to the offensive magnitude of the profits of this office, that we may proceed in reformation on the principles of reason, and not on the feelings of envy.

The treasurer of the navy is, mutatis mutandis, in the same circumstances. Indeed, all accountants are. Instead of the present mode, which is troublesome to the officer and unprofitable to the public, I propose to substitute something more effectual than rigor, which is the worst exactor in the world. I mean to remove the very temptations to delay; to facilitate the account; and to transfer this bank, now of private emolument, to the public. The crown will suffer no wrong at least from the pay offices; and its terrors will no longer reign over the families of those who hold or have held them. I propose that these offices should be no longer banks or treasuries, but mere offices of administration. I propose, first, that the present paymaster and the treasurer of the navy should carry into the Exchequer the whole body of the vouchers for what they have paid over to deputy-paymasters, to regimental agents, or to any of those to whom they have and ought to have paid money. I propose that those vouchers shall be admitted as actual payments in their accounts, and that the persons to whom the money has been paid shall then stand charged in the Exchequer in their place. After this process, they shall be debited or charged for nothing but the money-balance that remains in their hands.

I am conscious, Sir, that, if this balance (which they could not expect to be so suddenly demanded by any usual process of the Exchequer) should now be exacted all at once, not only their ruin, but a ruin of others to an extent which I do not like to think of, but which I can well conceive, and which you may well conceive, might be the consequence. I told you, Sir, when I promised before the holidays to bring in this plan, that I never would suffer any man or description of men to suffer from errors that naturally have grown out of the abusive constitution of those offices which I propose to regulate. If I cannot reform with equity, I will not reform at all.

For the regulation of past accounts, I shall therefore propose such a mode, as men, temperate and prudent, make use of in the management of their private affairs, when their accounts are various, perplexed, and of long standing. I would therefore, after their example, divide the public debts into three sorts,—good, bad, and doubtful. In looking over the public accounts, I should never dream of the blind mode of the Exchequer, which regards things in the abstract, and knows no difference in the quality of its debts or the circumstances of its debtors. By this means it fatigues itself, it vexes others, it often crushes the poor, it lets escape the rich, or, in a fit of mercy or carelessness, declines all means of recovering its just demands. Content with the eternity of its claims, it enjoys its Epicurean divinity with Epicurean languor. But it is proper that all sorts of accounts should be closed some time or other,—by payment, by composition, or by oblivion. Expedit reipublicæ ut sit finis litium. Constantly taking along with me, that an extreme rigor is sure to arm everything against it, and at length to relax into a supine neglect, I propose, Sir, that even the best, soundest, and the most recent dents should be put into instalments, for the mutual benefit of the accountant and the public.

In proportion, however, as I am tender of the past, I would be provident of the future. All money that was formerly imprested to the two great pay offices I would have imprested in future to the Bank of England. These offices should in future receive no more than cash sufficient for small payments. Their other payments ought to be made by drafts on the Bank, expressing the service. A check account from both offices, of drafts and receipts, should be annually made up in the Exchequer,—charging the Bank in account with the cash balance, but not demanding the payment until there is an order from the Treasury, in consequence of a vote of Parliament.

As I did not, Sir, deny to the paymaster the natural profits of the bank that was in his hands, so neither would I to the Bank of England. A share of that profit might be derived to the public in various ways. My favorite mode is this: that, in compensation for the use of this money, the bank may take upon themselves, first, the charge of the Mint, to which they are already, by their charter, obliged to bring in a great deal of bullion annually to be coined. In the next place, I mean that they should take upon themselves the charge of remittances to our troops abroad. This is a species of dealing from which, by the same charter, they are not debarred. One and a quarter per cent will be saved instantly thereby to the public on very large sums of money. This will be at once a matter of economy and a considerable reduction of influence, by taking away a private contract of an expensive nature. If the Bank, which is a great corporation, and of course receives the least profits from the money in their custody, should of itself refuse or be persuaded to refuse this offer upon those terms, I can speak with some confidence that one at least, if not both parts of the condition would be received, and gratefully received, by several bankers of eminence. There is no banker who will not be at least as good security as any paymaster of the forces, or any treasurer of the navy, that have ever been bankers to the public: as rich at least as my Lord Chatham, or my Lord Holland, or either of the honorable gentlemen who now hold the offices, were at the time that they entered into them; or as ever the whole establishment of the Mint has been at any period.

These, Sir, are the outlines of the plan I mean to follow, in suppressing these two large subordinate treasuries. I now come to another subordinate treasury,—I mean that of the paymaster of the pensions; for which purpose I reënter the limits of the civil establishment: I departed from those limits in pursuit of a principle; and, following the same game in its doubles, I am brought into those limits again. That treasury and that office I mean to take away, and to transfer the payment of every name, mode, and denomination of pensions to the Exchequer. The present course of diversifying the same object can answer no good purpose, whatever its use may be to purposes of another kind. There are also other lists of pensions; and I mean that they should all be hereafter paid at one and the same place. The whole of the new consolidated list I mean to reduce to 60,000l. a year, which sum I intend it shall never exceed. I think that sum will fully answer as a reward to all real merit and a provision for all real public charity that is ever like to be placed upon the list. If any merit of an extraordinary nature should emerge before that reduction is completed, I have left it open for an address of either House of Parliament to provide for the case. To all other demands it must be answered, with regret, but with firmness, "The public is poor."

I do not propose, as I told you before Christmas, to take away any pension. I know that the public seem to call for a reduction of such of them as shall appear unmerited. As a censorial act, and punishment of an abuse, it might answer some purpose. But this can make no part of my plan. I mean to proceed by bill; and I cannot stop for such an inquiry. I know some gentlemen may blame me. It is with great submission to better judgments that I recommend it to consideration, that a critical retrospective examination of the pension list, upon the principle of merit, can never serve for my basis. It cannot answer, according to my plan, any effectual purpose of economy, or of future, permanent reformation. The process in any way will be entangled and difficult, and it will be infinitely slow: there is a danger, that, if we turn our line of march, now directed towards the grand object, into this more laborious than useful detail of operations, we shall never arrive at our end.

The king, Sir, has been by the Constitution appointed sole judge of the merit for which a pension is to be given. We have a right, undoubtedly, to canvass this, as we have to canvass every act of government. But there is a material difference between an office to be reformed and a pension taken away for demerit. In the former case, no charge is implied against the holder; in the latter, his character is slurred, as well as his lawful emolument affected. The former process is against the thing; the second, against the person. The pensioner certainly, if he pleases, has a right to stand on his own defence, to plead his possession, and to bottom his title in the competency of the crown to give him what he holds. Possessed and on the defensive as he is, he will not be obliged to prove his special merit, in order to justify the act of legal discretion, now turned into his property, according to his tenure. The very act, he will contend, is a legal presumption, and an implication of his merit. If this be so, from the natural force of all legal presumption, he would put us to the difficult proof that he has no merit at all. But other questions would arise in the course of such an inquiry,—that is, questions of the merit when weighed against the proportion of the reward; then the difficulty will be much greater.

The difficulty will not, Sir, I am afraid, be much less, if we pass to the person really guilty in the question of an unmerited pension: the minister himself. I admit, that, when called to account for the execution of a trust, he might fairly be obliged to prove the affirmative, and to state the merit for which the pension is given, though on the pensioner himself such a process would be hard. If in this examination we proceed methodically, and so as to avoid all suspicion of partiality and prejudice, we must take the pensions in order of time, or merely alphabetically. The very first pension to which we come, in either of these ways, may appear the most grossly unmerited of any. But the minister may very possibly show that he knows nothing of the putting on this pension; that it was prior in time to his administration; that the minister who laid it on is dead: and then we are thrown back upon the pensioner himself, and plunged into all our former difficulties. Abuses, and gross ones, I doubt not, would appear, and to the correction of which I would readily give my hand: but when I consider that pensions have not generally been affected by the revolutions of ministry; as I know not where such inquiries would stop; and as an absence of merit is a negative and loose thing;—one might be led to derange the order of families founded on the probable continuance of their kind of income; I might hurt children; I might injure creditors;—I really think it the more prudent course not to follow the letter of the petitions. If we fix this mode of inquiry as a basis, we shall, I fear, end as Parliament has often ended under similar circumstances. There will be great delay, much confusion, much inequality in our proceedings. But what presses me most of all is this: that, though we should strike off all the unmerited pensions, while the power of the crown remains unlimited, the very same undeserving persons might afterwards return to the very same list; or, if they did not, other persons, meriting as little as they do, might be put upon it to an undefinable amount. This, I think, is the pinch of the grievance.

For these reasons, Sir, I am obliged to waive this mode of proceeding as any part of my plan. In a plan of reformation, it would be one of my maxims, that, when I know of an establishment which may be subservient to useful purposes, and which at the same time, from its discretionary nature, is liable to a very great perversion from those purposes, I would limit the quantity of the power that might be so abused. For I am sure that in all such cases the rewards of merit will have very narrow bounds, and that partial or corrupt favor will be infinite. This principle is not arbitrary, but the limitation of the specific quantity must be so in some measure. I therefore state 60,000l., leaving it open to the House to enlarge or contract the sum as they shall see, on examination, that the discretion I use is scanty or liberal. The whole amount of the pensions of all denominations which have been laid before us amount, for a period of seven years, to considerably more than 100,000l. a year. To what the other lists amount I know not. That will be seen hereafter. But from those that do appear, a saving will accrue to the public, at one time or other, of 40,000l. a year; and we had better, in my opinion, to let it fall in naturally than to tear it crude and unripe from the stalk.
%[40]
\footnote{ It was supposed by the Lord Advocate, in a subsequent debate, that Mr. Burke, because he objected to an inquiry into the pension list for the purpose of economy and relief of the public, would have it withheld from the judgment of Parliament for all purposes whatsoever. This learned gentleman certainly misunderstood him. His plan shows that he wished the whole list to be easily accessible; and he knows that the public eye is of itself a great guard against abuse.}


There is a great deal of uneasiness among the people upon an article which I must class under the head of pensions: I mean the great patent offices in the Exchequer. They are in reality and substance no other than pensions, and in no other light shall I consider them. They are sinecures; they are always executed by deputy; the duty of the principal is as nothing. They differ, however, from the pensions on the list in some particulars. They are held for life. I think, with the public, that the profits of those places are grown enormous; the magnitude of those profits, and the nature of them, both call for reformation. The nature of their profits, which grow out of the public distress, is itself invidious and grievous. But I fear that reform cannot be immediate. I find myself under a restriction. These places, and others of the same kind, which are held for life, have been considered as property. They have been given as a provision for children; they have been the subject of family settlements; they have been the security of creditors. What the law respects shall be sacred to me. If the barriers of law should be broken down, upon ideas of convenience, even of public convenience, we shall have no longer anything certain among us. If the discretion of power is once let loose upon property, we can be at no loss to determine whose power and what discretion it is that will prevail at last. It would be wise to attend upon the order of things, and not to attempt to outrun the slow, but smooth and even course of Nature. There are occasions, I admit, of public necessity, so vast, so clear, so evident, that they supersede all laws. Law, being only made for the benefit of the community, cannot in any one of its parts resist a demand which may comprehend the total of the public interest. To be sure, no law can set itself up against the cause and reason of all law; but such a case very rarely happens, and this most certainly is not such a case. The mere time of the reform is by no means worth the sacrifice of a principle of law. Individuals pass like shadows; but the commonwealth is fixed and stable. The difference, therefore, of to-day and to-morrow, which to private people is immense, to the state is nothing. At any rate, it is better, if possible, to reconcile our economy with our laws than to set them at variance,—a quarrel which in the end must be destructive to both.

My idea, therefore, is, to reduce those offices to fixed salaries, as the present lives and reversions shall successively fall. I mean, that the office of the great auditor (the auditor of the receipt) shall be reduced to 3000l. a year; and the auditors of the imprest, and the rest of the principal officers, to fixed appointments of 1,500l. a year each. It will not be difficult to calculate the value of this fall of lives to the public, when we shall have obtained a just account of the present income of those places; and we shall obtain that account with great facility, if the present possessors are not alarmed with any apprehension of danger to their freehold office.

I know, too, that it will be demanded of me, how it comes, that, since I admit these offices to be no better than pensions, I chose, after the principle of law had been satisfied, to retain them at all. To this, Sir, I answer, that, conceiving it to be a fundamental part of the Constitution of this country, and of the reason of state in every country, that there must be means of rewarding public service, those means will be incomplete, and indeed wholly insufficient for that purpose, if there should be no further reward for that service than the daily wages it receives during the pleasure of the crown.

Whoever seriously considers the excellent argument of Lord Somers, in the Bankers' Case, will see he bottoms himself upon the very same maxim which I do; and one of his principal grounds of doctrine for the alienability of the domain in England,
%[41]
\footnote{ Before the statute of Queen Anne, which limited the alienation of land.}
 contrary to the maxim of the law in France, he lays in the constitutional policy of furnishing a permanent reward to public service, of making that reward the origin of families, and the foundation of wealth as well as of honors. It is, indeed, the only genuine, unadulterated origin of nobility. It is a great principle in government, a principle at the very foundation of the whole structure. The other judges who held the same doctrine went beyond Lord Somers with regard to the remedy which they thought was given by law against the crown upon the grant of pensions. Indeed, no man knows, when he cuts off the incitements to a virtuous ambition, and the just rewards of public service, what infinite mischief he may do his country through all generations. Such saving to the public may prove the worst mode of robbing it. The crown, which has in its hands the trust of the daily pay for national service, ought to have in its hands also the means for the repose of public labor and the fixed settlement of acknowledged merit. There is a time when the weather-beaten, vessels of the state ought to come into harbor. They must at length have a retreat from the malice of rivals, from the perfidy of political friends, and the inconstancy of the people. Many of the persons who in all times have filled the great offices of state have been younger brothers, who had originally little, if any fortune. These offices do not furnish the means of amassing wealth. There ought to be some power in the crown of granting pensions out of the reach of its own caprices. An entail of dependence is a bad reward of merit.

I would therefore leave to the crown the possibility of conferring some favors, which, whilst they are received as a reward, do not operate as corruption. When men receive obligations from the crown, through the pious hands of fathers, or of connections as venerable as the paternal, the dependences which arise from thence are the obligations of gratitude, and not the fetters of servility. Such ties originate in virtue, and they promote it. They continue men in those habitudes of friendship, those political connections, and those political principles, in which they began life. They are antidotes against a corrupt levity, instead of causes of it. What an unseemly spectacle would it afford, what a disgrace would it be to the commonwealth that suffered such things, to see the hopeful son of a meritorious minister begging his bread at the door of that Treasury from whence his father dispensed the economy of an empire, and promoted the happiness and glory of his country! Why should he be obliged to prostrate his honor and to submit his principles at the levee of some proud favorite, shouldered and thrust aside by every impudent pretender on the very spot where a few days before he saw himself adored,—obliged to cringe to the author of the calamities of his house, and to kiss the hands that are red with his father's blood?—No, Sir, these things are unfit,—they are intolerable.

Sir, I shall be asked, why I do not choose to destroy those offices which are pensions, and appoint pensions under the direct title in their stead. I allow that in some cases it leads to abuse, to have things appointed for one purpose and applied to another. I have no great objection to such a change; but I do not think it quite prudent for me to propose it. If I should take away the present establishment, the burden of proof rests upon me, that so many pensions, and no more, and to such an amount each, and no more, are necessary for the public service. This is what I can never prove; for it is a thing incapable of definition. I do not like to take away an object that I think answers my purpose, in hopes of getting it back again in a better shape. People will bear an old establishment, when its excess is corrected, who will revolt at a new one. I do not think these office-pensions to be more in number than sufficient: but on that point the House will exercise its discretion. As to abuse, I am convinced that very few trusts in the ordinary course of administration have admitted less abuse than this. Efficient ministers have been their own paymasters, it is true; but their very partiality has operated as a kind of justice, and still it was service that was paid. When we look over this Exchequer list, we find it filled with the descendants of the Walpoles, of the Pelhams, of the Townshends,—names to whom this country owes its liberties, and to whom his Majesty owes his crown. It was in one of these lines that the immense and envied employment he now holds came to a certain duke, 
%[42]
\footnote{ Duke of Newcastle, whose dining-room is under the House of Commons.}
 who is now probably sitting quietly at a very good dinner directly under us, and acting high life below stairs, whilst we, his masters, are filling our mouths with unsubstantial sounds, and talking of hungry economy over his head. But he is the elder branch of an ancient and decayed house, joined to and repaired by the reward of services done by another. I respect the original title, and the first purchase of merited wealth and honor through all its descents, through all its transfers, and all its assignments. May such fountains never be dried up! May they ever flow with their original purity, and refresh and fructify the commonwealth for ages!

Sir, I think myself bound to give you my reasons as clearly and as fully for stopping in the course of reformation as for proceeding in it. My limits are the rules of law, the rules of policy, and the service of the state. This is the reason why I am not able to intermeddle with another article, which seems to be a specific object in several of the petitions: I mean the reduction of exorbitant emoluments to efficient offices. If I knew of any real efficient office which did possess exorbitant emoluments, I should be extremely desirous of reducing them. Others may know of them: I do not. I am not possessed of an exact common measure between real service and its reward. I am very sure that states do sometimes receive services which is hardly in their power to reward according to their worth. If I were to give my judgment with regard to this country, I do not think the great efficient offices of the state to be overpaid. The service of the public is a thing which cannot be put to auction and struck down to those who will agree to execute it the cheapest. When the proportion between reward and service is our object, we must always consider of what nature the service is, and what sort of men they are that must perform it. What is just payment for one kind of labor, and full encouragement for one kind of talents, is fraud and discouragement to others. Many of the great offices have much duty to do, and much expense of representation to maintain. A Secretary of State, for instance, must not appear sordid in the eyes of the ministers of other nations; neither ought our ministers abroad to appear contemptible in the courts where they reside. In all offices of duty, there is almost necessarily a great neglect of all domestic affairs. A person in high office can rarely take a view of his family-house. If he sees that the state takes no detriment, the state must see that his affairs should take as little.

I will even go so far as to affirm, that, if men were willing to serve in such situations without salary, they ought not to be permitted to do it. Ordinary service must be secured by the motives to ordinary integrity. I do not hesitate to say that that state which lays its foundation in rare and heroic virtues will be sure to have its superstructure in the basest profligacy and corruption. An honorable and fair profit is the best security against avarice and rapacity; as in all things else, a lawful and regulated enjoyment is the best security against debauchery and excess. For as wealth is power, so all power will infallibly draw wealth to itself by some means or other; and when men are left no way of ascertaining their profits but by their means of obtaining them, those means will be increased to infinity. This is true in all the parts of administration, as well as in the whole. If any individual were to decline his appointments, it might give an unfair advantage to ostentatious ambition over unpretending service; it might breed invidious comparisons; it might tend to destroy whatever little unity and agreement may be found among ministers. And, after all, when an ambitious man had run down his competitors by a fallacious show of disinterestedness, and fixed himself in power by that means, what security is there that he would not change his course, and claim as an indemnity ten times more than he has given up?

This rule, like every other, may admit its exceptions. When a great man has some one great object in view to be achieved in a given time, it may be absolutely necessary for him to walk out of all the common roads, and, if his fortune permits it, to hold himself out as a splendid example. I am told that something of this kind is now doing in a country near us. But this is for a short race, the training for a heat or two, and not the proper preparation for the regular stages of a methodical journey. I am speaking of establishments, and not of men.

It may be expected, Sir, that, when I am giving my reasons why I limit myself in the reduction of employments, or of their profits, I should say something of those which seem of eminent inutility in the state: I mean the number of officers who, by their places, are attendant on the person of the king. Considering the commonwealth merely as such, and considering those officers only as relative to the direct purposes of the state, I admit that they are of no use at all. But there are many things in the constitution of establishments, which appear of little value on the first view, which in a secondary and oblique manner produce very material advantages. It was on full consideration that I determined not to lessen any of the offices of honor about the crown, in their number or their emoluments. These emoluments, except in one or two cases, do not much more than answer the charge of attendance. Men of condition naturally love to be about a court; and women of condition love it much more. But there is in all regular attendance so much of constraint, that, if it wore a mere charge, without any compensation, you would soon have the court deserted by all the nobility of the kingdom.

Sir, the most serious mischiefs would follow from such a desertion. Kings are naturally lovers of low company. They are so elevated above all the rest of mankind that they must look upon all their subjects as on a level. They are rather apt to hate than to love their nobility, on account of the occasional resistance to their will which will be made by their virtue, their petulance, or their pride. It must, indeed, be admitted that many of the nobility are as perfectly willing to act the part of flatterers, tale-bearers, parasites, pimps, and buffoons, as any of the lowest and vilest of mankind can possibly be. But they are not properly qualified for this object of their ambition. The want of a regular education, and early habits, and some lurking remains of their dignity, will never permit them to become a match for an Italian eunuch, a mountebank, a fiddler, a player, or any regular practitioner of that tribe. The Roman emperors, almost from the beginning, threw themselves into such hands; and the mischief increased every day till the decline and final ruin of the empire. It is therefore of very great importance (provided the thing is not overdone) to contrive such an establishment as must, almost whether a prince will or not, bring into daily and hourly offices about his person a great number of his first nobility; and it is rather an useful prejudice that gives them a pride in such a servitude. Though they are not much the better for a court, a court will be much the better for them. I have therefore not attempted to reform any of the offices of honor about the king's person.

There are, indeed, two offices in his stables which are sinecures: by the change of manners, and indeed by the nature of the thing, they must be so: I mean the several keepers of buck-hounds, stag-hounds, foxhounds, and harriers. They answer no purpose of utility or of splendor. These I propose to abolish. It is not proper that great noblemen should be keepers of dogs, though they were the king's dogs.

In every part of the scheme, I have endeavored that no primary, and that even no secondary, service of the state should suffer by its frugality. I mean to touch no offices but such as I am perfectly sure are either of no use at all, or not of any use in the least assignable proportion to the burden with which they load the revenues of the kingdom, and to the influence with which they oppress the freedom of Parliamentary deliberation; for which reason there are but two offices, which are properly state offices, that I have a desire to reform.

The first of them is the new office of Third Secretary of State, which is commonly called Secretary of State for the Colonies.

We know that all the correspondence of the colonies had been, until within a few years, carried on by the Southern Secretary of State, and that this department has not been shunned upon account of the weight of its duties, but, on the contrary, much sought on account of its patronage. Indeed, he must be poorly acquainted with the history of office who does not know how very lightly the American functions have always leaned on the shoulders of the ministerial Atlas who has upheld that side of the sphere. Undoubtedly, great temper and judgment was requisite in the management of the colony politics; but the official detail was a trifle. Since the new appointment, a train of unfortunate accidents has brought before us almost the whole correspondence of this favorite secretary's office since the first day of its establishment. I will say nothing of its auspicious foundation, of the quality of its correspondence, or of the effects that have ensued from it. I speak merely of its quantity, which we know would have been little or no addition to the trouble of whatever office had its hands the fullest. But what has been the real condition of the old office of Secretary of State? Have their velvet bags and their red boxes been so full that nothing more could possibly be crammed into them?

A correspondence of a curious nature has been lately published.
%[43]
\footnote{ Letters between Dr. Addington and Sir James Wright.}
 In that correspondence, Sir, we find the opinion of a noble person who is thought to be the grand manufacturer of administrations, and therefore the best judge of the quality of his work. He was of opinion that there was but one man of diligence and industry in the whole administration: it was the late Earl of Suffolk. The noble lord lamented very justly, that this statesman, of so much mental vigor, was almost wholly disabled from the exertion of it by his bodily infirmities. Lord Suffolk, dead to the state long before he was dead to Nature, at last paid his tribute to the common treasury to which we must all be taxed. But so little want was found even of his intentional industry, that the office, vacant in reality to its duties long before, continued vacant even in nomination and appointment for a year after his death. The whole of the laborious and arduous correspondence of this empire rested solely upon the activity and energy of Lord Weymouth.

It is therefore demonstrable, since one diligent man was fully equal to the duties of the two offices, that two diligent men will be equal to the duty of three. The business of the new office, which I shall propose to you to suppress, is by no means too much to be returned to either of the secretaries which remain. If this dust in the balance should be thought too heavy, it may be divided between them both,—North America (whether free or reduced) to the Northern Secretary, the West Indies to the Southern. It is not necessary that I should say more upon the inutility of this office. It is burning daylight. But before I have done, I shall just remark that the history of this office is too recent to suffer us to forget that it was made for the mere convenience of the arrangements of political intrigue, and not for the service of the state,—that it was made in order to give a color to an exorbitant increase of the civil list, and in the same act to bring a new accession to the loaded compost-heap of corrupt influence.

There is, Sir, another office which was not long since closely connected with this of the American Secretary, but has been lately separated from it for the very same purpose for which it had been conjoined: I mean the sole purpose of all the separations and all the conjunctions that have been lately made,—a job. I speak, Sir, of the Board of Trade and Plantations. This board is a sort of temperate bed of influence, a sort of gently ripening hothouse, where eight members of Parliament receive salaries of a thousand a year for a certain given time, in order to mature, at a proper season, a claim to two thousand, granted for doing less, and on the credit of having toiled so long in that inferior, laborious department.

I have known that board, off and on, for a great number of years. Both of its pretended objects have been much the objects of my study, if I have a right to call any pursuits of mine by so respectable a name. I can assure the House, (and I hope they will not think that I risk my little credit lightly,) that, without meaning to convey the least reflection upon any one of its members, past or present, it is a board which, if not mischievous, is of no use at all.

You will be convinced, Sir, that I am not mistaken, if you reflect how generally it is true, that commerce, the principal object of that office, flourishes most when it is left to itself. Interest, the great guide of commerce, is not a blind one. It is very well able to find its own way; and its necessities are its best laws. But if it were possible, in the nature of things, that the young should direct the old, and the inexperienced instruct the knowing,—if a board in the state was the best tutor for the counting-house,—if the desk ought to read lectures to the anvil, and the pen to usurp the place of the shuttle,—yet in any matter of regulation we know that board must act with as little authority as skill. The prerogative of the crown is utterly inadequate to the object; because all regulations are, in their nature, restrictive of some liberty. In the reign, indeed, of Charles the First, the Council, or Committees of Council, were never a moment unoccupied with affairs of trade. But even where they had no ill intention, (which was sometimes the case,) trade and manufacture suffered infinitely from their injudicious tampering. But since that period, whenever regulation is wanting, (for I do not deny that sometimes it may be wanting,) Parliament constantly sits; and Parliament alone is competent to such regulation. We want no instruction from boards of trade, or from any other board; and God forbid we should give the least attention to their reports! Parliamentary inquiry is the only mode of obtaining Parliamentary information. There is more real knowledge to be obtained by attending the detail of business in the committees above stairs than ever did come, or ever will come, from any board in this kingdom, or from all of them together. An assiduous member of Parliament will not be the worse instructed there for not being paid a thousand a year for learning his lesson. And now that I speak of the committees above stairs, I must say, that, having till lately attended them a good deal, I have observed that no description of members give so little attendance, either to communicate or to obtain instruction upon matters of commerce, as the honorable members of the grave Board of Trade. I really do not recollect that I have ever seen one of them in that sort of business. Possibly some members may have better memories, and may call to mind some job that may have accidentally brought one or other of them, at one time or other, to attend a matter of commerce.

This board, Sir, has had both its original formation and its regeneration in a job. In a job it was conceived, and in a job its mother brought it forth. It made one among those showy and specious impositions which one of the experiment-making administrations of Charles the Second held out to delude the people, and to be substituted in the place of the real service which they might expect from a Parliament annually sitting. It was intended, also, to corrupt that body, whenever it should be permitted to sit. It was projected in the year 1668, and it continued in a tottering and rickety childhood for about three or four years: for it died in the year 1673, a babe of as little hopes as ever swelled the bills of mortality in the article of convulsed or overlaid children who have hardly stepped over the threshold of life.

It was buried with little ceremony, and never more thought of until the reign of King William, when, in the strange vicissitude of neglect and vigor, of good and ill success that attended his wars, in the year 1695, the trade was distressed beyond all example of former sufferings by the piracies of the French cruisers. This suffering incensed, and, as it should seem, very justly incensed, the House of Commons. In this ferment, they struck, not only at the administration, but at the very constitution of the executive government. They attempted to form in Parliament a board for the protection of trade, which, as they planned it, was to draw to itself a great part, if not the whole, of the functions and powers both of the Admiralty and of the Treasury; and thus, by a Parliamentary delegation of office and officers, they threatened absolutely to separate these departments from the whole system of the executive government, and of course to vest the most leading and essential of its attributes in this board. As the executive government was in a manner convicted of a dereliction of its functions, it was with infinite difficulty that this blow was warded off in that session. There was a threat to renew the same attempt in the next. To prevent the effect of this manoeuvre, the court opposed another manoeuvre to it, and, in the year 1696, called into life this Board of Trade, which had slept since 1673.

This, in a few words, is the history of the regeneration of the Board of Trade. It has perfectly answered its purposes. It was intended to quiet the minds of the people, and to compose the ferment that was then strongly working in Parliament. The courtiers were too happy to be able to substitute a board which they knew would be useless in the place of one that they feared would be dangerous. Thus the Board of Trade was reproduced in a job; and perhaps it is the only instance of a public body which has never degenerated, but to this hour preserves all the health and vigor of its primitive institution.

This Board of Trade and Plantations has not been of any use to the colonies, as colonies: so little of use, that the flourishing settlements of New England, of Virginia, and of Maryland, and all our wealthy colonies in the West Indies, were of a date prior to the first board of Charles the Second. Pennsylvania and Carolina were settled during its dark quarter, in the interval between the extinction of the first and the formation of the second board. Two colonies alone owe their origin to that board. Georgia, which, till lately, has made a very slow progress,—and never did make any progress at all, until it had wholly got rid of all the regulations which the Board of Trade had moulded into its original constitution. That colony has cost the nation very great sums of money; whereas the colonies which have had the fortune of not being godfathered by the Board of Trade never cost the nation a shilling, except what has been so properly spent in losing them. But the colony of Georgia, weak as it was, carried with it to the last hour, and carries, even in its present dead, pallid visage, the perfect resemblance of its parents. It always had, and it now has, an establishment, paid by the public of England, for the sake of the influence of the crown: that colony having never been able or willing to take upon itself the expense of its proper government or its own appropriated jobs.

The province of Nova Scotia was the youngest and the favorite child of the Board. Good God! what sums the nursing of that ill-thriven, hard-visaged, and ill-favored brat has cost to this wittol nation! Sir, this colony has stood us in a sum of not less than seven hundred thousand pounds. To this day it has made no repayment,—it does not even support those offices of expense which are miscalled its government; the whole of that job still lies upon the patient, callous shoulders of the people of England.

Sir, I am going to state a fact to you that will serve to set in full sunshine the real value of formality and official superintendence. There was in the province of Nova Scotia one little neglected corner, the country of the neutral French; which, having the good-fortune to escape the fostering care of both France and England, and to have been shut out from the protection and regulation of councils of commerce and of boards of trade, did, in silence, without notice, and without assistance, increase to a considerable degree. But it seems our nation had more skill and ability in destroying than in settling a colony. In the last war, we did, in my opinion, most inhumanly, and upon pretences that in the eye of an honest man are not worth a farthing, root out this poor, innocent, deserving people, whom our utter inability to govern, or to reconcile, gave us no sort of right to extirpate. Whatever the merits of that extirpation might have been, it was on the footsteps of a neglected people, it was on the fund of unconstrained poverty, it was on the acquisitions of unregulated industry, that anything which deserves the name of a colony in that province has been formed. It has been formed by overflowings from the exuberant population of New England, and by emigration from other parts of Nova Scotia of fugitives from the protection of the Board of Trade.

But if all of these things were not more than sufficient to prove to you the inutility of that expensive establishment, I would desire you to recollect, Sir, that those who may be very ready to defend it are very cautious how they employ it,—cautious how they employ it even in appearance and pretence. They are afraid they should lose the benefit of its influence in Parliament, if they deemed to keep it up for any other purpose. If ever there were commercial points of great weight, and most closely connected with our dependencies, they are those which have been agitated and decided in Parliament since I came into it. Which of the innumerable regulations since made had their origin or their improvement in the Board of Trade? Did any of the several East India bills which have been successively produced since 1767 originate there? Did any one dream of referring them, or any part of them, thither? Was anybody so ridiculous as even to think of it? If ever there was an occasion on which the Board was fit to be consulted, it was with regard to the acts that were preludes to the American war, or attendant on its commencement. Those acts were full of commercial regulations, such as they were: the Intercourse Bill; the Prohibitory Bill; the Fishery Bill. If the Board was not concerned in such things, in what particular was it thought fit that it should be concerned? In the course of all these bills through the House, I observed the members of that board to be remarkably cautious of intermeddling. They understood decorum better; they know that matters of trade and plantations are no business of theirs.

There were two very recent occasions, which, if the idea of any use for the Board had not been extinguished by prescription, appeared loudly to call for their interference.

When commissioners were sent to pay his Majesty's and our dutiful respects to the Congress of the United States, a part of their powers under the commission were, it seems, of a commercial nature. They were authorized, in the most ample and undefined manner, to form a commercial treaty with America on the spot. This was no trivial object. As the formation of such a treaty would necessarily have been no less than the breaking up of our whole commercial system, and the giving it an entire new form, one would imagine that the Board of Trade would have sat day and night to model propositions, which, on our side, might serve as a basis to that treaty. No such thing. Their learned leisure was not in the least interrupted, though one of the members of the Board was a commissioner, and might, in mere compliment to his office, have been supposed to make a show of deliberation on the subject. But he knew that his colleagues would have thought he laughed in their faces, had he attempted to bring anything the most distantly relating to commerce or colonies before them. A noble person, engaged in the same commission, and sent to learn his commercial rudiments in New York, (then under the operation of an act for the universal prohibition of trade,) was soon after put at the head of that board. This contempt from the present ministers of all the pretended functions of that board, and their manner of breathing into its very soul, of inspiring it with its animating and presiding principle, puts an end to all dispute concerning their opinion of the clay it was made of. But I will give them heaped measure.

It was but the other day, that the noble lord in the blue ribbon carried up to the House of Peers two acts, altering, I think much for the better, but altering in a great degree, our whole commercial system: those acts, I mean, for giving a free trade to Ireland in woollens, and in all things else, with independent nations, and giving them an equal trade to our own colonies. Here, too, the novelty of this great, but arduous and critical improvement of system, would make you conceive that the anxious solicitude of the noble lord in the blue ribbon would have wholly destroyed the plan of summer recreation of that board, by references to examine, compare, and digest matters for Parliament. You would imagine that Irish commissioners of customs, and English commissioners of customs, and commissioners of excise, that merchants and manufacturers of every denomination, had daily crowded their outer rooms. Nil horum. The perpetual virtual adjournment, and the unbroken sitting vacation of that board, was no more disturbed by the Irish than by the plantation commerce, or any other commerce. The same matter made a large part of the business which occupied the House for two sessions before; and as our ministers were not then mellowed by the mild, emollient, and engaging blandishments of our dear sister into all the tenderness of unqualified surrender, the bounds and limits of a restrained benefit naturally required much detailed management and positive regulation. But neither the qualified propositions which were received, nor those other qualified propositions which were rejected by ministers, were the least concern of theirs, or were they ever thought of in the business.

It is therefore, Sir, on the opinion of Parliament, on the opinion of the ministers, and even on their own opinion of their inutility, that I shall propose to you to suppress the Board of Trade and Plantations, and to recommit all its business to the Council, from whence it was very improvidently taken; and which business (whatever it might be) was much better done, and without any expense; and, indeed, where in effect it may all come at last. Almost all that deserves the name of business there is the reference of the plantation acts to the opinion of gentlemen of the law. But all this may be done, as the Irish business of the same nature has always been done, by the Council, and with a reference to the Attorney and Solicitor General.

There are some regulations in the household, relative to the officers of the yeomen of the guards, and the officers and band of gentlemen pensioners, which I shall likewise submit to your consideration, for the purpose of regulating establishments which at present are much abused.

I have now finished all that for the present I shall trouble you with on the plan of reduction. I mean next to propose to you the plan of arrangement, by which I mean to appropriate and fix the civil list money to its several services according to their nature: for I am thoroughly sensible, that, if a discretion wholly arbitrary can be exercised over the civil list revenue, although the most effectual methods may be taken to prevent the inferior departments from exceeding their bounds, the plan of reformation will still be left very imperfect. It will not, in my opinion, be safe to permit an entirely arbitrary discretion even in the First Lord of the Treasury himself; it will not be safe to leave with him a power of diverting the public money from its proper objects, of paying it in an irregular course, or of inverting perhaps the order of time, dictated by the proportion of value, which ought to regulate his application of payment to service.

I am sensible, too, that the very operation of a plan of economy which tends to exonerate the civil list of expensive establishments may in some sort defeat the capital end we have in view,—the independence of Parliament; and that, in removing the public and ostensible means of influence, we may increase the fund of private corruption. I have thought of some methods to prevent an abuse of surplus cash under discretionary application,—I mean the heads of secret service, special service, various payments, and the like,—which I hope will answer, and which in due time I shall lay before you. Where I am unable to limit the quantity of the sums to be applied, by reason of the uncertain quantity of the service, I endeavor to confine it to its line, to secure an indefinite application to the definite service to which it belongs,—not to stop the progress of expense in its line, but to confine it to that line in which it professes to move.

But that part of my plan, Sir, upon which I principally rest, that on which I rely for the purpose of binding up and securing the whole, is to establish a fixed and invariable order in all its payments, which it shall not be permitted to the First Lord of the Treasury, upon any pretence whatsoever, to depart from. I therefore divide the civil list payments into nine classes, putting each class forward according to the importance or justice of the demand, and to the inability of the persons entitled to enforce their pretensions: that is, to put those first who have the most efficient offices, or claim the justest debts, and at the same time, from the character of that description of men, from the retiredness or the remoteness of their situation, or from their want of weight and power to enforce their pretensions, or from their being entirely subject to the power of a minister, without any reciprocal power of awing, ought to be the most considered, and are the most likely to be neglected,—all these I place in the highest classes; I place in the lowest those whose functions are of the least importance, but whose persons or rank are often of the greatest power and influence.

In the first class I place the judges, as of the first importance. It is the public justice that holds the community together; the ease, therefore, and independence of the judges ought to supersede all other considerations, and they ought to be the very last to feel the necessities of the state, or to be obliged either to court or bully a minister for their right; they ought to be as weak solicitors on their own demands as strenuous assertors of the rights and liberties of others. The judges are, or ought to be, of a reserved and retired character, and wholly unconnected with the political world.

In the second class I place the foreign ministers. The judges are the links of our connections with one another; the foreign ministers are the links of our connection with other nations. They are not upon the spot to demand payment, and are therefore the most likely to be, as in fact they have sometimes been, entirely neglected, to the great disgrace and perhaps the great detriment of the nation.

In the third class I would bring all the tradesmen who supply the crown by contract or otherwise.

In the fourth class I place all the domestic servants of the king, and all persons in efficient offices whose salaries do not exceed two hundred pounds a year.

In the fifth, upon account of honor, which ought to give place to nothing but charity and rigid justice, I would place the pensions and allowances of his Majesty's royal family, comprehending of course the queen, together with the stated allowance of the privy purse.

In the sixth class I place those efficient offices of duty whose salaries may exceed the sum of two hundred pounds a year.

In the seventh class, that mixed mass, the whole pension list.

In the eighth, the offices of honor about the king.

In the ninth, and the last of all, the salaries and pensions of the First Lord of the Treasury himself, the Chancellor of the Exchequer, and the other Commissioners of the Treasury.

If, by any possible mismanagement of that part of the revenue which is left at discretion, or by any other mode of prodigality, cash should be deficient for the payment of the lowest classes, I propose that the amount of those salaries where the deficiency may happen to fall shall not be carried as debt to the account of the succeeding year, but that it shall be entirely lapsed, sunk, and lost; so that government will be enabled to start in the race of every new year wholly unloaded, fresh in wind and in vigor. Hereafter no civil list debt can ever come upon the public. And those who do not consider this as saving, because it is not a certain sum, do not ground their calculations of the future on their experience of the past.

I know of no mode of preserving the effectual execution of any duty, but to make it the direct interest of the executive officer that it shall be faithfully performed. Assuming, then, that the present vast allowance to the civil list is perfectly adequate to all its purposes, if there should be any failure, it must be from the mismanagement or neglect of the First Commissioner of the Treasury; since, upon the proposed plan, there can be no expense of any consequence which he is not himself previously to authorize and finally to control. It is therefore just, as well as politic, that the loss should attach upon the delinquency.

If the failure from the delinquency should be very considerable, it will fall on the class directly above the First Lord of the Treasury, as well as upon himself and his board. It will fall, as it ought to fall, upon offices of no primary importance in the state; but then it will fall upon persons whom it will be a matter of no slight importance for a minister to provoke: it will fall upon persons of the first rank and consequence in the kingdom,—upon those who are nearest to the king, and frequently have a more interior credit with him than the minister himself. It will fall upon masters of the horse, upon lord chamberlains, upon lord stewards, upon grooms of the stole, and lords of the bedchamber. The household troops form an army, who will be ready to mutiny for want of pay, and whose mutiny will be really dreadful to a commander-in-chief. A rebellion of the thirteen lords of the bedchamber would be far more terrible to a minister, and would probably affect his power more to the quick, than a revolt of thirteen colonies. What an uproar such an event would create at court! What petitions, and committees, and associations, would it not produce! Bless me! what a clattering of white sticks and yellow sticks would be about his head! what a storm of gold keys would fly about the ears of the minister! what a shower of Georges, and thistles, and medals, and collars of S.S. would assail him at his first entrance into the antechamber, after an insolvent Christmas quarter!—a tumult which could not be appeased by all the harmony of the new year's ode. Rebellion it is certain there would be; and rebellion may not now, indeed, be so critical an event to those who engage in it, since its price is so correctly ascertained at just a thousand pound.

Sir, this classing, in my opinion, is a serious and solid security for the performance of a minister's duty. Lord Coke says, that the staff was put into the Treasurer's hand to enable him to support himself when there was no money in the Exchequer, and to beat away importunate solicitors. The method which I propose would hinder him from the necessity of such a broken staff to lean on, or such a miserable weapon for repulsing the demands of worthless suitors, who, the noble lord in the blue ribbon knows, will bear many hard blows on the head, and many other indignities, before they are driven from the Treasury. In this plan, he is furnished with an answer to all their importunity,—an answer far more conclusive than if he had knocked them down with his staff:—"Sir, (or my Lord,) you are calling for my own salary,—Sir, you are calling for the appointments of my colleagues who sit about me in office,—Sir, you are going to excite a mutiny at court against me,—you are going to estrange his Majesty's confidence from me, through the chamberlain, or the master of the horse, or the groom of the stole."

As things now stand, every man, in proportion to his consequence at court, tends to add to the expenses of the civil list, by all manner of jobs, if not for himself, yet for his dependants. When the new plan is established, those who are now suitors for jobs will become the most strenuous opposers of them. They will have a common interest with the minister in public economy. Every class, as it stands low, will become security for the payment of the preceding class; and thus the persons whose insignificant services defraud those that are useful would then become interested in their payment. Then the powerful, instead of oppressing, would be obliged to support the weak; and idleness would become concerned in the reward of industry. The whole fabric of the civil economy would become compact and connected in all its parts; it would be formed into a well-organized body, where every member contributes to the support of the whole, and where even the lazy stomach secures the vigor of the active arm.

This plan, I really flatter myself, is laid, not in official formality, nor in airy speculation, but in real life, and in human nature, in what "comes home" (as Bacon says) "to the business and bosoms of men." You have now, Sir, before you, the whole of my scheme, as far as I have digested it into a form that might be in any respect worthy of your consideration. I intend to lay it before you in five bills.
%[44]
\footnote{ Titles of the bills read.}
 The plan consists, indeed, of many parts; but they stand upon a few plain principles. It is a plan which takes nothing from the civil list without discharging it of a burden equal to the sum carried to the public service. It weakens no one function necessary to government; but, on the contrary, by appropriating supply to service, it gives it greater vigor. It provides the means of order and foresight to a minister of finance, which may always keep all the objects of his office, and their state, condition, and relations, distinctly before him. It brings forward accounts without hurrying and distressing the accountants: whilst it provides for public convenience, it regards private rights. It extinguishes secret corruption almost to the possibility of its existence. It destroys direct and visible influence equal to the offices of at least fifty members of Parliament. Lastly, it prevents the provision for his Majesty's children from being diverted to the political purposes of his minister.

These are the points on which I rely for the merit of the plan. I pursue economy in a secondary view, and only as it is connected with these great objects. I am persuaded, that even for supply this scheme will be far from unfruitful, if it be executed to the extent I propose it. I think it will give to the public, at its periods, two or three hundred thousand pounds a year; if not, it will give them a system of economy, which is itself a great revenue. It gives me no little pride and satisfaction to find that the principles of my proceedings are in many respects the very same with those which are now pursued in the plans of the French minister of finance. I am sure that I lay before you a scheme easy and practicable in all its parts. I know it is common at once to applaud and to reject all attempts of this nature. I know it is common for men to say, that such and such things are perfectly right, very desirable,—but that, unfortunately, they are not practicable. Oh, no, Sir! no! Those things-which are not practicable are not desirable. There is nothing in the world really beneficial that does not lie within the reach of an informed understanding and a well-directed pursuit. There is nothing that God has judged good for us that He has not given us the means to accomplish, both in the natural and the moral world. If we cry, like children, for the moon, like children we must cry on.

We must follow the nature of our affairs, and conform ourselves to our situation. If we do, our objects are plain and compassable. Why should we resolve to do nothing, because what I propose to you may not be the exact demand of the petition, when we are far from resolved to comply even with what evidently is so? Does this sort of chicanery become us? The people are the masters. They have only to express their wants at large and in gross. We are the expert artists, we are the skilful workmen, to shape their desires into perfect form, and to fit the utensil to the use. They are the sufferers, they tell the symptoms of the complaint; but we know the exact seat of the disease, and how to apply the remedy according to the rules of art. How shocking would it be to see us pervert our skill into a sinister and servile dexterity, for the purpose of evading our duty, and defrauding our employers, who are our natural lords, of the object of their just expectations! I think the whole not only practicable, but practicable in a very short time. If we are in earnest about it, and if we exert that industry and those talents in forwarding the work, which, I am afraid, may be exerted in impeding it, I engage that the whole may be put in complete execution within a year. For my own part, I have very little to recommend me for this or for any task, but a kind of earnest and anxious perseverance of mind, which, with all its good and all its evil effects, is moulded into my constitution. I faithfully engage to the House, if they choose to appoint me to any part in the execution of this work, (which, when they have made it theirs by the improvements of their wisdom, will be worthy of the able assistance they may give me,) that by night and by day, in town or in country, at the desk or in the forest, I will, without regard to convenience, ease, or pleasure, devote myself to their service, not expecting or admitting any reward whatsoever. I owe to this country my labor, which is my all; and I owe to it ten times more industry, if ten times more I could exert. After all, I shall be an unprofitable servant.

At the same time, if I am able, and if I shall be permitted, I will lend an humble helping hand to any other good work which is going on. I have not, Sir, the frantic presumption to suppose that this plan contains in it the whole of what the public has a right to expect in the great work of reformation they call for. Indeed, it falls infinitely short of it. It falls short even of my own ideas. I have some thoughts, not yet fully ripened, relative to a reform in the customs and excise, as well as in some other branches of financial administration. There are other things, too, which form essential parts in a great plan for the purpose of restoring the independence of Parliament. The contractors' bill of last year it is fit to revive; and I rejoice that it is in better hands than mine. The bill for suspending the votes of custom-house officers, brought into Parliament several years ago by one of our worthiest and wisest members,
%[45]
\footnote{ W. Dowdeswell, Esq., Chancellor of the Exchequer, 1765.}
—would to God we could along with the plan revive the person who designed it! but a man of very real integrity, honor, and ability will be found to take his place, and to carry his idea into full execution. You all see how necessary it is to review our military expenses for some years past, and, if possible, to bind up and close that bleeding artery of profusion; but that business also, I have reason to hope, will be undertaken by abilities that are fully adequate to it. Something must be devised (if possible) to check the ruinous expense of elections.

Sir, all or most of these things must be done. Every one must take his part. If we should be able, by dexterity, or power, or intrigue, to disappoint the expectations of our constituents, what will it avail us? We shall never be strong or artful enough to parry, or to put by, the irresistible demands of our situation. That situation calls upon us, and upon our constituents too, with a voice which will be heard. I am sure no man is more zealously attached than I am to the privileges of this House, particularly in regard to the exclusive management of money. The Lords have no right to the disposition, in any sense, of the public purse; but they have gone further in self-denial 
%[46]
\footnote{ Rejection of Lord Shelburne's motion in the House of Lords.}
 than our utmost jealousy could have required. A power of examining accounts, to censure, correct, and punish, we never, that I know of, have thought of denying to the House of Lords. It is something more than a century since we voted that body useless: they have now voted themselves so. The whole hope of reformation is at length cast upon us; and let us not deceive the nation, which does us the honor to hope everything from our virtue. If all the nation are not equally forward to press this duty upon us, yet be assured that they all equally expect we should perform it. The respectful silence of those who wait upon your pleasure ought to be as powerful with you as the call of those who require your service as their right. Some, without doors, affect to feel hurt for your dignity, because they suppose that menaces are held out to you. Justify their good opinion by showing that no menaces are necessary to stimulate you to your duty. But, Sir, whilst we may sympathize with them in one point who sympathize with us in another, we ought to attend no less to those who approach us like men, and who, in the guise of petitioners, speak to us in the tone of a concealed authority. It is not wise to force them to speak out more plainly what they plainly mean.—But the petitioners are violent. Be it so. Those who are least anxious about your conduct are not those that love you most. Moderate affection and satiated enjoyment are cold and respectful; but an ardent and injured passion is tempered up with wrath, and grief, and shame, and conscious worth, and the maddening sense of violated right. A jealous love lights his torch from the firebrands of the furies. They who call upon you to belong wholly to the people are those who wish you to return to your proper home,—to the sphere of your duty, to the post of your honor, to the mansion-house of all genuine, serene, and solid satisfaction. We have furnished to the people of England (indeed we have) some real cause of jealousy. Let us leave that sort of company which, if it does not destroy our innocence, pollutes our honor; let us free ourselves at once from everything that can increase their suspicions and inflame their just resentment; let us cast away from us, with a generous scorn, all the love-tokens and symbols that we have been vain and light enough to accept,—all the bracelets, and snuff-boxes, and miniature pictures, and hair devices, and all the other adulterous trinkets that are the pledges of our alienation and the monuments of our shame. Let us return to our legitimate home, and all jars and all quarrels will be lost in embraces. Let the commons in Parliament assembled be one and the same thing with the commons at large. The distinctions that are made to separate us are unnatural and wicked contrivances. Let us identify, let us incorporate ourselves with the people. Let us cut all the cables and snap the chains which tie us to an unfaithful shore, and enter the friendly harbor that shoots far out into the main its moles and jetties to receive us. "War with the world, and peace with our constituents." Be this our motto, and our principle. Then, indeed, we shall be truly great. Respecting ourselves, we shall be respected by the world. At present all is troubled, and cloudy, and distracted, and full of anger and turbulence, both abroad and at home; but the air may be cleared by this storm, and light and fertility may follow it. Let us give a faithful pledge to the people, that we honor, indeed, the crown, but that we belong to them; that we are their auxiliaries, and not their task-masters,—the fellow-laborers in the same vineyard, not lording over their rights, but helpers of their joy; that to tax them is a grievance to ourselves, but to cut off from our enjoyments to forward theirs is the highest gratification we are capable of receiving. I feel, with comfort, that we are all warmed with these sentiments, and while we are thus warm, I wish we may go directly and with a cheerful heart to this salutary work.

Sir, I move for leave to bring in a bill, "For the better regulation of his Majesty's civil establishments, and of certain public offices; for the limitation of pensions, and the suppression of sundry useless, expensive, and inconvenient places, and for applying the moneys saved thereby to the public service."
%[47]
\footnote{ The motion was seconded by Mr. Fox.}


Lord North stated, that there was a difference between this bill for regulating the establishments and some of the others, as they affected the ancient patrimony of the crown, and therefore wished them to be postponed till the king's consent could be obtained. This distinction was strongly controverted; but when it was insisted on as a point of decorum only, it was agreed to postpone them to another day. Accordingly, on the Monday following, viz. Feb. 14, leave was given, on the motion of Mr. Burke, without opposition, to bring in—

1st, "A bill for the sale of the forest and other crown lands, rents, and hereditaments, with certain exceptions, and for applying the produce thereof to the public service; and for securing, ascertaining, and satisfying tenant rights, and common and other rights."

2nd, "A bill for the more perfectly uniting to the crown the Principality of Wales and the County Palatine of Chester, and for the more commodious administration of justice within the same; as also for abolishing certain offices now appertaining thereto, for quieting dormant claims, ascertaining and securing tenant rights, and for the sale of all forest lands, and other lands, tenements, and hereditaments, held by his Majesty in right of the said Principality, or County Palatine of Chester, and for applying the produce thereof to the public service."

3rd, "A bill for uniting to the crown the Duchy and County Palatine of Lancaster, for the suppression of unnecessary offices now belonging thereto, for the ascertainment and security of tenant and other rights, and for the sale of all rents, lands, tenements, and hereditaments, and forests, within the said Duchy and County Palatine, or either of them, and for applying the produce thereof to the public service."

And it was ordered that Mr. Burke, Mr. Fox, Lord John Cavendish, Sir George Savile, Colonel Barré, Mr. Thomas Townshend, Mr. Byng, Mr. Dunning, Sir Joseph Mawbey, Mr. Recorder of London, Sir Robert Clayton, Mr. Frederick Montagu, the Earl of Upper Ossory, Sir William Guise, and Mr. Gilbert do prepare and bring in the same.

At the same time, Mr. Burke moved for leave to bring in—

4th, "A bill for uniting the Duchy of Cornwall to the crown; for the suppression of certain unnecessary offices now belonging thereto; for the ascertainment and security of tenant and other rights; and for the sale of certain rents, lands, and tenements, within or belonging to the said Duchy; and for applying the produce thereof to the public service."

But some objections being made by the Surveyor-General of the Duchy concerning the rights of the Prince of Wales, now in his minority, and Lord North remaining perfectly silent, Mr. Burke, at length, though he strongly contended against the principle of the objection, consented to withdraw this last motion for the present, to be renewed upon an early occasion.

%FOOTNOTES:
%[31] This term comprehends various retributions made to persons whose offices are taken away, or who in any other way suffer by the new arrangements that are made.

%[32] Edict registered 29th January, 1780.

%[33] Thomas Gilbert, Esq., member for Lichfield.

% [34] Here Lord North shook his head, and told those who sat near him that Mr. Probert's pension was to depend on his success. It may be so. Mr. Probert's pension was, however, no essential part of the question; nor did Mr. B. care whether he still possessed it or not. His point was, to show the ridicule of attempting an improvement of the Welsh revenue under its present establishment.

% [35] Case of Richard Lee, Esq., appellant, against George Venables Lord Vernon, respondent, in the year 1775.

% [36] Vide Lord Talbot's speech in Almon's Parliamentary Register. Vol VII. p. 79, of the Proceedings of the Lords.

% [37] More exactly, 378,616l. 10 s. 1¾ d.

% [38] Et quaunt viscount ou baillif eit comence de acompter, nul autre ne seit resceu de aconter tanque le primer qe soit assis eit peraccompte, et qe la somme soit resceu.—Stat. 5. Ann Dom. 1266.

% [39] Summum jus summa injuria.

% [40] It was supposed by the Lord Advocate, in a subsequent debate, that Mr. Burke, because he objected to an inquiry into the pension list for the purpose of economy and relief of the public, would have it withheld from the judgment of Parliament for all purposes whatsoever. This learned gentleman certainly misunderstood him. His plan shows that he wished the whole list to be easily accessible; and he knows that the public eye is of itself a great guard against abuse.

% [41] Before the statute of Queen Anne, which limited the alienation of land.

% [42] Duke of Newcastle, whose dining-room is under the House of Commons.

% [43] Letters between Dr. Addington and Sir James Wright.

% [44] Titles of the bills read.

% [45] W. Dowdeswell, Esq., Chancellor of the Exchequer, 1765.

% [46] Rejection of Lord Shelburne's motion in the House of Lords.

% [47] The motion was seconded by Mr. Fox.



%%%%%%%%%%%%%%%%%%%%%%%%%%%%%%%%%%%%%%%%%%%%%%%%%%%%%%%%%%%%%%%%%%%%%%%
\chapter*[Speech at Bristol Previous to the Election]{
Speech at the Guildhall in Bristol, Previous to the Late Election in That City,
upon Certain Points Relative to His Parliamentary Conduct (1780)}
\addcontentsline{toc}{chapter}{SPEECH AT BRISTOL PREVIOUS TO THE ELECTION,
September 6,1780}

Mr. Mayor, and Gentlemen,—I am extremely pleased at the appearance of this large and respectable meeting. The steps I may be obliged to take will want the sanction of a considerable authority; and in explaining anything which may appear doubtful in my public conduct, I must naturally desire a very full audience.

I have been backward to begin my canvass. The dissolution of the Parliament was uncertain; and it did not become me, by an unseasonable importunity, to appear diffident of the effect of my six years' endeavors to please you. I had served the city of Bristol honorably, and the city of Bristol had no reason to think that the means of honorable service to the public were become indifferent to me.

I found, on my arrival here, that three gentlemen had been long in eager pursuit of an object which but two of us can obtain. I found that they had all met with encouragement. A contested election in such a city as this is no light thing. I paused on the brink of the precipice. These three gentlemen, by various merits, and on various titles, I made no doubt were worthy of your favor. I shall never attempt to raise myself by depreciating the merits of my competitors. In the complexity and confusion of these cross pursuits, I wished to take the authentic public sense of my friends upon a business of so much delicacy. I wished to take your opinion along with me, that, if I should give up the contest at the very beginning, my surrender of my post may not seem the effect of inconstancy, or timidity, or anger, or disgust, or indolence, or any other temper unbecoming a man who has engaged in the public service. If, on the contrary, I should undertake the election, and fail of success, I was full as anxious that it should be manifest to the whole world that the peace of the city had not been broken by my rashness, presumption, or fond conceit of my own merit.

I am not come, by a false and counterfeit show of deference to your judgment, to seduce it in my favor. I ask it seriously and unaffectedly. If you wish that I should retire, I shall not consider that advice as a censure upon my conduct, or an alteration in your sentiments, but as a rational submission to the circumstances of affairs. If, on the contrary, you should think it proper for me to proceed on my canvass, if you will risk the trouble on your part, I will risk it on mine. My pretensions are such as you cannot be ashamed of, whether they succeed or fail.

If you call upon me, I shall solicit the favor of the city upon manly ground. I come before you with the plain confidence of an honest servant in the equity of a candid and discerning master. I come to claim your approbation, not to amuse you with vain apologies, or with professions still more vain and senseless. I have lived too long to be served by apologies, or to stand in need of them. The part I have acted has been in open day; and to hold out to a conduct which stands in that clear and steady light for all its good and all its evil, to hold out to that conduct the paltry winking tapers of excuses and promises,—I never will do it. They may obscure it with their smoke, but they never can illumine sunshine by such a flame as theirs.

I am sensible that no endeavors have been left untried to injure me in your opinion. But the use of character is to be a shield against calumny. I could wish, undoubtedly, (if idle wishes were not the most idle of all things,) to make every part of my conduct agreeable to every one of my constituents; but in so great a city, and so greatly divided as this, it is weak to expect it.

In such a discordancy of sentiments it is better to look to the nature of things than to the humors of men. The very attempt towards pleasing everybody discovers a temper always flashy, and often false and insincere. Therefore, as I have proceeded straight onward in my conduct, so I will proceed in my account of those parts of it which have been most excepted to. But I must first beg leave just to hint to you that we may suffer very great detriment by being open to every talker. It is not to be imagined how much of service is lost from spirits full of activity and full of energy, who are pressing, who are rushing forward, to great and capital objects, when you oblige them to be continually looking back. Whilst they are defending one service, they defraud you of an hundred. Applaud us when we run, console us when we fall, cheer us when we recover; but let us pass on,—for God's sake, let us pass on!

Do you think, Gentlemen, that every public act in the six years since I stood in this place before you, that all the arduous things which have been done in this eventful period which has crowded into a few years' space the revolutions of an age, can be opened to you on their fair grounds in half an hour's conversation?

But it is no reason, because there is a bad mode of inquiry, that there should be no examination at all. Most certainly it is our duty to examine; it is our interest, too: but it must be with discretion, with an attention to all the circumstances and to all the motives; like sound judges, and not like cavilling pettifoggers and quibbling pleaders, prying into flaws and hunting for exceptions. Look, Gentlemen, to the whole tenor of your member's conduct. Try whether his ambition or his avarice have justled him out of the straight line of duty,—or whether that grand foe of the offices of active life, that master vice in men of business, a degenerate and inglorious sloth, has made him flag and languish in his course. This is the object of our inquiry. If our member's conduct can bear this touch, mark it for sterling. He may have fallen into errors, he must have faults; but our error is greater, and our fault is radically ruinous to ourselves, if we do not bear, if we do not even applaud, the whole compound and mixed mass of such a character. Not to act thus is folly; I had almost said it is impiety. He censures God who quarrels with the imperfections of man.

Gentlemen, we must not be peevish with those who serve the people; for none will serve us, whilst there is a court to serve, but those who are of a nice and jealous honor. They who think everything, in comparison of that honor, to be dust and ashes, will not bear to have it soiled and impaired by those for whose sake they make a thousand sacrifices to preserve it immaculate and whole. We shall either drive such men from the public stage, or we shall send them to the court for protection, where, if they must sacrifice their reputation, they will at least secure their interest. Depend upon it, that the lovers of freedom will be free. None will violate their conscience to please us, in order afterwards to discharge that conscience, which they have violated, by doing us faithful and affectionate service. If we degrade and deprave their minds by servility, it will be absurd to expect that they who are creeping and abject towards us will ever be bold and incorruptible assertors of our freedom against the most seducing and the most formidable of all powers. No! human nature is not so formed: nor shall we improve the faculties or better the morals of public men by our possession of the most infallible receipt in the world for making cheats and hypocrites.

Let me say, with plainness, I who am no longer in a public character, that, if, by a fair, by an indulgent, by a gentlemanly behavior to our representatives, we do not give confidence to their minds and a liberal scope to their understandings, if we do not permit our members to act upon a very enlarged view of things, we shall at length infallibly degrade our national representation into a confused and scuffling bustle of local agency. When the popular member is narrowed in his ideas and rendered timid in his proceedings, the service of the crown will be the sole nursery of statesmen. Among the frolics of the court, it may at length take that of attending to its business. Then the monopoly of mental power will be added to the power of all other kinds it possesses. On the side of the people there will be nothing but impotence: for ignorance is impotence; narrowness of mind is impotence; timidity is itself impotence, and makes all other qualities that go along with it impotent and useless.

At present it is the plan of the court to make its servants insignificant. If the people should fall into the same humor, and should choose their servants on the same principles of mere obsequiousness and flexibility and total vacancy or indifference of opinion in all public matters, then no part of the state will be sound, and it will be in vain to think of saving it.

I thought it very expedient at this time to give you this candid counsel; and with this counsel I would willingly close, if the matters which at various times have been objected to me in this city concerned only myself and my own election. These charges, I think, are four in number: my neglect of a due attention to my constituents, the not paying more frequent visits here; my conduct on the affairs of the first Irish Trade Acts; my opinion and mode of proceeding on Lord Beauchamp's Debtors' Bills; and my votes on the late affairs of the Roman Catholics. All of these (except perhaps the first) relate to matters of very considerable public concern; and it is not lest you should censure me improperly, but lest you should form improper opinions on matters of some moment to you, that I trouble you at all upon the subject. My conduct is of small importance.

With regard to the first charge, my friends have spoken to ms of it in the style of amicable expostulation,—not so much blaming the thing as lamenting the effects. Others, less partial to me, were less kind in assigning the motives. I admit, there is a decorum and propriety in a member of Parliament's paying a respectful court to his constituents. If I were conscious to myself that pleasure, or dissipation, or low, unworthy occupations had detained me from personal attendance on you, I would readily admit my fault, and quietly submit to the penalty. But, Gentlemen, I live at an hundred miles' distance from Bristol; and at the end of a session I come to my own house, fatigued in body and in mind, to a little repose, and to a very little attention to my family and my private concerns. A visit to Bristol is always a sort of canvass, else it will do more harm than good. To pass from the toils of a session to the toils of a canvass is the furthest thing in the world from repose. I could hardly serve you as I have done, and court you too. Most of you have heard that I do not very remarkably spare myself in public business; and in the private business of my constituents I have done very near as much as those who have nothing else to do. My canvass of you was not on the 'change, nor in the county meetings, nor in the clubs of this city: it was in the House of Commons; it was at the Custom-House; it was at the Council; it was at the Treasury; it was at the Admiralty. I canvassed you through your affairs, and not your persons. I was not only your representative as a body; I was the agent, the solicitor of individuals; I ran about wherever your affairs could call me; and in acting for you, I often appeared rather as a ship-broker than as a member of Parliament. There was nothing too laborious or too low for me to undertake. The meanness of the business was raised by the dignity of the object. If some lesser matters have slipped through my fingers, it was because I filled my hands too full, and, in my eagerness to serve you, took in more than any hands could grasp. Several gentlemen stand round me who are my willing witnesses; and there are others who, if they were here, would be still better, because they would be unwilling witnesses to the same truth. It was in the middle of a summer residence in London, and in the middle of a negotiation at the Admiralty for your trade, that I was called to Bristol; and this late visit, at this late day, has been possibly in prejudice to your affairs.

Since I have touched upon this matter, let me say, Gentlemen, that, if I had a disposition or a right to complain, I have some cause of complaint on my side. With a petition of this city in my hand, passed through the corporation without a dissenting voice, a petition in unison with almost the whole voice of the kingdom, (with whose formal thanks I was covered over,) whilst I labored on no less than five bills for a public reform, and fought, against the opposition of great abilities and of the greatest power, every clause and every word of the largest of those bills, almost to the very last day of a very long session,—all this time a canvass in Bristol was as calmly carried on as if I were dead. I was considered as a man wholly out of the question. Whilst I watched and fasted and sweated in the House of Commons, by the most easy and ordinary arts of election, by dinners and visits, by "How do you dos," and "My worthy friends," I was to be quietly moved out of my seat,—and promises were made, and engagements entered into, without any exception or reserve, as if my laborious zeal in my duty had been a regular abdication of my trust.

To open my whole heart to you on this subject, I do confess, however, that there were other times, besides the two years in which I did visit you, when I was not wholly without leisure for repeating that mark of my respect. But I could not bring my mind to see you. You remember that in the beginning of this American war (that era of calamity, disgrace, and downfall, an era which no feeling mind will ever mention without a tear for England) you were greatly divided,—and a very strong body, if not the strongest, opposed itself to the madness which every art and every power were employed to render popular, in order that the errors of the rulers might be lost in the general blindness of the nation. This opposition continued until after our great, but most unfortunate victory at Long Island. Then all the mounds and banks of our constancy were borne down, at once, and the frenzy of the American war broke in upon us like a deluge. This victory, which seemed to put an immediate end to all difficulties, perfected us in that spirit of domination which our unparalleled prosperity had but too long nurtured. We had been so very powerful, and so very prosperous, that even the humblest of us were degraded into the vices and follies of kings. We lost all measure between means and ends; and our headlong desires became our politics and our morals. All men who wished for peace, or retained any sentiments of moderation, were overborne or silenced; and this city was led by every artifice (and probably with the more management because I was one of your members) to distinguish itself by its zeal for that fatal cause. In this temper of yours and of my mind, I should sooner have fled to the extremities of the earth than hate shown myself here. I, who saw in every American victory (for you have had a long series of these misfortunes) the germ and seed of the naval power of France and Spain, which all our heat and warmth against America was only hatching into life,—I should not have been a welcome visitant, with the brow and the language of such feelings. When afterwards the other face of your calamity was turned upon you, and showed itself in defeat and distress, I shunned you full as much. I felt sorely this variety in our wretchedness; and I did not wish to have the least appearance of insulting you with that show of superiority, which, though it may not be assumed, is generally suspected, in a time of calamity, from those whose previous warnings have been despised. I could not bear to show you a representative whose face did not reflect that of his constituents,—a face that could not joy in your joys, and sorrow in your sorrows. But time at length has made us all of one opinion, and we have all opened our eyes on the true nature of the American war,—to the true nature of all its successes and all its failures.

In that public storm, too, I had my private feelings. I had seen blown down and prostrate on the ground several of those houses to whom I was chiefly indebted for the honor this city has done me. I confess, that, whilst the wounds of those I loved were yet green, I could not bear to show myself in pride and triumph in that place into which their partiality had brought me, and to appear at feasts and rejoicings in the midst of the grief and calamity of my warm friends, my zealous supporters, my generous benefactors. This is a true, unvarnished, undisguised state of the affair. You will judge of it.

This is the only one of the charges in which I am personally concerned. As to the other matters objected against me, which in their turn I shall mention to you, remember once more I do not mean to extenuate or excuse. Why should I, when the things charged are among those upon which I found all my reputation? What would be left to me, if I myself was the man who softened and blended and diluted and weakened all the distinguishing colors of my life, so as to leave nothing distinct and determinate in my whole conduct?

It has been said, and it is the second charge, that in the questions of the Irish trade I did not consult the interest of my constituents,—or, to speak out strongly, that I rather acted as a native of Ireland than as an English member of Parliament.

I certainly have very warm good wishes for the place of my birth. But the sphere of my duties is my true country. It was as a man attached to your interests, and zealous for the conservation of your power and dignity, that I acted on that occasion, and on all occasions. You were involved in the American war. A new world of policy was opened, to which it was necessary we should conform, whether we would or not; and my only thought was how to conform to our situation in such a manner as to unite to this kingdom, in prosperity and in affection, whatever remained of the empire. I was true to my old, standing, invariable principle, that all things which came from Great Britain should issue as a gift of her bounty and beneficence, rather than as claims recovered against a struggling litigant,—or at least, that, if your beneficence obtained no credit in your concessions, yet that they should appear the salutary provisions of your wisdom and foresight, not as things wrung from you with your blood by the cruel gripe of a rigid necessity. The first concessions, by being (much against my will) mangled and stripped of the parts which were necessary to make out their just correspondence and connection in trade, were of no use. The next year a feeble attempt was made to bring the thing into better shape. This attempt, (countenanced by the minister,) on the very first appearance of some popular uneasiness, was, after a considerable progress through the House, thrown out by him.

What was the consequence? The whole kingdom of Ireland was instantly in a flame. Threatened by foreigners, and, as they thought, insulted by England, they resolved at once to resist the power of France and to cast off yours. As for us, we were able neither to protect nor to restrain them. Forty thousand men were raised and disciplined without commission from the crown. Two illegal armies were seen with banners displayed at the same time and in the same country. No executive magistrate, no judicature, in Ireland, would acknowledge the legality of the army which bore the king's commission; and no law, or appearance of law, authorized the army commissioned by itself. In this unexampled state of things, which the least error, the least trespass on the right or left, would have hurried down the precipice into an abyss of blood and confusion, the people of Ireland demand a freedom of trade with arms in their hands. They interdict all commerce between the two nations. They deny all new supply in the House of Commons, although in time of war. They stint the trust of the old revenue, given for two years to all the king's predecessors, to six months. The British Parliament, in a former session, frightened into a limited concession by the menaces of Ireland, frightened out of it by the menaces of England, was now frightened back again, and made an universal surrender of all that had been thought the peculiar, reserved, uncommunicable rights of England: the exclusive commerce of America, of Africa, of the West Indies,—all the enumerations of the Acts of Navigation,—all the manufactures,—iron, glass, even the last pledge of jealousy and pride, the interest hid in the secret of our hearts, the inveterate prejudice moulded into the constitution of our frame, even the sacred fleece itself, all went together. No reserve, no exception; no debate, no discussion. A sudden light broke in upon us all. It broke in, not through well-contrived and well-disposed windows, but through flaws and breaches,—through the yawning chasms of our ruin. We were taught wisdom by humiliation. No town in England presumed to have a prejudice, or dared to mutter a petition. What was worse, the whole Parliament of England, which retained authority for nothing but surrenders, was despoiled of every shadow of its superintendence. It was, without any qualification, denied in theory, as it had been trampled upon in practice. This scene of shame and disgrace has, in a manner, whilst I am speaking, ended by the perpetual establishment of a military power in the dominions of this crown, without consent of the British legislature, 
%[48]
\footnote{ Irish Perpetual Mutiny Act.}
 contrary to the policy of the Constitution, contrary to the Declaration of Right; and by this your liberties are swept away along with your supreme authority,—and both, linked together from the beginning, have, I am afraid, both together perished forever.

What! Gentlemen, was I not to foresee, or foreseeing, was I not to endeavor to save you from all these multiplied mischiefs and disgraces? Would the little, silly, canvass prattle of obeying instructions, and having no opinions but yours, and such idle, senseless tales, which amuse the vacant ears of unthinking men, have saved you from "the pelting of that pitiless storm," to which the loose improvidence, the cowardly rashness, of those who dare not look danger in the face so as to provide against it in time, and therefore throw themselves headlong into the midst of it, have exposed this degraded nation, beat down and prostrate on the earth, unsheltered, unarmed, unresisting? Was I an Irishman on that day that I boldly withstood our pride? or on the day that I hung down my head, and wept in shame and silence over the humiliation of Great Britain? I became unpopular in England for the one, and in Ireland for the other. What then? What obligation lay on me to be popular? I was bound to serve both kingdoms. To be pleased with my service was their affair, not mine.

I was an Irishman in the Irish business, just as much as I was an American, when, on the same principles, I wished you to concede to America at a time when she prayed concession at our feet. Just as much was I an American, when I wished Parliament to offer terms in victory, and not to wait the well-chosen hour of defeat, for making good by weakness and by supplication a claim of prerogative, preëminence, and authority.

Instead of requiring it from me, as a point of duty, to kindle with your passions, had you all been as cool as I was, you would have been saved disgraces and distresses that are unutterable. Do you remember our commission? We sent out a solemn embassy across the Atlantic Ocean, to lay the crown, the peerage, the commons of Great Britain at the feet of the American Congress. That our disgrace might want no sort of brightening and burnishing, observe who they were that composed this famous embassy. My Lord Carlisle is among the first ranks of our nobility. He is the identical man who, but two years before, had been put forward, at the opening of a session, in the House of Lords, as the mover of an haughty and rigorous address against America. He was put in the front of the embassy of submission. Mr. Eden was taken from the office of Lord Suffolk, to whom he was then Under-Secretary of State,—from the office of that Lord Suffolk who but a few weeks before, in his place in Parliament, did not deign to inquire where a congress of vagrants was to be found. This Lord Suffolk sent Mr. Eden to find these vagrants, without knowing where his king's generals were to be found who were joined in the same commission of supplicating those whom they were sent to subdue. They enter the capital of America only to abandon it; and these assertors and representatives of the dignity of England, at the tail of a flying army, let fly their Parthian shafts of memorials and remonstrances at random behind them. Their promises and their offers, their flatteries and their menaces, were all despised; and we were saved the disgrace of their formal reception only because the Congress scorned to receive them; whilst the State-house of independent Philadelphia opened her doors to the public entry of the ambassador of France. From war and blood we went to submission, and from submission plunged back again to war and blood, to desolate and be desolated, without measure, hope, or end. I am a Royalist: I blushed for this degradation of the crown. I am a Whig: I blushed for the dishonor of Parliament. I am a true Englishman: I felt to the quick for the disgrace of England. I am a man: I felt for the melancholy reverse of human affairs in the fall of the first power in the world.

To read what was approaching in Ireland, in the black and bloody characters of the American war, was a painful, but it was a necessary part of my public duty. For, Gentlemen, it is not your fond desires or mine that can alter the nature of things; by contending against which, what have we got, or shall ever get, but defeat and shame? I did not obey your instructions. No. I conformed to the instructions of truth and Nature, and maintained your interest, against your opinions, with a constancy that became me. A representative worthy of you ought to be a person of stability. I am to look, indeed, to your opinions,—but to such opinions as you and I must have five years hence. I was not to look to the flash of the day. I knew that you chose me, in my place, along with others, to be a pillar of the state, and not a weathercock on the top of the edifice, exalted for my levity and versatility, and of no use but to indicate the shiftings of every fashionable gale. Would to God the value of my sentiments on Ireland and on America had been at this day a subject of doubt and discussion! No matter what my sufferings had been, so that this kingdom had kept the authority I wished it to maintain, by a grave foresight, and by an equitable temperance in the use of its power.

The next article of charge on my public conduct, and that which I find rather the most prevalent of all, is Lord Beauchamp's bill: I mean his bill of last session, for reforming the law-process concerning imprisonment. It is said, to aggravate the offence, that I treated the petition of this city with contempt even in presenting it to the House, and expressed myself in terms of marked disrespect. Had this latter part of the charge been true, no merits on the side of the question which I took could possibly excuse me. But I am incapable of treating this city with disrespect. Very fortunately, at this minute, (if my bad eyesight does not deceive me,) the worthy gentleman 
%[49]
\footnote{ Mr. Williams.}
 deputed on this business stands directly before me. To him I appeal, whether I did not, though it militated with my oldest and my most recent public opinions, deliver the petition with a strong and more than usual recommendation to the consideration of the House, on account of the character and consequence of those who signed it. I believe the worthy gentleman will tell you, that, the very day I received it, I applied to the Solicitor, now the Attorney General, to give it an immediate consideration; and he most obligingly and instantly consented to employ a great deal of his very valuable time to write an explanation of the bill. I attended the committee with all possible care and diligence, in order that every objection of yours might meet with a solution, or produce an alteration. I entreated your learned recorder (always ready in business in which you take a concern) to attend. But what will you say to those who blame me for supporting Lord Beauchamp's bill, as a disrespectful treatment of your petition, when you hear, that, out of respect to you, I myself was the cause of the loss of that very bill? For the noble lord who brought it in, and who, I must say, has much merit for this and some other measures, at my request consented to put it off for a week, which the Speaker's illness lengthened to a fortnight; and then the frantic tumult about Popery drove that and every rational business from the House. So that, if I chose to make a defence of myself, on the little principles of a culprit, pleading in his exculpation, I might not only secure my acquittal, but make merit with the opposers of the bill. But I shall do no such thing. The truth is, that I did occasion the loss of the bill, and by a delay caused by my respect to you. But such an event was never in my contemplation. And I am so far from taking credit for the defeat of that measure, that I cannot sufficiently lament my misfortune, if but one man, who ought to be at large, has passed a year in prison by my means. I am a debtor to the debtors. I confess judgment. I owe what, if ever it be in my power, I shall most certainly pay,—ample atonement and usurious amends to liberty and humanity for my unhappy lapse. For, Gentlemen, Lord Beauchamp's bill was a law of justice and policy, as far as it went: I say, as far as it went; for its fault was its being in the remedial part miserably defective.

There are two capital faults in our law with relation to civil debts. One is, that every man is presumed solvent: a presumption, in innumerable cases, directly against truth. Therefore the debtor is ordered, on a supposition of ability and fraud, to be coerced his liberty until he makes payment. By this means, in all cases of civil insolvency, without a pardon from his creditor, he is to be imprisoned for life; and thus a miserable mistaken invention of artificial science operates to change a civil into a criminal judgment, and to scourge misfortune or indiscretion with a punishment which the law does not inflict on the greatest crimes.

The next fault is, that the inflicting of that punishment is not on the opinion of an equal and public judge, but is referred to the arbitrary discretion of a private, nay, interested, and irritated, individual. He, who formally is, and substantially ought to be, the judge, is in reality no more than ministerial, a mere executive instrument of a private man, who is at once judge and party. Every idea of judicial order is subverted by this procedure. If the insolvency be no crime, why is it punished with arbitrary imprisonment? If it be a crime, why is it delivered into private hands to pardon without discretion, or to punish without mercy and without measure?

To these faults, gross and cruel faults in our law, the excellent principle of Lord Beauchamp's bill applied some sort of remedy. I know that credit must be preserved: but equity must be preserved, too; and it is impossible that anything should be necessary to commerce which is inconsistent with justice. The principle of credit was not weakened by that bill. God forbid! The enforcement of that credit was only put into the same public judicial hands on which we depend for our lives and all that makes life dear to us. But, indeed, this business was taken up too warmly, both here and elsewhere. The bill was extremely mistaken. It was supposed to enact what it never enacted; and complaints were made of clauses in it, as novelties, which existed before the noble lord that brought in the bill was born. There was a fallacy that ran through the whole of the objections. The gentlemen who opposed the bill always argued as if the option lay between that bill and the ancient law. But this is a grand mistake. For, practically, the option is between not that bill and the old law, but between that bill and those occasional laws called acts of grace. For the operation of the old law is so savage, and so inconvenient to society, that for a long time past, once in every Parliament, and lately twice, the legislature has been obliged to make a general arbitrary jail-delivery, and at once to set open, by its sovereign authority, all the prisons in England.

Gentlemen, I never relished acts of grace, nor ever submitted to them but from despair of better. They are a dishonorable invention, by which, not from humanity, not from policy, but merely because we have not room enough to hold these victims of the absurdity of our laws, we turn loose upon the public three or four thousand naked wretches, corrupted by the habits, debased by the ignominy of a prison. If the creditor had a right to those carcasses as a natural security for his property, I am sure we have no right to deprive him of that security. But if the few pounds of flesh were not necessary to his security, we had not a right to detain the unfortunate debtor, without any benefit at all to the person who confined him. Take it as you will, we commit injustice. Now Lord Beauchamp's bill intended to do deliberately, and with great caution and circumspection, upon each several case, and with all attention to the just claimant, what acts of grace do in a much greater measure, and with very little care, caution, or deliberation.

I suspect that here, too, if we contrive to oppose this bill, we shall be found in a struggle against the nature of things. For, as we grow enlightened, the public will not bear, for any length of time, to pay for the maintenance of whole armies of prisoners, nor, at their own expense, submit to keep jails as a sort of garrisons, merely to fortify the absurd principle of making men judges in their own cause. For credit has little or no concern in this cruelty. I speak in a commercial assembly. You know that credit is given because capital must be employed; that men calculate the chances of insolvency; and they either withhold the credit, or make the debtor pay the risk in the price. The counting-house has no alliance with the jail. Holland understands trade as well as we, and she has done much more than this obnoxious bill intended to do. There was not, when Mr. Howard visited Holland, more than one prisoner for debt in the great city of Rotterdam. Although Lord Beauchamp's act (which was previous to this bill, and intended to feel the way for it) has already preserved liberty to thousands, and though it is not three years since the last act of grace passed, yet, by Mr. Howard's last account, there were near three thousand again in jail. I cannot name this gentleman without remarking that his labors and writings have done much to open the eyes and hearts of mankind. He has visited all Europe,—not to survey the sumptuousness of palaces or the stateliness of temples, not to make accurate measurements of the remains of ancient grandeur nor to form a scale of the curiosity of modern art, not to collect medals or collate manuscripts,—but to dive into the depths of dungeons, to plunge into the infection of hospitals, to survey the mansions of sorrow and pain, to take the gauge and dimensions of misery, depression, and contempt, to remember the forgotten, to attend to the neglected, to visit the forsaken, and to compare and collate the distresses of all men in all countries. His plan is original; and it is as full of genius as it is of humanity. It was a voyage of discovery, a circumnavigation of charity. Already the benefit of his labor is felt more or less in every country; I hope he will anticipate his final reward by seeing all its effects fully realized in his own. He will receive, not by retail, but in gross, the reward of those who visit the prisoner; and he has so forestalled and monopolized this branch of charity, that there will be, I trust, little room to merit by such acts of benevolence hereafter.

Nothing now remains to trouble you with but the fourth charge against me,—the business of the Roman Catholics. It is a business closely connected with the rest. They are all on one and the same principle. My little scheme of conduct, such as it is, is all arranged. I could do nothing but what I have done on this subject, without confounding the whole train of my ideas and disturbing the whole order of my life. Gentlemen, I ought to apologize to you for seeming to think anything at all necessary to be said upon this matter. The calumny is fitter to be scrawled with the midnight chalk of incendiaries, with "No Popery," on walls and doors of devoted houses, than to be mentioned in any civilized company. I had heard that the spirit of discontent on that subject was very prevalent here. With pleasure I find that I have been grossly misinformed. If it exists at all in this city, the laws have crushed its exertions, and our morals have shamed its appearance in daylight. I have pursued this spirit wherever I could trace it; but it still fled from me. It was a ghost which all had heard of, but none had seen. None would acknowledge that he thought the public proceeding with regard to our Catholic dissenters to be blamable; but several were sorry it had made an ill impression upon others, and that my interest was hurt by my share in the business. I find with satisfaction and pride, that not above four or five in this city (and I dare say these misled by some gross misrepresentation) have signed that symbol of delusion and bond of sedition, that libel on the national religion and English character, the Protestant Association. It is, therefore, Gentlemen, not by way of cure, but of prevention, and lest the arts of wicked men may prevail over the integrity of any one amongst us, that I think it necessary to open to you the merits of this transaction pretty much at large; and I beg your patience upon it: for, although the reasonings that have been used to depreciate the act are of little force, and though the authority of the men concerned in this ill design is not very imposing, yet the audaciousness of these conspirators against the national honor, and the extensive wickedness of their attempts, have raised persons of little importance to a degree of evil eminence, and imparted a sort of sinister dignity to proceedings that had their origin in only the meanest and blindest malice.

In explaining to you the proceedings of Parliament which have been complained of, I will state to you,—first, the thing that was done,—next, the persons who did it,—and lastly, the grounds and reasons upon which the legislature proceeded in this deliberate act of public justice and public prudence.

Gentlemen, the condition of our nature is such that we buy our blessings at a price. The Reformation, one of the greatest periods of human improvement, was a time of trouble and confusion. The vast structure of superstition and tyranny which had been for ages in rearing, and which was combined with the interest of the great and of the many, which was moulded into the laws, the manners, and civil institutions of nations, and blended with the frame and policy of states, could not be brought to the ground without a fearful struggle; nor could it fall without a violent concussion of itself and all about it. When this great revolution was attempted in a more regular mode by government, it was opposed by plots and seditions of the people; when by popular efforts, it was repressed as rebellion by the hand of power; and bloody executions (often bloodily returned) marked the whole of its progress through all its stages. The affairs of religion, which are no longer heard of in the tumult of our present contentions, made a principal ingredient in the wars and politics of that time: the enthusiasm of religion threw a gloom over the politics; and political interests poisoned and perverted the spirit of religion upon all sides. The Protestant religion, in that violent struggle, infected, as the Popish had been before, by worldly interests and worldly passions, became a persecutor in its turn, sometimes of the new sects, which carried their own principles further than it was convenient to the original reformers, and always of the body from whom they parted: and this persecuting spirit arose, not only from the bitterness of retaliation, but from the merciless policy of fear.

It was long before the spirit of true piety and true wisdom, involved in the principles of the Reformation, could be depurated from the dregs and feculence of the contention with which it was carried through. However, until this be done, the Reformation is not complete: and those who think themselves good Protestants, from their animosity to others, are in that respect no Protestants at all. It was at first thought necessary, perhaps, to oppose to Popery another Popery, to get the better of it. Whatever was the cause, laws were made in many countries, and in this kingdom in particular, against Papists, which are as bloody as any of those which had been enacted by the Popish princes and states: and where those laws were not bloody, in my opinion, they were worse; as they were slow, cruel outrages on our nature, and kept men alive only to insult in their persons every one of the rights and feelings of humanity. I pass those statutes, because I would spare your pious ears the repetition of such shocking things; and I come to that particular law the repeal of which has produced so many unnatural and unexpected consequences.

A statute was fabricated in the year 1699, by which the saying mass (a church service in the Latin tongue, not exactly the same as our liturgy, but very near it, and containing no offence whatsoever against the laws, or against good morals) was forged into a crime, punishable with perpetual imprisonment. The teaching school, an useful and virtuous occupation, even the teaching in a private family, was in every Catholic subjected to the same unproportioned punishment. Your industry, and the bread of your children, was taxed for a pecuniary reward to stimulate avarice to do what Nature refused, to inform and prosecute on this law. Every Roman Catholic was, under the same act, to forfeit his estate to his nearest Protestant relation, until, through a profession of what he did not believe, he redeemed by his hypocrisy what the law had transferred to the kinsman as the recompense of his profligacy. When thus turned out of doors from his paternal estate, he was disabled from acquiring any other by any industry, donation, or charity; but was rendered a foreigner in his native land, only because he retained the religion, along with the property, handed down to him from those who had been the old inhabitants of that land before him.

Does any one who hears me approve this scheme of things, or think there is common justice, common sense, or common honesty in any part of it? If any does, let him say it, and I am ready to discuss the point with temper and candor. But instead of approving, I perceive a virtuous indignation beginning to rise in your minds on the mere cold stating of the statute.

But what will you feel, when you know from history how this statute passed, and what were the motives, and what the mode of making it? A party in this nation, enemies to the system of the Revolution, were in opposition to the government of King William. They knew that our glorious deliverer was an enemy to all persecution. They knew that he came to free us from slavery and Popery, out of a country where a third of the people are contented Catholics under a Protestant government. He came with a part of his army composed of those very Catholics, to overset the power of a Popish prince. Such is the effect of a tolerating spirit; and so much is liberty served in every way, and by all persons, by a manly adherence to its own principles. Whilst freedom is true to itself, everything becomes subject to it, and its very adversaries are an instrument in its hands.

The party I speak of (like some amongst us who would disparage the best friends of their country) resolved to make the king either violate his principles of toleration or incur the odium of protecting Papists. They therefore brought in this bill, and made it purposely wicked and absurd that it might be rejected. The then court party, discovering their game, turned the tables on them, and returned their bill to them stuffed with still greater absurdities, that its loss might lie upon its original authors. They, finding their own ball thrown back to them, kicked it back again to their adversaries. And thus this act, loaded with the double injustice of two parties, neither of whom intended to pass what they hoped the other would be persuaded to reject, went through the legislature, contrary to the real wish of all parts of it, and of all the parties that composed it. In this manner these insolent and profligate factions, as if they were playing with balls and counters, made a sport of the fortunes and the liberties of their fellow-creatures. Other acts of persecution have been acts of malice. This was a subversion of justice from wantonness and petulance. Look into the history of Bishop Burnet. He is a witness without exception.

The effects of the act have been as mischievous as its origin was ludicrous and shameful. From that time, every person of that communion, lay and ecclesiastic, has been obliged to fly from the face of day. The clergy, concealed in garrets of private houses, or obliged to take a shelter (hardly safe to themselves, but infinitely dangerous to their country) under the privileges of foreign ministers, officiated as their servants and under their protection. The whole body of the Catholics, condemned to beggary and to ignorance in their native land, have been obliged to learn the principles of letters, at the hazard of all their other principles, from the charity of your enemies. They have been taxed to their ruin at the pleasure of necessitous and profligate relations, and according to the measure of their necessity and profligacy. Examples of this are many and affecting. Some of them are known by a friend who stands near me in this hall. It is but six or seven years since a clergyman, of the name of Malony, a man of morals, neither guilty nor accused of anything noxious to the state, was condemned to perpetual imprisonment for exercising the functions of his religion; and after lying in jail two or three years, was relieved by the mercy of government from perpetual imprisonment, on condition of perpetual banishment. A brother of the Earl of Shrewsbury, a Talbot, a name respectable in this country whilst its glory is any part of its concern, was hauled to the bar of the Old Bailey, among common felons, and only escaped the same doom, either by some error in the process, or that the wretch who brought him there could not correctly describe his person,—I now forget which. In short, the persecution would never have relented for a moment, if the judges, superseding (though with an ambiguous example) the strict rule of their artificial duty by the higher obligation of their conscience, did not constantly throw every difficulty in the way of such informers. But so ineffectual is the power of legal evasion against legal iniquity, that it was but the other day that a lady of condition, beyond the middle of life, was on the point of being stripped of her whole fortune by a near relation to whom she had been a friend and benefactor; and she must have been totally ruined, without a power of redress or mitigation from the courts of law, had not the legislature itself rushed in, and by a special act of Parliament rescued her from the injustice of its own statutes. One of the acts authorizing such things was that which we in part repealed, knowing what our duty was, and doing that duty as men of honor and virtue, as good Protestants, and as good citizens. Let him stand forth that disapproves what we have done!

Gentlemen, bad laws are the worst sort of tyranny. In such a country as this they are of all bad things the worst,—worse by far than anywhere else; and they derive a particular malignity even from the wisdom and soundness of the rest of our institutions. For very obvious reasons you cannot trust the crown with a dispensing power over any of your laws. However, a government, be it as bad as it may, will, in the exercise of a discretionary power, discriminate times and persons, and will not ordinarily pursue any man, when its own safety is not concerned. A mercenary informer knows no distinction. Under such a system, the obnoxious people are slaves not only to the government, but they live at the mercy of every individual; they are at once the slaves of the whole community and of every part of it; and the worst and most unmerciful men are those on whose goodness they most depend.

In this situation, men not only shrink from the frowns of a stern magistrate, but they are obliged to fly from their very species. The seeds of destruction are sown in civil intercourse, in social habitudes. The blood of wholesome kindred is infected. Their tables and beds are surrounded with snares. All the means given by Providence to make life safe and comfortable are perverted into instruments of terror and torment. This species of universal subserviency, that makes the very servant who waits behind your chair the arbiter of your life and fortune, has such a tendency to degrade and abase mankind, and to deprive them of that assured and liberal state of mind which alone can make us what we ought to be, that I vow to God I would sooner bring myself to put a man to immediate death for opinions I disliked, and so to get rid of the man and his opinions at once, than to fret him with a feverish being, tainted with the jail-distemper of a contagious servitude, to keep him above ground an animated mass of putrefaction, corrupted himself, and corrupting all about him.

The act repealed was of this direct tendency; and it was made in the manner which I have related to you. I will now tell you by whom the bill of repeal was brought into Parliament. I find it has been industriously given out in this city (from kindness to me, unquestionably) that I was the mover or the seconder. The fact is, I did not once open my lips on the subject during the whole progress of the bill. I do not say this as disclaiming my share in that measure. Very far from it. I inform you of this fact, lest I should seem to arrogate to myself the merits which belong to others. To have been the man chosen out to redeem our fellow-citizens from slavery, to purify our laws from absurdity and injustice, and to cleanse our religion from the blot and stain of persecution, would be an honor and happiness to which my wishes would undoubtedly aspire, but to which nothing but my wishes could possibly have entitled me. That great work was in hands in every respect far better qualified than mine. The mover of the bill was Sir George Savile.

When an act of great and signal humanity was to be done, and done with all the weight and authority that belonged to it, the world could cast its eyes upon none but him. I hope that few things which have a tendency to bless or to adorn life have wholly escaped my observation in my passage through it. I have sought the acquaintance of that gentleman, and have seen him in all situations. He is a true genius; with an understanding vigorous, and acute, and refined, and distinguishing even to excess; and illuminated with a most unbounded, peculiar, and original cast of imagination. With these he possesses many external and instrumental advantages; and he makes use of them all. His fortune is among the largest,—a fortune which, wholly unincumbered as it is with one single charge from luxury, vanity, or excess, sinks under the benevolence of its dispenser. This private benevolence, expanding itself into patriotism, renders his whole being the estate of the public, in which he has not reserved a peculium for himself of profit, diversion, or relaxation. During the session the first in and the last out of the House of Commons, he passes from the senate to the camp; and seldom seeing the seat of his ancestors, he is always in Parliament to serve his country or in the field to defend it. But in all well-wrought compositions some particulars stand out more eminently than the rest; and the things which will carry his name to posterity are his two bills: I mean that for a limitation of the claims of the crown upon landed estates, and this for the relief of the Roman Catholics. By the former he has emancipated property; by the latter he has quieted conscience; and by both he has taught that grand lesson to government and subject,—no longer to regard each other as adverse parties.

Such was the mover of the act that is complained of by men who are not quite so good as he is,—an act most assuredly not brought in by him from any partiality to that sect which is the object of it. For among his faults I really cannot help reckoning a greater degree of prejudice against that people than becomes so wise a man. I know that he inclines to a sort of disgust, mixed with a considerable degree of asperity, to the system; and he has few, or rather no habits with any of its professors. What he has done was on quite other motives. The motives were these, which he declared in his excellent speech on his motion for the bill: namely, his extreme zeal to the Protestant religion, which he thought utterly disgraced by the act of 1699; and his rooted hatred to all kind of oppression, under any color, or upon any pretence whatsoever.

The seconder was worthy of the mover and the motion. I was not the seconder; it was Mr. Dunning, recorder of this city. I shall say the less of him because his near relation to you makes you more particularly acquainted with his merits. But I should appear little acquainted with them, or little sensible of them, if I could utter his name on this occasion without expressing my esteem for his character. I am not afraid of offending a most learned body, and most jealous of its reputation for that learning, when I say he is the first of his profession. It is a point settled by those who settle everything else; and I must add (what I am enabled to say from my own long and close observation) that there is not a man, of any profession, or in any situation, of a more erect and independent spirit, of a more proud honor, a more manly mind, a more firm and determined integrity. Assure yourselves, that the names of two such men will bear a great load of prejudice in the other scale before they can be entirely outweighed.

With this mover and this seconder agreed the whole House of Commons, the whole House of Lords, the whole Bench of Bishops, the king, the ministry, the opposition, all the distinguished clergy of the Establishment, all the eminent lights (for they were consulted) of the dissenting churches. This according voice of national wisdom ought to be listened to with reverence. To say that all these descriptions of Englishmen unanimously concurred in a scheme for introducing the Catholic religion, or that none of them understood the nature and effects of what they were doing so well as a few obscure clubs of people whose names you never heard of, is shamelessly absurd. Surely it is paying a miserable compliment to the religion we profess, to suggest that everything eminent in the kingdom is indifferent or even adverse to that religion, and that its security is wholly abandoned to the zeal of those who have nothing but their zeal to distinguish them. In weighing this unanimous concurrence of whatever the nation has to boast of, I hope you will recollect that all these concurring parties do by no means love one another enough to agree in any point which was not both evidently and importantly right.

To prove this, to prove that the measure was both clearly and materially proper, I will next lay before you (as I promised) the political grounds and reasons for the repeal of that penal statute, and the motives to its repeal at that particular time.

Gentlemen, America—When the English nation seemed to be dangerously, if not irrecoverably divided,—when one, and that the most growing branch, was torn from the parent stock, and ingrafted on the power of France, a great terror fell upon this kingdom. On a sudden we awakened from our dreams of conquest, and saw ourselves threatened with an immediate invasion, which we were at that time very ill prepared to resist. You remember the cloud that gloomed over us all. In that hour of our dismay, from the bottom of the hiding-places into which the indiscriminate rigor of our statutes had driven them, came out the body of the Roman Catholics. They appeared before the steps of a tottering throne, with one of the most sober, measured, steady, and dutiful addresses that was ever presented to the crown. It was no holiday ceremony, no anniversary compliment of parade and show. It was signed by almost every gentleman of that persuasion, of note or property, in England. At such a crisis, nothing but a decided resolution to stand or fall with their country could have dictated such an address, the direct tendency of which was to cut off all retreat, and to render them peculiarly obnoxious to an invader of their own communion. The address showed what I long languished to see, that all the subjects of England had cast off all foreign views and connections, and that every man looked for his relief from every grievance at the hands only of his own natural government.

It was necessary, on our part, that the natural government should show itself worthy of that name. It was necessary, at the crisis I speak of, that the supreme power of the state should meet the conciliatory dispositions of the subject. To delay protection would be to reject allegiance. And why should it be rejected, or even coldly and suspiciously received? If any independent Catholic state should choose to take part with this kingdom in a war with France and Spain, that bigot (if such a bigot could be found) would be heard with little respect, who could dream of objecting his religion to an ally whom the nation would not only receive with its freest thanks, but purchase with the last remains of its exhausted treasure. To such an ally we should not dare to whisper a single syllable of those base and invidious topics upon which some unhappy men would persuade the state to reject the duty and allegiance of its own members. Is it, then, because foreigners are in a condition to set our malice at defiance, that with them we are willing to contract engagements of friendship, and to keep them with fidelity and honor, but that, because we conceive some descriptions of our countrymen are not powerful enough to punish our malignity, we will not permit them to support our common interest? Is it on that ground that our anger is to be kindled by their offered kindness? Is it on that ground that they are to be subjected to penalties, because they are willing by actual merit to purge themselves from imputed crimes? Lest by an adherence to the cause of their country they should acquire a title to fair and equitable treatment, are we resolved to furnish them with causes of eternal enmity, and rather supply them with just and founded motives to disaffection than not to have that disaffection in existence to justify an oppression which, not from policy, but disposition, we have predetermined to exercise?

What shadow of reason could be assigned, why, at a time when the most Protestant part of this Protestant empire found it for its advantage to unite with the two principal Popish states, to unite itself in the closest bonds with France and Spain, for our destruction, that we should refuse to unite with our own Catholic countrymen for our own preservation? Ought we, like madmen, to tear off the plasters that the lenient hand of prudence had spread over the wounds and gashes which in our delirium of ambition we had given to our own body? No person ever reprobated the American war more than I did, and do, and ever shall. But I never will consent that we should lay additional, voluntary penalties on ourselves, for a fault which carries but too much of its own punishment in its own nature. For one, I was delighted with the proposal of internal peace. I accepted the blessing with thankfulness and transport. I was truly happy to find one good effect of our civil distractions: that they had put an end to all religious strife and heart-burning in our own bowels. What must be the sentiments of a man who would wish to perpetuate domestic hostility when the causes of dispute are at an end, and who, crying out for peace with one part of the nation on the most humiliating terms, should deny it to those who offer friendship without any terms at all?

But if I was unable to reconcile such a denial to the contracted principles of local duty, what answer could I give to the broad claims of general humanity? I confess to you freely, that the sufferings and distresses of the people of America in this cruel war have at times affected me more deeply than I can express. I felt every gazette of triumph as a blow upon my heart, which has an hundred times sunk and fainted within me at all the mischiefs brought upon those who bear the whole brunt of war in the heart of their country. Yet the Americans are utter strangers to me; a nation among whom I am not sure that I have a single acquaintance. Was I to suffer my mind to be so unaccountably warped, was I to keep such iniquitous weights and measures of temper and of reason, as to sympathize with those who are in open rebellion against an authority which I respect, at war with a country which by every title ought to be, and is, most dear to me,—and yet to have no feeling at all for the hardships and indignities suffered by men who by their very vicinity are bound up in a nearer relation to us, who contribute their share, and more than their share, to the common prosperity, who perform the common offices of social life, and who obey the laws, to the full as well as I do? Gentlemen, the danger to the state being out of the question, (of which, let me tell you, statesmen themselves are apt to have but too exquisite a sense,) I could assign no one reason of justice, policy, or feeling, for not concurring most cordially, as most cordially I did concur, in softening some part of that shameful servitude under which several of my worthy fellow-citizens were groaning.

Important effects followed this act of wisdom. They appeared at home and abroad, to the great benefit of this kingdom, and, let me hope, to the advantage of mankind at large. It betokened union among ourselves. It showed soundness, even on the part of the persecuted, which generally is the weak side of every community. But its most essential operation was not in England. The act was immediately, though very imperfectly, copied in Ireland; and this imperfect transcript of an imperfect act, this first faint sketch of toleration, which did little more than disclose a principle and mark out a disposition, completed in a most wonderful manner the reunion to the state of all the Catholics of that country. It made us what we ought always to have been, one family, one body, one heart and soul, against the family combination and all other combinations of our enemies. We have, indeed, obligations to that people, who received such small benefits with so much gratitude, and for which gratitude and attachment to us I am afraid they have suffered not a little in other places.

I dare say you have all hoard of the privileges indulged to the Irish Catholics residing in Spain. You have likewise heard with what circumstances of severity they have been lately expelled from the seaports of that kingdom, driven into the inland cities, and there detained as a sort of prisoners of state. I have good reason to believe that it was the zeal to our government and our cause (somewhat indiscreetly expressed in one of the addresses of the Catholics of Ireland) which has thus drawn down on their heads the indignation of the court of Madrid, to the inexpressible loss of several individuals, and, in future, perhaps to the great detriment of the whole of their body. Now that our people should be persecuted in Spain for their attachment to this country, and persecuted in this country for their supposed enmity to us, is such a jarring reconciliation of contradictory distresses, is a thing at once so dreadful and ridiculous, that no malice short of diabolical would wish to continue any human creatures in such a situation. But honest men will not forget either their merit or their sufferings. There are men (and many, I trust, there are) who, out of love to their country and their kind, would torture their invention to find excuses for the mistakes of their brethren, and who, to stifle dissension, would construe even doubtful appearances with the utmost favor: such men will never persuade themselves to be ingenious and refined in discovering disaffection and treason in the manifest, palpable signs of suffering loyalty. Persecution is so unnatural to them, that they gladly snatch the very first opportunity of laying aside all the tricks and devices of penal politics, and of returning home, after all their irksome and vexatious wanderings, to our natural family mansion, to the grand social principle that unites all men, in all descriptions, under the shadow of an equal and impartial justice.

Men of another sort, I mean the bigoted enemies to liberty, may, perhaps, in their politics, make no account of the good or ill affection of the Catholics of England, who are but an handful of people, (enough to torment, but not enough to fear,) perhaps not so many, of both sexes and of all ages, as fifty thousand. But, Gentlemen, it is possible you may not know that the people of that persuasion in Ireland amount at least to sixteen or seventeen hundred thousand souls. I do not at all exaggerate the number. A nation to be persecuted! Whilst we were masters of the sea, embodied with America, and in alliance with half the powers of the Continent, we might, perhaps, in that remote corner of Europe, afford to tyrannize with impunity. But there is a revolution in our affairs, which makes it prudent to be just. In our late awkward contest with Ireland about trade, had religion been thrown in, to ferment and embitter the mass of discontents, the consequences might have been truly dreadful. But, very happily, that cause of quarrel was previously quieted by the wisdom of the acts I am commending.

Even in England, where I admit the danger from the discontent of that persuasion to be less than in Ireland, yet even here, had we listened to the counsels of fanaticism and folly, we might have wounded ourselves very deeply, and wounded ourselves in a very tender part. You are apprised that the Catholics of England consist mostly of our best manufacturers. Had the legislature chosen, instead of returning their declarations of duty with correspondent good-will, to drive them to despair, there is a country at their very door to which they would be invited,—a country in all respects as good as ours, and with the finest cities in the world ready built to receive them. And thus the bigotry of a free country, and in an enlightened age, would have repeopled the cities of Flanders, which, in the darkness of two hundred years ago, had been desolated by the superstition of a cruel tyrant. Oar manufactures were the growth of the persecutions in the Low Countries. What a spectacle would it be to Europe, to see us at this time of day balancing the account of tyranny with those very countries, and by our persecutions driving back trade and manufacture, as a sort of vagabonds, to their original settlement! But I trust we shall be saved this last of disgraces.

So far as to the effect of the act on the interests of this nation. With regard to the interests of mankind at large, I am sure the benefit was very considerable. Long before this act, indeed, the spirit of toleration began to gain ground in Europe. In Holland the third part of the people are Catholics; they live at ease, and are a sound part of the state. In many parts of Germany, Protestants and Papists partake the same cities, the same councils, and even the same churches. The unbounded liberality of the king of Prussia's conduct on this occasion is known to all the world; and it is of a piece with the other grand maxims of his reign. The magnanimity of the Imperial court, breaking through the narrow principles of its predecessors, has indulged its Protestant subjects, not only with property, with worship, with liberal education, but with honors and trusts, both civil and military. A worthy Protestant gentleman of this country now fills, and fills with credit, an high office in the Austrian Netherlands. Even the Lutheran obstinacy of Sweden has thawed at length, and opened a toleration to all religions. I know, myself, that in France the Protestants begin to be at rest. The army, which in that country is everything, is open to them; and some of the military rewards and decorations which the laws deny are supplied by others, to make the service acceptable and honorable. The first minister of finance in that country is a Protestant. Two years' war without a tax is among the first fruits of their liberality. Tarnished as the glory of this nation is, and far as it has waded into the shades of an eclipse, some beams of its former illumination still play upon its surface; and what is done in England is still looked to, as argument, and as example. It is certainly true, that no law of this country ever met with such universal applause abroad, or was so likely to produce the perfection of that tolerating spirit which, as I observed, has been long gaining ground in Europe: for abroad it was universally thought that we had done what I am sorry to say we had not; they thought we had granted a full toleration. That opinion was, however, so far from hurting the Protestant cause, that I declare, with the most serious solemnity, my firm belief that no one thing done for these fifty years past was so likely to prove deeply beneficial to our religion at large as Sir George Savile's act. In its effects it was "an act for tolerating and protecting Protestantism throughout Europe"; and I hope that those who were taking steps for the quiet and settlement of our Protestant brethren in other countries will, even yet, rather consider the steady equity of the greater and better part of the people of Great Britain than the vanity and violence of a few.

I perceive, Gentlemen, by the manner of all about me, that you look with horror on the wicked clamor which has been raised on this subject, and that, instead of an apology for what was done, you rather demand from me an account, why the execution of the scheme of toleration was not made more answerable to the large and liberal grounds on which it was taken up. The question is natural and proper; and I remember that a great and learned magistrate,
%[50]
\footnote{ The Chancellor.}
 distinguished for his strong and systematic understanding, and who at that time was a member of the House of Commons, made the same objection to the proceeding. The statutes, as they now stand, are, without doubt, perfectly absurd. But I beg leave to explain the cause of this gross imperfection in the tolerating plan, as well and as shortly as I am able. It was universally thought that the session ought not to pass over without doing something in this business. To revise the whole body of the penal statutes was conceived to be an object too big for the time. The penal statute, therefore, which was chosen for repeal (chosen to show our disposition to conciliate, not to perfect a toleration) was this act of ludicrous cruelty of which I have just given you the history. It is an act which, though not by a great deal so fierce and bloody as some of the rest, was infinitely more ready in the execution. It was the act which gave the greatest encouragement to those pests of society, mercenary informers and interested disturbers of household peace; and it was observed with truth, that the prosecutions, either carried to conviction or compounded, for many years, had been all commenced upon that act. It was said, that, whilst we were deliberating on a more perfect scheme, the spirit of the age would never come up to the execution of the statutes which remained, especially as more steps, and a coöperation of more minds and powers, were required towards a mischievous use of them, than for the execution of the act to be repealed: that it was better to unravel this texture from below than from above, beginning with the latest, which, in general practice, is the severest evil. It was alleged, that this slow proceeding would be attended with the advantage of a progressive experience,—and that the people would grow reconciled to toleration, when they should find, by the effects, that justice was not so irreconcilable an enemy to convenience as they had imagined.

These, Gentlemen, were the reasons why we left this good work in the rude, unfinished state in which good works are commonly left, through the tame circumspection with which a timid prudence so frequently enervates beneficence. In doing good, we are generally cold, and languid, and sluggish, and of all things afraid of being too much in the right. But the works of malice and injustice are quite in another style. They are finished with a bold, masterly hand, touched as they are with the spirit of those vehement passions that call forth all our energies, whenever we oppress and persecute.

Thus this matter was left for the time, with a full determination in Parliament not to suffer other and worse statutes to remain for the purpose of counteracting the benefits proposed by the repeal of one penal law: for nobody then dreamed of defending what was done as a benefit, on the ground of its being no benefit at all. We were not then ripe for so mean a subterfuge.

I do not wish to go over the horrid scene that was afterwards acted. Would to God it could be expunged forever from the annals of this country! But since it must subsist for our shame, let it subsist for our instruction. In the year 1780 there were found in this nation men deluded enough, (for I give the whole to their delusion,) on pretences of zeal and piety, without any sort of provocation whatsoever, real or pretended, to make a desperate attempt, which would have consumed all the glory and power of this country in the flames of London, and buried all law, order, and religion under the ruins of the metropolis of the Protestant world. Whether all this mischief done, or in the direct train of doing, was in their original scheme, I cannot say; I hope it was not: but this would have been the unavoidable consequence of their proceedings, had not the flames they had lighted up in their fury been extinguished in their blood.

All the time that this horrid scene was acting, or avenging, as well as for some time before, and ever since, the wicked instigators of this unhappy multitude, guilty, with every aggravation, of all their crimes, and screened in a cowardly darkness from their punishment, continued, without interruption, pity, or remorse, to blow up the blind rage of the populace with a continued blast of pestilential libels, which infected and poisoned the very air we breathed in.

The main drift of all the libels and all the riots was, to force Parliament (to persuade us was hopeless) into an act of national perfidy which has no example. For, Gentlemen, it is proper you should all know what infamy we escaped by refusing that repeal, for a refusal of which, it seems, I, among others, stand somewhere or other accused. When we took away, on the motives which I had the honor of stating to you, a few of the innumerable penalties upon an oppressed and injured people, the relief was not absolute, but given on a stipulation and compact between them and us: for we bound down the Roman Catholics with the most solemn oaths to bear true allegiance to this government, to abjure all sort of temporal power in any other, and to renounce, under the same solemn obligations, the doctrines of systematic perfidy with which they stood (I conceive very unjustly) charged. Now our modest petitioners came up to us, most humbly praying nothing more than that we should break our faith, without any one cause whatsoever of forfeiture assigned; and when the subjects of this kingdom had, on their part, fully performed their engagement, we should refuse, on our part, the benefit we had stipulated on the performance of those very conditions that were prescribed by our own authority, and taken on the sanction of our public faith: that is to say, when we had inveigled them with fair promises within our door, we were to shut it on them, and, adding mockery to outrage, to tell them,—"Now we have got you fast: your consciences are bound to a power resolved on your destruction. We have made you swear that your religion obliges you to keep your faith: fools as you are! we will now let you see that our religion enjoins us to keep no faith with you." They who would advisedly call upon us to do such things must certainly have thought us not only a convention of treacherous tyrants, but a gang of the lowest and dirtiest wretches that ever disgraced humanity. Had we done this, we should have indeed proved that there were some in the world whom no faith could bind; and we should have convicted ourselves of that odious principle of which Papists stood accused by those very savages who wished us, on that accusation, to deliver them over to their fury.

In this audacious tumult, when our very name and character as gentlemen was to be cancelled forever, along with the faith and honor of the nation, I, who had exerted myself very little on the quiet passing of the bill, thought it necessary then to come forward. I was not alone; but though some distinguished members on all sides, and particularly on ours, added much to their high reputation by the part they took on that day, (a part which will be remembered as long as honor, spirit, and eloquence have estimation in the world,) I may and will value myself so far, that, yielding in abilities to many, I yielded in zeal to none. With warmth and with vigor, and animated with a just and natural indignation, I called forth every faculty that I possessed, and I directed it in every way in which I could possibly employ it. I labored night and day. I labored in Parliament; I labored out of Parliament. If, therefore, the resolution of the House of Commons, refusing to commit this act of unmatched turpitude, be a crime, I am guilty among the foremost. But, indeed, whatever the faults of that House may have been, no one member was found hardy enough to propose so infamous a thing; and on full debate we passed the resolution against the petitions with as much unanimity as we had formerly passed the law of which these petitions demanded the repeal.

There was a circumstance (justice will not suffer me to pass it over) which, if anything could enforce the reasons I have given, would fully justify the act of relief, and render a repeal, or anything like a repeal, unnatural, impossible. It was the behavior of the persecuted Roman Catholics under the acts of violence and brutal insolence which they suffered. I suppose there are not in London less than four or five thousand of that persuasion from my country, who do a great deal of the most laborious works in the metropolis; and they chiefly inhabit those quarters which were the principal theatre of the fury of the bigoted multitude. They are known to be men of strong arms and quick feelings, and more remarkable for a determined resolution than clear ideas or much foresight. But, though provoked by everything that can stir the blood of men, their houses and chapels in flames, and with the most atrocious profanations of everything which they hold sacred before their eyes, not a hand was moved to retaliate, or even to defend. Had a conflict once begun, the rage of their persecutors would have redoubled. Thus fury increasing by the reverberation of outrages, house being fired for house, and church for chapel, I am convinced that no power under heaven could have prevented a general conflagration, and at this day London would have been a tale. But I am well informed, and the thing speaks it, that their clergy exerted their whole influence to keep their people in such a state of forbearance and quiet, as, when I look back, fills me with astonishment,—but not with astonishment only. Their merits on that occasion ought not to be forgotten; nor will they, when Englishmen come to recollect themselves. I am sure it were far more proper to have called them forth, and given them the thanks of both Houses of Parliament, than to have suffered those worthy clergymen and excellent citizens to be hunted into holes and corners, whilst we are making low-minded inquisitions into the number of their people; as if a tolerating principle was never to prevail, unless we were very sure that only a few could possibly take advantage of it. But, indeed, we are not yet well recovered of our fright. Our reason, I trust, will return with our security, and this unfortunate temper will pass over like a cloud.

Gentlemen, I have now laid before you a few of the reasons for taking away the penalties of the act of 1699, and for refusing to establish them on the riotous requisition of 1780. Because I would not suffer anything which may be for your satisfaction to escape, permit me just to touch on the objections urged against our act and our resolves, and intended as a justification of the violence offered to both Houses. "Parliament," they assert, "was too hasty, and they ought, in so essential and alarming a change, to have proceeded with a far greater degree of deliberation." The direct contrary. Parliament was too slow. They took fourscore years to deliberate on the repeal of an act which ought not to have survived a second session. When at length, after a procrastination of near a century, the business was taken up, it proceeded in the most public manner, by the ordinary stages, and as slowly as a law so evidently right as to be resisted by none would naturally advance. Had it been read three times in one day, we should have shown only a becoming readiness to recognize, by protection, the undoubted dutiful behavior of those whom we had but too long punished for offences of presumption or conjecture. But for what end was that bill to linger beyond the usual period of an unopposed measure? Was it to be delayed until a rabble in Edinburgh should dictate to the Church of England what measure of persecution was fitting for her safety? Was it to be adjourned until a fanatical force could be collected in London, sufficient to frighten us out of all our ideas of policy and justice? Were we to wait for the profound lectures on the reason of state, ecclesiastical and political, which the Protestant Association have since condescended to read to us? Or were we, seven hundred peers and commoners, the only persons ignorant of the ribald invectives which occupy the place of argument in those remonstrances, which every man of common observation had heard a thousand times over, and a thousand times over had despised? All men had before heard what they dare to say, and all men at this day know what they dare to do; and I trust all honest men are equally influenced by the one and by the other.

But they tell us, that those our fellow-citizens whose chains we have a little relaxed are enemies to liberty and our free Constitution.—Not enemies, I presume, to their own liberty. And as to the Constitution, until we give them some share in it, I do not know on what pretence we can examine into their opinions about a business in which they have no interest or concern. But, after all, are we equally sure that they are adverse to our Constitution as that our statutes are hostile and destructive to them? For my part, I have reason to believe their opinions and inclinations in that respect are various, exactly like those of other men; and if they lean more to the crown than I and than many of you think we ought, we must remember that he who aims at another's life is not to be surprised, if he flies into any sanctuary that will receive him. The tenderness of the executive power is the natural asylum of those upon whom the laws have declared war; and to complain that men are inclined to favor the means of their own safety is so absurd, that one forgets the injustice in the ridicule.

I must fairly tell you, that so far as my principles are concerned, (principles that I hope will only depart with my last breath,) that I have no idea of a liberty unconnected with honesty and justice. Nor do I believe that any good constitutions of government, or of freedom, can find it necessary for their security to doom any part of the people to a permanent slavery. Such a constitution of freedom, if such can be, is in effect no more than another name for the tyranny of the strongest faction; and factions in republics have been, and are, full as capable as monarchs of the most cruel oppression and injustice. It is but too true, that the love, and even the very idea, of genuine liberty is extremely rare. It is but too true that there are many whose whole scheme of freedom is made up of pride, perverseness, and insolence. They feel themselves in a state of thraldom, they imagine that their souls are cooped and cabined in, unless they have some man or some body of men dependent on their mercy. This desire of having some one below them descends to those who are the very lowest of all; and a Protestant cobbler, debased by his poverty, but exalted by his share of the ruling church, feels a pride in knowing it is by his generosity alone that the peer whose footman's instep he measures is able to keep his chaplain from a jail. This disposition is the true source of the passion which many men in very humble life have taken to the American war. Our subjects in America; our colonies; our dependants. This lust of party power is the liberty they hunger and thirst for; and this Siren song of ambition has charmed ears that one would have thought were never organized to that sort of music.

This way of proscribing the citizens by denominations and general descriptions, dignified by the name of reason of state, and security for constitutions and commonwealths, is nothing better at bottom than the miserable invention of an ungenerous ambition which would fain hold the sacred trust of power, without any of the virtues or any of the energies that give a title to it,—a receipt of policy, made up of a detestable compound of malice, cowardice, and sloth. They would govern men against their will; but in that government they would be discharged from the exercise of vigilance, providence, and fortitude; and therefore, that they may sleep on their watch, they consent to take some one division of the society into partnership of the tyranny over the rest. But let government, in what form it may be, comprehend the whole in its justice, and restrain the suspicious by its vigilance,—let it keep watch and ward,—let it discover by its sagacity, and punish by its firmness, all delinquency against its power, whenever delinquency exists in the overt acts,—and then it will be as safe as ever God and Nature intended it should be. Crimes are the acts of individuals, and not of denominations: and therefore arbitrarily to class men under general descriptions, in order to proscribe and punish them in the lump for a presumed delinquency, of which perhaps but a part, perhaps none at all, are guilty, is indeed a compendious method, and saves a world of trouble about proof; but such a method, instead of being law, is an act of unnatural rebellion against the legal dominion of reason and justice; and this vice, in any constitution that entertains it, at one time or other will certainly bring on its ruin.

We are told that this is not a religious persecution; and its abettors are loud in disclaiming all severities on account of conscience. Very fine indeed! Then, let it be so: they are not persecutors; they are only tyrants. With all my heart. I am perfectly indifferent concerning the pretexts upon which we torment one another,—or whether it be for the constitution of the Church of England, or for the constitution of the State of England, that people choose to make their fellow-creatures wretched. When we were sent into a place of authority, you that sent us had yourselves but one commission to give. You could give us none to wrong or oppress, or even to suffer any kind of oppression or wrong, on any grounds whatsoever: not on political, as in the affairs of America; not on commercial, as in those of Ireland; not in civil, as in the laws for debt; not in religious, as in the statutes against Protestant or Catholic dissenters. The diversified, but connected, fabric of universal justice is well cramped and bolted together in all its parts; and depend upon it, I never have employed, and I never shall employ, any engine of power which may come into my hands to wrench it asunder. All shall stand, if I can help it, and all shall stand connected. After all, to complete this work, much remains to be done: much in the East, much in the West. But, great as the work is, if our will be ready, our powers are not deficient.

Since you have suffered me to trouble you so much on this subject, permit me, Gentlemen, to detain you a little longer. I am, indeed, most solicitous to give you perfect satisfaction. I find there are some of a better and softer nature than the persons with whom I have supposed myself in debate, who neither think ill of the act of relief, nor by any means desire the repeal,—yet who, not accusing, but lamenting, what was done, on account of the consequences, have frequently expressed their wish that the late act had never been made. Some of this description, and persons of worth, I have met with in this city. They conceive that the prejudices, whatever they might be, of a large part of the people, ought not to have been shocked,—that their opinions ought to have been previously taken, and much attended to,—and that thereby the late horrid scenes might have been prevented.

I confess, my notions are widely different; and I never was less sorry for any action of my life. I like the bill the better on account of the events of all kinds that followed it. It relieved the real sufferers; it strengthened the state; and, by the disorders that ensued, we had clear evidence that there lurked a temper somewhere which ought not to be fostered by the laws. No ill consequences whatever could be attributed to the act itself. We knew beforehand, or we were poorly instructed, that toleration is odious to the intolerant, freedom to oppressors, property to robbers, and all kinds and degrees of prosperity to the envious. We knew that all these kinds of men would gladly gratify their evil dispositions under the sanction of law and religion, if they could: if they could not, yet, to make way to their objects, they would do their utmost to subvert all religion and all law. This we certainly knew. But, knowing this, is there any reason, because thieves break in and steal, and thus bring detriment to you, and draw ruin on themselves, that I am to be sorry that you are in possession of shops, and of warehouses, and of wholesome laws to protect them? Are you to build no houses, because desperate men may pull them down upon their own heads? Or, if a malignant wretch will cut his own throat, because he sees you give alms to the necessitous and deserving, shall his destruction be attributed to your charity, and not to his own deplorable madness? If we repent of our good actions, what, I pray you, is left for our faults and follies? It is not the beneficence of the laws, it is the unnatural temper which beneficence can fret and sour, that is to be lamented. It is this temper which, by all rational means, ought to be sweetened and corrected. If froward men should refuse this cure, can they vitiate anything but themselves? Does evil so react upon good, as not only to retard its motion, but to change its nature? If it can so operate, then good men will always be in the power of the bad,—and virtue, by a dreadful reverse of order, must lie under perpetual subjection and bondage to vice.

As to the opinion of the people, which some think, in such cases, is to be implicitly obeyed,—near two years' tranquillity, which follows the act, and its instant imitation in Ireland, proved abundantly that the late horrible spirit was in a great measure the effect of insidious art, and perverse industry, and gross misrepresentation. But suppose that the dislike had been much more deliberate and much more general than I am persuaded it was,—when we know that the opinions of even the greatest multitudes are the standard of rectitude, I shall think myself obliged to make those opinions the masters of my conscience. But if it may be doubted whether Omnipotence itself is competent to alter the essential constitution of right and wrong, sure I am that such things as they and I are possessed of no such power. No man carries further than I do the policy of making government pleasing to the people. But the widest range of this politic complaisance is confined within the limits of justice. I would not only consult the interest of the people, but I would cheerfully gratify their humors. We are all a sort of children that must be soothed and managed. I think I am not austere or formal in my nature. I would bear, I would even play my part in, any innocent buffooneries, to divert them. But I never will act the tyrant for their amusement. If they will mix malice in their sports, I shall never consent to throw them any living, sentient creature whatsoever, no, not so much as a kitling, to torment.

"But if I profess all this impolitic stubbornness, I may chance never to be elected into Parliament."—It is certainly not pleasing to be put out of the public service. But I wish to be a member of Parliament to have my share of doing good and resisting evil. It would therefore be absurd to renounce my objects in order to obtain my seat. I deceive myself, indeed, most grossly, if I had not much rather pass the remainder of my life hidden in the recesses of the deepest obscurity, feeding my mind even with the visions and imaginations of such things, than to be placed on the most splendid throne of the universe, tantalized with a denial of the practice of all which can make the greatest situation any other than the greatest curse. Gentlemen, I have had my day. I can never sufficiently express my gratitude to you for having set me in a place wherein I could lend the slightest help to great and laudable designs. If I have had my share in any measure giving quiet to private property and private conscience,—if by my vote I have aided in securing to families the best possession, peace,—if I have joined in reconciling kings to their subjects, and subjects to their prince,—if I have assisted to loosen the foreign holdings of the citizen, and taught him to look for his protection to the laws of his country, and for his comfort to the good-will of his countrymen,—if I have thus taken my part with the best of men in the best of their actions, I can shut the book: I might wish to read a page or two more, but this is enough for my measure. I have not lived in vain.

And now, Gentlemen, on this serious day, when I come, as it were, to make up my account with you, let me take to myself some degree of honest pride on the nature of the charges that are against me. I do not here stand before you accused of venality, or of neglect of duty. It is not said, that, in the long period of my service, I have, in a single instance, sacrificed the slightest of your interests to my ambition or to my fortune. It is not alleged, that, to gratify any anger or revenge of my own, or of my party, I have had a share in wronging or oppressing any description of men, or any one man in any description. No! the charges against me are all of one kind: that I have pushed the principles of general justice and benevolence too far,—further than a cautious policy would warrant, and further than the opinions of many would go along with me. In every accident which may happen through life, in pain, in sorrow, in depression, and distress, I will call to mind this accusation, and be comforted.

Gentlemen, I submit the whole to your judgment. Mr. Mayor, I thank you for the trouble you have taken on this occasion: in your state of health it is particularly obliging. If this company should think it advisable for me to withdraw, I shall respectfully retire; if you think otherwise, I shall go directly to the Council-House and to the 'Change, and without a moment's delay begin my canvass.

\PRLsep

\begin{itpars}
\hfill
BRISTOL, September 6, 1780.

At a great and respectable meeting of the friends of EDMUND BURKE, Esq., held at the Guildhall this day, the Right Worshipful the Mayor in the chair:—Resolved, That Mr. Burke, as a representative for this city, has done all possible honor to himself as a senator and a man, and that we do heartily and honestly approve of his conduct, as the result of an enlightened loyalty to his sovereign, a warm and zealous love to his country through its widely extended empire, a jealous and watchful care of the liberties of his fellow-subjects, an enlarged and liberal understanding of our commercial interest, a humane attention to the circumstances of even the lowest ranks of the community, and a truly wise, politic, and tolerant spirit, in supporting the national church, with a reasonable indulgence to all who dissent from it; and we wish to express the most marked abhorrence of the base arts which have been employed, without regard to truth and reason, to misrepresent his eminent services to his country.

Resolved, That this resolution be copied out, and signed by the chairman, and be by him presented to Mr. Burke, as the fullest expression of the respectful and grateful sense we entertain of his merits and services, public and private, to the citizens of Bristol, as a man and a representative.

Resolved, That the thanks of this meeting be given to the Right Worshipful the Mayor, who so ably and worthily presided in this meeting.

Resolved, That it is the earnest request of this meeting to Mr. Burke, that he should again offer himself a candidate to represent this city in Parliament; assuring him of that full and strenuous support which is due to the merits of so excellent a representative.

\PRLsep

This business being over, Mr. Burke went to the Exchange, and offered himself as a candidate in the usual manner. He was accompanied to the Council-House, and from thence to the Exchange, by a large body of most respectable gentlemen, amongst whom were the following members of the corporation, viz.: Mr. Mayor, Mr. Alderman Smith, Mr. Alderman Deane, Mr. Alderman Gordon, William Weare, Samuel Munckley, John Merlott, John Crofts, Levy Ames, John Fisher Weare, Benjamin Loscombe, Philip Protheroe, Samuel Span, Joseph Smith, Richard Bright and John Noble, Esquires.
\end{itpars}

%FOOTNOTES:
% [48] Irish Perpetual Mutiny Act.

% [49] Mr. Williams.

% [50] The Chancellor.


%%%%%%%%%%%%%%%%%%%%%%%%%%%%%%%%%%%%%%%%%%%%%%%%%%%%%%%%%%%%%%%%%%%%%%%
\chapter*[Speech at Bristol on Declining the Poll]{
Speech at Bristol on Declining the Poll (1780)}
\addcontentsline{toc}{chapter}{SPEECH AT BRISTOL ON DECLINING THE POLL,
September 9,1780}

\hfill BRISTOL, Saturday, 9th Sept, 1780.

\textit{This morning the sheriff and candidates assembled as usual at the Council-House, and from thence proceeded to Guildhall. Proclamation being made for the electors to appear and give their votes, Mr. BURKE stood forward on the hustings, surrounded by a great number of the corporation and other principal citizens, and addressed himself to the whole assembly as follows.}

\PRLsep

\noindent
GENTLEMEN, --- I decline the election. It has ever been my rule through life to observe a proportion between my efforts and my objects. I have never been remarkable for a bold, active, and sanguine pursuit of advantages that are personal to myself.

I have not canvassed the whole of this city in form, but I have taken such a view of it as satisfies my own mind that your choice will not ultimately fall upon me. Your city, Gentlemen, is in a state of miserable distraction, and I am resolved to withdraw whatever share my pretensions may have had in its unhappy divisions. I have not been in haste; I have tried all prudent means; I have waited for the effect of all contingencies. If I were fond of a contest, by the partiality of my numerous friends (whom you know to be among the most weighty and respectable people of the city) I have the means of a sharp one in my hands. But I thought it far better, with my strength unspent, and my reputation unimpaired, to do, early and from foresight, that which I might be obliged to do from necessity at last.

I am not in the least surprised nor in the least angry at this view of things. I have read the book of life for a long time, and I have read other books a little. Nothing has happened to me, but what has happened to men much better than me, and in times and in nations full as good as the age and country that we live in. To say that I am no way concerned would be neither decent nor true. The representation of Bristol was an object on many accounts dear to me; and I certainly should very far prefer it to any other in the kingdom. My habits are made to it; and it is in general more unpleasant to be rejected after long trial than not to be chosen at all.

But, Gentlemen, I will see nothing except your former kindness, and I will give way to no other sentiments than those of gratitude. From the bottom of my heart I thank you for what you have done for me. You have given me a long term, which is now expired. I have performed the conditions, and enjoyed all the profits to the full; and I now surrender your estate into your hands, without being in a single tile or a single stone impaired or wasted by my use. I have served the public for fifteen years. I have served you in particular for six. What is past is well stored; it is safe, and out of the power of fortune. What is to come is in wiser hands than ours; and He in whose hands it is best knows whether it is best for you and me that I should be in Parliament, or even in the world.

Gentlemen, the melancholy event of yesterday reads to us an awful lesson against being too much troubled about any of the objects of ordinary ambition. The worthy gentleman 
%[51]
\footnote{ Mr. Coombe.}
 who has been snatched from us at the moment of the election, and in the middle of the contest, whilst his desires were as warm and his hopes as eager as ours, has feelingly told us what shadows we are and what shadows we pursue.

It has been usual for a candidate who declines to take his leave by a letter to the sheriffs: but I received your trust in the face of day, and in the face of day I accept your dismission. I am not—I am not at all ashamed to look upon you; nor can my presence discompose the order of business here. I humbly and respectfully take my leave of the sheriffs, the candidates, and the electors, wishing heartily that the choice may be for the best, at a time which calls, if ever time did call, for service that is not nominal. It is no plaything you are about. I tremble, when I consider the trust I have presumed to ask. I confided, perhaps, too much in my intentions. They were really fair and upright; and I am bold to say that I ask no ill thing for you, when, on parting from this place, I pray, that, whomever you choose to succeed me, he may resemble me exactly in all things, except in my abilities to serve, and my fortune to please you.


%FOOTNOTES:
% [51] Mr. Coombe.


%%%%%%%%%%%%%%%%%%%%%%%%%%%%%%%%%%%%%%%%%%%%%%%%%%%%%%%%%%%%%%%%%%%%%%%
\chapter*[Speech on Mr. Fox's East India Bill]{
Speech
\\(December 1, 1783)
\\Upon
the Question for the Speaker's Leaving the Chair in Order for the House
to Resolve Itself into a Committee
\\On
\\Mr. Fox's East India Bill}
\addcontentsline{toc}{chapter}{SPEECH ON MR. FOX'S EAST INDIA BILL, 
December 1,1783}

Mr. Speaker,—I thank you for pointing to me. I really wished much to engage your attention in an early stage of the debate. I have been long very deeply, though perhaps ineffectually, engaged in the preliminary inquiries, which have continued without intermission for some years. Though I have felt, with some degree of sensibility, the natural and inevitable impressions of the several matters of fact, as they have been successively disclosed, I have not at any time attempted to trouble you on the merits of the subject, and very little on any of the points which incidentally arose in the course of our proceedings. But I should be sorry to be found totally silent upon this day. Our inquiries are now come to their final issue. It is now to be determined whether the three years of laborious Parliamentary research, whether the twenty years of patient Indian suffering, are to produce a substantial reform in our Eastern administration; or whether our knowledge of the grievances has abated our zeal for the correction of them, and our very inquiry into the evil was only a pretext to elude the remedy which is demanded from us by humanity, by justice, and by every principle of true policy. Depend upon it, this business cannot be indifferent to our fame. It will turn out a matter of great disgrace or great glory to the whole British nation. We are on a conspicuous stage, and the world marks our demeanor.

I am therefore a little concerned to perceive the spirit and temper in which the debate has been all along pursued upon one side of the House. The declamation of the gentlemen who oppose the bill has been abundant and vehement; but they have been reserved and even silent about the fitness or unfitness of the plan to attain the direct object it has in view. By some gentlemen it is taken up (by way of exercise, I presume) as a point of law, on a question of private property and corporate franchise; by others it is regarded as the petty intrigue of a faction at court, and argued merely as it tends to set this man a little higher or that a little lower in situation and power. All the void has been filled up with invectives against coalition, with allusions to the loss of America, with the activity and inactivity of ministers. The total silence of these gentlemen concerning the interest and well-being of the people of India, and concerning the interest which this nation has in the commerce and revenues of that country, is a strong indication of the value which they set upon these objects.

It has been a little painful to me to observe the intrusion into this important debate of such company as quo warranto, and mandamus, and certiorari: as if we were on a trial about mayors and aldermen and capital burgesses, or engaged in a suit concerning the borough of Penryn, or Saltash, or St. Ives, or St. Mawes. Gentlemen have argued with as much heat and passion as if the first things in the world were at stake; and their topics are such as belong only to matter of the lowest and meanest litigation. It is not right, it is not worthy of us, in this manner to depreciate the value, to degrade the majesty, of this grave deliberation of policy and empire.

For my part, I have thought myself bound, when a matter of this extraordinary weight came before me, not to consider (as some gentlemen are so fond of doing) whether the bill originated from a Secretary of State for the Home Department or from a Secretary for the Foreign, from a minister of influence or a minister of the people, from Jacob or from Esau.
%[52]
\footnote{ An allusion made by Mr. Powis.}
 I asked myself, and I asked myself nothing else, what part it was fit for a member of Parliament, who has supplied a mediocrity of talents by the extreme of diligence, and who has thought himself obliged by the research of years to wind himself into the inmost recesses and labyrinths of the Indian detail,—what part, I say, it became such a member of Parliament to take, when a minister of state, in conformity to a recommendation from the throne, has brought before us a system for the better government of the territory and commerce of the East. In this light, and in this only, I will trouble you with my sentiments.

It is not only agreed, but demanded, by the right honorable gentleman,
%[53]
\footnote{ Mr. Pitt.}
 and by those who act with him, that a whole system ought to be produced; that it ought not to be an half-measure; that it ought to be no palliative, but a legislative provision, vigorous, substantial, and effective.—I believe that no man who understands the subject can doubt for a moment that those must be the conditions of anything deserving the name of a reform in the Indian government; that anything short of them would not only be delusive, but, in this matter, which admits no medium, noxious in the extreme.

To all the conditions proposed by his adversaries the mover of the bill perfectly agrees; and on his performance of them he rests his cause. On the other hand, not the least objection has been taken with regard to the efficiency, the vigor, or the completeness of the scheme. I am therefore warranted to assume, as a thing admitted, that the bills accomplish what both sides of the House demand as essential. The end is completely answered, so for as the direct and immediate object is concerned.

But though there are no direct, yet there are various collateral objections made: objections from the effects which this plan of reform for Indian administration may have on the privileges of great public bodies in England; from its probable influence on the constitutional rights, or on the freedom and integrity, of the several branches of the legislature.

Before I answer these objections, I must beg leave to observe, that, if we are not able to contrive some method of governing India well, which will not of necessity become the means of governing Great Britain ill, a ground is laid for their eternal separation, but none for sacrificing the people of that country to our Constitution. I am, however, far from being persuaded that any such incompatibility of interest does at all exist. On the contrary, I am certain that every means effectual to preserve India from oppression is a guard to preserve the British Constitution from its worst corruption. To show this, I will consider the objections, which, I think, are four.

1st, That the bill is an attack on the chartered rights of men.

2ndly, That it increases the influence of the crown.

3rdly, That it does not increase, but diminishes, the influence of the crown, in order to promote the interests of certain ministers and their party.

4thly, That it deeply affects the national credit.

As to the first of these objections, I must observe that the phrase of "the chartered rights of men" is full of affectation, and very unusual in the discussion of privileges conferred by charters of the present description. But it is not difficult to discover what end that ambiguous mode of expression, so often reiterated, is meant to answer.

The rights of men—that is to say, the natural rights of mankind—are indeed sacred things; and if any public measure is proved mischievously to affect them, the objection ought to be fatal to that measure, even if no charter at all could be set up against it. If these natural rights are further affirmed and declared by express covenants, if they are clearly defined and secured against chicane, against power and authority, by written instruments and positive engagements, they are in a still better condition: they partake not only of the sanctity of the object so secured, but of that solemn public faith itself which secures an object of such importance. Indeed, this formal recognition, by the sovereign power, of an original right in the subject, can never be subverted, but by rooting up the holding radical principles of government, and even of society itself. The charters which we call by distinction great are public instruments of this nature: I mean the charters of King John and King Henry the Third. The things secured by these instruments may, without any deceitful ambiguity, be very fitly called the chartered rights of men.

These charters have made the very name of a charter dear to the heart of every Englishman. But, Sir, there may be, and there are, charters, not only different in nature, but formed on principles the very reverse of those of the Great Charter. Of this kind is the charter of the East India Company. Magna Charta is a charter to restrain power and to destroy monopoly. The East India charter is a charter to establish monopoly and to create power. Political power and commercial monopoly are not the rights of men; and the rights to them derived from charters it is fallacious and sophistical to call "the chartered rights of men." These chartered rights (to speak of such charters and of their effects in terms of the greatest possible moderation) do at least suspend the natural rights of mankind at large, and in their very frame and constitution are liable to fall into a direct violation of them.

It is a charter of this latter description (that is to say, a charter of power and monopoly) which is affected by the bill before you. The bill, Sir, does without question affect it: it does affect it essentially and substantially. But, having stated to you of what description the chartered rights are which this bill touches, I feel no difficulty at all in acknowledging the existence of those chartered rights in their fullest extent. They belong to the Company in the surest manner, and they are secured to that body by every sort of public sanction. They are stamped by the faith of the king; they are stamped by the faith of Parliament: they have been bought for money, for money honestly and fairly paid; they have been bought for valuable consideration, over and over again.

I therefore freely admit to the East India Company their claim to exclude their fellow-subjects from the commerce of half the globe. I admit their claim to administer an annual territorial revenue of seven millions sterling, to command an army of sixty thousand men, and to dispose (under the control of a sovereign, imperial discretion, and with the due observance of the natural and local law) of the lives and fortunes of thirty millions of their fellow-creatures. All this they possess by charter, and by Acts of Parliament, (in my opinion,) without a shadow of controversy.

Those who carry the rights and claims of the Company the furthest do not contend for more than this; and all this I freely grant. But, granting all this, they must grant to me, in my turn, that all political power which is set over men, and that all privilege claimed or exercised in exclusion of them, being wholly artificial, and for so much a derogation from the natural equality of mankind at large, ought to be some way or other exercised ultimately for their benefit.

If this is true with regard to every species of political dominion and every description of commercial privilege, none of which can be original, self-derived rights, or grants for the mere private benefit of the holders, then such rights, or privileges, or whatever else you choose to call them, are all in the strictest sense a trust: and it is of the very essence of every trust to be rendered accountable,—and even totally to cease, when it substantially varies from the purposes for which alone it could have a lawful existence.

This I conceive, Sir, to be true of trusts of power vested in the highest hands, and of such, as seem to hold of no human creature. But about the application of this principle to subordinate derivative trusts I do not see how a controversy can be maintained. To whom, then, would I make the East India Company accountable? Why, to Parliament, to be sure,—to Parliament, from whom their trust was derived,—to Parliament, which alone is capable of comprehending the magnitude of its object, and its abuse, and alone capable of an effectual legislative remedy. The very charter, which is held out to exclude Parliament from correcting malversation with regard to the high trust vested in the Company, is the very thing which at once gives a title and imposes a duty on us to interfere with effect, wherever power and authority originating from ourselves are perverted from their purposes, and become instruments of wrong and violence.

If Parliament, Sir, had nothing to do with this charter, we might have some sort of Epicurean excuse to stand aloof, indifferent spectators of what passes in the Company's name in India and in London. But if we are the very cause of the evil, we are in a special manner engaged to the redress; and for us passively to bear with oppressions committed under the sanction of our own authority is in truth and reason for this House to be an active accomplice in the abuse.

That the power, notoriously grossly abused, has been bought from us is very certain. But this circumstance, which is urged against the bill, becomes an additional motive for our interference, lest we should be thought to have sold the blood of millions of men for the base consideration of money. We sold, I admit, all that we had to sell,—that is, our authority, not our control. We had not a right to make a market of our duties.

I ground myself, therefore, on this principle:—that, if the abuse is proved, the contract is broken, and we reënter into all our rights, that is, into the exercise of all our duties. Our own authority is, indeed, as much a trust originally as the Company's authority is a trust derivatively; and it is the use we make of the resumed power that must justify or condemn us in the resumption of it. When we have perfected the plan laid before us by the right honorable mover, the world will then see what it is we destroy, and what it is we create. By that test we stand or fall; and by that test I trust that it will be found, in the issue, that we are going to supersede a charter abused to the full extent of all the powers which it could abuse, and exercised in the plenitude of despotism, tyranny, and corruption,—and that in one and the same plan we provide a real chartered security for the rights of men, cruelly violated under that charter.

This bill, and those connected with it, are intended to form the Magna Charta of Hindostan. Whatever the Treaty of Westphalia is to the liberty of the princes and free cities of the Empire, and to the three religions there professed,—whatever the Great Charter, the Statute of Tallage, the Petition of Right, and the Declaration of Right are to Great Britain, these bills are to the people of India. Of this benefit I am certain their condition is capable: and when I know that they are capable of more, my vote shall most assuredly be for our giving to the full extent of their capacity of receiving; and no charter of dominion shall stand as a bar in my way to their charter of safety and protection.

The strong admission I have made of the Company's rights (I am conscious of it) binds me to do a great deal. I do not presume to condemn those who argue a priori against the propriety of leaving such extensive political powers in the hands of a company of merchants. I know much is, and much more may be, said against such a system. But, with my particular ideas and sentiments, I cannot go that way to work. I feel an insuperable reluctance in giving my hand to destroy any established institution of government, upon a theory, however plausible it may be. My experience in life teaches me nothing clear upon the subject. I have known merchants with the sentiments and the abilities of great statesmen, and I have seen persons in the rank of statesmen with the conceptions and character of peddlers. Indeed, my observation has furnished me with nothing that is to be found in any habits of life or education, which tends wholly to disqualify men for the functions of government, but that by which the power of exercising those functions is very frequently obtained: I mean a spirit and habits of low cabal and intrigue; which I have never, in one instance, seen united with a capacity for sound and manly policy.

To justify us in taking the administration of their affairs out of the hands of the East India Company, on my principles, I must see several conditions. 1st, The object affected by the abuse should be great and important. 2nd, The abuse affecting this great object ought to be a great abuse. 3d, It ought to be habitual, and not accidental. 4th, It ought to be utterly incurable in the body as it now stands constituted. All this ought to be made as visible to me as the light of the sun, before I should strike off an atom of their charter. A right honorable gentleman
%[54]
\footnote{ Mr. Pitt.}
 has said, and said, I think, but once, and that very slightly, (whatever his original demand for a plan might seem to require,) that "there are abuses in the Company's government." If that were all, the scheme of the mover of this bill, the scheme of his learned friend, and his own scheme of reformation, (if he has any,) are all equally needless. There are, and must be, abuses in all governments. It amounts to no more than a nugatory proposition. But before I consider of what nature these abuses are, of which the gentleman speaks so very lightly, permit me to recall to your recollection the map of the country which this abused chartered right affects. This I shall do, that you may judge whether in that map I can discover anything like the first of my conditions: that is, whether the object affected by the abuse of the East India Company's power be of importance sufficient to justify the measure and means of reform applied to it in this bill.

With very few, and those inconsiderable intervals, the British dominion, either in the Company's name, or in the names of princes absolutely dependent upon the Company, extends from the mountains that separate India from Tartary to Cape Comorin, that is, one-and-twenty degrees of latitude!

In the northern parts it is a solid mass of land, about eight hundred miles in length, and four or five hundred broad. As you go southward, it becomes narrower for a space. It afterwards dilates; but, narrower or broader, you possess the whole eastern and northeastern coast of that vast country, quite from the borders of Pegu.—Bengal, Bahar, and Orissa, with Benares, (now unfortunately in our immediate possession,) measure 161,978 square English miles: a territory considerably larger than the whole kingdom of France. Oude, with its dependent provinces, is 53,286 square miles: not a great deal less than England. The Carnatic, with Tanjore and the Circars, is 65,948 square miles: very considerably larger than England. And the whole of the Company's dominions, comprehending Bombay and Salsette, amounts to 281,412 square miles: which forms a territory larger than any European dominion, Russia and Turkey excepted. Through all that vast extent of country there is not a man who eats a mouthful of rice but by permission of the East India Company.

So far with regard to the extent. The population of this great empire is not easy to be calculated. When the countries of which it is composed came into our possession, they were all eminently peopled, and eminently productive,—though at that time considerably declined from their ancient prosperity. But since they are come into our hands!—--! However, if we make the period of our estimate immediately before the utter desolation of the Carnatic, and if we allow for the havoc which our government had even then made in these regions, we cannot, in my opinion, rate the population at much less than thirty millions of souls: more than four times the number of persons in the island of Great Britain.

My next inquiry to that of the number is the quality and description of the inhabitants. This multitude of men does not consist of an abject and barbarous populace; much less of gangs of savages, like the Guaranies and Chiquitos, who wander on the waste borders of the River of Amazons or the Plate; but a people for ages civilized and cultivated,—cultivated by all the arts of polished life, whilst we were yet in the woods. There have been (and still the skeletons remain) princes once of great dignity, authority, and opulence. There are to be found the chiefs of tribes and nations. There is to be found an ancient and venerable priesthood, the depository of their laws, learning, and history, the guides of the people whilst living and their consolation in death; a nobility of great antiquity and renown; a multitude of cities, not exceeded in population and trade by those of the first class in Europe; merchants and bankers, individual houses of whom have once vied in capital with the Bank of England, whose credit had often supported a tottering state, and preserved their governments in the midst of war and desolation; millions of ingenious manufacturers and mechanics; millions of the most diligent, and not the least intelligent, tillers of the earth. Here are to be found almost all the religions professed by men,—the Braminical, the Mussulman, the Eastern and the Western Christian.

If I were to take the whole aggregate of our possessions there, I should compare it, as the nearest parallel I can find, with the Empire of Germany. Our immediate possessions I should compare with the Austrian dominions: and they would not suffer in the comparison. The Nabob of Oude might stand for the King of Prussia; the Nabob of Arcot I would compare, as superior in territory, and equal in revenue, to the Elector of Saxony. Cheit Sing, the Rajah of Benares, might well rank with the Prince of Hesse, at least; and the Rajah of Tanjore (though hardly equal in extent of dominion, superior in revenue) to the Elector of Bavaria. The polygars and the Northern zemindars, and other great chiefs, might well class with the rest of the princes, dukes, counts, marquises, and bishops in the Empire; all of whom I mention to honor, and surely without disparagement to any or all of those most respectable princes and grandees.

All this vast mass, composed of so many orders and classes of men, is again infinitely diversified by manners, by religion, by hereditary employment, through all their possible combinations. This renders the handling of India a matter in an high degree critical and delicate. But, oh, it has been handled rudely indeed! Even some of the reformers seem to have forgot that they had anything to do but to regulate the tenants of a manor, or the shopkeepers of the next county town.

It is an empire of this extent, of this complicated nature, of this dignity and importance, that I have compared to Germany and the German government,—not for an exact resemblance, but as a sort of a middle term, by which India might be approximated to our understandings, and, if possible, to our feelings, in order to awaken something of sympathy for the unfortunate natives, of which I am afraid we are not perfectly susceptible, whilst we look at this very remote object through a false and cloudy medium.

My second condition necessary to justify me in touching the charter is, whether the Company's abuse of their trust with regard to this great object be an abuse of great atrocity. I shall beg your permission to consider their conduct in two lights: first the political, and then the commercial. Their political conduct (for distinctness) I divide again into two heads: the external, in which I mean to comprehend their conduct in their federal capacity, as it relates to powers and states independent, or that not long since were such; the other internal,—namely, their conduct to the countries, either immediately subject to the Company, or to those who, under the apparent government of native sovereigns, are in a state much lower and much more miserable than common subjection.

The attention, Sir, which I wish to preserve to method will not be considered as unnecessary or affected. Nothing else can help me to selection out of the infinite mass of materials which have passed under my eye, or can keep my mind steady to the great leading points I have in view.

With regard, therefore, to the abuse of the external federal trust, I engage myself to you to make good these three positions. First, I say, that from Mount Imaus, (or whatever else you call that large range of mountains that walls the northern frontier of India,) where it touches us in the latitude of twenty-nine, to Cape Comorin, in the latitude of eight, that there is not a single prince, state, or potentate, great or small, in India, with whom they have come into contact, whom they have not sold: I say sold, though sometimes they have not been able to deliver according to their bargain. Secondly, I say, that there is not a single treaty they have ever made which they have not broken. Thirdly, I say, that there is not a single prince or state, who ever put any trust in the Company, who is not utterly ruined; and that none are in any degree secure or flourishing, but in the exact proportion to their settled distrust and irreconcilable enmity to this nation.

These assertions are universal: I say, in the full sense, universal. They regard the external and political trust only; but I shall produce others fully equivalent in the internal. For the present, I shall content myself with explaining my meaning; and if I am called on for proof, whilst these bills are depending, (which I believe I shall not,) I will put my finger on the appendixes to the Reports, or on papers of record in the House or the Committees, which I have distinctly present to my memory, and which I think I can lay before you at half an hour's warning.

The first potentate sold by the Company for money was the Great Mogul,—the descendant of Tamerlane. This high personage, as high as human veneration can look at, is by every account amiable in his manners, respectable for his piety, according to his mode, and accomplished in all the Oriental literature. All this, and the title derived under his charter to all that we hold in India, could not save him from the general sale. Money is coined in his name; in his name justice is administered; he is prayed for in every temple through the countries we possess;—but he was sold.

It is impossible, Mr. Speaker, not to pause here for a moment, to reflect on the inconstancy of human greatness, and the stupendous revolutions that have happened in our age of wonders. Could it be believed, when I entered into existence, or when you, a younger man, were born, that on this day, in this House, we should be employed in discussing the conduct of those British subjects who had disposed of the power and person of the Grand Mogul? This is no idle speculation. Awful lessons are taught by it, and by other events, of which it is not yet too late to profit.

This is hardly a digression: but I return to the sale of the Mogul. Two districts, Corah and Allahabad, out of his immense grants, were reserved as a royal demesne to the donor of a kingdom, and the rightful sovereign of so many nations.—After withholding the tribute of 260,000l. a year, which the Company was, by the charter they had received from this prince, under the most solemn obligation to pay, these districts were sold to his chief minister, Sujah ul Dowlah; and what may appear to some the worst part of the transaction, these two districts were sold for scarcely two years' purchase. The descendant of Tamerlane now stands in need almost of the common necessaries of life; and in this situation we do not even allow him, as bounty, the smallest portion of what we owe him in justice.

The next sale was that of the whole nation of the Rohillas, which the grand salesman, without a pretence of quarrel, and contrary to his own declared sense of duty and rectitude, sold to the same Sujah ul Dowlah. He sold the people to utter extirpation, for the sum of four hundred thousand pounds. Faithfully was the bargain performed on our side. Hafiz Rhamet, the most eminent of their chiefs, one of the bravest men of his time, and as famous throughout the East for the elegance of his literature and the spirit of his poetical compositions (by which he supported the name of Hafiz) as for his courage, was invaded with an army of an hundred thousand men, and an English brigade. This man, at the head of inferior forces, was slain valiantly fighting for his country. His head was cut off, and delivered for money to a barbarian. His wife and children, persons of that rank, were seen begging an handful of rice through the English camp. The whole nation, with inconsiderable exceptions, was slaughtered or banished. The country was laid waste with fire and sword; and that land, distinguished above most others by the cheerful face of paternal government and protected labor, the chosen seat of cultivation and plenty, is now almost throughout a dreary desert, covered with rushes, and briers, and jungles full of wild beasts.

The British officer who commanded in the delivery of the people thus sold felt some compunction at his employment. He represented these enormous excesses to the President of Bengal, for which he received a severe reprimand from the civil governor; and I much doubt whether the breach caused by the conflict between the compassion of the military and the firmness of the civil governor be closed at this hour.

In Bengal, Surajah Dowlah was sold to Mir Jaffier; Mir Jaffier was sold to Mir Cossim; and Mir Cossim was sold to Mir Jaffier again. The succession to Mir Jaffier was sold to his eldest son;—another son of Mir Jaffier, Mobarech ul Dowlah, was sold to his step-mother. The Mahratta Empire was sold to Ragobah; and Ragobah was sold and delivered to the Peishwa of the Mahrattas. Both Ragobah and the Peishwa of the Mahrattas were offered to sale to the Rajah of Berar. Scindia, the chief of Malwa, was offered to sale to the same Rajah; and the Subah of the Deccan was sold to the great trader, Mahomet Ali, Nabob of Arcot. To the same Nabob of Arcot they sold Hyder Ali and the kingdom of Mysore. To Mahomet Ali they twice sold the kingdom of Tanjore. To the same Mahomet Ali they sold at least twelve sovereign princes, called the Polygars. But to keep things even, the territory of Tinnevelly, belonging to their nabob, they would have sold to the Dutch; and to conclude the account of sales, their great customer, the Nabob of Arcot himself, and his lawful succession, has been sold to his second son, Amir ul Omrah, whose character, views, and conduct are in the accounts upon your table. It remains with you whether they shall finally perfect this last bargain.

All these bargains and sales were regularly attended with the waste and havoc of the country,—always by the buyer, and sometimes by the object of the sale. This was explained to you by the honorable mover, when he stated the mode of paying debts due from the country powers to the Company. An honorable gentleman, who is not now in his place, objected to his jumping near two thousand miles for an example. But the southern example is perfectly applicable to the northern claim, as the northern is to the southern; for, throughout the whole space of these two thousand miles, take your stand where you will, the proceeding is perfectly uniform, and what is done in one part will apply exactly to the other.

My second assertion is, that the Company never has made a treaty which they have not broken. This position is so connected with that of the sales of provinces and kingdoms, with the negotiation of universal distraction in every part of India, that a very minute detail may well be spared on this point. It has not yet been contended, by any enemy to the reform, that they have observed any public agreement. When I hear that they have done so in any one instance, (which hitherto, I confess, I never heard alleged,) I shall speak to the particular treaty. The Governor General has even amused himself and the Court of Directors in a very singular letter to that board, in which he admits he has not been very delicate with regard to public faith; and he goes so far as to state a regular estimate of the sums which the Company would have lost, or never acquired, if the rigid ideas of public faith entertained by his colleagues had been observed. The learned gentleman 
%[55]
\footnote{ Mr. Dundas, Lord Advocate of Scotland.}
 over against me has, indeed, saved me much trouble. On a former occasion, he obtained no small credit for the clear and forcible manner in which he stated, what we have not forgot, and I hope he has not forgot, that universal, systematic breach of treaties which had made the British faith proverbial in the East.

It only remains, Sir, for me just to recapitulate some heads.—The treaty with the Mogul, by which we stipulated to pay him 260,000l. annually, was broken. This treaty they have broken, and not paid him a shilling. They broke their treaty with him, in which they stipulated to pay 400,000l. a year to the Subah of Bengal. They agreed with the Mogul, for services admitted to have been performed, to pay Nudjif Cawn a pension. They broke this article with the rest, and stopped also this small pension. They broke their treaties with the Nizam, and with Hyder Ali. As to the Mahrattas, they had so many cross treaties with the states-general of that nation, and with each of the chiefs, that it was notorious that no one of these agreements could be kept without grossly violating the rest. It was observed, that, if the terms of these several treaties had been kept, two British armies would at one and the same time have met in the field to cut each other's throats. The wars which desolate India originated from a most atrocious violation of public faith on our part. In the midst of profound peace, the Company's troops invaded the Mahratta territories, and surprised the island and fortress of Salsette. The Mahrattas nevertheless yielded to a treaty of peace by which solid advantages were procured to the Company. But this treaty, like every other treaty, was soon violated by the Company. Again the Company invaded the Mahratta dominions. The disaster that ensued gave occasion to a new treaty. The whole army of the Company was obliged in effect to surrender to this injured, betrayed, and insulted people. Justly irritated, however, as they were, the terms which they prescribed were reasonable and moderate, and their treatment of their captive invaders of the most distinguished humanity. But the humanity of the Mahrattas was of no power whatsoever to prevail on the Company to attend to the observance of the terms dictated by their moderation. The war was renewed with greater vigor than ever; and such was their insatiable lust of plunder, that they never would have given ear to any terms of peace, if Hyder Ali had not broke through the Ghauts, and, rushing like a torrent into the Carnatic, swept away everything in his career. This was in consequence of that confederacy which by a sort of miracle united the most discordant powers for our destruction, as a nation in which no other could put any trust, and who were the declared enemies of the human species.

It is very remarkable that the late controversy between the several presidencies, and between them and the Court of Directors, with relation to these wars and treaties, has not been, which of the parties might be defended for his share in them, but on which of the parties the guilt of all this load of perfidy should be fixed. But I am content to admit all these proceedings to be perfectly regular, to be full of honor and good faith; and wish to fix your attention solely to that single transaction which the advocates of this system select for so transcendent a merit as to cancel the guilt of all the rest of their proceedings: I mean the late treaties with the Mahrattas.

I make no observation on the total cession of territory, by which they surrendered all they had obtained by their unhappy successes in war, and almost all they had obtained under the treaty of Poorunder. The restitution was proper, if it had been voluntary and seasonable. I attach on the spirit of the treaty, the dispositions it showed, the provisions it made for a general peace, and the faith kept with allies and confederates,—in order that the House may form a judgment, from this chosen piece, of the use which has been made (and is likely to be made, if things continue in the same hands) of the trust of the federal powers of this country.

It was the wish of almost every Englishman that the Mahratta peace might lead to a general one; because the Mahratta war was only a part of a general confederacy formed against us, on account of the universal abhorrence of our conduct which prevailed in every state, and almost in every house in India. Mr. Hastings was obliged to pretend some sort of acquiescence in this general and rational desire. He therefore consented, in order to satisfy the point of honor of the Mahrattas, that an article should be inserted to admit Hyder Ali to accede to the pacification. But observe, Sir, the spirit of this man,—which, if it were not made manifest by a thousand things, and particularly by his proceedings with regard to Lord Macartney, would be sufficiently manifest by this. What sort of article, think you, does he require this essential head of a solemn treaty of general pacification to be? In his instruction to Mr. Anderson, he desires him to admit "a vague article" in favor of Hyder. Evasion and fraud were the declared basis of the treaty. These vague articles, intended for a more vague performance, are the things which have damned our reputation in India.

Hardly was this vague article inserted, than, without waiting for any act on the part of Hyder, Mr. Hastings enters into a negotiation with the Mahratta chief, Scindia, for a partition of the territories of the prince who was one of the objects to be secured by the treaty. He was to be parcelled out in three parts: one to Scindia; one to the Peishwa of the Mahrattas; and the third to the East India Company, or to (the old dealer and chapman) Mahomet Ali.

During the formation of this project, Hyder dies; and before his son could take any one step, either to conform to the tenor of the article or to contravene it, the treaty of partition is renewed on the old footing, and an instruction is sent to Mr. Anderson to conclude it in form.

A circumstance intervened, during the pendency of this negotiation, to set off the good faith of the Company with an additional brilliancy, and to make it sparkle and glow with a variety of splendid faces. General Matthews had reduced that most valuable part of Hyder's dominions called the country of Biddanore. When the news reached Mr. Hastings, he instructed Mr. Anderson to contend for an alteration in the treaty of partition, and to take the Biddanore country out of the common stock which was to be divided, and to keep it for the Company.

The first ground for this variation was its being a separate conquest made before the treaty had actually taken place. Here was a new proof given of the fairness, equity, and moderation of the Company. But the second of Mr. Hastings's reasons for retaining the Biddanore as a separate portion, and his conduct on that second ground, is still more remarkable. He asserted that that country could not be put into the partition stock, because General Matthews had received it on the terms of some convention which might be incompatible with the partition proposed. This was a reason in itself both honorable and solid; and it showed a regard to faith somewhere, and with some persons. But in order to demonstrate his utter contempt of the plighted faith which was alleged on one part as a reason for departing from it on another, and to prove his impetuous desire for sowing a new war even in the prepared soil of a general pacification, he directs Mr. Anderson, if he should find strong difficulties impeding the partition on the score of the subtraction of Biddanore, wholly to abandon that claim, and to conclude the treaty on the original terms. General Matthews's convention was just brought forward sufficiently to demonstrate to the Mahrattas the slippery hold which they had on their new confederate; on the other hand, that convention being instantly abandoned, the people of India were taught that no terms on which they can surrender to the Company are to be regarded, when farther conquests are in view.

Next, Sir, let me bring before you the pious care that was taken of our allies under that treaty which is the subject of the Company's applauses. These allies were Ragonaut Row, for whom we had engaged to find a throne; the Guickwar, (one of the Guzerat princes,) who was to be emancipated from the Mahratta authority, and to grow great by several accessions of dominion; and, lastly, the Rana of Gohud, with whom we had entered into a treaty of partition for eleven sixteenths of our joint conquests. Some of these inestimable securities called vague articles were inserted in favor of them all.

As to the first, the unhappy abdicated Peishwa, and pretender to the Mahratta throne, Ragonaut Row, was delivered up to his people, with an article for safety, and some provision. This man, knowing how little vague the hatred of his countrymen was towards him, and well apprised of what black crimes he stood accused, (among which our invasion of his country would not appear the least,) took a mortal alarm at the security we had provided for him. He was thunderstruck at the article in his favor, by which he was surrendered to his enemies. He never had the least notice of the treaty; and it was apprehended that he would fly to the protection of Hyder Ali, or some other, disposed or able to protect him. He was therefore not left without comfort; for Mr. Anderson did him the favor to send a special messenger, desiring him to be of good cheer and to fear nothing. And his old enemy, Scindia, at our request, sent him a message equally well calculated to quiet his apprehensions.

By the same treaty the Guickwar was to come again, with no better security, under the dominion of the Mahratta state. As to the Rana of Gohud, a long negotiation depended for giving him up. At first this was refused by Mr. Hastings with great indignation; at another stage it was admitted as proper, because he had shown himself a most perfidious person. But at length a method of reconciling these extremes was found out, by contriving one of the usual articles in his favor. What I believe will appear beyond all belief, Mr. Anderson exchanged the final ratifications of that treaty by which the Rana was nominally secured in his possessions, in the camp of the Mahratta chief, Scindia, whilst he was (really, and not nominally) battering the castle of Gwalior, which we had given, agreeably to treaty, to this deluded ally. Scindia had already reduced the town, and was at the very time, by various detachments, reducing, one after another, the fortresses of our protected ally, as well as in the act of chastising all the rajahs who had assisted Colonel Camac in his invasion. I have seen in a letter from Calcutta, that the Rana of Gohud's agent would have represented these hostilities (which went hand in hand with the protecting treaty) to Mr. Hastings, but he was not admitted to his presence.

In this manner the Company has acted with their allies in the Mahratta war. But they did not rest here. The Mahrattas were fearful lest the persons delivered to them by that treaty should attempt to escape into the British territories, and thus might elude the punishment intended for them, and, by reclaiming the treaty, might stir up new disturbances. To prevent this, they desired an article to be inserted in the supplemental treaty, to which they had the ready consent of Mr. Hastings, and the rest of the Company's representatives in Bengal. It was this: "That the English and Mahratta governments mutually agree not to afford refuge to any chiefs, merchants, or other persons, flying for protection to the territories of the other." This was readily assented to, and assented to without any exception whatever in favor of our surrendered allies. On their part a reciprocity was stipulated which was not unnatural for a government like the Company's to ask,—a government conscious that many subjects had been, and would in future be, driven to fly from its jurisdiction.

To complete the system of pacific intention and public faith which predominate in those treaties, Mr. Hastings fairly resolved to put all peace, except on the terms of absolute conquest, wholly out of his own power. For, by an article in this second treaty with Scindia, he binds the Company not to make any peace with Tippoo Sahib without the consent of the Peishwa of the Mahrattas, and binds Scindia to him by a reciprocal engagement. The treaty between France and England obliges us mutually to withdraw our forces, if our allies in India do not accede to the peace within four months; Mr. Hastings's treaty obliges us to continue the war as long as the Peishwa thinks fit. We are now in that happy situation, that the breach of the treaty with France, or the violation of that with the Mahrattas, is inevitable; and we have only to take our choice.

My third assertion, relative to the abuse made of the right of war and peace, is, that there are none who have ever confided in us who have not been utterly ruined. The examples I have given of Ragonaut Row, of Guickwar, of the Rana of Gohud, are recent. There is proof more than enough in the condition of the Mogul,—in the slavery and indigence of the Nabob of Oude,—the exile of the Rajah of Benares,—the beggary of the Nabob of Bengal,—the undone and captive condition of the Rajah and kingdom of Tanjore,—the destruction of the Polygars,—and, lastly, in the destruction of the Nabob of Arcot himself, who, when his dominions were invaded, was found entirely destitute of troops, provisions, stores, and (as he asserts) of money, being a million in debt to the Company, and four millions to others: the many millions which he had extorted from so many extirpated princes and their desolated countries having (as he has frequently hinted) been expended for the ground-rent of his mansion-house in an alley in the suburbs of Madras. Compare the condition of all these princes with the power and authority of all the Mahratta states, with the independence and dignity of the Subah of the Deccan, and the mighty strength, the resources, and the manly struggle of Hyder Ali,—and then the House will discover the effects, on every power in India, of an easy confidence or of a rooted distrust in the faith of the Company.

These are some of my reasons, grounded on the abuse of the external political trust of that body, for thinking myself not only justified, but bound, to declare against those chartered rights which produce so many wrongs. I should deem myself the wickedest of men, if any vote of mine could contribute to the continuance of so great an evil.

Now, Sir, according to the plan I proposed, I shall take notice of the Company's internal government, as it is exercised first on the dependent provinces, and then as it affects those under the direct and immediate authority of that body. And here, Sir, before I enter into the spirit of their interior government, permit me to observe to you upon a few of the many lines of difference which are to be found between the vices of the Company's government and those of the conquerors who preceded us in India, that we may be enabled a little the better to see our way in an attempt to the necessary reformation.

The several irruptions of Arabs, Tartars, and Persians into India were, for the greater part, ferocious, bloody, and wasteful in the extreme: our entrance into the dominion of that country was, as generally, with small comparative effusion of blood,—being introduced by various frauds and delusions, and by taking advantage of the incurable, blind, and senseless animosity which the several country powers bear towards each other, rather than by open force. But the difference in favor of the first conquerors is this. The Asiatic conquerors very soon abated of their ferocity, because they made the conquered country their own. They rose or fell with the rise or fall of the territory they lived in. Fathers there deposited the hopes of their posterity; and children there beheld the monuments of their fathers. Here their lot was finally cast; and it is the natural wish of all that their lot should not be cast in a bad land. Poverty, sterility, and desolation are not a recreating prospect to the eye of man; and there are very few who can bear to grow old among the curses of a whole people. If their passion or their avarice drove the Tartar lords to acts of rapacity or tyranny, there was time enough, even in the short life of man, to bring round the ill effects of an abuse of power upon the power itself. If hoards were made by violence and tyranny, they were still domestic hoards; and domestic profusion, or the rapine of a more powerful and prodigal hand, restored them to the people. With many disorders, and with few political checks upon power, Nature had still fair play; the sources of acquisition were not dried up; and therefore the trade, the manufactures, and the commerce of the country flourished. Even avarice and usury itself operated both for the preservation and the employment of national wealth. The husbandman and manufacturer paid heavy interest, but then they augmented the fund from whence they were again to borrow. Their resources were dearly bought, but they were sure; and the general stock of the community grew by the general effort.

But under the English government all this order is reversed. The Tartar invasion was mischievous; but it is our protection that destroys India. It was their enmity; but it is our friendship. Our conquest there, after twenty years, is as crude as it was the first day. The natives scarcely know what it is to see the gray head of an Englishman. Young men (boys almost) govern there, without society and without sympathy with the natives. They have no more social habits with the people than if they still resided in England,—nor, indeed, any species of intercourse, but that which is necessary to making a sudden fortune, with a view to a remote settlement. Animated with all the avarice of age and all the impetuosity of youth, they roll in one after another, wave after wave; and there is nothing before the eyes of the natives but an endless, hopeless prospect of new flights of birds of prey and passage, with appetites continually renewing for a food that is continually wasting. Every rupee of profit made by an Englishman is lost forever to India. With us are no retributory superstitions, by which a foundation of charity compensates, through ages, to the poor, for the rapine and injustice of a day. With us no pride erects stately monuments which repair the mischiefs which pride had produced, and which adorn a country out of its own spoils. England has erected no churches, no hospitals,
%[56]
\footnote{ The paltry foundation at Calcutta is scarcely worth naming as an exception.}
 no palaces, no schools; England has built no bridges, made no high-roads, cut no navigations, dug out no reservoirs. Every other conqueror of every other description has left some monument, either of state or beneficence, behind him. Were we to be driven out of India this day, nothing would remain to tell that it had been possessed, during the inglorious period of our dominion, by anything better than the orang-outang or the tiger.

There is nothing in the boys we send to India worse than in the boys whom we are whipping at school, or that we see trailing a pike or bending over a desk at home. But as English youth in India drink the intoxicating draught of authority and dominion before their heads are able to bear it, and as they are full grown in fortune long before they are ripe in principle, neither Nature nor reason have any opportunity to exert themselves for remedy of the excesses of their premature power. The consequences of their conduct, which in good minds (and many of theirs are probably such) might produce penitence or amendment, are unable to pursue the rapidity of their flight. Their prey is lodged in England; and the cries of India are given to seas and winds, to be blown about, in every breaking up of the monsoon, over a remote and unhearing ocean. In India all the vices operate by which sudden fortune is acquired: in England are often displayed, by the same persons, the virtues which dispense hereditary wealth. Arrived in England, the destroyers of the nobility and gentry of a whole kingdom will find the best company in this nation at a board of elegance and hospitality. Here the manufacturer and husbandman will bless the just and punctual hand that in India has torn the cloth from the loom, or wrested the scanty portion of rice and salt from the peasant of Bengal, or wrung from him the very opium in which he forgot his oppressions and his oppressor. They marry into your families; they enter into your senate; they ease your estates by loans; they raise their value by demand; they cherish and protect your relations which lie heavy on your patronage; and there is scarcely an house in the kingdom that does not feel some concern and interest that makes all reform of our Eastern government appear officious and disgusting, and, on the whole, a most discouraging attempt. In such an attempt you hurt those who are able to return kindness or to resent injury. If you succeed, you save those who cannot so much as give you thanks. All these things show the difficulty of the work we have on hand: but they show its necessity, too. Our Indian government is in its best state a grievance. It is necessary that the correctives should be uncommonly vigorous, and the work of men sanguine, warm, and even impassioned in the cause. But it is an arduous thing to plead against abuses of a power which originates from your own country, and affects those whom we are used to consider as strangers.

I shall certainly endeavor to modulate myself to this temper; though I am sensible that a cold style of describing actions, which appear to me in a very affecting light, is equally contrary to the justice due to the people and to all genuine human feelings about them. I ask pardon of truth and Nature for this compliance. But I shall be very sparing of epithets either to persons or things. It has been said, (and, with regard to one of them, with truth,) that Tacitus and Machiavel, by their cold way of relating enormous crimes, have in some sort appeared not to disapprove them; that they seem a sort of professors of the art of tyranny; and that they corrupt the minds of their readers by not expressing the detestation and horror that naturally belong to horrible and detestable proceedings. But we are in general, Sir, so little acquainted with Indian details, the instruments of oppression under which the people suffer are so hard to be understood, and even the very names of the sufferers are so uncouth and strange to our ears, that it is very difficult for our sympathy to fix upon these objects. I am sure that some of us have come down stairs from the committee-room with impressions on our minds which to us were the inevitable results of our discoveries, yet, if we should venture to express ourselves in the proper language of our sentiments to other gentlemen not at all prepared to enter into the cause of them, nothing could appear more harsh and dissonant, more violent and unaccountable, than our language and behavior. All these circumstances are not, I confess, very favorable to the idea of our attempting to govern India at all. But there we are; there we are placed by the Sovereign Disposer; and we must do the best we can in our situation. The situation of man is the preceptor of his duty.

Upon the plan which I laid down, and to which I beg leave to return, I was considering the conduct of the Company to those nations which are indirectly subject to their authority. The most considerable of the dependent princes is the Nabob of Oude. My right honorable friend, 
%[57]
\footnote{ Mr. Fox.}
 to whom we owe the remedial bills on your table, has already pointed out to you, in one of the reports, the condition of that prince, and as it stood in the time he alluded to. I shall only add a few circumstances that may tend to awaken some sense of the manner in which the condition of the people is affected by that of the prince, and involved in it,—and to show you, that, when we talk of the sufferings of princes, we do not lament the oppression of individuals,—and that in these cases the high and the low suffer together.

In the year 1779, the Nabob of Oude represented, through the British resident at his court, that the number of Company's troops stationed in his dominions was a main cause of his distress,—and that all those which he was not bound by treaty to maintain should be withdrawn, as they had greatly diminished his revenue and impoverished his country. I will read you, if you please, a few extracts from these representations.

He states, "that the country and cultivation are abandoned, and this year in particular, from the excessive drought of the season, deductions of many lacs having been allowed to the farmers, who are still left unsatisfied"; and then he proceeds with a long detail of his own distress, and that of his family and all his dependants; and adds, "that the new-raised brigade is not only quite useless to my government, but is, moreover, the cause of much loss both in revenues and customs. The detached body of troops under European officers bring nothing but confusion to the affairs of my government, and are entirely their own masters." Mr. Middleton, Mr. Hastings's confidential resident, vouches for the truth of this representation in its fullest extent. "I am concerned to confess that there is too good ground for this plea. The misfortune hat been general throughout the whole of the vizier's [the Nabob of Oude] dominions, obvious to everybody; and so fatal have been its consequences, that no person of either credit or character would enter into engagements with government for farming the country." He then proceeds to give strong instances of the general calamity, and its effects.

It was now to be seen what steps the Governor-General and Council took for the relief of this distressed country, long laboring under the vexations of men, and now stricken by the hand of God. The case of a general famine is known to relax the severity even of the most rigorous government.—Mr. Hastings does not deny or show the least doubt of the fact. The representation is humble, and almost abject. On this representation from a great prince of the distress of his subjects, Mr. Hastings falls into a violent passion,—such as (it seems) would be unjustifiable in any one who speaks of any part of his conduct. He declares "that the demands, the tone in which they were asserted, and the season in which they were made, are all equally alarming, and appear to him to require an adequate degree of firmness in this board in opposition to them." He proceeds to deal out very unreserved language on the person and character of the Nabob and his ministers. He declares, that, in a division between him and the Nabob, "the strongest must decide." With regard to the urgent and instant necessity from the failure of the crops, he says, "that perhaps expedients may be found for affording a gradual relief from the burden of which he so heavily complains, and it shall be my endeavor to seek them out": and lest he should be suspected of too much haste to alleviate sufferings and to remove violence, he says, "that these must be gradually applied, and their complete effect may be distant; and this, I conceive, is all he can claim of right."

This complete effect of his lenity is distant indeed. Rejecting this demand, (as he calls the Nabob's abject supplication,) he attributes it, as he usually does all things of the kind, to the division in their government, and says, "This is a powerful motive with me (however inclined I might be, upon any other occasion, to yield to somepart of his demand) to give them an absolute and unconditional refusal upon the present,—and even to bring to punishment, if my influence can produce that effect, those incendiaries who have endeavored to make themselves the instruments of division between us."

Here, Sir, is much heat and passion,—but no more consideration of the distress of the country, from a failure of the means of subsistence, and (if possible) the worse evil of an useless and licentious soldiery, than if they were the most contemptible of all trifles. A letter is written, in consequence, in such a style of lofty despotism as I believe has hitherto been unexampled and unheard of in the records of the East. The troops were continued. The gradual relief, whose effect was to be so distant, has never been substantially and beneficially applied,—and the country is ruined.

Mr. Hastings, two years after, when it was too late, saw the absolute necessity of a removal of the intolerable grievance of this licentious soldiery, which, under pretence of defending it, held the country under military execution. A new treaty and arrangement, according to the pleasure of Mr. Hastings, took place; and this new treaty was broken in the old manner, in every essential article. The soldiery were again sent, and again set loose. The effect of all his manoeuvres, from which it seems he was sanguine enough to entertain hopes, upon the state of the country, he himself informs us,—"The event has proved the reverse of these hopes, and accumulation of distress, debasement, and dissatisfaction to the Nabob, and disappointment and disgrace to me.—Every measure [which he had himself proposed] has been so conducted as to give him cause of displeasure. There are no officers established by which his affairs could be regularly conducted: mean, incapable, and indigent men have been appointed. A number of the districts without authority, and without the means of personal protection; some of them have been murdered by the zemindars, and those zemindars, instead of punishment, have been permitted to retain their zemindaries, with independent authority; all the other zemindars suffered to rise up in rebellion, and to insult the authority of the sircar, without any attempt made to suppress them; and the Company's debt, instead of being discharged by the assignments and extraordinary sources of money provided for that purpose, is likely to exceed even the amount at which it stood at the time in which the arrangement with his Excellency was concluded." The House will smile at the resource on which the Directors take credit as such a certainty in their curious account.

This is Mr. Hastings's own narrative of the effects of his own settlement. This is the state of the country which we have been told is in perfect peace and order; and, what is curious, he informs us, that every part of this was foretold to him in the order and manner in which it happened, at the very time he made his arrangement of men and measures.

The invariable course of the Company's policy is this: either they set up some prince too odious to maintain himself without the necessity of their assistance, or they soon render him odious by making him the instrument of their government. In that case troops are bountifully sent to him to maintain his authority. That he should have no want of assistance, a civil gentleman, called a Resident, is kept at his court, who, under pretence of providing duly for the pay of these troops, gets assignments on the revenue into his hands. Under his provident management, debts soon accumulate; new assignments are made for these debts; until, step by step, the whole revenue, and with it the whole power of the country, is delivered into his hands. The military do not behold without a virtuous emulation the moderate gains of the civil department. They feel that in a country driven to habitual rebellion by the civil government the military is necessary; and they will not permit their services to go unrewarded. Tracts of country are delivered over to their discretion. Then it is found proper to convert their commanding officers into farmers of revenue. Thus, between the well-paid civil and well-rewarded military establishment, the situation of the natives may be easily conjectured. The authority of the regular and lawful government is everywhere and in every point extinguished. Disorders and violences arise; they are repressed by other disorders and other violences. Wherever the collectors of the revenue and the farming colonels and majors move, ruin is about them, rebellion before and behind them. The people in crowds fly out of the country; and the frontier is guarded by lines of troops, not to exclude an enemy, but to prevent the escape of the inhabitants.

By these means, in the course of not more than four or five years, this once opulent and flourishing country, which, by the accounts given in the Bengal consultations, yielded more than three crore of sicca rupees, that is, above three millions sterling, annually, is reduced, as far as I can discover, in a matter purposely involved in the utmost perplexity, to less than one million three hundred thousand pounds, and that exacted by every mode of rigor that can be devised. To complete the business, most of the wretched remnants of this revenue are mortgaged, and delivered into the hands of the usurers at Benares (for there alone are to be found some lingering remains of the ancient wealth of these regions) at an interest of near thirty per cent per annum.

The revenues in this manner failing, they seized upon the estates of every person of eminence in the country, and, under the name of resumption, confiscated their property. I wish, Sir, to be understood universally and literally, when I assert that there is not left one man of property and substance for his rank in the whole of these provinces, in provinces which are nearly the extent of England and Wales taken together: not one landholder, not one banker, not one merchant, not one even of those who usually perish last, the ultimum moriens in a ruined state, not one farmer of revenue.

One country for a while remained, which stood as an island in the midst of the grand waste of the Company's dominion. My right honorable friend, in his admirable speech on moving the bill, just touched the situation, the offences, and the punishment of a native prince, called Fizulla Khân. This man, by policy and force, had protected himself from the general extirpation of the Rohilla chiefs. He was secured (if that were any security) by a treaty. It was stated to you, as it was stated by the enemies of that unfortunate man, "that the whole of his country is what the whole country of the Rohillas was, cultivated like a garden, without one neglected spot in it." Another accuser says,—"Fyzoolah Khan, though a bad soldier, [that is the true source of his misfortune,] has approved himself a good aumil,—having, it is supposed, in the course of a few years, at least doubled the population and revenue of his country." In another part of the correspondence he is charged with making his country an asylum for the oppressed peasants who fly from the territories of Oude. The improvement of his revenue, arising from this single crime, (which Mr. Hastings considers as tantamount to treason,) is stated at an hundred and fifty thousand pounds a year.

Dr. Swift somewhere says, that he who could make two blades of grass grow where but one grew before was a greater benefactor to the human race than all the politicians that ever existed. This prince, who would have been deified by antiquity, who would have been ranked with Osiris, and Bacchus, and Ceres, and the divinities most propitious to men, was, for those very merits, by name attacked by the Company's government, as a cheat, a robber, a traitor. In the same breath in which he was accused as a rebel, he was ordered at once to furnish five thousand horse. On delay, or (according to the technical phrase, when any remonstrance is made to them) "on evasion," he was declared a violator of treaties, and everything he had was to be taken from him. Not one word, however, of horse in this treaty.

The territory of this Fizulla Khân, Mr. Speaker, is less than the County of Norfolk. It is an inland country, full seven hundred miles from any seaport, and not distinguished for any one considerable branch of manufacture whatsoever. From this territory several very considerable sums had at several times been paid to the British resident. The demand of cavalry, without a shadow or decent pretext of right, amounted to three hundred thousand a year more, at the lowest computation; and it is stated, by the last person sent to negotiate, as a demand of little use, if it could be complied with,—but that the compliance was impossible, as it amounted to more than his territories could supply, if there had been no other demand upon him. Three hundred thousand pounds a year from an inland country not so large as Norfolk!

The thing most extraordinary was to hear the culprit defend himself from the imputation of his virtues, as if they had been the blackest offences. He extenuated the superior cultivation of his country. He denied its population. He endeavored to prove that he had often sent back the poor peasant that sought shelter with him.—I can make no observation on this.

After a variety of extortions and vexations, too fatiguing to you, too disgusting to me, to go through with, they found "that they ought to be in a better state to warrant forcible means"; they therefore contented themselves with a gross sum of one hundred and fifty thousand pounds for their present demand. They offered him, indeed, an indemnity from their exactions in future for three hundred thousand pounds more. But he refused to buy their securities,—pleading (probably with truth) his poverty; but if the plea were not founded, in my opinion very wisely: not choosing to deal any more in that dangerous commodity of the Company's faith; and thinking it better to oppose distress and unarmed obstinacy to uncolored exaction than to subject himself to be considered as a cheat, if he should make a treaty in the least beneficial to himself.

Thus they executed an exemplary punishment on Fizulla Khân for the culture of his country. But, conscious that the prevention of evils is the great object of all good regulation, they deprived him of the means of increasing that criminal cultivation in future, by exhausting his coffers; and that the population of his country should no more be a standing reproach and libel on the Company's government, they bound him by a positive engagement not to afford any shelter whatsoever to the farmers and laborers who should seek refuge in his territories from the exactions of the British residents in Oude. When they had done all this effectually, they gave him a full and complete acquittance from all charges of rebellion, or of any intention to rebel, or of his having originally had any interest in, or any means of, rebellion.

These intended rebellions are one of the Company's standing resources. When money has been thought to be heaped up anywhere, its owners are universally accused of rebellion, until they are acquitted of their money and their treasons at once. The money once taken, all accusation, trial, and punishment ends. It is so settled a resource, that I rather wonder how it comes to be omitted in the Directors' account; but I take it for granted this omission will be supplied in their next edition.

The Company stretched this resource to the full extent, when they accused two old women, in the remotest corner of India, (who could have no possible view or motive to raise disturbances,) of being engaged in rebellion, with an intent to drive out the English nation, in whose protection, purchased by money and secured by treaty, rested the sole hope of their existence. But the Company wanted money, and the old women must be guilty of a plot. They were accused of rebellion, and they were convicted of wealth. Twice had great sums been extorted from them, and as often had the British faith guarantied the remainder. A body of British troops, with one of the military farmers-general at their head, was sent to seize upon the castle in which these helpless women resided. Their chief eunuchs, who were their agents, their guardians, protectors, persons of high rank according to the Eastern manners, and of great trust, were thrown into dungeons, to make them discover their hidden treasures; and there they lie at present. The lands assigned for the maintenance of the women were seized and confiscated. Their jewels and effects were taken, and set up to a pretended auction in an obscure place, and bought at such a price as the gentlemen thought proper to give. No account has ever been transmitted of the articles or produce of this sale. What money was obtained is unknown, or what terms were stipulated for the maintenance of these despoiled and forlorn creatures: for by some particulars it appears as if an engagement of the kind was made.

Let me here remark, once for all, that though the act of 1773 requires that an account of all proceedings should be diligently transmitted, that this, like all the other injunctions of the law, is totally despised, and that half at least of the most important papers are intentionally withheld.

I wish you, Sir, to advert particularly, in this transaction, to the quality and the numbers of the persons spoiled, and the instrument by whom that spoil was made. These ancient matrons, called the Begums, or Princesses, were of the first birth and quality in India: the one mother, the other wife, of the late Nabob of Oude, Sujah Dowlah, a prince possessed of extensive and flourishing dominions, and the second man in the Mogul Empire. This prince (suspicious, and not unjustly suspicious, of his son and successor) at his death committed his treasures and his family to the British faith. That family and household consisted of two thousand women, to which were added two other seraglios of near kindred, and said to be extremely numerous, and (as I am well informed) of about fourscore of the Nabob's children, with all the eunuchs, the ancient servants, and a multitude of the dependants of his splendid court. These were all to be provided, for present maintenance and future establishment, from the lands assigned as dower, and from the treasures which he left to these matrons, in trust for the whole family.

So far as to the objects of the spoil. The instrument chosen by Mr. Hastings to despoil the relict of Sujah Dowlah was her own son, the reigning Nabob of Oude. It was the pious hand of a son that was selected to tear from his mother and grandmother the provision of their age, the maintenance of his brethren, and of all the ancient household of his father. [Here a laugh, from some young members.] The laugh is seasonable, and the occasion decent and proper.

By the last advices, something of the sum extorted remained unpaid. The women, in despair, refuse to deliver more, unless their lands are restored, and their ministers released from prison; but Mr. Hastings and his council, steady to their point, and consistent to the last in their conduct, write to the resident to stimulate the son to accomplish the filial acts he had brought so near to their perfection. "We desire," say they in their letter to the resident, (written so late as March last,) "that you will inform us if any, and what means, have been taken for recovering the balance due from the Begum [Princess] at Fyzabad; and that, if necessary, you recommend it to the vizier to enforce the most effectual means for that purpose."

What their effectual means of enforcing demands on women of high rank and condition are I shall show you, Sir, in a few minutes, when I represent to you another of these plots and rebellions, which always in India, though so rarely anywhere else, are the offspring of an easy condition and hoarded riches.

Benares is the capital city of the Indian religion. It is regarded as holy by a particular and distinguished sanctity; and the Gentoos in general think themselves as much obliged to visit it once in their lives as the Mahometans to perform their pilgrimage to Mecca. By this means that city grew great in commerce and opulence; and so effectually was it secured by the pious veneration of that people, that in all wars and in all violences of power there was so sure an asylum both for poverty and wealth, (as it were under a divine protection,) that the wisest laws and best assured free constitution could not better provide for the relief of the one or the safety of the other; and this tranquillity influenced to the greatest degree the prosperity of all the country, and the territory of which it was the capital. The interest of money there was not more than half the usual rate in which it stood in all other places. The reports have fully informed you of the means and of the terms in which this city and the territory called Ghazipoor, of which it was the head, came under the sovereignty of the East India Company.

If ever there was a subordinate dominion pleasantly circumstanced to the superior power, it was this. A large rent or tribute, to the amount of two hundred and sixty thousand pounds a year, was paid in monthly instalments with the punctuality of a dividend at the Bank. If ever there was a prince who could not have an interest in disturbances, it was its sovereign, the Rajah Cheit Sing. He was in possession of the capital of his religion, and a willing revenue was paid by the devout people who resorted to him from all parts. His sovereignty and his independence, except his tribute, was secured by every tie. His territory was not much less than half of Ireland, and displayed in all parts a degree of cultivation, ease, and plenty, under his frugal and paternal management, which left him nothing to desire, either for honor or satisfaction.

This was the light in which this country appeared to almost every eye. But Mr. Hastings beheld it askance. Mr. Hastings tells us that it was reported of this Cheit Sing, that his father left him a million sterling, and that he made annual accessions to the hoard. Nothing could be so obnoxious to indigent power. So much wealth could not be innocent. The House is fully acquainted with the unfounded and unjust requisitions which were made upon this prince. The question has been most ably and conclusively cleared up in one of the reports of the select committee, and in an answer of the Court of Directors to an extraordinary publication against them by their servant, Mr. Hastings. But I mean to pass by these exactions as if they were perfectly just and regular; and having admitted them, I take what I shall now trouble you with only as it serves to show the spirit of the Company's government, the mode in which it is carried on, and the maxims on which it proceeds.

Mr. Hastings, from whom I take the doctrine, endeavors to prove that Cheit Sing was no sovereign prince, but a mere zemindar, or common subject, holding land by rent. If this be granted to him, it is next to be seen under what terms he is of opinion such a landholder, that is a British subject, holds his life and property under the Company's government. It is proper to understand well the doctrines of the person whose administration has lately received such distinguished approbation from the Company. His doctrine is,—"That the Company, or the person delegated by it, holds an absolute authority over such zemindars;—that he [such a subject] owes an implicit and unreserved obedience to its authority, at the forfeiture even of his life and property, at the DISCRETION of those who held or fully represented the sovereign authority;—and that these rights are fully delegated to him, Mr. Hastings."

Such is a British governor's idea of the condition of a great zemindar holding under a British authority; and this kind of authority he supposes fully delegated to him,—though no such delegation appears in any commission, instruction, or act of Parliament. At his discretion he may demand of the substance of any zemindar, over and above his rent or tribute, even, what he pleases, with a sovereign authority; and if he does not yield an implicit, unreserved obedience to all his commands, he forfeits his lands, his life, and his property, at Mr. Hastings's discretion. But, extravagant, and even frantic, as these positions appear, they are less so than what I shall now read to you; for he asserts, that, if any one should urge an exemption from more than a stated payment, or should consider the deeds which passed between him and the Board "as bearing the quality and force of a treaty between equal states," he says, "that such an opinion is itself criminal to the state of which he is a subject; and that he was himself amenable to its justice, if he gave countenance to such a belief." Here is a new species of crime invented, that of countenancing a belief,—but a belief of what? A belief of that which the Court of Directors, Hastings's masters, and a committee of this House, have decided as this prince's indisputable right.

But supposing the Rajah of Benares to be a mere subject, and that subject a criminal of the highest form; let us see what course was taken by an upright English magistrate. Did he cite this culprit before his tribunal? Did he make a charge? Did he produce witnesses? These are not forms; they are parts of substantial and eternal justice. No, not a word of all this. Mr. Hastings concludes him, in his own mind, to be guilty: he makes this conclusion on reports, on hearsays, on appearances, on rumors, on conjectures, on presumptions; and even these never once hinted to the party, nor publicly to any human being, till the whole business was done.

But the Governor tells you his motive for this extraordinary proceeding, so contrary to every mode of justice towards either a prince or a subject, fairly and without disguise; and he puts into your hands the key of his whole conduct:—"I will suppose, for a moment, that I have acted with unwarrantable rigor towards Cheit Sing, and even with injustice.—Let my MOTIVE be consulted. I left Calcutta, impressed with a belief that extraordinary means were necessary, and those exerted with a steady hand, to preserve the Company's interests from sinking under the accumulated weight which oppressed them. I saw a political necessity for curbing the overgrown power of a great member of their dominion, and for making it contribute to the relief of their pressing exigencies." This is plain speaking; after this, it is no wonder that the Rajah's wealth and his offence, the necessities of the judge and the opulence of the delinquent, are never separated, through the whole of Mr. Hastings's apology. "The justice and policy of exacting a large pecuniary mulct." The resolution "to draw from his guilt the means of relief to the Company's distresses." His determination "to make him pay largely for his pardon, or to execute a severe vengeance for past delinquency." That "as his wealth was great, and the Company's exigencies pressing, he thought it a measure of justice and policy to exact from him a large pecuniary mulct for their relief."—"The sum" (says Mr. Wheler, bearing evidence, at his desire, to his intentions) "to which the Governor declared his resolution to extend his fine was forty or fifty lacs, that is, four or five hundred thousand pounds; and that, if he refused, he was to be removed from his zemindary entirely; or by taking possession of his forts, to obtain, out of the treasure deposited in them, the above sum for the Company."

Crimes so convenient, crimes so politic, crimes so necessary, crimes so alleviating of distress, can never be wanting to those who use no process, and who produce no proofs.

But there is another serious part (what is not so?) in this affair. Let us suppose that the power for which Mr. Hastings contends, a power which no sovereign ever did or ever can vest in any of his subjects, namely, his own sovereign authority, to be conveyed by the act of Parliament to any man or body of men whatsoever; it certainly was never given to Mr. Hastings. The powers given by the act of 1773 were formal and official; they were given, not to the Governor-General, but to the major vote of the board, as a board, on discussion amongst themselves, in their public character and capacity; and their acts in that character and capacity were to be ascertained by records and minutes of council. The despotic acts exercised by Mr. Hastings were done merely in his private character; and, if they had been moderate and just, would still be the acts of an usurped authority, and without any one of the legal modes of proceeding which could give him competence for the most trivial exertion of power. There was no proposition or deliberation whatsoever in council, no minute on record, by circulation or otherwise, to authorize his proceedings; no delegation of power to impose a fine, or to take any step to deprive the Rajah of Benares of his government, his property, or his liberty. The minutes of consultation assign to his journey a totally different object, duty, and destination. Mr. Wheler, at his desire, tells us long after, that he had a confidential conversation with him on various subjects, of which this was the principal, in which Mr. Hastings notified to him his secret intentions; "and that he bespoke his support of the measures which he intended to pursue towards him (the Rajah)." This confidential discourse, and bespeaking of support, could give him no power, in opposition to an express act of Parliament, and the whole tenor of the orders of the Court of Directors.

In what manner the powers thus usurped were employed is known to the whole world. All the House knows that the design on the Rajah proved as unfruitful as it was violent. The unhappy prince was expelled, and his more unhappy country was enslaved and ruined; but not a rupee was acquired. Instead of treasure to recruit the Company's finances, wasted by their wanton wars and corrupt jobs, they were plunged into a new war, which shook their power in India to its foundation, and, to use the Governor's own happy simile, might have dissolved it like a magic structure, if the talisman had been broken.

But the success is no part of my consideration, who should think just the same of this business, if the spoil of one rajah had been fully acquired, and faithfully applied to the destruction of twenty other rajahs. Not only the arrest of the Rajah in his palace was unnecessary and unwarrantable, and calculated to stir up any manly blood which remained in his subjects, but the despotic style and the extreme insolence of language and demeanor, used to a person of great condition among the politest people in the world, was intolerable. Nothing aggravates tyranny so much as contumely. Quicquid superbia in contumeliis was charged by a great man of antiquity, as a principal head of offence against the Governor-General of that day. The unhappy people were still more insulted. A relation, but an enemy to the family, a notorious robber and villain, called Ussaun Sing, kept as a hawk in a mew, to fly upon this nation, was set up to govern there, instead of a prince honored and beloved. But when the business of insult was accomplished, the revenue was too serious a concern to be intrusted to such hands. Another was set up in his place, as guardian to an infant.

But here, Sir, mark the effect of all these extraordinary means, of all this policy and justice. The revenues, which had been hitherto paid with such astonishing punctuality, fell into arrear. The new prince guardian was deposed without ceremony,—and with as little, cast into prison. The government of that once happy country has been in the utmost confusion ever since such good order was taken about it. But, to complete the contumely offered to this undone people, and to make them feel their servitude in all its degradation and all its bitterness, the government of their sacred city, the government of that Benares which had been so respected by Persian and Tartar conquerors, though of the Mussulman persuasion, that, even in the plenitude of their pride, power, and bigotry, no magistrate of that sect entered the place, was now delivered over by English hands to a Mahometan; and an Ali Ibrahim Khân was introduced, under the Company's authority, with power of life and death, into the sanctuary of the Gentoo religion. After this, the taking off a slight payment, cheerfully made by pilgrims to a chief of their own rites, was represented as a mighty benefit.

It remains only to show, through the conduct in this business, the spirit of the Company's government, and the respect they pay towards other prejudices, not less regarded in the East than those of religion: I mean the reverence paid to the female sex in general, and particularly to women of high rank and condition. During the general confusion of the country of Ghazipoor, Panna, the mother of Cheit Sing, was lodged with her train in a castle called Bidgé Gur, in which were likewise deposited a large portion of the treasures of her son, or more probably her own. To whomsoever they belonged was indifferent: for, though no charge of rebellion was made on this woman, (which was rather singular, as it would have cost nothing,) they were resolved to secure her with her fortune. The castle was besieged by Major Popham.

There was no great reason to apprehend that soldiers ill paid, that soldiers who thought they had been defrauded of their plunder on former services of the same kind, would not have been sufficiently attentive to the spoil they were expressly come for; but the gallantry and generosity of the profession was justly suspected, as being likely to set bounds to military rapaciousness. The Company's first civil magistrate discovered the greatest uneasiness lest the women should have anything preserved to them. Terms tending to put some restraint on military violence were granted. He writes a letter to Mr. Popham, referring to some letter written before to the same effect, which I do not remember to have seen; but it shows his anxiety on this subject. Hear himself:—"I think every demand she has made on you, except that of safety and respect to her person, is unreasonable. If the reports brought to me are true, your rejecting her offers, or any negotiation, would soon obtain you the fort upon your own terms. I apprehend she will attempt to defraud the captors of a considerable part of their booty, by being suffered to retire without examination. But this is your concern, not mine. I should be very sorry that your officers and soldiers lost any part of the reward to which they are so well entitled; but you must be the best judge of the promised indulgence to the Ranny: what you have engaged for I will certainly ratify; but as to suffering the Ranny to hold the purgunna of Hurlich, or any other zemindary, without being subject to the authority of the zemindar, or any lands whatsoever, or indeed making any condition with her for a provision, I will never consent."

Here your Governor stimulates a rapacious and licentious soldiery to the personal search of women, lest these unhappy creatures should avail themselves of the protection of their sex to secure any supply for their necessities; and he positively orders that no stipulation should be made for any provision for them. The widow and mother of a prince, well informed of her miserable situation, and the cause of it, a woman of this rank became a suppliant to the domestic servant of Mr. Hastings, (they are his own words that I read,) "imploring his intercession that she may be relieved from the hardships and dangers of her present situation, and offering to surrender the fort, and the treasure and valuable effects contained in it, provided she can be assured of safety and protection to her person and honor, and to that of her family and attendants." He is so good as to consent to this, "provided she surrenders everything of value, with the reserve only of such articles as you shall think necessary to her condition, or as you yourself shall be disposed to indulge her with.—But should she refuse to execute the promise she has made, or delay it beyond the term of twenty-four hours, it is my positive injunction that you immediately put a stop to any further intercourse or negotiation with her, and on no pretext renew it. If she disappoints or trifles with me, after I have subjected my duan to the disgrace of returning ineffectually, and of course myself to discredit, I shall consider it as a wanton affront and indignity which I can never forgive; nor will I grant her any conditions whatever, but leave her exposed to those dangers which she has chosen to risk, rather than trust to the clemency and generosity of our government. I think she cannot be ignorant of these consequences, and will not venture to incur them; and it is for this reason I place a dependence on her offers, and have consented to send my duan to her." The dreadful secret hinted at by the merciful Governor in the latter part of the letter is well understood in India, where those who suffer corporeal indignities generally expiate the offences of others with their own blood. However, in spite of all these, the temper of the military did, some way or other, operate. They came to terms which have never been transmitted. It appears that a fifteenth per cent of the plunder was reserved to the captives, of which the unhappy mother of the Prince of Benares was to have a share. This ancient matron, born to better things [A laugh from certain young gentlemen]—I see no cause for this mirth. A good author of antiquity reckons among the calamities of his time "nobilissimarum fæminarum exilia et fugas." I say, Sir, this ancient lady was compelled to quit her house, with three hundred helpless women and a multitude of children in her train. But the lower sort in the camp, it seems, could not be restrained. They did not forget the good lessons of the Governor-General. They were unwilling "to be defrauded of a considerable part of their booty by suffering them to pass without examination."—They examined them, Sir, with a vengeance; and the sacred protection of that awful character, Mr. Hastings's maître d'hôtel, could not secure them from insult and plunder. Here is Popham's narrative of the affair:—

"The Ranny came out of the fort, with her family and dependants, the tenth, at night, owing to which such attention was not paid to her as I wished; and I am exceedingly sorry to inform you that the licentiousness of our followers was beyond the bounds of control; for, notwithstanding all I could do, her people were plundered on the road of most of the things which they brought out of the fort, by which means one of the articles of surrender has been much infringed. The distress I have felt upon this occasion cannot be expressed, and can only be allayed by a firm performance of the other articles of the treaty, which I shall make it my business to enforce.—The suspicions which the officers had of treachery, and the delay made to our getting possession, had enraged them, as well as the troops, so much, that the treaty was at first regarded as void; but this determination was soon succeeded by pity and compassion for the unfortunate besieged."—After this comes, in his due order, Mr. Hastings; who is full of sorrow and indignation, \&c., \&c., \&c., according to the best and most authentic precedents established upon such occasions.

The women being thus disposed of, that is, completely despoiled, and pathetically lamented, Mr. Hastings at length recollected the great object of his enterprise, which, during his zeal lest the officers and soldiers should lose any part of their reward, he seems to have forgot,—that is to say, "to draw from the Rajah's guilt the means of relief to the Company's distresses." This was to be the stronghold of his defence. This compassion to the Company, he knew by experience, would sanctify a great deal of rigor towards the natives. But the military had distresses of their own, which they considered first. Neither Mr. Hastings's authority, nor his supplications, could prevail on them to assign a shilling to the claim he made on the part of the Company. They divided the booty amongst themselves. Driven from his claim, he was reduced to petition for the spoil as a loan. But the soldiers were too wise to venture as a loan what the borrower claimed as a right. In defiance of all authority, they shared among themselves about two hundred thousand pounds sterling, besides what had been taken from the women.

In all this there is nothing wonderful. We may rest assured, that, when the maxims of any government establish among its resources extraordinary means, and those exerted with a strong hand, that strong hand will provide those extraordinary means for itself. Whether the soldiers had reason or not, (perhaps much might be said for them,) certain it is, the military discipline of India was ruined from that moment; and the same rage for plunder, the same contempt of subordination, which blasted all the hopes of extraordinary means from your strong hand at Benares, have very lately lost you an army in Mysore. This is visible enough from the accounts in the last gazette.

There is no doubt but that the country and city of Benares, now brought into the same order, will very soon exhibit, if it does not already display, the same appearance with those countries and cities which are under better subjection. A great master, Mr. Hastings, has himself been at the pains of drawing a picture of one of these countries: I mean the province and city of Furruckabad. There is no reason to question his knowledge of the facts; and his authority (on this point at least) is above all exception, as well for the state of the country as for the cause. In his minute of consultation, Mr. Hastings describes forcibly the consequences which arise from the degradation into which we have sunk the native government. "The total want (says he) of all order, regularity, or authority, in his (the Nabob of Furruckabad's) government, and to which, among other obvious causes, it may no doubt be owing that the country of Furruckabad is become almost an entire waste, without cultivation or inhabitants,—that the capital, which but a very short time ago was distinguished as one of the most populous and opulent commercial cities in Hindostan, at present exhibits nothing but scenes of the most wretched poverty, desolation, and misery,—and that the Nabob himself, though in the possession of a tract of country which, with only common care, is notoriously capable of yielding an annual revenue of between thirty and forty lacs, (three or four hundred thousand pounds,) with no military establishment to maintain, scarcely commands the means of a bare subsistence."

This is a true and unexaggerated picture, not only of Furruckabad, but of at least three fourths of the country which we possess, or rather lay waste, in India. Now, Sir, the House will be desirous to know for what purpose this picture was drawn. It was for a purpose, I will not say laudable, but necessary: that of taking the unfortunate prince and his country out of the hands of a sequestrator sent thither by the Nabob of Oude, the mortal enemy of the prince thus ruined, and to protect him by means of a British resident, who might carry his complaints to the superior resident at Oude, or transmit them to Calcutta. But mark how the reformer persisted in his reformation. The effect of the measure was better than was probably expected. The prince began to be at ease; the country began to recover; and the revenue began to be collected. These were alarming circumstances. Mr. Hastings not only recalled the resident, but he entered into a formal stipulation with the Nabob of Oude never to send an English subject again to Furruckabad; and thus the country, described as you have heard by Mr. Hastings, is given up forever to the very persons to whom he had attributed its ruin,—that is, to the sezawals or sequestrators of the Nabob of Oude.

Such was the issue of the first attempt to relieve the distresses of the dependent provinces. I shall close what I have to say on the condition of the northern dependencies with the effect of the last of these attempts. You will recollect, Sir, the account I have not long ago stated to you, as given by Mr. Hastings, of the ruined condition of the destroyer of others, the Nabob of Oude, and of the recall, in consequence, of Hannay, Middleton, and Johnson. When the first little sudden gust of passion against these gentlemen was spent, the sentiments of old friendship began to revive. Some healing conferences were held between them and the superior government. Mr. Hannay was permitted to return to Oude; but death prevented the further advantages intended for him, and the future benefits proposed for the country by the provident cars of the Council-General.

One of these gentlemen was accused of the grossest peculations; two of them by Mr. Hastings himself, of what he considered as very gross offences. The Court of Directors were informed, by the Governor-General and Council, that a severe inquiry would be instituted against the two survivors; and they requested that court to suspend its judgment, and to wait the event of their proceedings. A mock inquiry has been instituted, by which the parties could not be said to be either acquitted or condemned. By means of the bland and conciliatory dispositions of the charter-governors, and proper private explanations, the public inquiry has in effect died away; the supposed peculators and destroyers of Oude repose in all security in the bosoms of their accusers; whilst others succeed to them to be instructed by their example.

It is only to complete the view I proposed of the conduct of the Company with regard to the dependent provinces, that I shall say any thing at all of the Carnatic, which is the scene, if possible, of greater disorder than the northern provinces. Perhaps it were better to say of this centre and metropolis of abuse, whence all the rest in India and in England diverge, from whence they are fed and methodized, what was said of Carthage,—"De Carthagine satius est silere quam parum dicere." This country, in all its denominations, is about 46,000 square miles. It may be affirmed universally, that not one person of substance or property, landed, commercial, or moneyed, excepting two or three bankers, who are necessary deposits and distributors of the general spoil, is left in all that region. In that country, the moisture, the bounty of Heaven, is given but at a certain season. Before the era of our influence, the industry of man carefully husbanded that gift of God. The Gentoos preserved, with a provident and religious care, the precious deposit of the periodical rain in reservoirs, many of them works of royal grandeur; and from these, as occasion demanded, they fructified the whole country. To maintain these reservoirs, and to keep up an annual advance to the cultivators for seed and cattle, formed a principal object of the piety and policy of the priests and rulers of the Gentoo religion.

This object required a command of money; and there was no pollam, or castle, which in the happy days of the Carnatic was without some hoard of treasure, by which the governors were enabled to combat with the irregularity of the seasons, and to resist or to buy off the invasion of an enemy. In all the cities were multitudes of merchants and bankers, for all occasions of moneyed assistance; and on the other hand, the native princes were in condition to obtain credit from them. The manufacturer was paid by the return of commodities, or by imported money, and not, as at present, in the taxes that had been originally exacted from his industry. In aid of casual distress, the country was full of choultries, which were inns and hospitals, where the traveller and the poor were relieved. All ranks of people had their place in the public concern, and their share in the common stock and common prosperity. But the chartered rights of men, and the right which it was thought proper to set up in the Nabob of Arcot, introduced a new system. It was their policy to consider hoards of money as crimes,—to regard moderate rents as frauds on the sovereign,—and to view, in the lesser princes, any claim of exemption from more than settled tribute as an act of rebellion. Accordingly, all the castles were, one after the other, plundered and destroyed; the native princes were expelled; the hospitals fell to ruin; the reservoirs of water went to decay; the merchants, bankers, and manufacturers disappeared; and sterility, indigence, and depopulation overspread the face of these once flourishing provinces.

The Company was very early sensible of these mischiefs, and of their true cause. They gave precise orders, "that the native princes, called polygars, should not be extirpated." "The rebellion" (so they choose to call it) "of the polygars may, they fear, with, too much justice, be attributed to the maladministration of the Nabob's collectors." "They observe with concern, that their troops have been put to disagreeable services." They might have used a stronger expression without impropriety. But they make amends in another place. Speaking of the polygars, the Directors say that "it was repugnant to humanity to force them to such dreadful extremities as they underwent"; that some examples of severity might be necessary, "when they fell into the Nabob's hands," and not by the destruction of the country; "that they fear his government is none of the mildest, and that there is great oppression in collecting his revenues." They state, that the wars in which he has involved the Carnatic had been a cause of its distresses; "that those distresses have been certainly great, but those by the Nabob's oppressions they believe to be greater than all." Pray, Sir, attend to the reason for their opinion that the government of this their instrument is more calamitous to the country than the ravages of war:—Because, say they, his oppressions are "without intermission; the others are temporary;—by all which oppressions we believe the Nabob has great wealth in store." From this store neither he nor they could derive any advantage whatsoever, upon the invasion of Hyder Ali, in the hour of their greatest calamity and dismay.

It is now proper to compare these declarations with the Company's conduct. The principal reason which they assigned against the extirpation of the polygars was, that the weavers were protected in their fortresses. They might have added, that the Company itself, which stung them to death, had been warmed in the bosom of these unfortunate princes: for, on the taking of Madras by the French, it was in their hospitable pollams that most of the inhabitants found refuge and protection. But notwithstanding all these orders, reasons, and declarations, they at length gave an indirect sanction, and permitted the use of a very direct and irresistible force, to measures which they had over and over again declared to be false policy, cruel, inhuman, and oppressive. Having, however, forgot all attention to the princes and the people, they remembered that they had some sort of interest in the trade of the country; and it is matter of curiosity to observe the protection which they afforded to this their natural object.

Full of anxious cares on this head, they direct, "that, in reducing the polygars, they [their servants] were to be cautious not to deprive the weavers and manufacturers of the protection they often met with in the strongholds of the polygar countries"; and they write to their instrument, the Nabob of Arcot, concerning these poor people in a most pathetic strain. "We entreat your Excellency," (say they,) "in particular, to make the manufacturers the object of your tenderest care; particularly when you root out the polygars, you do not deprive the weavers of the protection they enjoyed under them." When they root out the protectors in favor of the oppressor, they show themselves religiously cautious of the rights of the protected. When they extirpate the shepherd and the shepherd's dog, they piously recommend the helpless flock to the mercy, and even to the tenderest care, of the wolf. This is the uniform strain of their policy,—strictly forbidding, and at the same time strenuously encouraging and enforcing, every measure that can ruin and desolate the country committed to their charge. After giving the Company's idea of the government of this their instrument, it may appear singular, but it is perfectly consistent with their system, that, besides wasting for him, at two different times, the most exquisite spot upon the earth, Tanjore, and all the adjacent countries, they have even voluntarily put their own territory, that is, a large and fine country adjacent to Madras, called their jaghire, wholly out of their protection,—and have continued to farm their subjects, and their duties towards these subjects, to that very Nabob whom they themselves constantly represent as an habitual oppressor and a relentless tyrant. This they have done without any pretence of ignorance of the objects of oppression for which this prince has thought fit to become their renter; for he has again and again told them that it is for the sole purpose of exercising authority he holds the jaghire lands; and he affirms (and I believe with truth) that he pays more for that territory than the revenues yield. This deficiency he must make up from his other territories; and thus, in order to furnish the means of oppressing one part of the Carnatic, he is led to oppress all the rest.

The House perceives that the livery of the Company's government is uniform. I have described the condition of the countries indirectly, but most substantially, under the Company's authority. And now I ask, whether, with this map of misgovernment before me, I can suppose myself bound by my vote to continue, upon any principles of pretended public faith, the management of these countries in those hands. If I kept such a faith (which in reality is no better than a fides latronum) with what is called the Company, I must break the faith, the covenant, the solemn, original, indispensable oath, in which I am bound, by the eternal frame and constitution of things, to the whole human race.

As I have dwelt so long on these who are indirectly under the Company's administration, I will endeavor to be a little shorter upon the countries immediately under this charter-government. These are the Bengal provinces. The condition of these provinces is pretty fully detailed in the Sixth and Ninth Reports, and in their Appendixes. I will select only such principles and instances as are broad and general. To your own thoughts I shall leave it to furnish the detail of oppressions involved in them. I shall state to you, as shortly as I am able, the conduct of the Company:—1st, towards the landed interests;—next, the commercial interests;—3rdly, the native government;—and lastly, to their own government.

Bengal, and the provinces that are united to it, are larger than the kingdom of France, and once contained, as France does contain, a great and independent landed interest, composed of princes, of great lords, of a numerous nobility and gentry, of freeholders, of lower tenants, of religious communities, and public foundations. So early as 1769, the Company's servants perceived the decay into which these provinces had fallen under English administration, and they made a strong representation upon this decay, and what they apprehended to be the causes of it. Soon after this representation, Mr. Hastings became President of Bengal. Instead of administering a remedy to this melancholy disorder, upon the heels of a dreadful famine, in the year 1772, the succor which the new President and the Council lent to this afflicted nation was—shall I be believed in relating it?—the landed interest of a whole kingdom, of a kingdom to be compared to France, was set up to public auction! They set up (Mr. Hastings set up) the whole nobility, gentry, and freeholders to the highest bidder. No preference was given to the ancient proprietors. They must bid against every usurer, every temporary adventurer, every jobber and schemer, every servant of every European,—or they were obliged to content themselves, in lieu of their extensive domains, with their house, and such a pension as the state auctioneers thought fit to assign. In this general calamity, several of the first nobility thought (and in all appearance justly) that they had better submit to the necessity of this pension, than continue, under the name of zemindars, the objects and instruments of a system by which they ruined their tenants and were ruined themselves. Another reform has since come upon the back of the first; and a pension having been assigned to these unhappy persons, in lieu of their hereditary lands, a new scheme of economy has taken place, and deprived them of that pension.

The menial servants of Englishmen, persons (to use the emphatical phrase of a ruined and patient Eastern chief) "whose fathers they would not have set with the dogs of their flock" entered into their patrimonial lands. Mr. Hastings's banian was, after this auction, found possessed of territories yielding a rent of one hundred and forty thousand pounds a year.

Such an universal proscription, upon any pretence, has few examples. Such a proscription, without even a pretence of delinquency, has none. It stands by itself. It stands as a monument to astonish the imagination, to confound the reason of mankind. I confess to you, when I first came to know this business in its true nature and extent, my surprise did a little suspend my indignation. I was in a manner stupefied by the desperate boldness of a few obscure young men, who, having obtained, by ways which they could not comprehend, a power of which they saw neither the purposes nor the limits, tossed about, subverted, and tore to pieces, as if it were in the gambols of a boyish unluckiness and malice, the most established rights, and the most ancient and most revered institutions, of ages and nations. Sir, I will not now trouble you with any detail with regard to what they have since done with these same lands and landholders, only to inform you that nothing has been suffered to settle for two seasons together upon any basis, and that the levity and inconstancy of these mock legislators were not the least afflicting parts of the oppressions suffered under their usurpation; nor will anything give stability to the property of the natives, but an administration in England at once protecting and stable. The country sustains, almost every year, the miseries of a revolution. At present, all is uncertainty, misery, and confusion. There is to be found through these vast regions no longer one landed man who is a resource for voluntary aid or an object for particular rapine. Some of them were not long since great princes; they possessed treasures, they levied armies. There was a zemindar in Bengal, (I forget his name,) that, on the threat of an invasion, supplied the subah of these provinces with the loan of a million sterling. The family at this day wants credit for a breakfast at the bazaar.

I shall now say a word or two on the Company's care of the commercial interest of those kingdoms. As it appears in the Reports that persons in the highest stations in Bengal have adopted, as a fixed plan of policy, the destruction of all intermediate dealers between the Company and the manufacturer, native merchants have disappeared of course. The spoil of the revenues is the sole capital which purchases the produce and manufactures, and through three or four foreign companies transmits the official gains of individuals to Europe. No other commerce has an existence in Bengal. The transport of its plunder is the only traffic of the country. I wish to refer you to the Appendix to the Ninth Report for a full account of the manner in which the Company have protected the commercial interests of their dominions in the East.

As to the native government and the administration of justice, it subsisted in a poor, tottering manner for some years. In the year 1781 a total revolution took place in that establishment. In one of the usual freaks of legislation of the Council of Bengal, the whole criminal jurisdiction of these courts, called the Phoujdary Judicature, exercised till then by the principal Mussulmen, was in one day, without notice, without consultation with the magistrates or the people there, and without communication with the Directors or Ministers here, totally subverted. A new institution took place, by which this jurisdiction was divided between certain English servants of the Company and the Gentoo zemindars of the country, the latter of whom never petitioned for it, nor, for aught that appears, ever desired this boon. But its natural use was made of it: it was made a pretence for new extortions of money.

The natives had, however, one consolation in the ruin of their judicature: they soon saw that it fared no better with the English government itself. That, too, after destroying every other, came to its period. This revolution may well be rated for a most daring act, even among the extraordinary things that have been doing in Bengal since our unhappy acquisition of the means of so much mischief.

An establishment of English government for civil justice, and for the collection of revenue, was planned and executed by the President and Council of Bengal, subject to the pleasure of the Directors, in the year 1772. According to this plan, the country was divided into six districts, or provinces. In each of these was established a provincial council, which administered the revenue; and of that council, one member, by monthly rotation, presided in the courts of civil resort, with an appeal to the council of the province, and thence to Calcutta. In this system (whether in other respects good or evil) there were some capital advantages. There was, in the very number of persons in each provincial council, authority, communication, mutual check, and control. They were obliged, on their minutes of consultation, to enter their reasons and dissents; so that a man of diligence, of research, and tolerable sagacity, sitting in London, might, from these materials, be enabled to form some judgment of the spirit of what was going on on the furthest banks of the Ganges and Burrampooter.

The Court of Directors so far ratified this establishment, (which was consonant enough to their general plan of government,) that they gave precise orders that no alteration should be made in it without their consent. So far from being apprised of any design against this constitution, they had reason to conceive that on trial it had been more and more approved by their Council-General, at least by the Governor-General, who had planned it. At the time of the revolution, the Council-General was nominally in two persons, virtually in one. At that time measures of an arduous and critical nature ought to have been forborne, even if, to the fullest council, this specific measure had not been prohibited by the superior authority. It was in this very situation that one man had the hardiness to conceive and the temerity to execute a total revolution in the form and the persons composing the government of a great kingdom. Without any previous step, at one stroke, the whole constitution, of Bengal, civil and criminal, was swept away. The counsellors were recalled from their provinces; upwards of fifty of the principal officers of government were turned out of employ, and rendered dependent on Mr. Hastings for their immediate subsistence, and for all hope of future provision. The chief of each council, and one European collector of revenue, was left in each province.

But here, Sir, you may imagine a new government, of some permanent description, was established in the place of that which had been thus suddenly overturned. No such thing. Lest these chiefs, without councils, should be conceived to form the ground-plan of some future government, it was publicly declared that their continuance was only temporary and permissive. The whole subordinate British administration of revenue was then vested in a committee in Calcutta, all creatures of the Governor-General; and the provincial management, under the permissive chief, was delivered over to native officers.

But that the revolution and the purposes of the revolution might be complete, to this committee were delegated, not only the functions of all the inferior, but, what will surprise the House, those of the supreme administration of revenue also. Hitherto the Governor-General and Council had, in their revenue department, administered the finances of those kingdoms. By the new scheme they are delegated to this committee, who are only to report their proceedings for approbation.

The key to the whole transaction is given in one of the instructions to the committee,—"that it is not necessary that they should enter dissents." By this means the ancient plan of the Company's administration was destroyed; but the plan of concealment was perfected. To that moment the accounts of the revenues were tolerably clear,—or at least means were furnished for inquiries, by which they might be rendered satisfactory. In the obscure and silent gulf of this committee everything is now buried. The thickest shades of night surround all their transactions. No effectual means of detecting fraud, mismanagement, or misrepresentation exist. The Directors, who have dared to talk with such confidence on their revenues, know nothing about them. What used to fill volumes is now comprised under a few dry heads on a sheet of paper. The natives, a people habitually made to concealment, are the chief managers of the revenue throughout the provinces. I mean by natives such wretches as your rulers select out of them as most fitted for their purposes. As a proper keystone to bind the arch, a native, one Gunga Govind Sing, a man turned out of his employment by Sir John Clavering for malversation in office, is made the corresponding secretary, and, indeed, the great moving principle of their new board.

As the whole revenue and civil administration was thus subverted, and a clandestine government substituted in the place of it, the judicial institution underwent a like revolution. In 1772 there had been six courts, formed out of the six provincial councils. Eighteen new ones are appointed in their place, with each a judge, taken from the junior servants of the Company. To maintain these eighteen courts, a tax is levied on the sums in litigation, of two and one half per cent on the great, and of five per cent on the less. This money is all drawn from the provinces to Calcutta. The chief justice (the same who stays in defiance of a vote of this House, and of his Majesty's recall) is appointed at once the treasurer and disposer of these taxes, levied without any sort of authority from the Company, from the Crown, or from Parliament.

In effect, Sir, every legal, regular authority, in matters of revenue, of political administration, of criminal law, of civil law, in many of the most essential parts of military discipline, is laid level with the ground; and an oppressive, irregular, capricious, unsteady, rapacious, and peculating despotism, with a direct disavowal of obedience to any authority at home, and without any fixed maxim, principle, or rule of proceeding to guide them in India, is at present the state of your charter-government over great kingdoms.

As the Company has made this use of their trust, I should ill discharge mine, if I refused to give my most cheerful vote for the redress of these abuses, by putting the affairs of so large and valuable a part of the interests of this nation and of mankind into some steady hands, possessing the confidence and assured of the support of this House, until they can be restored to regularity, order, and consistency.

I have touched the heads of some of the grievances of the people and the abuses of government. But I hope and trust you will give me credit, when I faithfully assure you that I have not mentioned one fourth part of what has come to my knowledge in your committee; and further, I have full reason to believe that not one fourth part of the abuses are come to my knowledge, by that or by any other means. Pray consider what I have said only as an index to direct you in your inquiries.

If this, then, Sir, has been the use made of the trust of political powers, internal and external, given by you in the charter, the next thing to be seen is the conduct of the Company with regard to the commercial trust. And here I will make a fair offer:—If it can be proved that they have acted wisely, prudently, and frugally, as merchants, I shall pass by the whole mass of their enormities as statesmen. That they have not done this their present condition is proof sufficient. Their distresses are said to be owing to their wars. This is not wholly true. But if it were, is not that readiness to engage in wars, which distinguishes them, and for which the Committee of Secrecy has so branded their politics, founded on the falsest principles of mercantile speculation?

The principle of buying cheap and selling dear is the first, the great foundation of mercantile dealing. Have they ever attended to this principle? Nay, for years have they not actually authorized in their servants a total indifference as to the prices they were to pay?

A great deal of strictness in driving bargains for whatever we contract is another of the principles of mercantile policy. Try the Company by that test. Look at the contracts that are made for them. Is the Company so much as a good commissary to their own armies? I engage to select for you, out of the innumerable mass of their dealings, all conducted very nearly alike, one contract only the excessive profits on which during a short term would pay the whole of their year's dividend. I shall undertake to show that upon two others the inordinate profits given, with the losses incurred in order to secure those profits, would pay a year's dividend more.

It is a third property of trading-men to see that their clerks do not divert the dealings of the master to their own benefit. It was the other day only, when their Governor and Council taxed the Company's investment with a sum of fifty thousand pounds, as an inducement to persuade only seven members of their Board of Trade to give their honor that they would abstain from such profits upon that investment, as they must have violated their oaths, if they had made at all.

It is a fourth quality of a merchant to be exact in his accounts. What will be thought, when you have fully before you the mode of accounting made use of in the Treasury of Bengal? I hope you will have it soon. With regard to one of their agencies, when it came to the material part, the prime cost of the goods on which a commission of fifteen per cent was allowed, to the astonishment of the factory to whom the commodities were sent, the Accountant-General reports that he did not think himself authorized to call for vouchers relative to this and other particulars,—because the agent was upon his honor with regard to them. A new principle of account upon honor seems to be regularly established in their dealings and their treasury, which in reality amounts to an entire annihilation of the principle of all accounts.

It is a fifth property of a merchant, who does not meditate a fraudulent bankruptcy, to calculate his probable profits upon the money he takes up to vest in business. Did the Company, when they bought goods on bonds bearing eight per cent interest, at ten and even twenty per cent discount, even ask themselves a question concerning the possibility of advantage from dealing on these terms?

The last quality of a merchant I shall advert to is the taking care to be properly prepared, in cash or goods in the ordinary course of sale, for the bills which are drawn on them. Now I ask, whether they have ever calculated the clear produce of any given sales, to make them tally with the four million of bills which are come and coming upon them, so as at the proper periods to enable the one to liquidate the other. No, they have not. They are now obliged to borrow money of their own servants to purchase their investment. The servants stipulate five per cent on the capital they advance, if their bills should not be paid at the time when they become due; and the value of the rupee on which they charge this interest is taken at two shillings and a penny. Has the Company ever troubled themselves to inquire whether their sales can bear the payment of that interest, and at that rate of exchange? Have they once considered the dilemma in which they are placed,—the ruin of their credit in the East Indies, if they refuse the bills,—the ruin of their credit and existence in England, if they accept them?

Indeed, no trace of equitable government is found in their politics, not one trace of commercial principle in their mercantile dealing: and hence is the deepest and maturest wisdom of Parliament demanded, and the best resources of this kingdom must be strained, to restore them,—that is, to restore the countries destroyed by the misconduct of the Company, and to restore the Company itself, ruined by the consequences of their plans for destroying what they were bound to preserve.

I required, if you remember, at my outset, a proof that these abuses were habitual. But surely this is not necessary for me to consider as a separate head; because I trust I have made it evident beyond a doubt, in considering the abuses themselves, that they are regular, permanent, and systematical.

I am now come to my last condition, without which, for one, I will never readily lend my hand to the destruction of any established government, which is,—that, in its present state, the government of the East India Company is absolutely incorrigible.

Of this great truth I think there can be little doubt, after all that has appeared in this House. It is so very clear, that I must consider the leaving any power in their hands, and the determined resolution to continue and countenance every mode and every degree of peculation, oppression, and tyranny, to be one and the same thing. I look upon that body incorrigible, from the fullest consideration both of their uniform conduct and their present real and virtual constitution.

If they had not constantly been apprised of all the enormities committed in India under their authority, if this state of things had been as much a discovery to them as it was to many of us, we might flatter ourselves that the detection of the abuses would lead to their reformation. I will go further. If the Court of Directors had not uniformly condemned every act which this House or any of its committees had condemned, if the language in which they expressed their disapprobation against enormities and their authors had not been much more vehement and indignant than any ever used in this House, I should entertain some hopes. If they had not, on the other hand, as uniformly commended all their servants who had done their duty and obeyed their orders as they had heavily censured those who rebelled, I might say, These people have been in an error, and when they are sensible of it they will mend. But when I reflect on the uniformity of their support to the objects of their uniform censure, and the state of insignificance and disgrace to which all of those have been reduced whom they approved, and that even utter ruin and premature death have been among the fruits of their favor, I must be convinced, that in this case, as in all others, hypocrisy is the only vice that never can be cured.

Attend, I pray you, to the situation and prosperity of Benfield, Hastings, and others of that sort. The last of these has been treated by the Company with an asperity of reprehension that has no parallel. They lament "that the power of disposing of their property for perpetuity should fall into such hands." Yet for fourteen years, with little interruption, he has governed all their affairs, of every description, with an absolute sway. He has had himself the means of heaping up immense wealth; and during that whole period, the fortunes of hundreds have depended on his smiles and frowns. He himself tells you he is incumbered with two hundred and fifty young gentlemen, some of them of the best families in England, all of whom aim at returning with vast fortunes to Europe in the prime of life. He has, then, two hundred and fifty of your children as his hostages for your good behavior; and loaded for years, as he has been, with the execrations of the natives, with the censures of the Court of Directors, and struck and blasted with resolutions of this House, he still maintains the most despotic power ever known in India. He domineers with an overbearing sway in the assemblies of his pretended masters; and it is thought in a degree rash to venture to name his offences in this House, even as grounds of a legislative remedy.

On the other hand, consider the fate of those who have met with the applauses of the Directors. Colonel Monson, one of the best of men, had his days shortened by the applauses, destitute of the support, of the Company. General Clavering, whose panegyric was made in every dispatch from England, whose hearse was bedewed with the tears and hung round with the eulogies of the Court of Directors, burst an honest and indignant heart at the treachery of those who ruined him by their praises. Uncommon patience and temper supported Mr. Francis a while longer under the baneful influence of the commendation of the Court of Directors. His health, however, gave way at length; and in utter despair, he returned to Europe. At his return, the doors of the India House were shut to this man who had been the object of their constant admiration. He has, indeed, escaped with life; but he has forfeited all expectation of credit, consequence, party, and following. He may well say, "Me nemo ministro fur erit, atque ideo nulli comes exeo." This man, whose deep reach of thought, whose large legislative conceptions, and whose grand plans of policy make the most shining part of our Reports, from whence we have all learned our lessons, if we have learned any good ones,—this man, from whose materials those gentlemen who have least acknowledged it have yet spoken as from a brief,—this man, driven from his employment, discountenanced by the Directors, has had no other reward, and no other distinction, but that inward "sunshine of the soul" which a good conscience can always bestow upon itself. He has not yet had so much as a good word, but from a person too insignificant to make any other return for the means with which he has been furnished for performing his share of a duty which is equally urgent on us all.

Add to this, that, from the highest in place to the lowest, every British subject, who, in obedience to the Company's orders, has been active in the discovery of peculations, has been ruined. They have been driven from India. When they made their appeal at home, they were not heard; when they attempted to return, they were stopped. No artifice of fraud, no violence of power, has been omitted to destroy them in character as well as in fortune.

Worse, far worse, has been the fate of the poor creatures, the natives of India, whom the hypocrisy of the Company has betrayed into complaint of oppression and discovery of peculation. The first women in Bengal, the Ranny of Rajeshahi, the Ranny of Burdwan, the Ranny of Ambooah, by their weak and thoughtless trust in the Company's honor and protection, are utterly ruined: the first of these women, a person of princely rank, and once of correspondent fortune, who paid above two hundred thousand a year quit-rent to the state, is, according to very credible information, so completely beggared as to stand in need of the relief of alms. Mahomed Reza Khân, the second Mussulman in Bengal, for having been distinguished by the ill-omened honor of the countenance and protection of the Court of Directors, was, without the pretence of any inquiry whatsoever into his conduct, stripped of all his employments, and reduced to the lowest condition. His ancient rival for power, the Rajah Nundcomar, was, by an insult on everything which India holds respectable and sacred, hanged in the face of all his nation by the judges you sent to protect that people: hanged for a pretended crime, upon an ex post facto British act of Parliament, in the midst of his evidence against Mr. Hastings. The accuser they saw hanged. The culprit, without acquittal or inquiry, triumphs on the ground of that murder: a murder, not of Nundcomar only, but of all living testimony, and even of evidence yet unborn. From that time not a complaint has been heard from the natives against their governors. All the grievances of India have found a complete remedy.

Men will not look to acts of Parliament, to regulations, to declarations, to votes, and resolutions. No, they are not such fools. They will ask, What is the road to power, credit, wealth, and honors? They will ask, What conduct ends in neglect, disgrace, poverty, exile, prison, and gibbet? These will teach them the course which they are to follow. It is your distribution of these that will give the character and tone to your government. All the rest is miserable grimace.

When I accuse the Court of Directors of this habitual treachery in the use of reward and punishment, I do not mean to include all the individuals in that court. There have been, Sir, very frequently men of the greatest integrity and virtue amongst them; and the contrariety in the declarations and conduct of that court has arisen, I take it, from this,—that the honest Directors have, by the force of matter of fact on the records, carried the reprobation of the evil measures of the servants in India. This could not be prevented, whilst these records stared them in the face; nor were the delinquents, either here or there, very solicitous about their reputation, as long as they were able to secure their power. The agreement of their partisans to censure them blunted for a while the edge of a severe proceeding. It obtained for them a character of impartiality, which enabled them to recommend with some sort of grace, what will always carry a plausible appearance, those treacherous expedients called moderate measures. Whilst these were under discussion, new matter of complaint came over, which seemed to antiquate the first. The same circle was here trod round once more; and thus through years they proceeded in a compromise of censure for punishment, until, by shame and despair, one after another, almost every man who preferred his duty to the Company to the interest of their servants has been driven from that court.

This, Sir, has been their conduct: and it has been the result of the alteration which was insensibly made in their constitution. The change was made insensibly; but it is now strong and adult, and as public and declared as it is fixed beyond all power of reformation: so that there is none who hears me that is not as certain as I am, that the Company, in the sense in which it was formerly understood, has no existence.

The question is not, what injury you may do to the proprietors of India stock; for there are no such men to be injured. If the active, ruling part of the Company, who form the General Court, who fill the offices and direct the measures, (the rest tell for nothing,) were persons who held their stock as a means of their subsistence, who in the part they took were only concerned in the government of India for the rise or fall of their dividend, it would be indeed a defective plan of policy. The interest of the people who are governed by them would not be their primary object,—perhaps a very small part of their consideration at all. But then they might well be depended on, and perhaps more than persons in other respects preferable, for preventing the peculations of their servants to their own prejudice. Such a body would not easily have left their trade as a spoil to the avarice of those who received their wages. But now things are totally reversed. The stock is of no value, whether it be the qualification of a Director or Proprietor; and it is impossible that it should. A Director's qualification may be worth about two thousand five hundred pounds,—and the interest, at eight per cent, is about one hundred and sixty pounds a year. Of what value is that, whether it rise to ten, or fall to six, or to nothing; to him whose son, before he is in Bengal two months, and before he descends the stops of the Council-Chamber, sells the grant of a single contract for forty thousand pounds? Accordingly, the stock is bought up in qualifications. The vote is not to protect the stock, but the stock is bought to acquire the vote; and the end of the vote is to cover and support, against justice, some man of power who has made an obnoxious fortune in India, or to maintain in power those who are actually employing it in the acquisition of such a fortune,—and to avail themselves, in return, of his patronage, that he may shower the spoils of the East, "barbaric pearl and gold," on them, their families, and dependants. So that all the relations of the Company are not only changed, but inverted. The servants in India are not appointed by the Directors, but the Directors are chosen by them. The trade is carried on with their capitals. To them the revenues of the country are mortgaged. The seat of the supreme power is in Calcutta. The house in Leadenhall Street is nothing more than a 'change for their agents, factors, and deputies to meet in, to take care of their affairs and support their interests,—and this so avowedly, that we see the known agents of the delinquent servants marshalling and disciplining their forces, and the prime spokesmen in all their assemblies.

Everything has followed in this order, and according to the natural train of events. I will close what I have to say on the incorrigible condition of the Company, by stating to you a few facts that will leave no doubt of the obstinacy of that corporation, and of their strength too, in resisting the reformation of their servants. By these facts you will be enabled to discover the sole grounds upon which they are tenacious of their charter.

It is now more than two years, that upon account of the gross abuses and ruinous situation of the Company's affairs, (which occasioned the cry of the whole world long before it was taken up here,) that we instituted two committees to inquire into the mismanagements by which the Company's affairs had been brought to the brink of ruin. These inquiries had been pursued with unremitting diligence, and a great body of facts was collected and printed for general information. In the result of those inquiries, although the committees consisted of very different descriptions, they were unanimous. They joined in censuring the conduct of the Indian administration, and enforcing the responsibility upon two men, whom this House, in consequence of these reports, declared it to be the duty of the Directors to remove from their stations, and recall to Great Britain,—"because they had acted in a manner repugnant to the honor and policy of this nation, and thereby brought great calamities on India and enormous expenses on the East India Company."

Here was no attempt on the charter. Here was no question of their privileges. To vindicate their own honor, to support their own interests, to enforce obedience to their own orders,—these were the sole object of the monitory resolution of this House. But as soon as the General Court could assemble, they assembled to demonstrate who they really were. Regardless of the proceedings of this House, they ordered the Directors not to carry into effect any resolution they might come to for the removal of Mr. Hastings and Mr. Hornby. The Directors, still retaining some shadow of respect to this House, instituted an inquiry themselves, which continued from June to October, and, after an attentive perusal and full consideration of papers, resolved to take steps for removing the persons who had been the objects of our resolution, but not without a violent struggle against evidence. Seven Directors went so far as to enter a protest against the vote of their court. Upon this the General Court takes the alarm: it reassembles; it orders the Directors to rescind their resolution, that is, not to recall Mr. Hastings and Mr. Hornby, and to despise the resolution of the House of Commons. Without so much as the pretence of looking into a single paper, without the formality of instituting any committee of inquiry, they superseded all the labors of their own Directors and of this House.

It will naturally occur to ask, how it was possible that they should not attempt some sort of examination into facts, as a color for their resistance to a public authority proceeding so very deliberately, and exerted, apparently at least, in favor of their own. The answer, and the only answer which can be given, is, that they were afraid that their true relation should be mistaken. They were afraid that their patrons and masters in India should attribute their support of them to an opinion of their cause, and not to an attachment to their power. They were afraid it should be suspected that they did not mean blindly to support them in the use they made of that power. They determined to show that they at least were set against reformation: that they were firmly resolved to bring the territories, the trade, and the stock of the Company to ruin, rather than be wanting in fidelity to their nominal servants and real masters, in the ways they took to their private fortunes.

Even since the beginning of this session, the same act of audacity was repeated, with the same circumstances of contempt of all the decorum of inquiry on their part, and of all the proceedings of this House. They again made it a request to their favorite, and your culprit, to keep his post,—and thanked and applauded him, without calling for a paper which could afford light into the merit or demerit of the transaction, and without giving themselves a moment's time to consider, or even to understand, the articles of the Mahratta peace. The fact is, that for a long time there was a struggle, a faint one indeed, between the Company and their servants. But it is a struggle no longer. For some time the superiority has been decided. The interests abroad are become the settled preponderating weight both in the Court of Proprietors and the Court of Directors. Even the attempt you have made to inquire into their practices and to reform abuses has raised and piqued them to a far more regular and steady support. The Company has made a common cause and identified themselves with the destroyers of India. They have taken on themselves all that mass of enormity; they are supporting what you have reprobated; those you condemn they applaud, those you order home to answer for their conduct they request to stay, and thereby encourage to proceed in their practices. Thus the servants of the East India Company triumph, and the representatives of the people of Great Britain are defeated.

I therefore conclude, what you all conclude, that this body, being totally perverted from the purposes of its institution, is utterly incorrigible; and because they are incorrigible, both in conduct and constitution, power ought to be taken out of their hands,—just on the same principles on which have been made all the just changes and revolutions of government that have taken place since the beginning of the world.

I will now say a few words to the general principle of the plan which is set up against that of my right honorable friend. It is to recommit the government of India to the Court of Directors. Those who would commit the reformation of India to the destroyers of it are the enemies to that reformation. They would make a distinction between Directors and Proprietors, which, in the present state of things, does not, cannot exist. But a right honorable gentleman says, he would keep the present government of India in the Court of Directors, and would, to curb them, provide salutary regulations. Wonderful! That is, he would appoint the old offenders to correct the old offences; and he would render the vicious and the foolish wise and virtuous by salutary regulations. He would appoint the wolf as guardian of the sheep; but he has invented a curious muzzle, by which this protecting wolf shall not be able to open his jaws above an inch or two at the utmost. Thus his work is finished. But I tell the right honorable gentleman, that controlled depravity is not innocence, and that it is not the labor of delinquency in chains that will correct abuses. Will these gentlemen of the direction animadvert on the partners of their own guilt? Never did a serious plan of amending of any old tyrannical establishment propose the authors and abettors of the abuses as the reformers of them. If the undone people of India see their old oppressors in confirmed power, even by the reformation, they will expect nothing but what they will certainly feel,—continuance, or rather an aggravation, of all their former sufferings. They look to the seat of power, and to the persons who fill it; and they despise those gentlemen's regulations as much as the gentlemen do who talk of them.

But there is a cure for everything. Take away, say they, the Court of Proprietors, and the Court of Directors will do their duty. Yes,—as they have done it hitherto. That the evils in India have solely arisen from the Court of Proprietors is grossly false. In many of them the Directors were heartily concurring; in most of them they were encouraging, and sometimes commanding; in all they were conniving.

But who are to choose this well-regulated and reforming Court of Directors?—Why, the very Proprietors who are excluded from all management, for the abuse of their power. They will choose, undoubtedly, out of themselves, men like themselves; and those who are most forward in resisting your authority, those who are most engaged in faction or interest with the delinquents abroad, will be the objects of their selection. But gentlemen say, that, when this choice is made, the Proprietors are not to interfere in the measures of the Directors, whilst those Directors are busy in the control of their common patrons and masters in India. No, indeed, I believe they will not desire to interfere. They will choose those whom they know may be trusted, safely trusted, to act in strict conformity to their common principles, manners, measures, interests, and connections. They will want neither monitor nor control. It is not easy to choose men to act in conformity to a public interest against their private; but a sure dependence may be had on those who are chosen to forward their private interest at the expense of the public. But if the Directors should slip, and deviate into rectitude, the punishment is in the hands of the General Court, and it will surely be remembered to them at their next election.

If the government of India wants no reformation, but gentlemen are amusing themselves with a theory, conceiving a more democratic or aristocratic mode of government for these dependencies, or if they are in a dispute only about patronage, the dispute is with me of so little concern that I should not take the pains to utter an affirmative or negative to any proposition in it. If it be only for a theoretical amusement that they are to propose a bill, the thing is at best frivolous and unnecessary. But if the Company's government is not only full of abuse, but is one of the most corrupt and destructive tyrannies that probably ever existed in the world, (as I am sure it is,) what a cruel mockery would it be in me, and in those who think like me, to propose this kind of remedy for this kind of evil!

I now come to the third objection,—that this bill will increase the influence of the crown. An honorable gentleman has demanded of me, whether I was in earnest when I proposed to this House a plan for the reduction of that influence. Indeed, Sir, I was much, very much, in earnest my heart was deeply concerned in it; and I hope the public has not lost the effect of it. How far my judgment was right, for what concerned personal favor and consequence to myself, I shall not presume to determine; nor is its effect upon me, of any moment. But as to this bill, whether it increases the influence of the crown, or not, is a question I should be ashamed to ask. If I am not able to correct a system of oppression and tyranny, that goes to the utter ruin of thirty millions of my fellow-creatures and fellow-subjects, but by some increase to the influence of the crown, I am ready here to declare that I, who have been active to reduce it, shall be at least as active and strenuous to restore it again. I am no lover of names; I contend for the substance of good and protecting government, let it come from what quarter it will.

But I am not obliged to have recourse to this expedient. Much, very much, the contrary. I am sure that the influence of the crown will by no means aid a reformation of this kind, which can neither be originated nor supported but by the uncorrupt public virtue of the representatives of the people of England. Let it once get into the ordinary course of administration, and to me all hopes of reformation are gone. I am far from knowing or believing that this bill will increase the influence of the crown. We all know that the crown has ever had some influence in the Court of Directors, and that it has been extremely increased by the acts of 1773 and 1780. The gentlemen who, as part of their reformation, propose "a more active control on the part of the crown," which is to put the Directors under a Secretary of State specially named for that purpose, must know that their project will increase it further. But that old influence has had, and the new will have, incurable inconveniences, which cannot happen under the Parliamentary establishment proposed in this bill. An honorable gentleman, 
%[58]
\footnote{ Governor Johnstone.}
 not now in his place, but who is well acquainted with the India Company, and by no means a friend to this bill, has told you that a ministerial influence has always been predominant in that body,—and that to make the Directors pliant to their purposes, ministers generally caused persons meanly qualified to be chosen Directors. According to his idea, to secure subserviency, they submitted the Company's affairs to the direction of incapacity. This was to ruin the Company in order to govern it. This was certainly influence in the very worst form in which it could appear. At best it was clandestine and irresponsible. Whether this was done so much upon system as that gentleman supposes, I greatly doubt. But such in effect the operation of government on that court unquestionably was; and such, under a similar constitution, it will be forever. Ministers must be wholly removed from the management of the affairs of India, or they will have an influence in its patronage. The thing is inevitable. Their scheme of a new Secretary of State, "with a more vigorous control," is not much better than a repetition of the measure which we know by experience will not do. Since the year 1773 and the year 1780, the Company has been under the control of the Secretary of State's office, and we had then three Secretaries of State. If more than this is done, then they annihilate the direction which they pretend to support; and they augment the influence of the crown, of whose growth they affect so great an horror. But in truth this scheme of reconciling a direction really and truly deliberative with an office really and substantially controlling is a sort of machinery that can be kept in order but a very short time. Either the Directors will dwindle into clerks, or the Secretary of State, as hitherto has been the course, will leave everything to them, often through design, often through neglect. If both should affect activity, collision, procrastination, delay, and, in the end, utter confusion, must ensue.

But, Sir, there is one kind of influence far greater than that of the nomination to office. This gentlemen in opposition have totally overlooked, although it now exists in its full vigor; and it will do so, upon their scheme, in at least as much force as it does now. That influence this bill cuts up by the roots. I mean the influence of protection. I shall explain myself.—The office given to a young man going to India is of trifling consequence. But he that goes out an insignificant boy in a few years returns a great nabob. Mr. Hastings says he has two hundred and fifty of that kind of raw materials, who expect to be speedily manufactured into the merchantable quality I mention. One of these gentlemen, suppose, returns hither laden with odium and with riches. When he comes to England, he comes as to a prison, or as to a sanctuary; and either is ready for him, according to his demeanor. What is the influence in the grant of any place in India, to that which is acquired by the protection or compromise with such guilt, and with the command of such riches, under the dominion of the hopes and fears which power is able to hold out to every man in that condition? That man's whole fortune, half a million perhaps, becomes an instrument of influence, without a shilling of charge to the civil list: and the influx of fortunes which stand in need of this protection is continual. It works both ways: it influences the delinquent, and it may corrupt the minister. Compare the influence acquired by appointing, for instance, even a Governor-General, and that obtained by protecting him. I shall push this no further. But I wish gentlemen to roll it a little in their own minds.

The bill before you cuts off this source of influence. Its design and main scope is, to regulate the administration of India upon the principles of a court of judicature,—and to exclude, as far as human prudence can exclude, all possibility of a corrupt partiality, in appointing to office, or supporting in office, or covering from inquiry and punishment, any person who has abused or shall abuse his authority. At the board, as appointed and regulated by this bill, reward and punishment cannot be shifted and reversed by a whisper. That commission becomes fatal to cabal, to intrigue, and to secret representation, those instruments of the ruin of India. He that cuts off the means of premature fortune, and the power of protecting it when acquired, strikes a deadly blow at the great fund, the bank, the capital stock of Indian influence, which cannot be vested anywhere, or in any hands, without most dangerous consequences to the public.

The third and contradictory objection is, that this bill does not increase the influence of the crown; on the contrary, that the just power of the crown will be lessened, and transferred to the use of a party, by giving the patronage of India to a commission nominated by Parliament and independent of the crown. The contradiction is glaring, and it has been too well exposed to make it necessary for me to insist upon it. But passing the contradiction, and taking it without any relation, of all objections that is the most extraordinary. Do not gentlemen know that the crown has not at present the grant of a single office under the Company, civil or military, at home or abroad? So far as the crown is concerned, it is certainly rather a gainer; for the vacant offices in the new commission are to be filled up by the king.

It is argued, as a part of the bill derogatory to the prerogatives of the crown, that the commissioners named in the bill are to continue for a short term of years, too short in my opinion,—and because, during that time, they are not at the mercy of every predominant faction of the court. Does not this objection lie against the present Directors,—none of whom are named by the crown, and a proportion of whom hold for this very term of four years? Did it not lie against the Governor-General and Council named in the act of 1773,—who were invested by name, as the present commissioners are to be appointed in the body of the act of Parliament, who were to hold their places for a term of years, and were not removable at the discretion of the crown? Did it not lie against the reappointment, in the year 1780, upon the very same terms? Yet at none of these times, whatever other objections the scheme might be liable to, was it supposed to be a derogation to the just prerogative of the crown, that a commission created by act of Parliament should have its members named by the authority which called it into existence. This is not the disposal by Parliament of any office derived from the authority of the crown, or now disposable by that authority. It is so far from being anything new, violent, or alarming, that I do not recollect, in any Parliamentary commission, down to the commissioners of the land-tax, that it has ever been otherwise.

The objection of the tenure for four years is an objection to all places that are not held during pleasure; but in that objection I pronounce the gentlemen, from my knowledge of their complexion and of their principles, to be perfectly in earnest. The party (say these gentlemen) of the minister who proposes this scheme will be rendered powerful by it; for he will name his party friends to the commission. This objection against party is a party objection; and in this, too, these gentlemen are perfectly serious. They see, that, if, by any intrigue, they should succeed to office, they will lose the clandestine patronage, the true instrument of clandestine influence, enjoyed in the name of subservient Directors, and of wealthy, trembling Indian delinquents. But as often as they are beaten off this ground, they return to it again. The minister will name his friends, and persons of his own party. Whom should he name? Should he name his adversaries? Should he name those whom he cannot trust? Should he name those to execute his plans who are the declared enemies to the principles of his reform? His character is here at stake. If he proposes for his own ends (but he never will propose) such names as, from their want of rank, fortune, character, ability, or knowledge, are likely to betray or to fall short of their trust, he is in an independent House of Commons,—in an House of Commons which has, by its own virtue, destroyed the instruments of Parliamentary subservience. This House of Commons would not endure the sound of such names. He would perish by the means which he is supposed to pursue for the security of his power. The first pledge he must give of his sincerity in this great reform will be in the confidence which ought to be reposed in those names.

For my part, Sir, in this business I put all indirect considerations wholly out of my mind. My sole question, on each clause of the bill, amounts to this:—Is the measure proposed required by the necessities of India? I cannot consent totally to lose sight of the real wants of the people who are the objects of it, and to hunt after every matter of party squabble that may be started on the several provisions. On the question of the duration of the commission I am clear and decided. Can I, can any one who has taken the smallest trouble to be informed concerning the affairs of India, amuse himself with so strange an imagination as that the habitual despotism and oppression, that the monopolies, the peculations, the universal destruction of all the legal authority of this kingdom, which have been for twenty years maturing to their present enormity, combined with the distance of the scene, the boldness and artifice of delinquents, their combination, their excessive wealth, and the faction they have made in England, can be fully corrected in a shorter term than four years? None has hazarded such an assertion; none who has a regard for his reputation will hazard it.

Sir, the gentlemen, whoever they are, who shall be appointed to this commission, have an undertaking of magnitude on their hands, and their stability must not only be, but it must be thought, real; and who is it will believe that anything short of an establishment made, supported, and fixed in its duration, with all the authority of Parliament, can be thought secure of a reasonable stability? The plan of my honorable friend is the reverse of that of reforming by the authors of the abuse. The best we could expect from them is, that they should not continue their ancient, pernicious activity. To those we could think of nothing but applying control; as we are sure that even a regard to their reputation (if any such thing exists in them) would oblige them to cover, to conceal, to suppress, and consequently to prevent all cure of the grievances of India. For what can be discovered which is not to their disgrace? Every attempt to correct an abuse would be a satire on their former administration. Every man they should pretend to call to an account would be found their instrument, or their accomplice. They can never see a beneficial regulation, but with a view to defeat it. The shorter the tenure of such persons, the better would be the chance of some amendment.

But the system of the bill is different. It calls in persons in no wise concerned with any act censured by Parliament,—persons generated with, and for, the reform, of which they are themselves the most essential part. To these the chief regulations in the bill are helps, not fetters: they are authorities to support, not regulations to restrain them. From these we look for much more than innocence. From these we expect zeal, firmness, and unremitted activity. Their duty, their character, binds them to proceedings of vigor; and they ought to have a tenure in their office which precludes all fear, whilst they are acting up to the purposes of their trust,—a tenure without which none will undertake plans that require a series and system of acts. When they know that they cannot be whispered out of their duty, that their public conduct cannot be censured without a public discussion, that the schemes which they have begun will not be committed to those who will have an interest and credit in defeating and disgracing them, then we may entertain hopes. The tenure is for four years, or during their good behavior. That good behavior is as long as they are true to the principles of the bill; and the judgment is in either House of Parliament. This is the tenure of your judges; and the valuable principle of the bill is to make a judicial administration for India. It is to give confidence in the execution of a duty which requires as much perseverance and fortitude as can fall to the lot of any that is born of woman.

As to the gain by party from the right honorable gentleman's bill, let it be shown that this supposed party advantage is pernicious to its object, and the objection is of weight; but until this is done, (and this has not been attempted,) I shall consider the sole objection from its tendency to promote the interest of a party as altogether contemptible. The kingdom is divided into parties, and it ever has been so divided, and it ever will be so divided; and if no system for relieving the subjects of this kingdom from oppression, and snatching its affairs from ruin, can be adopted, until it is demonstrated that no party can derive an advantage from it, no good can ever be done in this country. If party is to derive an advantage from the reform of India, (which is more than I know or believe,) it ought to be that party which alone in this kingdom has its reputation, nay, its very being, pledged to the protection and preservation of that part of the empire. Great fear is expressed that the commissioners named in this bill will show some regard to a minister out of place. To men made like the objectors this must appear criminal. Let it, however, be remembered by others, that, if the commissioners should be his friends, they cannot be his slaves. But dependants are not in a condition to adhere to friends, nor to principles, nor to any uniform line of conduct. They may begin censors, and be obliged to end accomplices. They may be even put under the direction of those whom they were appointed to punish.

The fourth and last objection is, that the bill will hurt public credit. I do not know whether this requires an answer. But if it does, look to your foundations. The sinking fund is the pillar of credit in this country; and let it not be forgot, that the distresses, owing to the mismanagement, of the East India Company, have already taken a million from that fund by the non-payment of duties. The bills drawn upon the Company, which are about four millions, cannot be accepted without the consent of the Treasury. The Treasury, acting under a Parliamentary trust and authority, pledges the public for these millions. If they pledge the public, the public must have a security in its hands for the management of this interest, or the national credit is gone. For otherwise it is not only the East India Company, which is a great interest, that is undone, but, clinging to the security of all your funds, it drags down the rest, and the whole fabric perishes in one ruin. If this bill does not provide a direction of integrity and of ability competent to that trust, the objection is fatal; if it does, public credit must depend on the support of the bill.

It has been said, If you violate this charter, what security has the charter of the Bank, in which public credit is so deeply concerned, and even the charter of London, in which the rights of so many subjects are involved? I answer, In the like case they have no security at all,—no, no security at all. If the Bank should, by every species of mismanagement, fall into a state similar to that of the East India Company,—if it should be oppressed with demands it could not answer, engagements which it could not perform, and with bills for which it could not procure payment,—no charter should protect the mismanagement from correction, and such public grievances from redress. If the city of London had the means and will of destroying an empire, and of cruelly oppressing and tyrannizing over millions of men as good as themselves, the charter of the city of London should prove no sanction to such tyranny and such oppression. Charters are kept, when their purposes are maintained: they are violated, when the privilege is supported against its end and its object.

Now, Sir, I have finished all I proposed to say, as my reasons for giving my vote to this bill. If I am wrong, it is not for want of pains to know what is right. This pledge, at least, of my rectitude I have given to my country.

And now, having done my duty to the bill, let me say a word to the author. I should leave him to his own noble sentiments, if the unworthy and illiberal language with which he has been treated, beyond all example of Parliamentary liberty, did not make a few words necessary,—not so much in justice to him as to my own feelings. I must say, then, that it will be a distinction honorable to the age, that the rescue of the greatest number of the human race that ever were so grievously oppressed from the greatest tyranny that was ever exercised has fallen to the lot of abilities and dispositions equal to the task,—that it has fallen to one who has the enlargement to comprehend, the spirit to undertake, and the eloquence to support so great a measure of hazardous benevolence. His spirit is not owing to his ignorance of the state of men and things: he well knows what snares are spread about his path, from personal animosity, from court intrigues, and possibly from popular delusion. But he has put to hazard his ease, his security, his interest, his power, even his darling popularity, for the benefit of a people whom he has never seen. This is the road that all heroes have trod before him. He is traduced and abused for his supposed motives. He will remember that obloquy is a necessary ingredient in the composition of all true glory: he will remember that it was not only in the Roman customs, but it is in the nature and constitution of things, that calumny and abuse are essential parts of triumph. These thoughts will support a mind which only exists for honor under the burden of temporary reproach. He is doing, indeed, a great good,—such as rarely falls to the lot, and almost as rarely coincides with the desires, of any man. Let him use his time. Let him give the whole length of the reins to his benevolence. He is now on a great eminence, where the eyes of mankind are turned to him. He may live long, he may do much; but here is the summit: he never can exceed what he does this day.

He has faults; but they are faults that, though they may in a small degree tarnish the lustre and sometimes impede the march of his abilities, have nothing in them to extinguish the fire of great virtues. In those faults there is no mixture of deceit, of hypocrisy, of pride, of ferocity, of complexional despotism, or want of feeling for the distresses of mankind. His are faults which might exist in a descendant of Henry the Fourth of France, as they did exist in that father of his country. Henry the Fourth wished that he might live to see a fowl in the pot of every peasant in his kingdom. That sentiment of homely benevolence was worth all the splendid sayings that are recorded of kings. But he wished perhaps for more than could be obtained, and the goodness of the man exceeded the power of the king. But this gentleman, a subject, may this day say this at least with truth,—that he secures the rice in his pot to every man in India. A poet of antiquity thought it one of the first distinctions to a prince whom he meant to celebrate, that through a long succession of generations he had been the progenitor of an able and virtuous citizen who by force of the arts of peace had corrected governments of oppression and suppressed wars of rapine.

\begin{verse}
Indole proh quanta juvenis, quantumque daturus \\
Ausoniæ populis ventura in sæcula civem! \\
Ille super Gangem, super exauditus et Indos, \\
Implebit terras voce, et furialia bella \\
Fulmine compescet linguæ.—
\end{verse}

This was what was said of the predecessor of the only person to whose eloquence it does not wrong that of the mover of this bill to be compared. But the Ganges and the Indus are the patrimony of the fame of my honorable friend, and not of Cicero. I confess I anticipate with joy the reward of those whose whole consequence, power, and authority exist only for the benefit of mankind; and I carry my mind to all the people, and all the names and descriptions, that, relieved by this bill, will bless the labors of this Parliament, and the confidence which the best House of Commons has given to him who the best deserves it. The little cavils of party will not be heard where freedom and happiness will be felt. There is not a tongue, a nation, or religion in India, which will not bless the presiding care and manly beneficence of this House, and of him who proposes to you this great work. Your names will never be separated before the throne of the Divine Goodness, in whatever language, or with whatever rites, pardon is asked for sin, and reward for those who imitate the Godhead in His universal bounty to His creatures. These honors you deserve, and they will surely be paid, when all the jargon of influence and party and patronage are swept into oblivion.

I have spoken what I think, and what I feel, of the mover of this bill. An honorable friend of mine, speaking of his merits, was charged with having made a studied panegyric. I don't know what his was. Mine, I am sure, is a studied panegyric,—the fruit of much meditation, the result of the observation of near twenty years. For my own part, I am happy that I have lived to see this day; I feel myself overpaid for the labors of eighteen years, when, at this late period, I am able to take my share, by one humble vote, in destroying a tyranny that exists to the disgrace of this nation and the destruction of so large a part of the human species.

%FOOTNOTES:
% [52] An allusion made by Mr. Powis.

% [53] Mr. Pitt.

% [54] Mr. Pitt.

% [55] Mr. Dundas, Lord Advocate of Scotland.

% [56] The paltry foundation at Calcutta is scarcely worth naming as an exception.

% [57] Mr. Fox.

% [58] Governor Johnstone.


%%%%%%%%%%%%%%%%%%%%%%%%%%%%%%%%%%%%%%%%%%%%%%%%%%%%%%%%%%%%%%%%%%%%%%%
\chapter*[A Representation to His Majesty]{
A
\\Representation to His Majesty,
\\Moved in
\\The House of Commons
\\By the Right Hon. Edmund Burke, and Seconded by William Windham, Esq.,
\\On Monday, June 14, 1784,
\\And Negatived.
\\With a Preface and Notes}
\addcontentsline{toc}{chapter}{
A REPRESENTATION TO HIS MAJESTY, MOVED IN THE HOUSE OF COMMONS,
June 14,1784}

%%%%%%%%%%%%%%%%%%%%%%%%%%%%%%%%%%%%%%%%%
\begin{center}
  \textbf{\large PREFACE} \par 
\end{center}
\addcontentsline{toc}{section}{PREFACE}

The representation now given to the public relates to some of the most essential privileges of the House of Commons. It would appear of little importance, if it were to be judged by its reception in the place where it was proposed. There it was rejected without debate. The subject matter may, perhaps, hereafter appear to merit a more serious consideration. Thinking men will scarcely regard the penal dissolution of a Parliament as a very trifling concern. Such a dissolution must operate forcibly as an example; and it much imports the people of this kingdom to consider what lesson that example is to teach.

The late House of Commons was not accused of an interested compliance to the will of a court. The charge against them was of a different nature. They were charged with being actuated by an extravagant spirit of independency. This species of offence is so closely connected with merit, this vice bears so near a resemblance to virtue, that the flight of a House of Commons above the exact temperate medium of independence ought to be correctly ascertained, lest we give encouragement to dispositions of a less generous nature, and less safe for the people; we ought to call for very solid and convincing proofs of the existence, and of the magnitude, too, of the evils which are charged to an independent spirit, before we give sanction to any measure, that, by checking a spirit so easily damped, and so hard to be excited, may affect the liberty of a part of our Constitution, which, if not free, is worse than useless.

The Editor does not deny that by possibility such an abuse may exist: but, primâ fronte, there is no reason to presume it. The House of Commons is not, by its complexion, peculiarly subject to the distempers of an independent habit. Very little compulsion is necessary, on the part of the people, to render it abundantly complaisant to ministers and favorites of all descriptions. It required a great length of time, very considerable industry and perseverance, no vulgar policy, the union of many men and many tempers, and the concurrence of events which do not happen every day, to build up an independent House of Commons. Its demolition was accomplished in a moment; and it was the work of ordinary hands. But to construct is a matter of skill; to demolish, force and fury are sufficient.

The late House of Commons has been punished for its independence. That example is made. Have we an example on record of a House of Commons punished for its servility? The rewards of a senate so disposed are manifest to the world. Several gentlemen are very desirous of altering the constitution of the House of Commons; but they must alter the frame and constitution of human nature itself, before they can so fashion it, by any mode of election, that its conduct will not be influenced by reward and punishment, by fame and by disgrace. If these examples take root in the minds of men, what members hereafter will be bold enough not to be corrupt, especially as the king's highway of obsequiousness is so very broad and easy? To make a passive member of Parliament, no dignity of mind, no principles of honor, no resolution, no ability, no industry, no learning, no experience, are in the least degree necessary. To defend a post of importance against a powerful enemy requires an Eliot; a drunken invalid is qualified to hoist a white flag, or to deliver up the keys of the fortress on his knees.

The gentlemen chosen into this Parliament, for the purpose of this surrender, were bred to better things, and are no doubt qualified for other service. But for this strenuous exertion of inactivity, for the vigorous task of submission and passive obedience, all their learning and ability are rather a matter of personal ornament to themselves than of the least use in the performance of their duty.

The present surrender, therefore, of rights and privileges without examination, and the resolution to support any minister given by the secret advisers of the crown, determines not only on all the power and authority of the House, but it settles the character and description of the men who are to compose it, and perpetuates that character as long as it may be thought expedient to keep up a phantom of popular representation.

It is for the chance of some amendment before this new settlement takes a permanent form, and while the matter is yet soft and ductile, that the Editor has republished this piece, and added some notes and explanations to it. His intentions, he hopes, will excuse him to the original mover, and to the world. He acts from a strong sense of the incurable ill effects of holding out the conduct of the late House of Commons as an example to be shunned by future representatives of the people.

%%%%%%%%%%%%%%%%%%%%%%%%%%%%%%%%%%%%%%%%%
\begin{center}
  \textbf{{\large Motion} \\Relative To {\large The Speech from the Throne}} \par 
\end{center}
\addcontentsline{toc}{section}{MOTION RELATED TO THE SPEECH FROM THE THRONE}

%\PRLsep

\begin{center}
LUNÆ, 14° DIE JUNII, 1784.
\end{center}

A motion was made, That a representation be presented to his Majesty, most humbly to offer to his royal consideration, that the address of this House, upon his Majesty's speech from the throne, was dictated solely by our conviction of his Majesty's own most gracious intentions towards his people, which, as we feel with gratitude, so we are ever ready to acknowledge with cheerfulness and satisfaction.

Impressed with these sentiments, we were willing to separate from our general expressions of duty, respect, and veneration to his Majesty's royal person and his princely virtues all discussion whatever with relation to several of the matters suggested and several of the expressions employed in that speech.

That it was not fit or becoming that any decided opinion should be formed by his faithful Commons on that speech, without a degree of deliberation adequate to the importance of the object. Having afforded ourselves due time for that deliberation, we do now most humbly beg leave to represent to his Majesty, that, in the speech from the throne, his ministers have thought proper to use a language of a very alarming import, unauthorized by the practice of good times, and irreconcilable to the principles of this government.

Humbly to express to his Majesty, that it is the privilege and duty of this House to guard the Constitution from all infringement on the part of ministers, and, whenever the occasion requires it, to warn them against any abuse of the authorities committed to them; but it is very lately,
%[59]
\footnote{ See King's Speech, Dec. 5, 1782, and May 19, 1784.}
 that, in a manner not more unseemly than irregular and preposterous, ministers have thought proper, by admonition from the throne, implying distrust and reproach, to convey the expectations of the people to us, their sole representatives, 
%[60]
\footnote{ "I shall never submit to the doctrines I have heard this day from the woolsack, that the other House [House of Commons] are the only representatives and guardians of the people's rights. I boldly maintain the contrary. I say this House [House of Lords] is equally the representatives of the people."—Lord Shelburne's Speech, April 8, 1778. Vide Parliamentary Register, Vol. X. p. 892.}
 and have presumed to caution us, the natural guardians of the Constitution, against any infringement of it on our parts.

This dangerous innovation we, his faithful Commons, think it our duty to mark; and as these admonitions from the throne, by their frequent repetition, seem intended to lead gradually to the establishment of an usage, we hold ourselves bound thus solemnly to protest against them.

This House will be, as it ever ought to be, anxiously attentive to the inclinations and interests of its constituents; nor do we desire to straiten any of the avenues to the throne, or to either House of Parliament. But the ancient order in which the rights of the people have been exercised is not a restriction of these rights. It is a method providently framed in favor of those privileges which it preserves and enforces, by keeping in that course which has been found the most effectual for answering their ends. His Majesty may receive the opinions and wishes of individuals under their signatures, and of bodies corporate under their seals, as expressing their own particular sense; and he may grant such redress as the legal powers of the crown enable the crown to afford. This, and the other House of Parliament, may also receive the wishes of such corporations and individuals by petition. The collective sense of his people his Majesty is to receive from his Commons in Parliament assembled. It would destroy the whole spirit of the Constitution, if his Commons were to receive that sense from the ministers of the crown, or to admit them to be a proper or a regular channel for conveying it.

That the ministers in the said speech declare, "His Majesty has a just and confident reliance that we (his faithful Commons) are animated with the same sentiments of loyalty, and the same attachment to our excellent Constitution which he had the happiness to see so fully manifested in every part of the kingdom."

To represent, that his faithful Commons have never foiled in loyalty to his Majesty. It is new to them to be reminded of it. It is unnecessary and invidious to press it upon them by any example. This recommendation of loyalty, after his Majesty has sat for so many years, with the full support of all descriptions of his subjects, on the throne of this kingdom, at a time of profound peace, and without any pretence of the existence or apprehension of war or conspiracy, becomes in itself a source of no small jealousy to his faithful Commons; as many circumstances lead us to apprehend that therein the ministers have reference to some other measures and principles of loyalty, and to some other ideas of the Constitution, than the laws require, or the practice of Parliament will admit.

No regular communication of the proofs of loyalty and attachment to the Constitution, alluded to in the speech from the throne, have been laid before this House, in order to enable us to judge of the nature, tendency, or occasion of them, or in what particular acts they were displayed; but if we are to suppose the manifestations of loyalty (which are held out to us as an example for imitation) consist in certain addresses delivered to his Majesty, promising support to his Majesty in the exercise of his prerogative, and thanking his Majesty for removing certain of his ministers, on account of the votes they have given upon bills depending in Parliament,—if this be the example of loyalty alluded to in the speech from the throne, then we must beg leave to express our serious concern for the impression which has been made on any of our fellow-subjects by misrepresentations which have seduced them into a seeming approbation of proceedings subversive of their own freedom. We conceive that the opinions delivered in these papers were not well considered; nor were the parties duly informed of the nature of the matters on which they were called to determine, nor of those proceedings of Parliament which they were led to censure.

We shall act more advisedly.—The loyalty we shall manifest will not be the same with theirs; but, we trust, it will be equally sincere, and more enlightened. It is no slight authority which shall persuade us (by receiving as proofs of loyalty the mistaken principles lightly taken up in these addresses) obliquely to criminate, with the heavy and ungrounded charge of disloyalty and disaffection, an uncorrupt, independent, and reforming Parliament. 
%[61]
\footnote{ In that Parliament the House of Commons by two several resolutions put an end to the American war. Immediately on the change of ministry which ensued, in order to secure their own independence, and to prevent the accumulation of new burdens on the people by the growth of a civil list debt, they passed the Establishment Bill. By that bill thirty-six offices tenable by members of Parliament were suppressed, and an order of payment was framed by which the growth of any fresh debt was rendered impracticable. The debt on the civil list from the beginning of the present reign had amounted to one million three hundred thousand pounds and upwards. Another act was passed for regulating the office of the Paymaster-General and the offices subordinate to it. A million of public money had sometimes been in the hands of the paymasters: this act prevented the possibility of any money whatsoever being accumulated in that office in future. The offices of the Exchequer, whose emoluments in time of war were excessive, and grew in exact proportion to the public burdens, were regulated,—some of them suppressed, and the rest reduced to fixed salaries. To secure the freedom of election against the crown, a bill was passed to disqualify all officers concerned in the collection of the revenue in any of its branches from voting in elections: a most important act, not only with regard to its primary object, the freedom of election, but as materially forwarding the due collection of revenue. For the same end, (the preserving the freedom of election,) the House rescinded the famous judgment relative to the Middlesex election, and expunged it from the journals. On the principle of reformation of their own House, connected with a principle of public economy, an act passed for rendering contractors with government incapable of a seat in Parliament. The India Bill (unfortunately lost in the House of Lords) pursued the same idea to its completion, and disabled all servants of the East India Company from a seat in that House for a certain time, and until their conduct was examined into and cleared. The remedy of infinite corruptions and of infinite disorders and oppressions, as well as the security of the most important objects of public economy, perished with that bill and that Parliament. That Parliament also instituted a committee to inquire into the collection of the revenue in all its branches, which prosecuted its duty with great vigor, and suggested several material improvements.}
 Above all, we shall take care that none of the rights and privileges, always claimed, and since the accession of his Majesty's illustrious family constantly exercised by this House, (and which we hold and exercise in trust for the Commons of Great Britain, and for their benefit,) shall be constructively surrendered, or even weakened and impaired, under ambiguous phrases and implications of censure on the late Parliamentary proceedings. If these claims are not well founded, they ought to be honestly abandoned; if they are just, they ought to be steadily and resolutely maintained.

Of his Majesty's own gracious disposition towards the true principles of our free Constitution his faithful Commons never did or could entertain a doubt; but we humbly beg leave to express to his Majesty our uneasiness concerning other new and unusual expressions of his ministers, declaratory of a resolution "to support in their just balance the rights and privileges of every branch of the legislature."

It were desirable that all hazardous theories concerning a balance of rights and privileges (a mode of expression wholly foreign to Parliamentary usage) might have been forborne. His Majesty's faithful Commons are well instructed in their own rights and privileges, which they are determined to maintain on the footing upon which they were handed down from their ancestors; they are not unacquainted with the rights and privileges of the House of Peers; and they know and respect the lawful prerogatives of the crown: but they do not think it safe to admit anything concerning the existence of a balance of those rights, privileges, and prerogatives; nor are they able to discern to what objects ministers would apply their fiction of a balance, nor what they would consider as a just one. These unauthorized doctrines have a tendency to stir improper discussions, and to lead to mischievous innovations in the Constitution. 
%[62]
\footnote{ If these speculations are let loose, the House of Lords may quarrel with their share of the legislature, as being limited with regard to the origination of grants to the crown and the origination of money bills. The advisers of the crown may think proper to bring its negative into ordinary use,—and even to dispute, whether a mere negative, compared with the deliberative power exercised in the other Houses, be such a share in the legislature as to produce a due balance in favor of that branch, and thus justify the previous interference of the crown in the manner lately used. The following will serve to show how much foundation there is for great caution concerning these novel speculations. Lord Shelburne, in his celebrated speech, April 8th, 1778, expresses himself as follows. (Vide Parliamentary Register, Vol. X.)

"The noble and learned lord on the woolsack, in the debate which opened the business of this day, asserted that your Lordships were incompetent to make any alteration in a money bill or a bill of supply, I should be glad to see the matter fairly and fully discussed, and the subject brought forward and argued upon precedent, as well as all its collateral relations. I should be pleased to see the question fairly committed, were it for no other reason but to hear the sleek, smooth contractors from the other House come to this bar and declare, that they, and they only, could frame a money bill, and they, and they only, could dispose of the property of the peers of Great Britain. Perhaps some arguments more plausible than those I heard this day from the woolsack, to show that the Commons have an uncontrollable, unqualified right to bind your Lordships' property, may be urged by them. At present, I beg leave to differ from the noble and learned lord; for, until the claim, after a solemn discussion of this House, is openly and directly relinquished, I shall continue to be of opinion that your Lordships have a right to after, amend, or reject a money bill."

The Duke of Richmond also, in his letter to the volunteers of Ireland, speaks of several of the powers exercised by the House of Commons in the light of usurpations; and his Grace is of opinion, that, when the people are restored to what he conceives to be their rights, in electing the House of Commons, the other branches of the legislature ought to be restored to theirs.—Vide Remembrancer, Vol. XVI.
}

That his faithful Commons most humbly recommend, instead of the inconsiderate speculations of unexperienced men, that, on all occasions, resort should be had to the happy practice of Parliament, and to those solid maxims of government which have prevailed since the accession of his Majesty's illustrious family, as furnishing the only safe principles on which the crown and Parliament can proceed.

We think it the more necessary to be cautious on this head, as, in the last Parliament, the present ministers had thought proper to countenance, if not to suggest, an attack upon the most clear and undoubted rights and privileges of this House.
%[63]
\footnote{ By an act of Parliament, the Directors of the East India Company are restrained from acceptance of bills drawn, from India, beyond a certain amount, without the consent of the Commissioners of the Treasury. The late House of Commons, finding bills to an immense amount drawn upon that body by their servants abroad, and knowing their circumstances to be exceedingly doubtful, came to a resolution providently, cautioning the Lords of the Treasury against the acceptance of these bills, until the House should otherwise direct. The Court Lords then took occasion to declare against the resolution as illegal, by the Commons undertaking to direct in the execution of a trust created by act of Parliament. The House, justly alarmed at this resolution, which went to the destruction of the whole of its superintending capacity, and particularly in matters relative to its own province of money, directed a committee to search the journals, and they found a regular series of precedents, commencing from the remotest of those records, and carried on to that day, by which it appeared that the House interfered, by an authoritative advice and admonition, upon every act of executive government without exception, and in many much stronger cases than that which the Lords thought proper to quarrel with.}


Fearing, from these extraordinary admonitions, and from the new doctrines, which seem to have dictated several unusual expressions, that his Majesty has been abused by false representations of the late proceedings in Parliament, we think it our duty respectfully to inform his Majesty, that no attempt whatever has been made against his lawful prerogatives, or against the rights and privileges of the Peers, by the late House of Commons, in any of their addresses, votes, or resolutions; neither do we know of any proceeding by bill, in which it was proposed to abridge the extent of his royal prerogative: but, if such provision had existed in any bill, we protest, and we declare, against all speeches, acts, or addresses, from any persons whatsoever, which have a tendency to consider such bills, or the persons concerned in them, as just objects of any kind of censure and punishment from the throne. Necessary reformations may hereafter require, as they have frequently done in former times, limitations and abridgments, and in some cases an entire extinction, of some branch of prerogative. If bills should be improper in the form in which they appear in the House where they originate, they are liable, by the wisdom of this Constitution, to be corrected, and even to be totally set aside, elsewhere. This is the known, the legal, and the safe remedy; but whatever, by the manifestation of the royal displeasure, tends to intimidate individual members from proposing, or this House from receiving, debating, and passing bills, tends to prevent even the beginning of every reformation in the state, and utterly destroys the deliberative capacity of Parliament. We therefore claim, demand, and insist upon it, as our undoubted right, that no persons shall be deemed proper objects of animadversion by the crown, in any mode whatever, for the votes which they give or the propositions which they make in Parliament.

We humbly conceive, that besides its share of the legislative power, and its right of impeachment, that, by the law and usage of Parliament, this House has other powers and capacities, which it is bound to maintain. This House is assured that our humble advice on the exercise of prerogative will be heard with the same attention with which it has ever been regarded, and that it will be followed by the same effects which it has ever produced, during the happy and glorious reigns of his Majesty's royal progenitors,—not doubting but that, in all those points, we shall be considered as a council of wisdom and weight to advise, and not merely as an accuser of competence to criminate. 
%[64]
\footnote{ "I observe, at the same time, that there is no charge or complaint suggested against my present ministers."—The King's Answer, 25th February, 1784, to the Address of the House of Common. Vide Resolutions of the House of Commons, printed for Debrett, p. 31.}
 This House claims both capacities; and we trust that we shall be left to our free discretion which of them we shall employ as best calculated for his Majesty's and the national service. Whenever we shall see it expedient to offer our advice concerning his Majesty's servants, who are those of the public, we confidently hope that the personal favor of any minister, or any set of ministers, will not be more dear to his Majesty than the credit and character of a House of Commons. It is an experiment full of peril to put the representative wisdom and justice of his Majesty's people in the wrong; it is a crooked and desperate design, leading to mischief, the extent of which no human wisdom can foresee, to attempt to form a prerogative party in the nation, to be resorted to as occasion shall require, in derogation, from the authority of the Commons of Great Britain in Parliament assembled; it is a contrivance full of danger, for ministers to set up the representative and constituent bodies of the Commons of this kingdom as two separate and distinct powers, formed to counterpoise each other, leaving the preference in the hands of secret advisers of the crown. In such a situation of things, these advisers, taking advantage of the differences which may accidentally arise or may purposely be fomented between them, will have it in their choice to resort to the one or the other, as may best suit the purposes of their sinister ambition. By exciting an emulation and contest between the representative and the constituent bodies, as parties contending for credit and influence at the throne, sacrifices will be made by both; and the whole can end in nothing else than the destruction of the dearest rights and liberties of the nation. If there must be another mode of conveying the collective sense of the people to the throne than that by the House of Commons, it ought to be fixed and defined, and its authority ought to be settled: it ought not to exist in so precarious and dependent a state as that ministers should have it in their power, at their own mere pleasure, to acknowledge it with respect or to reject it with scorn.

It is the undoubted prerogative of the crown to dissolve Parliament; but we beg leave to lay before his Majesty, that it is, of all the trusts vested in his Majesty, the most critical and delicate, and that in which this House has the most reason to require, not only the good faith, but the favor of the crown. His Commons are not always upon a par with his ministers in an application to popular judgment; it is not in the power of the members of this House to go to their election at the moment the most favorable for them. It is in the power of the crown to choose a time for their dissolution whilst great and arduous matters of state and legislation are depending, which may be easily misunderstood, and which cannot be fully explained before that misunderstanding may prove fatal to the honor that belongs and to the consideration that is due to members of Parliament.

With his Majesty is the gift of all the rewards, the honors, distinctions, favors, and graces of the state; with his Majesty is the mitigation of all the rigors of the law: and we rejoice to see the crown possessed of trusts calculated to obtain good-will, and charged with duties which are popular and pleasing. Our trusts are of a different kind. Our duties are harsh and invidious in their nature; and justice and safety is all we can expect in the exercise of them. We are to offer salutary, which is not always pleasing counsel: we are to inquire and to accuse; and the objects of our inquiry and charge will be for the most part persons of wealth, power, and extensive connections: we are to make rigid laws for the preservation of revenue, which of necessity more or less confine some action or restrain some function which before was free: what is the most critical and invidious of all, the whole body of the public impositions originate from us, and the hand of the House of Commons is seen and felt in every burden that presses on the people. Whilst ultimately we are serving them, and in the first instance whilst we are serving his Majesty, it will be hard indeed, if we should see a House of Commons the victim of its zeal and fidelity, sacrificed by his ministers to those very popular discontents which shall be excited by our dutiful endeavors for the security and greatness of his throne. No other consequence can result from such an example, but that, in future, the House of Commons, consulting its safety at the expense of its duties, and suffering the whole energy of the state to be relaxed, will shrink from every service which, however necessary, is of a great and arduous nature,—or that, willing to provide for the public necessities, and at the same time to secure the means of performing that task, they will exchange independence for protection, and will court a subservient existence through the favor of those ministers of state or those secret advisers who ought themselves to stand in awe of the Commons of this realm.

A House of Commons respected by his ministers is essential to his Majesty's service: it is fit that they should yield to Parliament, and not that Parliament should be new-modelled until it is fitted to their purposes. If our authority is only to be held up when we coincide in opinion with his Majesty's advisers, but is to be set at nought the moment it differs from them, the House of Commons will sink into a mere appendage of administration, and will lose that independent character which, inseparably connecting the honor and reputation with the acts of this House, enables us to afford a real, effective, and substantial support to his government. It is the deference shown to our opinion, when we dissent from the servants of the crown, which alone can give authority to the proceedings of this House, when it concurs with their measures.

That authority once lost, the credit of his Majesty's crown will be impaired in the eyes of all nations. Foreign powers, who may yet wish to revive a friendly intercourse with this nation, will look in vain for that hold which gave a connection with Great Britain the preference to an affiance with any other state. A House of Commons of which ministers were known to stand in awe, where everything was necessarily discussed on principles fit to be openly and publicly avowed, and which could not be retracted or varied without danger, furnished a ground of confidence in the public faith which the engagement of no state dependent on the fluctuation of personal favor and private advice can ever pretend to. If faith with the House of Commons, the grand security for the national faith itself, can be broken with impunity, a wound is given to the political importance of Great Britain which will not easily be healed.

That there was a great variance between the late House of Commons and certain persons, whom his Majesty has been advised to make and continue as ministers, in defiance of the advice of that House, is notorious to the world. That House did not confide in those ministers; and they withheld their confidence from them for reasons for which posterity will honor and respect the names of those who composed that House of Commons, distinguished for its independence. They could not confide in persons who have shown a disposition to dark and dangerous intrigues. By these intrigues they have weakened, if not destroyed, the clear assurance which his Majesty's people, and which all nations, ought to have of what are and what are not the real acts of his government.

If it should be seen that his ministers may continue in their offices without any signification to them of his Majesty's displeasure at any of their measures, whilst persons considerable for their rank, and known to have had access to his Majesty's sacred person, can with impunity abuse that advantage, and employ his Majesty's name to disavow and counteract the proceedings of his official servants, nothing but distrust, discord, debility, contempt of all authority, and general confusion, can prevail in his government.

This we lay before his Majesty, with humility and concern, as the inevitable effect of a spirit of intrigue in his executive government: an evil which we have but too much reason to be persuaded exists and increases. During the course of the last session it broke out in a manner the most alarming. This evil was infinitely aggravated by the unauthorized, but not disavowed, use which has been made of his Majesty's name, for the purpose of the most unconstitutional, corrupt, and dishonorable influence on the minds of the members of Parliament that ever was practised in this kingdom. No attention even to exterior decorum, in the practice of corruption and intimidation employed on peers, was observed: several peers were obliged under menaces to retract their declarations and to recall their proxies.

The Commons have the deepest interest in the purity and integrity of the Peerage. The Peers dispose of all the property in the kingdom, in the last resort; and they dispose of it on their honor, and not on their oaths, as all the members of every other tribunal in the kingdom must do,—though in them the proceeding is not conclusive. We have, therefore, a right to demand that no application shall be made to peers of such a nature as may give room to call in question, much less to attaint, our sole security for all that we possess. This corrupt proceeding appeared to the House of Commons, who are the natural guardians of the purity of Parliament, and of the purity of every branch of judicature, a most reprehensible and dangerous practice, tending to shake the very foundation of the authority of the House of Peers; and they branded it as such by their resolution.

The House had not sufficient evidence to enable them legally to punish this practice, but they had enough to caution them against all confidence in the authors and abettors of it. They performed their duty in humbly advising his Majesty against the employment of such ministers; but his Majesty was advised to keep those ministers, and to dissolve that Parliament. The House, aware of the importance and urgency of its duty with regard to the British interests in India, which were and are in the utmost disorder, and in the utmost peril, most humbly requested his Majesty not to dissolve the Parliament during the course of their very critical proceedings on that subject. His Majesty's gracious condescension to that request was conveyed in the royal faith, pledged to a House of Parliament, and solemnly delivered from the throne. It was but a very few days after a committee had been, with the consent and concurrence of the Chancellor of the Exchequer, appointed for an inquiry into certain accounts delivered to the House by the Court of Directors, and then actually engaged in that inquiry, that the ministers, regardless of the assurance given from the crown to a House of Commons, did dissolve that Parliament. We most humbly submit to his Majesty's consideration the consequences of this their breach of public faith.

Whilst the members of the House of Commons, under that security, were engaged in his Majesty's and the national business, endeavors were industriously used to calumniate those whom it was found impracticable to corrupt. The reputation of the members, and the reputation of the House itself, was undermined in every part of the kingdom.

In the speech from the throne relative to India, we are cautioned by the ministers "not to lose sight of the effect any measure may have on the Constitution of our country." We are apprehensive that a calumnious report, spread abroad, of an attack upon his Majesty's prerogative by the late House of Commons, may have made an impression on his royal mind, and have given occasion to this unusual admonition to the present. This attack is charged to have been made in the late Parliament by a bill which passed the House of Commons, in the late session of that Parliament, for the regulation of the affairs, for the preservation of the commerce, and for the amendment of the government of this nation, in the East Indies.

That his Majesty and his people may have an opportunity of entering into the ground of this injurious charge, we beg leave humbly to acquaint his Majesty, that, far from having made any infringement whatsoever on any part of his royal prerogative, that bill did, for a limited time, give to his Majesty certain powers never before possessed by the crown; and for this his present ministers (who, rather than fall short in the number of their calumnies, employ some that are contradictory) have slandered this House, as aiming at the extension of an unconstitutional influence in his Majesty's crown. This pretended attempt to increase the influence of the crown they were weak enough to endeavor to persuade his Majesty's people was amongst the causes which excited his Majesty's resentment against his late ministers.

Further, to remove the impressions of this calumny concerning an attempt in the House of Commons against his prerogative, it is proper to inform his Majesty, that the territorial possessions in the East Indies never have been declared by any public judgment, act, or instrument, or any resolution of Parliament whatsoever, to be the subject matter of his Majesty's prerogative; nor have they ever been understood as belonging to his ordinary administration, or to be annexed or united to his crown; but that they are acquisitions of a new and peculiar description, 
%[65]
\footnote{ The territorial possessions in the East Indies were acquired to the Company, in virtue of grants from the Great Mogul, in the nature of offices and jurisdictions, to be held under him, and dependent upon his crown, with the express condition of being obedient to orders from his court, and of paying an annual tribute to his treasury. It is true that no obedience is yielded to these orders, and for some time past there has been no payment made of this tribute. But it is under a grant so conditioned that they still hold. To subject the King of Great Britain as tributary to a foreign power by the acts of his subjects; to suppose the grant valid, and yet the condition void; to suppose it good for the king, and insufficient for the Company; to suppose it an interest divisible between the parties: these are some few of the many legal difficulties to be surmounted, before the Common Law of England can acknowledge the East India Company's Asiatic affairs to be a subject matter of prerogative, so as to bring it within the verge of English jurisprudence. It is a very anomalous species of power and property which is held by the East India Company. Our English prerogative law does not furnish principles, much less precedents, by which it can be defined or adjusted. Nothing but the eminent dominion of Parliament over every British subject, in every concern, and in every circumstance in which he is placed, can adjust this new, intricate matter. Parliament may act wisely or unwisely, justly or unjustly; but Parliament alone is competent to it.}
 unknown to the ancient executive constitution of this country.

From time to time, therefore, Parliament provided for their government according to its discretion, and to its opinion of what was required by the public necessities. We do not know that his Majesty was entitled, by prerogative, to exercise any act of authority whatsoever in the Company's affairs, or that, in effect, such authority has ever been exercised. His Majesty's patronage was not taken away by that bill; because it is notorious that his Majesty never originally had the appointment of a single officer, civil or military, in the Company's establishment in India: nor has the least degree of patronage ever been acquired to the crown in any other manner or measure than as the power was thought expedient to be granted by act of Parliament,—that is, by the very same authority by which the offices were disposed of and regulated in the bill which his Majesty's servants have falsely and injuriously represented as infringing upon the prerogative of the crown.

Before the year 1773 the whole administration of India, and the whole patronage to office there, was in the hands of the East India Company. The East India Company is not a branch of his Majesty's prerogative administration, nor does that body exercise any species of authority under it, nor indeed from any British title that does not derive all its legal validity from acts of Parliament.

When a claim was asserted to the India territorial possessions in the occupation of the Company, these possessions were not claimed as parcel of his Majesty's patrimonial estate, or as a fruit of the ancient inheritance of his crown: they were claimed for the public. And when agreements were made with the East India Company concerning any composition for the holding, or any participation of the profits, of those territories, the agreement was made with the public; and the preambles of the several acts have uniformly so stated it. These agreements were not made (even nominally) with his Majesty, but with Parliament: and the bills making and establishing such agreements always originated in this House; which appropriated the money to await the disposition of Parliament, without the ceremony of previous consent from the crown even so much as suggested by any of his ministers: which previous consent is an observance of decorum, not indeed of strict right, but generally paid, when a new appropriation takes place in any part of his Majesty's prerogative revenues.

In pursuance of a right thus uniformly recognized and uniformly acted on, when Parliament undertook the reformation of the East India Company in 1773, a commission was appointed, as the commission in the late bill was appointed; and it was made to continue for a term of years, as the commission in the late bill was to continue; all the commissioners were named in Parliament, as in the late bill they were named. As they received, so they held their offices, wholly independent of the crown; they held them for a fixed term; they were not removable by an address of either House or even of both Houses of Parliament, a precaution observed in the late bill relative to the commissioners proposed therein; nor were they bound by the strict rules of proceeding which regulated and restrained the late commissioners against all possible abuse of a power which could not fail of being diligently and zealously watched by the ministers of the crown, and the proprietors of the stock, as well as by Parliament. Their proceedings were, in that bill, directed to be of such a nature as easily to subject them to the strictest revision of both, in case of any malversation.

In the year 1780, an act of Parliament again made provision for the government of those territories for another four years, without any sort of reference to prerogative; nor was the least objection taken at the second, more than at the first of those periods, as if an infringement had been made upon the rights of the crown: yet his Majesty's ministers have thought fit to represent the late commission as an entire innovation on the Constitution, and the setting up a new order and estate in the nation, tending to the subversion of the monarchy itself.

If the government of the East Indies, other than by his Majesty's prerogative, be in effect a fourth order in the commonwealth, this order has long existed; because the East India Company has for many years enjoyed it in the fullest extent, and does at this day enjoy the whole administration of those provinces, and the patronage to offices throughout that great empire, except as it is controlled by act of Parliament.

It was the ill condition and ill administration of the Company's affairs which induced this House (merely as a temporary establishment) to vest the same powers which the Company did before possess, (and no other,) for a limited time, and under very strict directions, in proper hands, until they could be restored, or farther provision made concerning them. It was therefore no creation whatever of a new power, but the removal of an old power, long since created, and then existing, from the management of those persons who had manifestly and dangerously abused their trust. This House, which well knows the Parliamentary origin of all the Company's powers and privileges, and is not ignorant or negligent of the authority which may vest those powers and privileges in others, if justice and the public safety so require, is conscious to itself that it no more creates a new order in the state, by making occasional trustees for the direction of the Company, than it originally did in giving a much more permanent trust to the Directors or to the General Court of that body. The monopoly of the East India Company was a derogation from the general freedom of trade belonging to his Majesty's people. The powers of government, and of peace and war, are parts of prerogative of the highest order. Of our competence to restrain the rights of all his subjects by act of Parliament, and to vest those high and eminent prerogatives even in a particular company of merchants, there has been no question. We beg leave most humbly to claim as our right, and as a right which this House has always used, to frame such bills for the regulation of that commerce, and of the territories held by the East India Company, and everything relating to them, as to our discretion shall seem fit; and we assert and maintain that therein we follow, and do not innovate on, the Constitution.

That his Majesty's ministers, misled by their ambition, have endeavored, if possible, to form a faction in the country against the popular part of the Constitution; and have therefore thought proper to add to their slanderous accusation against a House of Parliament, relative to his Majesty's prerogative, another of a different nature, calculated for the purpose of raising fears and jealousies among the corporate bodies of the kingdom, and of persuading uninformed persons belonging to those corporations to look to and to make addresses to them, as protectors of their rights, under their several charters, from the designs which they, without any ground, charged the then House of Commons to have formed against charters in general. For this purpose they have not scrupled to assert that the exertion of his Majesty's prerogative in the late precipitate change in his administration, and the dissolution of the late Parliament, were measures adopted in order to rescue the people and their rights out of the hands of the House of Commons, their representatives.

We trust that his Majesty's subjects are not yet so far deluded as to believe that the charters, or that any other of their local or general privileges, can have a solid security in any place but where that security has always been looked for, and always found,—in the House of Commons. Miserable and precarious indeed would be the state of their franchises, if they were to find no defence but from that quarter from whence they have always been attacked! 
%[66]
\footnote{ The attempt upon charters and the privileges of the corporate bodies of the kingdom in the reigns of Charles the Second and James the Second was made by the crown. It was carried on by the ordinary course of law, in courts instituted for the security of the property and franchises of the people. This attempt made by the crown was attended with complete success. The corporate rights of the city of London, and of all the companies it contains, were by solemn judgment of law declared forfeited, and all their franchises, privileges, properties, and estates were of course seized into the hands of the crown. The injury was from the crown: the redress was by Parliament. A bill was brought into the House of Commons, by which the judgment against the city of London, and against the companies, was reversed: and this bill passed the House of Lords without any complaint of trespass on their jurisdiction, although the bill was for a reversal of a judgment in law. By this act, which is in the second of William and Mary, chap. 8, the question of forfeiture of that charter is forever taken out of the power of any court of law: no cognizance can be taken of it except in Parliament.

Although the act above mentioned has declared the judgment against the corporation of London to be illegal yet Blackstone makes no scruple of asserting, that, "perhaps, in strictness of law, the proceedings in most of them [the Quo Warranto causes] were sufficiently regular," leaving it in doubt, whether this regularity did not apply to the corporation of London, as well as to any of the rest; and he seems to blame the proceeding (as most blamable it was) not so much on account of illegality as for the crown's having employed a legal proceeding for political purposes. He calls it "an exertion of an act of law for the purposes of the state."

The same security which was given to the city of London, would have been extended to all the corporations, if the House of Commons could have prevailed. But the bill for that purpose passed but by a majority of one in the Lords; and it was entirely lost by a prorogation, which is the act of the crown. Small, indeed, was the security which the corporation of London enjoyed before the act of William and Mary, and which all the other corporations, secured by no statute, enjoy at this hour, if strict law was employed against them. The use of strict law has always been rendered very delicate by the same means by which the almost unmeasured legal powers residing (and in many instances dangerously residing) in the crown are kept within due bounds: I mean, that strong superintending power in the House of Commons which inconsiderate people have been prevailed on to condemn as trenching on prerogative. Strict law is by no means such a friend to the rights of the subject as they have been taught to believe. They who have been most conversant in this kind of learning will be most sensible of the danger of submitting corporate rights of high political importance to these subordinate tribunals. The general heads of law on that subject are vulgar and trivial. On them there is not much question. But it is far from easy to determine what special acts, or what special neglect of action, shall subject corporations to a forfeiture. There is so much laxity in this doctrine, that great room is left for favor or prejudice, which might give to the crown an entire dominion over those corporations. On the other hand, it is undoubtedly true that every subordinate corporate right ought to be subject to control, to superior direction, and even to forfeiture upon just cause. In this reason and law agree. In every judgment given on a corporate right of great political importance, the policy and prudence make no small part of the question. To these considerations a court of law is not competent; and, indeed, an attempt at the least intermixture of such ideas with the matter of law could have no other effect than wholly to corrupt the judicial character of the court in which such a cause should come to be tried. It is besides to be remarked, that, if, in virtue of a legal process, a forfeiture should be adjudged, the court of law has no power to modify or mitigate. The whole franchise is annihilated, and the corporate property goes into the hands of the crown. They who hold the new doctrines concerning the power of the House of Commons ought well to consider in such a case by what means the corporate rights could be revived, or the property could be recovered out of the hands of the crown. But Parliament can do what the courts neither can do nor ought to attempt. Parliament is competent to give due weight to all political considerations. It may modify, it may mitigate, and it may render perfectly secure, all that it does not think fit to take away. It is not likely that Parliament will ever draw to itself the cognizance of questions concerning ordinary corporations, farther than to protect them, in case attempts are made to induce a forfeiture of their franchises.

The case of the East India Company is different even from that of the greatest of these corporations. No monopoly of trade, beyond their own limits, is vested in the corporate body of any town or city in the kingdom. Even within these limits the monopoly is not general. The Company has the monopoly of the trade of half the world. The first corporation of the kingdom has for the object of its jurisdiction only a few matters of subordinate police. The East India Company governs an empire, through all its concerns and all its departments, from the lowest office of economy to the highest councils of state,—an empire to which Great Britain is in comparison but a respectable province. To leave these concerns without superior cognizance would be madness; to leave them to be judged in the courts below, on the principles of a confined jurisprudence, would be folly. It is well, if the whole legislative power is competent to the correction of abuses which are commensurate to the immensity of the object they affect. The idea of an absolute power has, indeed, its terrors; but that objection lies to every Parliamentary proceeding; and as no other can regulate the abuses of such a charter, it is fittest that sovereign authority should be exercised, where it is most likely to be attended with the most effectual correctives. These correctives are furnished by the nature and course of Parliamentary proceedings, and by the infinitely diversified characters who compose the two Houses. In effect and virtually, they form a vast number, variety, and succession of judges and jurors. The fulness, the freedom, and publicity of discussion leaves it easy to distinguish what are acts of power, and what the determinations of equity and reason. There prejudice corrects prejudice, and the different asperities of party zeal mitigate and neutralize each other. So far from violence being the general characteristic of the proceedings of Parliament, whatever the beginnings of any Parliamentary process may be, its general fault in the end is, that it is found incomplete and ineffectual.
}
 But the late House of Commons, in passing that bill, made no attack upon any powers or privileges, except such as a House of Commons has frequently attacked, and will attack, (and they trust, in the end, with their wonted success,)—that is, upon those which are corruptly and oppressively administered; and this House do faithfully assure his Majesty, that we will correct, and, if necessary for the purpose, as far as in us lies, will wholly destroy, every species of power and authority exercised by British subjects to the oppression, wrong, and detriment of the people, and to the impoverishment and desolation of the countries subject to it.

The propagators of the calumnies against that House of Parliament have been indefatigable in exaggerating the supposed injury done to the East India Company by the suspension of the authorities which they have in every instance abused,—as if power had been wrested by wrong and violence from just and prudent hands; but they have, with equal care, concealed the weighty grounds and reasons on which that House had adopted the most moderate of all possible expedients for rescuing the natives of India from oppression, and for saving the interests of the real and honest proprietors of their stock, as well as that great national, commercial concern, from imminent ruin.

The ministers aforesaid have also caused it to be reported that the House of Commons have confiscated the property of the East India Company. It is the reverse of truth. The whole management was a trust for the proprietors, under their own inspection, (and it was so provided for in the bill,) and under the inspection of Parliament. That bill, so far from confiscating the Company's property, was the only one which, for several years past, did not, in some shape or other, affect their property, or restrain them in the disposition of it.

It is proper that his Majesty and all his people should be informed that the House of Commons have proceeded, with regard to the East India Company, with a degree of care, circumspection, and deliberation, which has not been equalled in the history of Parliamentary proceedings. For sixteen years the state and condition of that body has never been wholly out of their view. In the year 1767 the House took those objects into consideration, in a committee of the whole House. The business was pursued in the following year. In the year 1772 two committees were appointed for the same purpose, which examined into their affairs with much diligence, and made very ample reports. In the year 1773 the proceedings were carried to an act of Parliament, which proved ineffectual to its purpose. The oppressions and abuses in India have since rather increased than diminished, on account of the greatness of the temptations, and convenience of the opportunities, which got the better of the legislative provisions calculated against ill practices then in their beginnings; insomuch that, in 1781, two committees were again instituted, who have made seventeen reports. It was upon the most minute, exact, and laborious collection and discussion of facts, that the late House of Commons proceeded in the reform which they attempted in the administration of India, but which has been frustrated by ways and means the most dishonorable to his Majesty's government, and the most pernicious to the Constitution of this kingdom. His Majesty was so sensible of the disorders in the Company's administration, that the consideration of that subject was no less than six times recommended to this House in speeches from the throne.

The result of the Parliamentary inquiries has been, that the East India Company was found totally corrupted, and totally perverted from the purposes of its institution, whether political or commercial; that the powers of war and peace given by the charter had been abused, by kindling hostilities in every quarter for the purposes of rapine; that almost all the treaties of peace they have made have only given cause to so many breaches of public faith; that countries once the most flourishing are reduced to a state of indigence, decay, and depopulation, to the diminution of our strength, and to the infinite dishonor of our national character; that the laws of this kingdom are notoriously, and almost in every instance, despised; that the servants of the Company, by the purchase of qualifications to vote in the General Court, and, at length, by getting the Company itself deeply in their debt, have obtained the entire and absolute mastery in the body by which they ought to have been ruled and coerced. Thus their malversations in office are supported, instead of being checked by the Company. The whole of the affairs of that body are reduced to a most perilous situation; and many millions of innocent and deserving men, who are under the protection of this nation, and who ought to be protected by it, are oppressed by a most despotic and rapacious tyranny. The Company and their servants, having strengthened themselves by this confederacy, set at defiance the authority and admonitions of this House employed to reform them; and when this House had selected certain principal delinquents, whom they declared it the duty of the Company to recall, the Company held out its legal privileges against all reformation, positively refused to recall them, and supported those who had fallen under the just censure of this House with new and stronger marks of countenance and approbation.

The late House, discovering the reversed situation of the Company, by which the nominal servants are really the masters, and the offenders are become their own judges, thought fit to examine into the state of their commerce; and they have also discovered that their commercial affairs are in the greatest disorder; that their debts have accumulated beyond any present or obvious future means of payment, at least under the actual administration of their affairs; that this condition of the East India Company has begun to affect the sinking fund itself, on which the public credit of the kingdom rests,—a million and upwards being due to the customs, which that House of Commons whose intentions towards the Company have been so grossly misrepresented were indulgent enough to respite. And thus, instead of confiscating their property, the Company received without interest (which in such a case had been before charged) the use of a very large sum of the public money. The revenues are under the peculiar care of this House, not only as the revenues originate from us, but as, on every failure if the funds set apart for the support of the national credit, or to provide for the national strength and safety, the task of supplying every deficiency falls upon his Majesty's faithful Commons, this House must, in effect, tax the people. The House, therefore, at every moment, incurs the hazard of becoming obnoxious to its constituents.

The enemies of the late House of Commons resolved, if possible, to bring on that event. They therefore endeavored to misrepresent the provident means adopted by the House of Commons for keeping off this invidious necessity, as an attack on the rights of the East India Company: for they well knew, that, on the one hand, if, for want of proper regulation and relief, the Company should become insolvent, or even stop payment, the national credit and commerce would sustain a heavy blow; and that calamity would be justly imputed to Parliament, which, after such long inquiries, and such frequent admonitions from his Majesty, had neglected so essential and so urgent an article of their duty: on the other hand, they knew, that, wholly corrupted as the Company is, nothing effectual could be done to preserve that interest from ruin, without taking for a time the national objects of their trust out of their hands; and then a cry would be industriously raised against the House of Commons, as depriving British subjects of their legal privileges. The restraint, being plain and simple, must be easily understood by those who would be brought with great difficulty to comprehend the intricate detail of matters of fact which rendered this suspension of the administration of India absolutely necessary on motives of justice, of policy, of public honor, and public safety.

The House of Commons had not been able to devise a method by which the redress of grievances could be effected through the authors of those grievances; nor could they imagine how corruptions could be purified by the corrupters and the corrupted; nor do we now conceive how any reformation can proceed from the known abettors and supporters of the persons who have been guilty of the misdemeanors which Parliament has reprobated, and who for their own ill purposes have given countenance to a false and delusive state of the Company's affairs, fabricated to mislead Parliament and to impose upon the nation.
%[67]
\footnote{ The purpose of the misrepresentation being now completely answered, there is no doubt but the committee in this Parliament, appointed by the ministers themselves, will justify the grounds upon which the last Parliament proceeded, and will lay open to the world the dreadful state of the Company's affairs, and the grossness of their own calumnies upon this head. By delay the new assembly is come into the disgraceful situation of allowing a dividend of eight per cent by act of Parliament, without the least matter before them to justify the granting of any dividend at all.}

Your Commons feel, with a just resentment, the inadequate estimate which your ministers have formed of the importance of this great concern. They call on us to act upon the principles of those who have not inquired into the subject, and to condemn those who with the most laudable diligence have examined and scrutinized every part of it. The deliberations of Parliament have been broken; the season of the year is unfavorable; many of us are new members, who must be wholly unacquainted with the subject, which lies remote from the ordinary course of general information.

We are cautioned against an infringement of the Constitution; and it is impossible to know what the secret advisers of the crown, who have driven out the late ministers for their conduct in Parliament, and have dissolved the late Parliament for a pretended attack upon prerogative, will consider as such an infringement. We are not furnished with a rule, the observance of which can make us safe from the resentment of the crown, even by an implicit obedience to the dictates of the ministers who have advised that speech; we know not how soon those ministers may be disavowed, and how soon the members of this House, for our very agreement with them, may be considered as objects of his Majesty's displeasure. Until by his Majesty's goodness and wisdom the late example is completely done away, we are not free.

We are well aware, in providing for the affairs of the East, with what an adult strength of abuse, and of wealth and influence growing out of that abuse, his Majesty's Commons had, in the last Parliament, and still have, to struggle. We are sensible that the influence of that wealth, in a much larger degree and measure than at any former period, may have penetrated into the very quarter from whence alone any real reformation can be expected.
%[68]
\footnote{ This will be evident to those who consider the number and description of Directors and servants of the East India Company chosen into the present Parliament. The light in which the present ministers hold the labors of the House of Commons in searching into the disorders in the Indian administration, and all its endeavors for the reformation of the government there, without any distinction of times, or of the persons concerned, will appear from the following extract from a speech of the present Lord Chancellor. After making a high-flown panegyric on those whom the House of Commons had condemned by their resolutions, he said:—"Let us not be misled by reports from committees of another House, to which, I again repeat, I pay as much attention as I would do to the history of Robinson Crusoe, Let the conduct of the East India Company be fairly and fully inquired into. Let it be acquitted or condemned by evidence brought to the bar of the House. Without entering very deeply into the subject, let me reply in a few words to an observation which fell from a noble and learned lord, that the Company's finances are distressed, and that they owe at this moment a million sterling to the nation. When such a charge is brought, will Parliament in its justice forget that the Company is restricted from employing that credit which its great and flourishing situation gives to it?"}

If, therefore, in the arduous affairs recommended to us, our proceedings should be ill adapted, feeble, and ineffectual,—if no delinquency should be prevented, and no delinquent should be called to account,—if every person should be caressed, promoted, and raised in power, in proportion to the enormity of his offences,—if no relief should be given to any of the natives unjustly dispossessed of their rights, jurisdictions, and properties,—if no cruel and unjust exactions should be forborne,—if the source of no peculation or oppressive gain should be cut off,—if, by the omission of the opportunities that were in our hands, our Indian empire should fall into ruin irretrievable, and in its fall crush the credit and overwhelm the revenues of this country,—we stand acquitted to our honor and to our conscience, who have reluctantly seen the weightiest interests of our country, at times the most critical to its dignity and safety, rendered the sport of the inconsiderate and unmeasured ambition of individuals, and by that means the wisdom of his Majesty's government degraded in the public estimation, and the policy and character of this renowned nation rendered contemptible in the eyes of all Europe.

\vspace{0.2cm}
It passed in the negative.

%FOOTNOTES:
% [59] See King's Speech, Dec. 5, 1782, and May 19, 1784.

% [60] "I shall never submit to the doctrines I have heard this day from the woolsack, that the other House [House of Commons] are the only representatives and guardians of the people's rights. I boldly maintain the contrary. I say this House [House of Lords] is equally the representatives of the people."—Lord Shelburne's Speech, April 8, 1778. Vide Parliamentary Register, Vol. X. p. 892.

% [61] In that Parliament the House of Commons by two several resolutions put an end to the American war. Immediately on the change of ministry which ensued, in order to secure their own independence, and to prevent the accumulation of new burdens on the people by the growth of a civil list debt, they passed the Establishment Bill. By that bill thirty-six offices tenable by members of Parliament were suppressed, and an order of payment was framed by which the growth of any fresh debt was rendered impracticable. The debt on the civil list from the beginning of the present reign had amounted to one million three hundred thousand pounds and upwards. Another act was passed for regulating the office of the Paymaster-General and the offices subordinate to it. A million of public money had sometimes been in the hands of the paymasters: this act prevented the possibility of any money whatsoever being accumulated in that office in future. The offices of the Exchequer, whose emoluments in time of war were excessive, and grew in exact proportion to the public burdens, were regulated,—some of them suppressed, and the rest reduced to fixed salaries. To secure the freedom of election against the crown, a bill was passed to disqualify all officers concerned in the collection of the revenue in any of its branches from voting in elections: a most important act, not only with regard to its primary object, the freedom of election, but as materially forwarding the due collection of revenue. For the same end, (the preserving the freedom of election,) the House rescinded the famous judgment relative to the Middlesex election, and expunged it from the journals. On the principle of reformation of their own House, connected with a principle of public economy, an act passed for rendering contractors with government incapable of a seat in Parliament. The India Bill (unfortunately lost in the House of Lords) pursued the same idea to its completion, and disabled all servants of the East India Company from a seat in that House for a certain time, and until their conduct was examined into and cleared. The remedy of infinite corruptions and of infinite disorders and oppressions, as well as the security of the most important objects of public economy, perished with that bill and that Parliament. That Parliament also instituted a committee to inquire into the collection of the revenue in all its branches, which prosecuted its duty with great vigor, and suggested several material improvements.

% [62] If these speculations are let loose, the House of Lords may quarrel with their share of the legislature, as being limited with regard to the origination of grants to the crown and the origination of money bills. The advisers of the crown may think proper to bring its negative into ordinary use,—and even to dispute, whether a mere negative, compared with the deliberative power exercised in the other Houses, be such a share in the legislature as to produce a due balance in favor of that branch, and thus justify the previous interference of the crown in the manner lately used. The following will serve to show how much foundation there is for great caution concerning these novel speculations. Lord Shelburne, in his celebrated speech, April 8th, 1778, expresses himself as follows. (Vide Parliamentary Register, Vol. X.)

%"The noble and learned lord on the woolsack, in the debate which opened the business of this day, asserted that your Lordships were incompetent to make any alteration in a money bill or a bill of supply, I should be glad to see the matter fairly and fully discussed, and the subject brought forward and argued upon precedent, as well as all its collateral relations. I should be pleased to see the question fairly committed, were it for no other reason but to hear the sleek, smooth contractors from the other House come to this bar and declare, that they, and they only, could frame a money bill, and they, and they only, could dispose of the property of the peers of Great Britain. Perhaps some arguments more plausible than those I heard this day from the woolsack, to show that the Commons have an uncontrollable, unqualified right to bind your Lordships' property, may be urged by them. At present, I beg leave to differ from the noble and learned lord; for, until the claim, after a solemn discussion of this House, is openly and directly relinquished, I shall continue to be of opinion that your Lordships have a right to after, amend, or reject a money bill."

%The Duke of Richmond also, in his letter to the volunteers of Ireland, speaks of several of the powers exercised by the House of Commons in the light of usurpations; and his Grace is of opinion, that, when the people are restored to what he conceives to be their rights, in electing the House of Commons, the other branches of the legislature ought to be restored to theirs.—Vide Remembrancer, Vol. XVI.

% [63] By an act of Parliament, the Directors of the East India Company are restrained from acceptance of bills drawn, from India, beyond a certain amount, without the consent of the Commissioners of the Treasury. The late House of Commons, finding bills to an immense amount drawn upon that body by their servants abroad, and knowing their circumstances to be exceedingly doubtful, came to a resolution providently, cautioning the Lords of the Treasury against the acceptance of these bills, until the House should otherwise direct. The Court Lords then took occasion to declare against the resolution as illegal, by the Commons undertaking to direct in the execution of a trust created by act of Parliament. The House, justly alarmed at this resolution, which went to the destruction of the whole of its superintending capacity, and particularly in matters relative to its own province of money, directed a committee to search the journals, and they found a regular series of precedents, commencing from the remotest of those records, and carried on to that day, by which it appeared that the House interfered, by an authoritative advice and admonition, upon every act of executive government without exception, and in many much stronger cases than that which the Lords thought proper to quarrel with.

% [64] "I observe, at the same time, that there is no charge or complaint suggested against my present ministers."—The King's Answer, 25th February, 1784, to the Address of the House of Common. Vide Resolutions of the House of Commons, printed for Debrett, p. 31.

% [65] The territorial possessions in the East Indies were acquired to the Company, in virtue of grants from the Great Mogul, in the nature of offices and jurisdictions, to be held under him, and dependent upon his crown, with the express condition of being obedient to orders from his court, and of paying an annual tribute to his treasury. It is true that no obedience is yielded to these orders, and for some time past there has been no payment made of this tribute. But it is under a grant so conditioned that they still hold. To subject the King of Great Britain as tributary to a foreign power by the acts of his subjects; to suppose the grant valid, and yet the condition void; to suppose it good for the king, and insufficient for the Company; to suppose it an interest divisible between the parties: these are some few of the many legal difficulties to be surmounted, before the Common Law of England can acknowledge the East India Company's Asiatic affairs to be a subject matter of prerogative, so as to bring it within the verge of English jurisprudence. It is a very anomalous species of power and property which is held by the East India Company. Our English prerogative law does not furnish principles, much less precedents, by which it can be defined or adjusted. Nothing but the eminent dominion of Parliament over every British subject, in every concern, and in every circumstance in which he is placed, can adjust this new, intricate matter. Parliament may act wisely or unwisely, justly or unjustly; but Parliament alone is competent to it.

% [66] The attempt upon charters and the privileges of the corporate bodies of the kingdom in the reigns of Charles the Second and James the Second was made by the crown. It was carried on by the ordinary course of law, in courts instituted for the security of the property and franchises of the people. This attempt made by the crown was attended with complete success. The corporate rights of the city of London, and of all the companies it contains, were by solemn judgment of law declared forfeited, and all their franchises, privileges, properties, and estates were of course seized into the hands of the crown. The injury was from the crown: the redress was by Parliament. A bill was brought into the House of Commons, by which the judgment against the city of London, and against the companies, was reversed: and this bill passed the House of Lords without any complaint of trespass on their jurisdiction, although the bill was for a reversal of a judgment in law. By this act, which is in the second of William and Mary, chap. 8, the question of forfeiture of that charter is forever taken out of the power of any court of law: no cognizance can be taken of it except in Parliament.

%Although the act above mentioned has declared the judgment against the corporation of London to be illegal yet Blackstone makes no scruple of asserting, that, "perhaps, in strictness of law, the proceedings in most of them [the Quo Warranto causes] were sufficiently regular," leaving it in doubt, whether this regularity did not apply to the corporation of London, as well as to any of the rest; and he seems to blame the proceeding (as most blamable it was) not so much on account of illegality as for the crown's having employed a legal proceeding for political purposes. He calls it "an exertion of an act of law for the purposes of the state."

%The same security which was given to the city of London, would have been extended to all the corporations, if the House of Commons could have prevailed. But the bill for that purpose passed but by a majority of one in the Lords; and it was entirely lost by a prorogation, which is the act of the crown. Small, indeed, was the security which the corporation of London enjoyed before the act of William and Mary, and which all the other corporations, secured by no statute, enjoy at this hour, if strict law was employed against them. The use of strict law has always been rendered very delicate by the same means by which the almost unmeasured legal powers residing (and in many instances dangerously residing) in the crown are kept within due bounds: I mean, that strong superintending power in the House of Commons which inconsiderate people have been prevailed on to condemn as trenching on prerogative. Strict law is by no means such a friend to the rights of the subject as they have been taught to believe. They who have been most conversant in this kind of learning will be most sensible of the danger of submitting corporate rights of high political importance to these subordinate tribunals. The general heads of law on that subject are vulgar and trivial. On them there is not much question. But it is far from easy to determine what special acts, or what special neglect of action, shall subject corporations to a forfeiture. There is so much laxity in this doctrine, that great room is left for favor or prejudice, which might give to the crown an entire dominion over those corporations. On the other hand, it is undoubtedly true that every subordinate corporate right ought to be subject to control, to superior direction, and even to forfeiture upon just cause. In this reason and law agree. In every judgment given on a corporate right of great political importance, the policy and prudence make no small part of the question. To these considerations a court of law is not competent; and, indeed, an attempt at the least intermixture of such ideas with the matter of law could have no other effect than wholly to corrupt the judicial character of the court in which such a cause should come to be tried. It is besides to be remarked, that, if, in virtue of a legal process, a forfeiture should be adjudged, the court of law has no power to modify or mitigate. The whole franchise is annihilated, and the corporate property goes into the hands of the crown. They who hold the new doctrines concerning the power of the House of Commons ought well to consider in such a case by what means the corporate rights could be revived, or the property could be recovered out of the hands of the crown. But Parliament can do what the courts neither can do nor ought to attempt. Parliament is competent to give due weight to all political considerations. It may modify, it may mitigate, and it may render perfectly secure, all that it does not think fit to take away. It is not likely that Parliament will ever draw to itself the cognizance of questions concerning ordinary corporations, farther than to protect them, in case attempts are made to induce a forfeiture of their franchises.

%The case of the East India Company is different even from that of the greatest of these corporations. No monopoly of trade, beyond their own limits, is vested in the corporate body of any town or city in the kingdom. Even within these limits the monopoly is not general. The Company has the monopoly of the trade of half the world. The first corporation of the kingdom has for the object of its jurisdiction only a few matters of subordinate police. The East India Company governs an empire, through all its concerns and all its departments, from the lowest office of economy to the highest councils of state,—an empire to which Great Britain is in comparison but a respectable province. To leave these concerns without superior cognizance would be madness; to leave them to be judged in the courts below, on the principles of a confined jurisprudence, would be folly. It is well, if the whole legislative power is competent to the correction of abuses which are commensurate to the immensity of the object they affect. The idea of an absolute power has, indeed, its terrors; but that objection lies to every Parliamentary proceeding; and as no other can regulate the abuses of such a charter, it is fittest that sovereign authority should be exercised, where it is most likely to be attended with the most effectual correctives. These correctives are furnished by the nature and course of Parliamentary proceedings, and by the infinitely diversified characters who compose the two Houses. In effect and virtually, they form a vast number, variety, and succession of judges and jurors. The fulness, the freedom, and publicity of discussion leaves it easy to distinguish what are acts of power, and what the determinations of equity and reason. There prejudice corrects prejudice, and the different asperities of party zeal mitigate and neutralize each other. So far from violence being the general characteristic of the proceedings of Parliament, whatever the beginnings of any Parliamentary process may be, its general fault in the end is, that it is found incomplete and ineffectual.

% [67] The purpose of the misrepresentation being now completely answered, there is no doubt but the committee in this Parliament, appointed by the ministers themselves, will justify the grounds upon which the last Parliament proceeded, and will lay open to the world the dreadful state of the Company's affairs, and the grossness of their own calumnies upon this head. By delay the new assembly is come into the disgraceful situation of allowing a dividend of eight per cent by act of Parliament, without the least matter before them to justify the granting of any dividend at all.

% [68] This will be evident to those who consider the number and description of Directors and servants of the East India Company chosen into the present Parliament. The light in which the present ministers hold the labors of the House of Commons in searching into the disorders in the Indian administration, and all its endeavors for the reformation of the government there, without any distinction of times, or of the persons concerned, will appear from the following extract from a speech of the present Lord Chancellor. After making a high-flown panegyric on those whom the House of Commons had condemned by their resolutions, he said:—"Let us not be misled by reports from committees of another House, to which, I again repeat, I pay as much attention as I would do to the history of Robinson Crusoe, Let the conduct of the East India Company be fairly and fully inquired into. Let it be acquitted or condemned by evidence brought to the bar of the House. Without entering very deeply into the subject, let me reply in a few words to an observation which fell from a noble and learned lord, that the Company's finances are distressed, and that they owe at this moment a million sterling to the nation. When such a charge is brought, will Parliament in its justice forget that the Company is restricted from employing that credit which its great and flourishing situation gives to it?"

